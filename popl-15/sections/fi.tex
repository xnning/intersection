\section{The \name calculus} \label{sec:fi}

This section presents the syntax, subtyping, and typing of \name.  The
semantics of \name will be defined by a type-directed translation
to a simple variant of System $F$ in the next section.

\subsection{Syntax}

Figure~\ref{fig:fi-syntax} shows the syntax of \name (with the addition to
System $F$ highlighted). As a convention in this paper, we will be using
lowercase letters as meta-variables for sorts in \name, and uppercase letters
for those in the target language.

% To System $ F $, we add two features: intersection types and single-field
% records.
% % ~\bruno{labelled types (single field records are fine too)}
% We include only single records because single record types as the multi-records
% can be desugared into the merge of multiple single records.

\begin{figure}
  \[
  \begin{array}{l}
    \begin{array}{llrll}
      \text{Types}
      & T & \Coloneqq & \alpha     & \text{Type variable} \\
      &         & \mid & \bot            & \text{Bottom type} \\
      &        & \mid & A \to B         & \text{Function type}\\
      &         & \mid & \for {\alpha * B} A   & \text{Universal quantification} \\
      &         & \mid & A \intersect B  & \text{Intersection type} \\
%      &         & \mid & \constraints {A \disjoint B} C & \text{Disjoint constraint} \\
      % &         & \mid & \recordType l A & \text{Record type} \\
%     \\
%     \text{Types}
%     & A, B, C, D & \Coloneqq & \alpha     & \text{Type variable} \\
%     &  & \mid & A \intersect B  & \text{Intersection type} \\
%     &         & \mid & A \intersect B  & \text{Intersection type} \\
%     &         & \mid & T  & \text{Atomic type} \\

      \\
      \text{Expressions}
      & e & \Coloneqq & x            & \text{Variable} \\
%      &   & \mid & \top              & \text{Top} \\
      &   & \mid & \lam x A e        & \text{Lambda} \\
      &   & \mid & \app {e_1} {e_2}  & \text{Application} \\
      &   & \mid & \blam {\alpha * A}  e    & \text{Big lambda} \\
      &   & \mid & \tapp e A         & \text{Type application} \\
      &   & \mid &  e_1 \mergeOp e_2 & \text{Merge} \\
      % &   & \mid & {\_}                 & \text{Disjointness evidence}
%      &   & \mid & \assume {(A \disjoint B)} e & \text{Constraint intro} \\
%      &   & \mid & \app e {\_}                 & \text{Constraint elim} \\
      % &   & \mid & \recordCon l e    & \text{Record} \\
      % &   & \mid & e.l               & \text{Record selection} \\
      % &   & \mid & e \restrictOp l   & \text{Record restriction} \\

      \\
      \text{Contexts}
      & \Gamma & \Coloneqq & \epsilon         & \\
      &        & \mid & \Gamma, \alpha * A       & \\
      &        & \mid & \Gamma, x \hast A     & \\
%      &        & \mid & \Gamma, A \disjoint B & \\

      % \\
      \text{Sugar} & \blam {\alpha} e & \equiv & \blam {\alpha * \bot} e & \\
                   & e : A &  \equiv & (\lambda z : A . z) e &
    \end{array}
  \end{array}
  \]

  \caption{Syntax.}
  \label{fig:fi-syntax}
\end{figure}

% \bruno{I am not sure if the highlighting will be
%   visible. Use gray?}\bruno{there is no \lstinline{with} anymore.
% Make sure the syntax is consistent with what is presented in the type rules!}

Meta-variables $\tau$ range over types. Types include System $F$ constructs:
type variables $\alpha$; function types $\tau_1 \to \tau_2$; and type
abstraction $ \for \alpha \tau $. The top type $\top$ is the supertype of all
types. $\tau_1 \intersect \tau_2$ denotes
the intersection of types $\tau_1$ and $\tau_2$; and $\recordType l \tau$ the types
for single-field records. Interestingly, single-field record types can be viewed as types
tagged with a label $l$, while the other type forms are untagged types. We omit
type constants such as \lstinline$Int$ and \lstinline$String$.
% \bruno{be consistent: in the source language we use \&, and in the
% formalization we use $\intersect$}

Expressions include standard constructs in System
$F$: variables $ x $; abstraction of expressions over variables of a given type
$\lam x \tau e$ (concretely written \lstinline$\(x:T) -> e$);
abstraction of expressions over types $\blam \alpha e$ (concretely written \lstinline$/\A -> e$);
application of expressions to expressions $\app {e_1} {e_2}$;
and application of expressions to types $\tapp e \tau$ (concretely written \lstinline$e[T]$).
The last four constructs are novel.
The canonical term that inhabits the top type is also written as $\top$.
$e_1 \mergeOp e_2$ is the \emph{merge} of two expressions $e_1$ and $e_2$.
% \bruno{explain further, after all this is new.}
It can be used as either $ e_1 $ or $ e_2 $. In particular, if one regards $e_1$
and $e_2$ as objects, their merge will respond to every method that one or
both of them have. Merge of expressions correspond to intersection types
$ \tau_1 \intersect \tau_2 $. The expression $ \recordCon l e $ constructs a
single-field record, whereas $ e.l $ accesses the field labelled $ l $ in $ e $.
\begin{comment}
Note that $ e $ does not
need to be a record type in this case. For example, although the merge of two
records
\[
x = \recordCon {l_1} {e_1} \mergeOp \recordCon {l_1} {e_2}
\]
is of an intersection type, $ x.{l_1} $ still gives $ e_1 $. On the other hand,
$ x.{l_3} $ will be rejected by the type system.
\end{comment}
Restriction $e \restrictOp l$ clears the field $l$ inside $e$. In order to
focus on the most essential features, we do not include other forms such as
fixpoints here, although they are supported in our implementation and can
be included in formalization in standard ways.

Typing contexts $ \gamma $ track bound type variables and variables (and their
type $\tau$). We use $l$ for labels and use
$\subst {\tau_1} \alpha {\tau_2}$ for the capture-avoiding substitution of
$\tau_1$ for $\alpha$ inside $\tau_2$ and $\ftv \cdot$ for sets of free
variables.

\begin{comment}
\paragraph{Discussion.} A natural question the reader might ask is that why we
have excluded union types from the language. The answer is we found that
intersection types alone are enough support extensible designs. To focus on the
key features that make this language interesting, we also omit other common
constructs. For example, fixpoints can be added in standard ways.
% Dunfield has described a language that includes a ``top'' type but it does not
% appear in our language. Our work differs from Dunfield in that ...
\end{comment}

\subsection{Subtyping}

\begin{comment}
In some calculi, the subtyping relation is external to the language: those
calculi are indifferent to how the subtyping relation is defined. In \name, we
take a syntatic approach, that is, subtyping is due to the syntax of types.
However, this approach does not preclude integrating other forms of subtyping
into our system. One is ``primitive'' subtyping relations such as natural
numbers being a subtype of integers. The other is nominal subtyping relations
that are explicitly declared by the programmer.
\end{comment}

\begin{figure*}
  \begin{mathpar}
    \framebox{$ A \subtype B \yields F $} \\

    \subVar

    \subFun

    \subForall

    \subAnd

    \subAndleft

    \subAndright
  \end{mathpar}

  \footnote{miss a rule for $\bot$}

  \caption{Subtyping in \name.}
  \label{fig:fi-subtyping}
\end{figure*}

The subtyping rules of \name, shown in Figure~\ref{fig:fi-subtyping}, are syntax-directed
(different from the approach by Davies and Pfenning~\cite{davies2000intersection},
and Frisch et. al~\cite{frisch2008semantic}). The rule $\rulelabelsubFun$ says that a function is
contravariant in its parameter type and covariant in its return type. A
universal quantifier ($\forall$) is covariant in its body. A single-field record
type is also covariant, which becomes obvious if we regard it as a labelled
type. The three rules dealing with intersection types are just what one would
expect when interpreting types as sets. Under this interpretation, for example,
the rule $\rulelabelsubAnd$ says that if $\tau_1$ is both the subset of $\tau_2$
and the subset of $\tau_3$, then $\tau_1$ is also the subset of the intersection of $\tau_2$
and $\tau_3$.

% Intersection types introduce natural subtyping relations among types. For
% example, $ \tyint \intersect \tybool $ should be a subtype of $ \tyint $, since the former
% can be viewed as either $ \tyint $ or $ \tybool $. To summarize, the subtyping rules
% are standard except for three points listed below:
% \begin{enumerate}
% \item $ \tau_1 \intersect \tau_2 $ is a subtype of $ \tau_3 $, if \emph{either} $ \tau_1 $ or
%   $ \tau_2 $ are subtypes of $ \tau_3 $,

% \item $ \tau_1 $ is a subtype of $ \tau_2 \intersect \tau_3 $, if $ \tau_1 $ is a subtype of
%   both $ \tau_2 $ and $ \tau_3 $.

% \item $ \recordType {l_1} {\tau_1} $ is a subtype of $ \recordType {l_2} {\tau_2} $, if
%   $ l_1 $ and $ l_2 $ are identical and $ \tau_1 $ is a subtype of $ \tau_2 $.
% \end{enumerate}
% The first point is captured by two rules $ \rulelabelsubAndLeft $ and
% $ \rulelabelsubAndRight $, whereas the second point by $ \rulelabelsubAnd $.
% Note that the last point means that record types are covariant in the type of
% the fields.

It is easy to see that subtyping is reflexive and transitive.

\begin{lemma}[Subtyping is reflexive] \label{sub-refl}
Given a type $ \tau $, $ \tau \subtype \tau $.
\end{lemma}

\begin{lemma}[Subtyping is transitive] \label{sub-trans}
If $ \tau_1 \subtype \tau_2 $ and $ \tau_2 \subtype \tau_3 $,
then $ \tau_1 \subtype \tau_3 $.
\end{lemma}

For the corresponding mechanized proofs in Coq, we refer to the supplementary
materials submitted with the paper.

\subsection{Typing}

\begin{figure*}
  \begin{mathpar}
    \framebox{$ \hastype \Gamma e A \yields E $} \\

    \tyvar

    \tylam

    \tyapp

    \tyblam

    \tytapp

    \tymerge

  \end{mathpar}

  \caption{The type system of \name.}
  \label{fig:fi-typing}
\end{figure*}

The syntax-directed typing rules of \name are shown in Figure~\ref{fig:fi-typing}. They consist of one
main typing judgment and two auxiliary judgments. The main typing judgment is of
the form: $ \hastype \gamma e \tau $. It reads: ``in the typing context
$\gamma$, the expression $e$ is of type $\tau$''. The rules that are the same as
in System $F$ are rules for variables (\rulelabel{Var}), lambda abstractions
(\rulelabel{Lam}), type abstractions (\rulelabel{BLam}), and type applications
(\rulelabel{TApp}). The rule \rulelabel{App} needs special attention as we add
a subtyping requirement: the type of the argument ($\tau_3$) is a
subtype of that of the parameter ($\tau_1$).
%The advantage is that it then
%becomes easier to derive an algorithm for typechecking.
For merges
$e_1 \mergeOp e_2$, we typecheck $e_1$ and $e_2$, and give it the
intersection of the resulting types. The rule for
single-field record construction is standard. The rules
for record selection and restriction
are expectedly the most complicated. They turn to the
auxiliary ``select'' and ``restrict'' rules to statically check operations and
to obtain resulting types.

% The last two rules make use of the $ \rulename{fields} $ function just to make
% sure that the field being accessed ($ \rulelabelRecSelect $) or updated
% ($ \rulelabelRecUpd $) actually exists. The function is defined recursively, in
% Haskell pseudocode, as:
% \[ \begin{array}{rll}
%   \fields{\alpha} & = & \rel{\cdot} {\alpha} \\
%   \fields{\tau_1 \to \tau_2} & = & \rel{\cdot} {\tau_1 \to \tau_2} \\
%   \fields{\forall \alpha. \tau} & = & \rel{\cdot} {\forall \alpha. \tau} \\
%   \fields{\tau_1 \intersect \tau_2} & = & \fields{\tau_1} \dplus \fields{\tau_2} \\
%   \fields{\recordType l \tau} & = & \rel l t
% \end{array} \]
% where $ \cdot $ means empty list, $ \dplus $ list concatenation, and $ : $ is an
% infix operator that prepend the first argument to the second. The function
% returns an associative list whose domain is field labels and range types.
