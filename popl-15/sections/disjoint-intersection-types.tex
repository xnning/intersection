\section{Disjoint intersection types and disjoint quantification}

This section shows how to restrict the system presented before
so that it supports coherence as well as type soundness.
The keys aspects are the notion of disjoint intersections,
and disjoint quantification for polymorphic types.

In type systems with a top type (such as \lstinline@Object@ in some OO languages), the definition of disjointness can be modified to:

We say two types are \emph{disjoint} if their only common supertype is the top type.

The intuition can be found in figure \ldots.

% http://tex.stackexchange.com/questions/158876/drawing-subgroup-lattices-in-tikz
\begin{figure}

% center everything in the figure
\centering
% horizontal node distance
\newcommand{\mydistance}{.6cm}
\begin{tikzpicture}[node distance=2cm]
\title{Untergruppenverband der $A_4$}
\node(A4)                           {$A_4$};
\node(V4)       [below right=2cm and 2cm of A4] {$V_4$};
\node(C31)      [below left=2cm and 0cm of A4]  {$C_3$};
\node(C32)      [left=\mydistance of C31]       {$C_3$};
\node(C33)      [left=\mydistance of C32]       {$C_3$};
\node(C34)      [left=\mydistance of C33]       {$C_3$};
\node(C22)      [below=2cm of V4]       {$C_2$};
\node(C21)      [left=\mydistance of C22]       {$C_2$};
\node(C23)      [right=\mydistance of C22]      {$C_2$};
\node(1)            [below=6cm of A4]     {$\left\{1\right\}$};
\draw(A4)       -- (V4);
\foreach \x\y in {1,2,3,4} {
    \draw (A4) -- (C3\x) node [midway, fill=white] {3};
    \draw (C3\x) -- (1);

}
\foreach \x\y in {1/2,2/3,3/4} {
    \draw(V4) -- (C2\x) node [midway, fill=white] {2};
\draw (C3\x) -- (C3\y);
\draw (C2\x) -- (1);
}
\draw(C21)      -- (C22);
\draw(C22)      -- (C23);
\end{tikzpicture}
\caption{Untergruppenverband}
\end{figure}

\paragraph{Restrictions on subtyping}:

The subtyping rules, without the atomic condition are overlapping. With the atomic constraint, one can guarantee that at any moment during the derivation of a subtyping relation, at most one rule can be used. Indeed, our restrictions on subtyping do not make the subtyping relation less expressive to one without such restrictions.\todo{Point to proofs}

If we would like to have a deterministic elaboration result, another idea is to
tweak the rules a little bit so that given a term, it is no longer possible that
both of the twin rules described above can be used. For example, if
$A_1 \inter A_2 \subtype
A_3$, we would like to be certain that either $A_1 \subtype
A_3$ holds or $A_2 \subtype A_3$ holds, but not both.

Note that $A$ \emph{exclusive} or $B$ is true if and only if their truth value
differ. Next, we are going to investigate the minimal requirement (necessary and
sufficient conditions) such that the theorem holds.

If $A_1$ and $A_2$ in this setting are the same, for example,
$\tyint \inter \tyint \subtype \tyint$, obviously the theorem will
not hold since both the left $\tyint$ and the right $\tyint$ are a
subtype of $\tyint$.

If our types include primitive subtyping such as
$\code{Nat} \subtype_\text{prim} \tyint$ (a natural number is also an
integer), which can be promoted to the normal subtyping with this rule:
\begin{mathpar}
  \inferrule
  {A_1 \subtype_\text{prim} A_2}
  {A_1 \subtype A_2}
\end{mathpar}
the theorem will also not hold because
$\tyint \inter \code{Nat} \subtype \tyint$ and yet
$\tyint \subtype \tyint$ and $\code{Nat} \subtype \tyint$.

We can try to rule out such possibilities by making the requirement of
well-formedness stronger. This suggests that the two types on the sides of
$\inter$ should not ``overlap''. In other words, they should be ``disjoint''. It
is easy to determine if two base types are disjoint. For example, $\tyint$
and $\tyint$ are not disjoint. Neither do $\tyint$ and $\code{Nat}$.
Also, types built with different constructors are disjoint. For example,
$\tyint$ and $\tyint \to \tyint$. For function types, disjointness
is harder to visualise. But bear in the mind that disjointness can defined by
the very requirement that the theorem holds.

With the change, we need $\tyint \subtype \tyint \inter \code{Char}$ to
hold in order to get the premise, which does not. So it can be shown that
$(\tyint \inter \code{Char}) ((1 \mergeop 'c') : \tyint \inter
\code{Char}) \hookrightarrow 1$ is not derivable.

\george{Add interpretation of the theorem}

\begin{theorem}
  If $A \subtype C$, then $A \inter B \subtype C$.
  If $B \subtype C$, then $A \inter B \subtype C$.
\end{theorem}

\begin{proof}
  By induction on $C$.
  If $C \neq C_1 \inter C_2$, trivial.
  If $C = C_1 \inter C_2$,
  Need to show $A \subtype C_1 \inter C_2$ implies $A \inter B \subtype C_1 \inter C_2$.
  By inversion $A \subtype C_1$ and $A \subtype C_2$.
  By the i.h., $A \inter B \subtype C_1$ and $A \inter B \subtype C_2$.
  By \rulelabel{SubAnd}, $A \inter B \subtype C_1 \inter C_2$.
\end{proof}

\subsection{Disjointness}

Spec of disjointness/intuition ...

We say two types are \emph{disjoint} if they do not share a common supertype.

\begin{definition}[Disjointness]
$A \bot B = \not \exists C. A <: C \wedge B <: C$
\end{definition}

\subsection{Well-formed types}

\subsection{Subtyping}

\subsection{Metatheory}

\begin{definition}{Type variable constraint}
We say the \emph{constraint} of a type variable $\alpha$ inside the context $\Gamma$ is $A$ if $\alpha \disjoint A \in \Gamma$.
\end{definition}

% \begin{lemma}
% If $A \subtype B$ where both $A$ and $B$ are well-formed, then $A$ and $B$ cannot be disjoint.
% \end{lemma}
%
% \begin{proof}
% $A \subtype B$ implies $B$ is a common supertype of $A$ and $B$. As a result, $A$ and $B$ are not disjoint by definition.
% \end{proof}

\begin{lemma}[Free type variables of disjoint bounds] \label{free-var-disjoint-bounds}
  If $\isdisjoint \Gamma \alpha A$, then $\alpha \not \in \ftv A$.
\end{lemma}

\begin{lemma}[Unique subtype contributor] \label{unique-subtype-contributor}
If $A \inter B \subtype C$, where $A \inter B$ and $C$ are well-formed types, then it is not possible that the following hold at the same time:
\begin{enumerate}
\item $A \subtype C$
\item $B \subtype C$
\end{enumerate}
\end{lemma}

If $A \inter B \subtype C$, then either $A$ or $B$ contributes to that subtyping relation, but not both. The implication of this lemma is that during the derivation, it is not possible that two rules are applicable.

\newcommand{\wfinterlabel}{\textsc{WFInter}}

\begin{proof}
Since $A \inter B$ is well-formed, $A \disjoint B$ by the formation rule of intersection types \wfinterlabel. Then by the definition of disjointness, there does not exist a type $C$ such that $A \subtype C$ and $B \subtype C$. It follows that $A \subtype C$ and $B \subtype C$ cannot hold simultaneously.
\end{proof}

The coercion of a subtyping relation $A \subtype B$ is uniquely determined.

\begin{lemma}[Unique coercion] \label{unique-coercion}
If $A \subtype B \yields {C_1}$ and $A \subtype B \yields {C_2}$, where $A$ and $B$ are well-formed types, then $C_1 \equiv C_2$
\end{lemma}

\begin{proof}
The set of rules for generating coercions is syntax-directed except for the three rules that involve intersection types in the conclusion. Therefore it suffices to show that if well-formed types $A$ and $B$ satisfy $A \subtype B$, where $A$ or $B$ is an intersection type, then at most one of the three rules applies. In the following, we do a case analysis on the shape of $A$ and $B$:

\begin{itemize}
  \item \textbf{Case} $A \neq A_1 \inter A_2$ and $B = B_1 \inter B_2$: Clearly only \textsc{SubAnd} can apply.
  \item \textbf{Case} $A = A_1 \inter A_2$ and $B \neq B_1 \inter B_2$: Only two rules can apply, \textsc{SubAnd1} and \textsc{SubAnd2}. Further, by the unique subtype contributor lemma, it is not possible that $A_1 \subtype B$ and that $A_2 \subtype B$. Thus we are certain that at most one rule of \textsc{SubAnd1} and \textsc{SubAnd2} will apply.
  \item \textbf{Case} $A = A_1 \inter A_2$ and $B = B_1 \inter B_2$\footnote{An example of this case is:
    \[ (\integer \inter \bool) \inter \character \subtype \bool \inter \character \]}: Since $B$ is not atomic, only \rulelabel{SubAnd} apply.

  %   Suppose the contrary, that is, more than one of the three rules apply. Since it is not possible that both \textsc{SubAnd1} and \textsc{SubAnd2} apply by the unique subtype contributor lemma, only one of \textsc{SubAnd1} and \textsc{SubAnd2} apply. Therefore \textsc{SubAnd} has to hold. Without the loss of generality, assume \textsc{SubAnd1} apply. Then we have:
  % \[ A_1 \subtype B_1 \inter B_2 \]
  % \[ A_1 \inter A_2 \subtype B_1 \]
  % \[ A_1 \inter A_2 \subtype B_2 \]
\end{itemize}
\end{proof}

In general, disjointness judgements are not invariant with respect to free-variable substitution. In other words, a careless substitution can violate the disjoint constraint in the context. For example, in the context $\alpha \disjoint \tyint$, $\alpha$ and $\tyint$ are disjoint:
\begin{mathpar}
\isdisjoint {\alpha \disjoint \tyint} \alpha \tyint
\end{mathpar}
But after the substitution of $\tyint$ for $\alpha$ on the two types, the sentence
\begin{mathpar}
\isdisjoint {\alpha \disjoint \tyint} \tyint \tyint
\end{mathpar}
is longer true since $\tyint$ is clearly not disjoint with itself.

\begin{lemma}{Invariance of disjointness} \label{invariance-of-disjointness}
If $\isdisjoint \Gamma A B$ and $R$ respects the constraints of $\beta$, then $\isdisjoint \Gamma {\subst R \beta A} {\subst R \beta B}$.
\end{lemma}

This lemma says that substitution for free type variables preserves disjointness of types if the combination of the replacement type and the type variable is proven disjoint.

\begin{proof}
By induction on the derivation of $\isdisjoint \Gamma A B$.
\begin{itemize}
  \item Case \[ \disjointvar \]
  We need to show \[ \isdisjoint \Gamma {\subst R \beta \alpha} {\subst R \beta B} \]
  If $\beta$ is not equivalent to $\alpha$ and is not free in $B$, then the above trivially holds by the def. of the substitution function. Otherwise, if $\beta$ is equivalent to $\alpha$, then we need to show
  \[ \isdisjoint \Gamma R {\subst R \beta B} \]

  % Note that $\beta \not \in \ftv B$. Thus $B$ is equivalent to $\subst R \beta B$.
  %
  % If $\beta$ is not equivalent to $\alpha$, $\subst R \beta \alpha$ is equivalent to $\alpha$. Therefore $\isdisjoint \Gamma {\subst R \beta \alpha} {\subst R \beta B}$ is true.
  % If $\beta$ is equivalent to $\alpha$, then $\subst R \beta \alpha$ is equivalent to $R$ by the def. of the substitution function. It now remains to show \[ \isdisjoint \Gamma R B \].

  \item Case \[ \disjointinterleft \]
  By applying the i.h. and the def. of the substitution function.

  \item Case \[ \disjointinterright \]
  Similar.

  \item Case \[ \disjointfun \]
  By applying the i.h. and the def. of the substitution function.

  \item Case \[ \disjointforall \]
  By applying the i.h. and the def. of the substitution function. Note that $\alpha$ is fresh.

  \item Case \[ \disjointatomic \]
  Substitution does not change the shape of types when the variable case is excluded. Therefore, the relation in the premise of the rule continue to hold and hence the conclusion.

\end{itemize}
\end{proof}

\begin{lemma}{Substitution} \label{substitution}
If $\istype \Gamma R$, $\istype \Gamma S$, and $R$ respects the constraints of $\beta$, then $\istype \Gamma {\subst R \beta S}$.
\end{lemma}

\begin{proof}
By induction on the derivation of $\istype \Gamma {\subst R \beta S}$.

\begin{itemize}
  \item Case \[ \wfvar \]
  If $\alpha$ happens to be the same as $\beta$, then by the def. of substitution $\subst R \beta \alpha = R$. Since $\istype \Gamma R$, we have $\istype \Gamma {\subst R \beta \alpha}$; On the other hand, if not, then by the def. of substitution $\subst R \beta S = S$. Since $\istype \Gamma S$, we also have $\istype \Gamma {\subst R \beta \alpha}$.

  \item Case
  \begin{mathpar}
    \wfbot
  \end{mathpar}
  Trivial.

  \item Case
  \begin{mathpar}
    \wffun
  \end{mathpar}
  By i.h., $\istype \Gamma {\subst R \beta A}$ and $\istype \Gamma {\subst R \beta B}$. By the def. of substitution, $\istype \Gamma {\subst R \beta {A \to B}}$.

  \item Case
  \begin{mathpar}
    \wfforall
  \end{mathpar}
  By the premise and the i.h.,
  \[ \istype {\Gamma} {\subst R \beta A} \]
  \[ \istype {\Gamma, \alpha \disjoint A} {\subst R \beta B} \]
  which by \rulelabel{WFForall} implies
  \[ \istype \Gamma {\for {\alpha \disjoint A} {\subst R \beta B}} \]
  By the def. of substitution, $\istype \Gamma {\subst R \beta {\for {\alpha \disjoint A} B}}$~\todo{Subst. of $A$}.

  \item Case
  \begin{mathpar}
    \wfinter
  \end{mathpar}
  By i.h., $\istype \Gamma {\subst R \beta A}$ and $\istype \Gamma {\subst R \beta B}$. By Lemma~\ref{invariance-of-disjointness}, we also have $\isdisjoint \Gamma {\subst R \beta A} {\subst R \beta B}$. Therefore by \rulelabel{WFInter}, $\istype \Gamma {\subst R \beta {A \inter B}}$.
\end{itemize}
\end{proof}

\begin{lemma}{Instantiation} \label{instantiation}
If
  $\istype {\Gamma, \alpha \disjoint B} C$,
  $\istype \Gamma A$,
  $\isdisjoint \Gamma A B$
then
  $\istype \Gamma {\subst A \alpha C}$.
\end{lemma}

\begin{proof}
By induction.

\begin{itemize}
  \item Case \[ \wfvar \]
  If $C = \alpha$, then $\subst A \alpha \alpha = A$. Since $\istype \Gamma A$, it follows that $\istype \Gamma {\subst A \alpha \alpha}$; otherwise, let $C = \beta$, where $\beta$ is a type variable distinct from $\alpha$. Since $\istype {\Gamma, \alpha \disjoint B} \beta$ and $\alpha$ and $\beta$ are distinct, $\beta$ must be in $\Gamma$ and therefore $\istype {\Gamma} \beta$, which is equivalent to $\istype {\Gamma} {\subst A \alpha \beta}$.

  \item Case \[ \wffun \]
  By straightforwardly applying the i.h and the rule itself.

  \item Case \[ \wfbot \]
  Trivial.

  \item Case \[ \wfforall \]
  By straightforwardly applying the i.h and the rule itself.

  \item Case \[ \wfinter \]
  Let $C$ in the statement of this lemma be $C_1 \inter C_2$.
  By the condition we know
  \[ \istype {\Gamma, \alpha \disjoint B} {C_1 \inter C_2} \]
  Thus we must have,
  \[ \istype {\Gamma, \alpha \disjoint B} {C_1} \]
  By the i.h., $\istype \Gamma {\subst A \alpha {C_1}}$ and similarly $\istype \Gamma {\subst A \alpha {C_2}}$. By \rulelabel{WFInter}\todo{Show disjointness},
  \[ \istype \Gamma {\subst A \alpha {C_1} \inter \subst A \alpha {C_2}} \]
  and hence
  \[ \istype \Gamma {\subst A \alpha {(C_1 \inter C_2)}} \]

\end{itemize}

\end{proof}

\begin{lemma}{Well-formed typing} \label{wf-typing}
If $\hastype \Gamma e A$, then $\istype \Gamma e$.
\end{lemma}
Typing always produces a well-formed type.
\begin{proof}
By induction on the derivation of $\hastype \Gamma e A$. The case of \rulelabel{TyTApp} needs special attention
\begin{mathpar}
  \tytapp
\end{mathpar}
because we need to show that the result of substitution ($\subst A \alpha C$) is well-formed, which is evident by Lemma~\ref{instantiation}.
\end{proof}

\begin{theorem}[Unique elaboration] \label{unique-elaboration}
If $\hastype \Gamma e {A_1} \yields {E_1}$ and $\hastype \Gamma e {A_2} \yields {E_2}$, then $E_1 \equiv E_2$.
\end{theorem}
Given a source term $e$, elaboration always produces the same target term $E$.
\begin{proof}
The typing rules are syntax-directed. The case of \rulelabel{TyApp} needs special attention since we still need to show that the generated coercion $C$ is unique.
\begin{mathpar}
  \tyapp
\end{mathpar}
By Lemma~\ref{wf-typing}, we have $\istype \Gamma {A_1}$ and $\istype \Gamma {A_3}$. Therefore we are able to apply Lemma~\ref{unique-coercion} and conclude that $C$ is unique.
\end{proof}

\begin{figure}
  \begin{mathpar}
    \framebox{$\isatomic A$} \\

    \inferrule*
      {}
      {\isatomic \bot}

    \inferrule*
      {}
      {\isatomic {A \to B}}

    \inferrule*
      {}
      {\isatomic {\for {\alpha \disjoint B} A}}

  \end{mathpar}
  \caption{Atomic types.}
\end{figure}

\subsection{Disjointness and well-formedness}

\begin{figure}
  \begin{mathpar}
    \framebox{$\istype \Gamma A$} \\

    \wfvar

    \wfbot

    \wffun

    \wfforall

    \wfinter
  \end{mathpar}

  \label{fig:wf}
  \caption{Well-formed types.}
\end{figure}

The well-formedness of types is standard except that the two components of an
intersection type must be disjoint.