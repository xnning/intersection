\section{Introduction}

There has been a remarkable number of works aimed at improving support
for extensibility in programming languages. The motivation behind this
line of work is simple, and it is captured quite elegantly by the
infamous \emph{Expression Problem~}\cite{wadler1998expression}: there
are \emph{two} common and desirable forms of extensibility, but most
mainstream languages can only support one form well. Unfortunately
the lack of support in the other form has significant
consequences in terms of code maintenance and software evolution.  As a
result researchers proposed various approaches to address the problem,
including: visions of new programming
models~\cite{Prehofer97,Tarr99ndegrees,Harrison93subject}; new
programming languages or language
extensions~\cite{McDirmid01Jiazzi,Aracic06CaesarJ,Smaragdakis98mixin,nystrom2006j},
and \emph{design patterns} that can be used with existing mainstream
languages~\cite{togersen:2004,Zenger-Odersky2005,oliveira09modular,oliveira2012extensibility}.

Some of the more recent work on extensibility is focused on design
patterns. Examples include \emph{Object
  Algebras}~\cite{oliveira2012extensibility}, \emph{Modular Visitors}~\cite{oliveira09modular,togersen:2004} or
Torgersen's~\cite{togersen:2004} four design patterns using generics. In those
approaches the idea is to use some advanced (but already available)
features, such as \emph{generics}~\cite{Bracha98making}, in combination with conventional
OOP features to model more extensible designs.  Those designs work in
modern OOP languages such as Java, C\#, or Scala.

Although such design patterns give practical benefits in terms of
extensibility, they also expose limitations in existing mainstream OOP
languages. In particular there are three pressing limitations:
1) lack of good mechanisms for
  \emph{object-level} composition; 2) \emph{conflation of
    (type) inheritance with subtyping}; 3) \emph{heavy reliance on generics}.

  The first limitation shows up, for example, in encodings of Feature-Oriented
  Programming~\cite{Prehofer97} or Attribute Grammars~\cite{Knuth1968} using Object
  Algebras~\cite{oliveira2013feature,rendel14attributes}. These programs are best
  expressed using a form of \emph{type-safe}, \emph{dynamic},
  \emph{delegation}-based composition. Although such form of
  composition can be encoded in languages like Scala, it requires the
  use of low-level reflection techniques, such as dynamic proxies,
  reflection or other forms of meta-programming. It is clear
  that better language support would be desirable.

  The second limitation shows up in designs for modelling
  modular or extensible visitors~\cite{togersen:2004,oliveira09modular}.  The vast majority of modern
  OOP languages combines type inheritance and subtyping.
  That is, a type extension induces a subtype. However
  as Cook et al.~\cite{cook1989inheritance} famously argued there are programs where
  ``\emph{subtyping is not inheritance}''. Interestingly
  not many programs have been previously reported in the literature
  where the distinction between subtyping and inheritance is
  relevant in practice. However, as shown in this paper, it turns out that this
  difference does show up in practice when designing modular
  (extensible) visitors.  We believe that modular visitors provide a
  compelling example where inheritance and subtyping should
  not be conflated!

  Finally, the third limitation is prevalent in many extensible
  designs~\cite{togersen:2004,Zenger-Odersky2005,oliveira09modular,oliveira2013feature,rendel14attributes}.
  Such designs rely on advanced features of generics,
  such as \emph{F-bounded polymorphism}~\cite{Canning89f-bounded}, \emph{variance
    annotations}~\cite{Igarashi06variant}, \emph{wildcards}~\cite{Torgersen04wildcards} and/or \emph{higher-kinded
    types}~\cite{Moors08generics} to achieve type-safety. Sadly, the amount of
  type-annotations, combined with the lack of understanding of these
  features, usually deters programmers from using such designs.

\begin{comment}
Motivated by the insights gained in previous work, this paper presents
a minimal core calculus that addresses current limitations and
provides a better foundational model for statically typed
delegation-based OOP? We show that Object Algebras fit nicely in this
model.
\end{comment}

This paper presents System \name (pronounced \emph{f-and}): an extension of System F~\cite{Reynolds74f}
with intersection types and a merge operator~\cite{dunfield2014elaborating}.  The goal of
System \name is to study the \emph{minimal} foundational language
constructs that are needed to support various extensible designs,
while at the same time addressing the limitations of existing OOP
languages. To address the lack of good object-level composition
mechanisms, System \name uses the merge operator for dynamic
composition of values/objects. Moreover, in System \name (type-level)
extension is independent of subtyping, and it is possible for an
extension to be a supertype of a base object type. Furthermore,
intersection types and conventional subtyping can be used in many
cases instead of advanced features of generics. Indeed this paper
shows many previous designs in the literature can be encoded
without such advanced features of generics.


Technically speaking System \name is mainly inspired by the work of
Dundfield~\cite{dunfield2014elaborating}.  Dundfield showed how to model a simply typed
calculus with intersection types and a merge operator. The presence of
a merge operator adds significant expressiveness to the language,
allowing encodings for many other language constructs as syntactic
sugar. System \name differs from Dundfield's work in a few
ways. Firstly, it adds parametric polymorphism and formalizes an
extension for records to support a basic form of objects. Secondly,
the elaboration semantics into System F is done directly from the
source calculus with subtyping.
% In contrast, Dunfield has an additional step which eliminates subtyping.
Finally, a non-technical difference
is that System \name is aimed at studying issues of OOP languages and
extensibility, whereas Dunfield's work was aimed at Functional
Programming and he did not consider application to extensibility.
Like many other foundational formal models for OOP (for
example $F_{<:}$~\cite{CMMS}), System \name is purely functional and it uses
structural typing.

%%System \name is
%%formalized and implemented. Furthermore the paper illustrates how
%%various extensible designs can be encoded in System \name.

\begin{comment}
We present a polymorphic calculus containing intersection types and records, and show
how this language can be used to solve various common tasks in functional
programming in a nicer way.Intersection types provides a power mechanism for functional programming, in
particular for extensibility and allowing new forms of composition.

Prototype-based programming is one of the two major styles of object-oriented
programming, the other being class-based programming which is featured in
languages such as Java and C\#. It has gained increasing popularity recently
with the prominence of JavaScript in web applications. Prototype-based
programming supports highly dynamic behaviors at run time that are not possible
with traditional class-based programming. However, despite its flexibility,
prototype-based programming is often criticized over concerns of correctness and
safety. Furthermore, almost all prototype-based systems rely on the fact that
the language is dynamically typed and interpreted.
\end{comment}

In summary, the contributions of this paper are:

\begin{comment}
\george{Typing extensible records with a minimalistic design might also be of
  interest. We manage to achieve all what Daan Leijen achieved in
  \url{http://research.microsoft.com/pubs/65409/scopedlabels.pdf} except field
  renaming, with simpler approach.}
\end{comment}

\begin{itemize*}

\item {\bf A Minimal Core Language for Extensibility:} This paper
  identifies a minimal core language, System \name, capable of
  expressing various extensibility designs in the literature.
  System \name also addresses limitations of existing OOP
  languages that complicate extensible designs.

\item {\bf Formalization of System \name:} An elaboration semantics of
  System \name into System F is given, and type-soundness is proved.

\item {\bf Encodings of Extensible Designs:} Various encodings of
  extensible designs into System \name, including \emph{Object
    Algebras} and \emph{Modular Visitors}.

\item {\bf A Practical Example where ``Inheritance is not Subtyping''
    Matters:} This paper shows that modular/extensible visitors
  suffer from the ``inheritance is not subtyping problem''.
%% Moreover
%% with extensible visitors the extension should become a
%% \emph{supertype}, not a subtype. \bruno{extension with accept method}

\item {\bf Implementation:} An implementation of an
  extension of System \name, as well as the examples presented in the
  paper, are publicly available\footnote{{\bf Note to reviewers:} Due
    to the anonymous submission process, the code (and some machine
    checked proofs) is submitted as supplementary material.}.

\begin{comment}

\item{elaboration typing rules which given a source term with intersection
    types, typecheck and translate it into an ordinary F term. Prove a type
    preservation result: if a term $ e $ has type $ \tau $ in the source language,
    then the translated term $ \im e $ is well-typed and has type $ \im \tau $ in the
    target language.}

\item{present an algorithm for detecting incoherence which can be very important
    in practice.}

\item{explores the connection between intersection types and object algebra by
    showing various examples of encoding object algebra with intersection
    types.}

\end{comment}

\end{itemize*}

\begin{comment}
\subsection{Other Notes}

finitary overloading: yes
but have other merits of intersection been explored?

-- Compare Scala:
-- merge[A,B] = new A with B

-- type IEval  = { eval :  Int }
-- type IPrint = { print : String }

-- F[\_]
\end{comment}
