\section{Introduction}

Previous work by Dunfield~\cite{dunfield2014elaborating} has shown the power of intersection
types and a merge operator. The presence of a merge operator in a core
calculus provides significant expressiveness, allowing encodings for many
other language constructs as syntactic sugar. For example single-field
records are easily encoded as types with a label, and multi-field
records are encoded as the concatenation of single-field
records. Concatenation of records is expressed using intersection
types at the type-level and the corresponding merge operator at the
term level. Dunfield formalized a simply typed lambda calculus with
intersection types a the merge operator, and showed how to give a
semantics to the calculus by a type-directed translation to a simply
typed lambda calculus extended with pairs. The type-directed
translation is simple, elegant, and type-safe.

Intersection types and the merge operator are also useful in the
context of software \emph{extensibility}. In recent years there has
been a wide interest in presenting solutions to the \emph{expression
  problem}~\cite{wadler1998expression} in various
communities. Currently there are various solutions in functional
programming languages~\cite{swierstra:la-carte,emgm}, object-oriented
programming
languages~\cite{togersen:2004,Zenger-Odersky2005,oliveira09modular,oliveira2012extensibility}
and theorem provers~\cite{DelawareOS13,SchwaabS13}.  
Many of the proposed solutions for extensibility are closely related
to type-theoretic encodings of datatypes~\cite{BoehmBerarducci},
except that some form of subtyping is also involved.  Various
language-specific mechanisms are used to combine ideas from
type-theoretic encodings of datatypes with subtyping, but the essence
of the solutions is hidden behind the peculiarities of particular
programming languages.  Calculi with intersection types have a natural
subtyping relation that is helpful to model problems related to
extensibility.  Moreover, intersection types and an \emph{encoding} of
a merge operator have been shown to be useful to solve additional
challenges related to extensibility~\cite{oliveira2013feature}.  
Therefore it is natural to wonder if a
core calculus supporting parametric polymorphism,
intersection types and a merge operator, can be used to capture the
essence of various solutions to extensibility problems.

%Clearly this seems to indicate that a more
%foundational approach is lacking. What is currently missing is a
%foundational core calculus that can capture the key ideas behind the
%various solutions. 

Dunfield calculus seems to provide a good basis for a foundational
calculus for studying extensibility.  However, his calculus is still
insufficient for extensibility for two different reasons.  Firstly it does not
support parametric polymorhism. This is a pressing limitation because
type-theoretic encodings of datatypes fundamentally rely on parametric
polymorphism.  Secondly, and more importantly, while Dunfield calculus
is type-safe, it lacks the property of \emph{coherence}: different
derivations for the same expression can lead to different results. The
lack of coherence is an important disavantage for adoption of his core
calculus in implementations of programming languages, as the semantics
of the programming language becomes implementation dependent.
Moreover, from the theoretic point-of-view, the ambiguity that arizes
from the lack of coherence makes the calculus unsatisfying when the
goal is to precisely capture the essence of solutions to
extensibility.

This paper presents \name: a core calculus with a variant of
\emph{intersection types}, \emph{parametric polymorphism} and a
\emph{merge operator}. The semantics \name is both type-safe and
coherent. Thus \name addresses the two limitations of Dunfield
calculus and can be used to express the key ideas of extensible
type-theoretic encodings of datatypes.

Coherence is achieved by ensuring that intersection types are
\emph{disjoint}. Given two types $A$ and $B$, two types are disjoint
($A * B$) if there is no type $C$ such that both $A$ and $B$ are
subtypes of $C$. Formally this definition is captured as follows:
\[A * B \equiv \not\exists C.~A <: C \wedge B <: C\]
With this definition of disjointness we present a formal specification
of a type system that prevents intersection types that are not
disjoint.  However, the formal definition of disjointness does
not lend itself directly to an algorithmic implementation. Therefore,
we also present an algorithmic specification to determine whether two
types are disjoint. Moreover, this algorithmic specification is shown to be
sound and complete with respect to the formal definition of
disjointness.

We also show that disjoint intersection types can be extended to
support parametric polymorphism. Parametric polymorphism makes the
problem of coherence significantly harder. When a type variable occurs
in an intersection type, it is not statically known whether the
instantiated type will share a common supertype with other components
of the intersection. To address this problem we propose
\emph{disjoint quantification}: a constrained form of parametric
polymorphism, that allows programmers to specify disjointness
constraints for type variables. With disjoint quantification the
calculus remains very flexible in terms of programs that can be
written with intersection types, while retaining coherence.

In summary, the contributions of this paper are:

\begin{itemize*}

\item {\bf Disjoint Intersection Types:} A new form of intersection
  type where only disjoint types are allowed. A sound and complete
  algorithmic specification of disjointness (with respect to the
  corresponding formal definition) is presented.

\item {\bf Disjoint Quantification:} A novel form of universal
quantification where type variables can have disjointness
constraints.

\item {\bf Formalization of System \name and Proof of Coherence:} An
  elaboration semantics of System \name into System F is
  given. Type-soundness and coherence are proved.

\item {\bf Extensible Type-Theoretic Encodings:} We show that in \name 
 type-theoretic encodings can be combined with subtyping to provide extensibility.

\item {\bf Implementation:} An implementation of an
  extension of System \name, as well as the examples presented in the
  paper, are publicly available\footnote{{\bf Note to reviewers:} Due
    to the anonymous submission process, the code (and some machine
    checked proofs) is submitted as supplementary material.}.

\end{itemize*}

\begin{comment}

There has been a remarkable number of works aimed at improving support
for extensibility in programming languages. The motivation behind this
line of work is simple, and it is captured quite elegantly by the
infamous \emph{Expression Problem~}\cite{wadler1998expression}: there
are \emph{two} common and desirable forms of extensibility, but most
mainstream languages can only support one form well. Unfortunately
the lack of support in the other form has significant
consequences in terms of code maintenance and software evolution.  As a
result researchers proposed various approaches to address the problem,
including: visions of new programming
models~\cite{Prehofer97,Tarr99ndegrees,Harrison93subject}; new
programming languages or language
extensions~\cite{McDirmid01Jiazzi,Aracic06CaesarJ,Smaragdakis98mixin,nystrom2006j},
and \emph{design patterns} that can be used with existing mainstream
languages~\cite{togersen:2004,Zenger-Odersky2005,oliveira09modular,oliveira2012extensibility}.

Some of the more recent work on extensibility is focused on design
patterns. Examples include \emph{Object
  Algebras}~\cite{oliveira2012extensibility}, \emph{Modular Visitors}~\cite{oliveira09modular,togersen:2004} or
Torgersen's~\cite{togersen:2004} four design patterns using generics. In those
approaches the idea is to use some advanced (but already available)
features, such as \emph{generics}~\cite{Bracha98making}, in combination with conventional
OOP features to model more extensible designs.  Those designs work in
modern OOP languages such as Java, C\#, or Scala.

Although such design patterns give practical benefits in terms of
extensibility, they also expose limitations in existing mainstream OOP
languages. In particular there are three pressing limitations:
1) lack of good mechanisms for
  \emph{object-level} composition; 2) \emph{conflation of
    (type) inheritance with subtyping}; 3) \emph{heavy reliance on generics}.

  The first limitation shows up, for example, in encodings of Feature-Oriented
  Programming~\cite{Prehofer97} or Attribute Grammars~\cite{Knuth1968} using Object
  Algebras~\cite{oliveira2013feature,rendel14attributes}. These programs are best
  expressed using a form of \emph{type-safe}, \emph{dynamic},
  \emph{delegation}-based composition. Although such form of
  composition can be encoded in languages like Scala, it requires the
  use of low-level reflection techniques, such as dynamic proxies,
  reflection or other forms of meta-programming. It is clear
  that better language support would be desirable.

  The second limitation shows up in designs for modelling
  modular or extensible visitors~\cite{togersen:2004,oliveira09modular}.  The vast majority of modern
  OOP languages combines type inheritance and subtyping.
  That is, a type extension induces a subtype. However
  as Cook et al.~\cite{cook1989inheritance} famously argued there are programs where
  ``\emph{subtyping is not inheritance}''. Interestingly
  not many programs have been previously reported in the literature
  where the distinction between subtyping and inheritance is
  relevant in practice. However, as shown in this paper, it turns out that this
  difference does show up in practice when designing modular
  (extensible) visitors.  We believe that modular visitors provide a
  compelling example where inheritance and subtyping should
  not be conflated!

  Finally, the third limitation is prevalent in many extensible
  designs~\cite{togersen:2004,Zenger-Odersky2005,oliveira09modular,oliveira2013feature,rendel14attributes}.
  Such designs rely on advanced features of generics,
  such as \emph{F-bounded polymorphism}~\cite{Canning89f-bounded}, \emph{variance
    annotations}~\cite{Igarashi06variant}, \emph{wildcards}~\cite{Torgersen04wildcards} and/or \emph{higher-kinded
    types}~\cite{Moors08generics} to achieve type-safety. Sadly, the amount of
  type-annotations, combined with the lack of understanding of these
  features, usually deters programmers from using such designs.

This paper presents System \name (pronounced \emph{f-and}): an extension of System F~\cite{Reynolds74f}
with intersection types and a merge operator~\cite{dunfield2014elaborating}.  The goal of
System \name is to study the \emph{minimal} foundational language
constructs that are needed to support various extensible designs,
while at the same time addressing the limitations of existing OOP
languages. To address the lack of good object-level composition
mechanisms, System \name uses the merge operator for dynamic
composition of values/objects. Moreover, in System \name (type-level)
extension is independent of subtyping, and it is possible for an
extension to be a supertype of a base object type. Furthermore,
intersection types and conventional subtyping can be used in many
cases instead of advanced features of generics. Indeed this paper
shows many previous designs in the literature can be encoded
without such advanced features of generics.


Technically speaking System \name is mainly inspired by the work of
Dunfield~\cite{dunfield2014elaborating}.  Dunfield showed how to model a simply typed
calculus with intersection types and a merge operator. The presence of
a merge operator adds significant expressiveness to the language,
allowing encodings for many other language constructs as syntactic
sugar. System \name differs from Dunfield's work in a few
ways. Firstly, it adds parametric polymorphism and formalizes an
extension for records to support a basic form of objects. Secondly,
the elaboration semantics into System F is done directly from the
source calculus with subtyping.
% In contrast, Dunfield has an additional step which eliminates subtyping.
Finally, a non-technical difference
is that System \name is aimed at studying issues of OOP languages and
extensibility, whereas Dunfield's work was aimed at Functional
Programming and he did not consider application to extensibility.
Like many other foundational formal models for OOP (for
example $F_{<:}$~\cite{CMMS}), System \name is purely functional and it uses
structural typing.

%%System \name is
%%formalized and implemented. Furthermore the paper illustrates how
%%various extensible designs can be encoded in System \name.


In summary, the contributions of this paper are:


\begin{itemize*}

\item {\bf A Minimal Core Language for Extensibility:} This paper
  identifies a minimal core language, System \name, capable of
  expressing various extensibility designs in the literature.
  System \name also addresses limitations of existing OOP
  languages that complicate extensible designs.

\item {\bf Formalization of System \name:} An elaboration semantics of
  System \name into System F is given, and type-soundness is proved.

\item {\bf Encodings of Extensible Designs:} Various encodings of
  extensible designs into System \name, including \emph{Object
    Algebras} and \emph{Modular Visitors}.

\item {\bf A Practical Example where ``Inheritance is not Subtyping''
    Matters:} This paper shows that modular/extensible visitors
  suffer from the ``inheritance is not subtyping problem''.
%% Moreover
%% with extensible visitors the extension should become a
%% \emph{supertype}, not a subtype. \bruno{extension with accept method}

\item {\bf Implementation:} An implementation of an
  extension of System \name, as well as the examples presented in the
  paper, are publicly available\footnote{{\bf Note to reviewers:} Due
    to the anonymous submission process, the code (and some machine
    checked proofs) is submitted as supplementary material.}.

\end{itemize*}

\end{comment}

\begin{comment}
\subsection{Other Notes}

finitary overloading: yes
but have other merits of intersection been explored?

-- Compare Scala:
-- merge[A,B] = new A with B

-- type IEval  = { eval :  Int }
-- type IPrint = { print : String }

-- F[\_]
\end{comment}

% \section{Introduction}

% Dunfield's work showed how many language features can be encoded in terms
% of intersection types with a merge operator. However two important
% questions were left open by Dunfield:

% \begin{enumerate}
% \item How to allow coherent programs only?

% \item If a restriction that allows coherent programs is in place, can
%   all coherent programs conform to the restriction?
% \end{enumerate}

% In other words question 1) asks whether we can find sufficient
% conditions to guarantee coherency; whereas question 2) asks
% whether those conditions are also necessary. In terms of technical
% lemmas that would correspond to:

% \begin{enumerate}

% \item Coherency theorem: $\Gamma \turns e : A \leadsto E_1 \wedge
%   \Gamma \turns e : A \leadsto E_2~\to~E_1 = E_2$.

% \item Completness of Coherency: ($\Gamma \turns_{old} e : A \leadsto E_1 \wedge
%   \Gamma \turns_{old} e : A \leadsto E_2~\to~E_1 = E_2) \to \Gamma
%   \turns e : A$.

% \end{enumerate}

% For these theorems we assume two type systems. On liberal type system
% that ensures type-safety, but not coherence ($\Gamma \turns_{old} e :
% A$); and another one that is both type-safe and coherent  ($\Gamma \turns e :
% A$). What needs to be shown for completness is that if a coherent
% program type-checks in the liberal type system, then it also
% type-checks in the restricted system.


% \special{papersize=8.5in,11in}
% \setlength{\pdfpageheight}{\paperheight}
% \setlength{\pdfpagewidth}{\paperwidth}

% \title{\name}

% \subsection{``Testsuite'' of examples}

% \begin{enumerate}

% \item $\lambda (x : Int * Int). (\lambda (z : Int) . z)~x$: This
%   example should not type-check because it leads to an ambigous choice
%   in the body of the lambda. In the current system the well-formedness
%   checks forbid such example.

% \item $\Lambda A.\Lambda B.\lambda (x:A).\lambda (y:B). (\lambda (z:A)
%   . z) (x,,y)$: This example should not type-check because it is not
%   guaranteed that the instantiation of A and B produces a well-formed
%   type. The TyMerge rule forbids it with the disjointness check.

% \item $\Lambda A.\Lambda B * A.\lambda (x:A).\lambda (y:B). (\lambda
%   (z:A) . z) (x,,y)$: This example should type-check because B is
%   guaranteed to be disjoint with A. Therefore instantiation should
%   produce a well-formed type.

% \item $(\lambda (z:Int) . z) ((1,,'c'),,(2,False))$: This example
%   should not type-check, since it leads to an ambigous lookup of
%   integers (can either be 1 or 2). The definition of disjointness is
%   crutial to prevent this example from type-checking. When
%   type-checking the large merge, the disjointness predicate will
%   detect that more than one integer exists in the merge.

% \item $(\lambda (f: Int \to Int \& Bool) . \lambda (g : Int \to Char \& Bool) . ((f,,g) : Int \to Bool)$:
%   This example
%   should not type-check, since it leads to an ambigous lookup of
%   functions. It shows that in order to check disjointness
%   of functions we must also check disjointness of the subcomponents.

% \item $(\lambda (f: Int \to Int) . \lambda (g : Bool \to Int) . ((f,,g) : Bool \& Int \to Int)$:
%   This example shows that whenever the return types overlap, so does the function type:
%   we can always find a common subtype for the argument types.
% \end{enumerate}

% \subsection{Achieving coherence}

% The crutial challenge lies in the generation of coercions, which can lead
% to different results due to multiple possible choices in the rules that
% can be used. In particular the rules subinter1 and subinter2 overlap and
% can result in coercions that are not equivalent. A simple example is:

% $(\lambda (x:Int) . x) (1,,2)$

% The result of this program can be either 1 or 2 depending on whether
% we chose subinter1 or subinter2.

% Therefore the challenge of coherence lies in ensuring that, for any given
% types A and B, the result of $A <: B$ always leads to the same (or semantically
% equivalent) coercions.

% It is clear that, in general, the following does not hold:

% $if~A <: B \leadsto C1~and~A <: B \leadsto C2~then~C1 = C2$

% We can see this with the example above. There are two possible coercions:\\

% \noindent $(Int\&Int) <: Int \leadsto \lambda (x,y). x$\\
% $(Int\&Int) <: Int \leadsto \lambda (x,y). y$\\

% However $\lambda (x,y). x$ and $\lambda (x,y). y$ are not semantically equivalent.

% One simple observation is that the use of the subtyping relation on the
% example uses an ill-formed type ($Int\&Int$). Since the type system can prevent
% such bad uses of ill-formed types, it could be that if we only allow well-formed
% types then the uses of the subtyping relation do produce equivalent coercions.
% Therefore the we postulate the following conjecture:

% $if~A <: B \leadsto C1~and~A <: B \leadsto C2~and~A, B~well~formed~then~C1 = C2$

% If the following conjecture does hold then it should be easy to prove that
% the translation is coherent.

% % \begin{mathpar}
% %   \inferrule
% %   {}
% %   {\hastype \epsilon {1 \mergeop 2} {\constraints {\integer \disjoint \integer} \integer \inter \integer}}
% % \end{mathpar}

% % \begin{definition}{(Disjointness)}
% % Two sets $S$ and $T$ are \emph{disjoint} if there does not exist an element $x$, such that $x \in S$ and $x \in T$.
% % \end{definition}

% % \begin{definition}{(Disjointness)}
% % Two types $A$ and $B$ are \emph{disjoint} if there does not exist an term $e$, which is not a merge, such that $\hastype \epsilon e A'$, $\hastype \epsilon e B'$, $A' \subtype A$, and $B' \subtype B$.
% % \end{definition}

% \section{Introduction}

% The benefit of a merge, compared to a pair, is that you don't need to explicitly extract an item out. For example, \lstinline@fst (1,'c')@

% \begin{definition}{Determinism}
% If $e : A_1 \hookrightarrow E_1$ and $e : A_2 \hookrightarrow E_2$,
% then $A_1 = A_2$ and $E_1 = E_2$.
% \end{definition}

% \emph{Coherence} is a property about the relation between syntax and semantics. We say a semantics is \emph{coherent} if the syntax of a term uniquely determines its semantics.

% \begin{definition}{Coherence}
% If $e_1 : A_1 \hookrightarrow E_1$ and $e_2 : A_2 \hookrightarrow E_2$,
% $E_1 \Downarrow v_1$ and $E_2 \Downarrow v_2$,
% then $v_1 = v_2$.
% \end{definition}

% \begin{definition}{Disjointness}
% Two types $A$ and $B$ are \emph{disjoint} (written as ``$\disjoint A B$'') if there does not exist a type $C$ such that $C \subtype A$ and $C \subtype B$ and $C \subtype A \inter B$.
% \end{definition}
