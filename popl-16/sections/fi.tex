\section{The \namenaive Calculus} \label{sec:finaive}

This section presents the syntax, subtyping, and typing of \namenaive. The
semantics of \namenaive is defined by a type-directed translation to a simple
extension of System $F$.

\subsection{Syntax}

Figure~\ref{fig:finaive-syntax} shows the syntax of \namenaive (with the
addition to System $F$ highlighted).
% As a convention in this paper, we will be using
% lowercase letters as meta-variables for sorts in \name, and uppercase letters
% for those in the target language.

% To System $ F $, we add two features: intersection types and single-field
% records.
% % ~\bruno{labelled types (single field records are fine too)}
% We include only single records because single record types as the multi-records
% can be desugared into the merge of multiple single records.

\begin{figure}[h]
  \[
    \begin{array}{l}
      \begin{array}{llrll}
        \text{Types}
        & A, B & \Coloneqq & \alpha      & \\
        &      & \mid & A \to B          & \\
        &      & \mid & \for {\alpha} A  & \\
        &      & \mid & \highlight{$ A \inter B $}  & \\

        \\
        \text{Terms}
        & e & \Coloneqq & x            & \\
        &   & \mid & \lamty x A e      & \\
        &   & \mid & \app {e_1} {e_2}  & \\
        &   & \mid & \blam {\alpha} e  & \\
        &   & \mid & \tapp e A         & \\
        &   & \mid & \highlight{$ e_1 \mergeop e_2 $}  & \\

        \\
        \text{Contexts}
        & \Gamma & \Coloneqq & \cdot
                   \mid \Gamma, \alpha
                   \mid \Gamma, x : A  & \\
      \end{array}
    \end{array}
  \]

  \label{fig:finaive-syntax}
  \caption{Syntax.}
\end{figure}

% \bruno{I am not sure if the highlighting will be
%   visible. Use gray?}\bruno{there is no \lstinline{with} anymore.
% Make sure the syntax is consistent with what is presented in the type rules!}

\paragraph{Types} Meta-variables $A$, $B$ range over types. Types include
standard constructs in System $F$: type variables $\alpha$; function types $A
\to B$; and type abstraction $\for \alpha A$. $A \inter B$ denotes the
intersection of types $A$ and $B$. We omit type constants such as $\tyint$ and
$\tystring$.

\paragraph{Terms} Meta-variables $e$ range over terms.  Terms include standard
constructs in System $F$: variables $x$; abstraction of terms over variables of
a given type $\lamty x A e$; application of terms to terms $\app {e_1} {e_2}$;
abstraction of type variables over terms $\blam {\alpha \disjoint A} e$;  and
application of terms to types $\tapp e A$.  $e_1 \mergeop e_2$ is the
\emph{merge} of two terms $e_1$ and $e_2$. It can be used as either $ e_1 $ or $
e_2 $. In particular, if one regards $e_1$ and $e_2$ as objects, their merge
will respond to every method that one or both of them have. Merge of terms
correspond to intersection types $A \inter B$.

% Note that $ e $ does not
% need to be a record type in this case. For example, although the merge of two
% records
% \[
% x = \recordCon {l_1} {e_1} \mergeop \recordCon {l_1} {e_2}
% \]
% is of an intersection type, $ x.{l_1} $ still gives $ e_1 $. On the other hand,
% $ x.{l_3} $ will be rejected by the type system.

In order to focus on the most essential features, we do not include other forms
such as fixpoints here, although they are supported in our implementation and
can be included in formalization in standard ways.

Typing contexts $ \Gamma $ track bound type variables with their disjointness
constraint, and variables with their type $A$. We use use $\subst {A} \alpha
{B}$ for the capture-avoiding substitution of $A$ for $\alpha$ inside $B$ and
$\ftv \cdot$ for sets of free variables.

% \paragraph{Discussion.} A natural question the reader might ask is that why we
% have excluded union types from the language. The answer is we found that
% intersection types alone are enough support extensible designs. To focus on the
% key features that make this language interesting, we also omit other common
% constructs. For example, fixpoints can be added in standard ways.
% Dunfield has described a language that includes a ``top'' type but it does not
% appear in our language. Our work differs from Dunfield in that ...

\subsection{Subtyping}

% In some calculi, the subtyping relation is external to the language: those
% calculi are indifferent to how the subtyping relation is defined. In \name, we
% take a syntatic approach, that is, subtyping is due to the syntax of types.
% However, this approach does not preclude integrating other forms of subtyping
% into our system. One is ``primitive'' subtyping relations such as natural
% numbers being a subtype of integers. The other is nominal subtyping relations
% that are explicitly declared by the programmer.

\begin{figure*}
  \begin{mathpar}
    \framebox{$ A \subtype B \yields F $} \\

    \subvar

    \subfun

    \subforall

    \subinter

    \subinterleft

    \subinterright
  \end{mathpar}

  \footnote{miss a rule for $\bot$}

  \caption{Subtyping in \name.}
  \label{fig:fi-subtyping}
\end{figure*}

The subtyping rules of \name, shown in Figure~\ref{fig:fi-subtyping}, are
syntax-directed (different from the approach by Davies and
Pfenning~\cite{davies2000intersection}, and Frisch et.
al~\cite{frisch2008semantic}). The rule \rulelabel{subfun} says that a function
is contravariant in its parameter type and covariant in its return type. A
universal quantifier ($\forall$) is covariant in its body. The three rules
dealing with intersection types are just what one would expect when interpreting
types as sets. Under this interpretation, for example, the rule
\rulelabel{subinter} says that if $A_1$ is both the subset of $A_2$ and the
subset of $A_3$, then $A_1$ is also the subset of the intersection of
$A_2$ and $A_3$. In order to achieve coherence, \rulelabel{subinter1} and
\rulelabel{subinter2} additionally require the type on the right-hand side is
atomic.

% Intersection types introduce natural subtyping relations among types. For
% example, $ \tyint \inter \tybool $ should be a subtype of $ \tyint $, since the former
% can be viewed as either $ \tyint $ or $ \tybool $. To summarize, the subtyping rules
% are standard except for three points listed below:
% \begin{enumerate}
% \item $ A_1 \inter A_2 $ is a subtype of $ A_3 $, if \emph{either} $ A_1 $ or
%   $ A_2 $ are subtypes of $ A_3 $,

% \item $ A_1 $ is a subtype of $ A_2 \inter A_3 $, if $ A_1 $ is a subtype of
%   both $ A_2 $ and $ A_3 $.

% \item $ \recordType {l_1} {A_1} $ is a subtype of $ \recordType {l_2} {A_2} $, if
%   $ l_1 $ and $ l_2 $ are identical and $ A_1 $ is a subtype of $ A_2 $.
% \end{enumerate}
% The first point is captured by two rules $ \rulelabelsubinterLeft $ and
% $ \rulelabelsubinterRight $, whereas the second point by $ \rulelabelsubinter $.
% Note that the last point means that record types are covariant in the type of
% the fields.

It is easy to see that subtyping is reflexive and transitive.

\begin{lemma}[Subtyping is reflexive] \label{sub-refl}
Given a type $ A $, $ A \subtype A $.
\end{lemma}

\begin{lemma}[Subtyping is transitive] \label{sub-trans}
If $ A_1 \subtype A_2 $ and $ A_2 \subtype A_3 $,
then $ A_1 \subtype A_3 $.
\end{lemma}

For the corresponding mechanized proofs in Coq, we refer to the supplementary
materials submitted with the paper.

\subsection{Typing}

\begin{figure*}
  \begin{mathpar}
    \framebox{$ \hastype \Gamma e A \yields E $} \\

    \tyvar

    \tylam

    \tyapp

    \tyblam

    \tytapp

    \tymerge

  \end{mathpar}

  \caption{The type system of \name.}
  \label{fig:fi-typing}
\end{figure*}

The syntax-directed typing rules of \name are shown in
Figure~\ref{fig:fi-typing}. They consist of one main typing judgment and two
auxiliary judgments. The main typing judgment is of the form: $ \hastype \Gamma
e A $. It reads: ``in the typing context $\Gamma$, the term $e$ is of type
$A$''. The rules that are the same as in System $F$ are rules for variables
(\rulelabel{Var}), lambda abstractions (\rulelabel{Lam}), and type applications
(\rulelabel{TApp}). For the ease of discussion, in \rulelabel{BLam}, we require
the type variable introduced by the quantifier is fresh. For programs with type
variable shadowing, this requirement can be met straighforwardly by variable
renaming. The rule \rulelabel{App} needs special attention as we add a subtyping
requirement: the type of the argument ($A_3$) is a subtype of that of the
parameter ($A_1$).
%The advantage is that it then
%becomes easier to derive an algorithm for typechecking.
For merges $e_1 \mergeop e_2$, we typecheck $e_1$ and $e_2$, check that the two
resulting types are disjoint, and give it the intersection of the resulting
types.

% The last two rules make use of the $ \rulename{fields} $ function just to make
% sure that the field being accessed ($ \rulelabelRecSelect $) or updated
% ($ \rulelabelRecUpd $) actually exists. The function is defined recursively, in
% Haskell pseudocode, as:
% \[ \begin{array}{rll}
%   \fields{\alpha} & = & \rel{\cdot} {\alpha} \\
%   \fields{A_1 \to A_2} & = & \rel{\cdot} {A_1 \to A_2} \\
%   \fields{\forall \alpha. A} & = & \rel{\cdot} {\forall \alpha. A} \\
%   \fields{A_1 \inter A_2} & = & \fields{A_1} \dplus \fields{A_2} \\
%   \fields{\recordType l A} & = & \rel l t
% \end{array} \]
% where $ \cdot $ means empty list, $ \dplus $ list concatenation, and $ : $ is an
% infix operator that prepend the first argument to the second. The function
% returns an associative list whose domain is field labels and range types.

\subsection{Type-directed Translation to System $ F $}

In this section we define the dynamic semantics of the call-by-value \name by
means of a type-directed translation to a variant of System $F$. This
translation turns merges into usual pairs, similar to Dunfield's elaboration
approach~\cite{dunfield2014elaborating}. In the end the translated terms can be typed and interpreted
within System $F$. We add the blue-color part to our rules presented in the
previous section. Besides that, they stay the same. We also tacitly assume the
variables introduced in the blue part are generated from a unique name supply and
are always fresh.

\subsection{Informal Discussion}

This subsection presents the translation informally. The idea is turning merges
into pairs. For example,
\[
1 \mergeop \code{"one"}
\]
becomes \pair 1 {\code{"one"}}.
In usage, the pair will be coerced according to type information. For example,
consider the function application:
\[
\app {(\lam x \tystring x)} {(1 \mergeop \code{"one"})}
\]
It will be translated to
\[
\app {(\lam x \tystring x)} {(\app {(\lam x {\pair \tyint \tystring} {\proj 2 x})} {\pair 1 {\code{"one"}}})}
\]
The coercion in this case is $(\lam x {\pair \tyint \tystring} {\proj 2 x})$.

\noindent It extracts the second item from the pair since the function expects a $\tystring$
but the translated argument is of type $\pair \tyint \tystring$.

\subsection{Target Language}

Our target language is System $F$ extended with pair and unit types. The syntax
and typing is completely standard. The syntax of the target language is shown in
Figure~\ref{fig:f-syntax} and the typing rules in the appendix.
% \bruno{fill!}
\begin{figure}[h]
  \[
\begin{array}{llrl}
  \text{Types}       & T    & \Coloneqq & \alpha \mid () \mid {T_1} \to {T_2} \mid \for \alpha T \mid \pair {T_1} {T_2} \\
  \text{Terms} & E, C & \Coloneqq & x \mid () \mid \lam x T E \mid \app {E_1} {E_2} \mid \blam \alpha E \\ 
                     &      & \mid      & \tapp E T \mid \pair {E_1} {E_2} \mid \proj k E \\
  \text{Contexts} & \Gamma & \Coloneqq & \epsilon \mid \Gamma, \alpha \mid \Gamma, x \hast T \\
\end{array}
\]

  \caption{Target language syntax.}
  \label{fig:f-syntax}
\end{figure}

% \bruno{Why is this lemma placed here?}
% \bruno{Generaly Speaking this text seems out of place.Move to 5.4, maybe?}

% The main translation judgment is $ \hastype \gamma e A \yields E $ which
% states that with respect to the typing context $ \gamma $, the \name term
% $e$ is of $A$ and its translation is a target term $ E $.

\subsection{Type Translation}

\begin{figure}[h]
  \framebox{$ \im \tau = T $}

\[
\begin{array}{rcl}
  \im \alpha                     & = & \alpha \\
  \im \emptyrec                  & = & () \\
  \im {\tau_1} \to \im {\tau_2} & = & \im {\tau_1} \to \im {\tau_2} \\
  \im {\for \alpha \tau}      & = & \for \alpha \im \tau \\
  \im {\tau_1 \andop \tau_2}   & = & \pair {\im {\tau_1}} {\im {\tau_2}} \\
  \im {\recty l \tau}            & = & \im \tau
\end{array}
\]
  \begin{align*}
  \im \epsilon                    &= \epsilon \\
  \im {\gamma, \alpha}            &= \im \gamma, \alpha \\
  \im {\gamma, \alpha \hast \tau} &= \im \gamma, \alpha \hast \im \tau
\end{align*}
  \caption{Type and context translation.}
  \label{fig:type-and-context-translation}
\end{figure}

Figure~\ref{fig:type-and-context-translation} defines the type translation
function $\im \cdot$ from \name types $A$ to target language types $T$. The
notation $\im \cdot$ is also overloaded for context translation from \name
contexts $\gamma$ to target language contexts $\Gamma$.

% The rules given in this section are identical with those in
% Section~\ref{sec:fi}, except for the light blue part. The translation consists
% of four sets of rules, which are explained below:

\subsection{Coercive Subtyping}

Figure~\george{fig:elab-subtyping} shows subtyping with coercions. The judgment
\[
A_1 \subtype A_2 \yields C
\]
extends the subtyping judgment in Figure~\ref{fig:fi-subtyping} with a coercion
on the right hand side of $ \yields {} $. A coercion $ C $ is just an term
in the target language and is ensured to have type
$ \im {A_1} \to \im {A_2} $ (Lemma~\ref{lemma:sub})\bruno{ref
  now showing}. For example,
\[
\tyint \inter \tybool \subtype \tybool \yields {\lam x {\im {\tyint \inter \tybool}} {\proj 2 x}}
\]

\noindent generates a coercion function from $\tyint \inter \tybool$ to $\tybool$.

In rules \rulelabel{subvar}, \rulelabel{SubTop}, \rulelabel{subforall},
coercions are just identity functions. In \rulelabel{subfun}, we elaborate the
subtyping of parameter and return types by $\eta$-expanding $f$ to
$\lam x {\im {A_3}} {\app f x}$, applying $C_1$ to the argument and $C_2$ to
the result. Rules \rulelabel{subinter1}, \rulelabel{subinter2}, and
\rulelabel{subinter} elaborate with intersection types. \rulelabel{subinter} uses
both coercions to form a pair. Rules \rulelabel{subinter1} and
\rulelabel{subinter2} reuse the coercion from the premises and create new ones
that cater to the changes of the argument type in the conclusions. Note that the
two rules are syntatically the same and hence a program can be elaborated
differently, depending on which rule is used. But in the implementation one
usually applies the rules sequentially with pattern matching, essentially
defining a deterministic order of lookup.
\begin{comment}
if we know $A_1$ is a subtype of $A_3$ and $C$ is a coercion from $A_1$
to $A_3$, then we can conclude that $A_1 \inter A_2$ is also a subtype
of $A_3$ and the new coercion is a function that takes a value $ x $ of type
$A_1\inter A_2$, project $x$ on the first item, and apply $ C $ to it.
\end{comment}

\begin{restatable}[Subtyping rules produce type-correct coercion]{lemma}{lemmasub}
  \label{lemma:sub}
  If $ A_1 \subtype A_2 \yields C $, then $ \judgeTarget \epsilon C {\im {A_1} \to \im {A_2}} $.
\end{restatable}

\begin{proof}
  By a straighforward induction on the derivation\footnote{The proofs of major lemmata and theorems can be found in the appendix.}.
\end{proof}

\subsection{Main Translation}

\begin{comment}
In this subsection we now present formally the translation rules that convert
\name terms into System $ F $ ones. This set of rules essentially extends
those in the previous section with the light-blue part for the translation.
\end{comment}

% \bruno{Badly structured. Don't mention Coercion here, as it was already
% explained in the previous section.}
% \bruno{Don't use itemize and items. Use paragraphs instead!}

\paragraph{Main translation judgment.} The main translation judgment
$\hastype \gamma e A \yields E$ extends the typing judgment with an elaborated
term on the right hand side of $\yields {}$. The translation ensures
that $E$ has type $\im A$. In \name, one may pass more information to a
function than what is required; but not in System $F$. To account for this
difference, in \rulelabel{App}, the coercion $C$ from the subtyping relation is
applied to the argument. \rulelabel{Merge} straighforwardly translates merges
into pairs.

% Consider the source program:
% \begin{lstlisting}
%   ({ name = "Isaac", age = 10 }).name
% \end{lstlisting}

%   Multi-field records are desugared into merge of single-field records:
%   \begin{lstlisting}
%     ({ name = "Isaac"} ,, { age = 10 }).name
%   \end{lstlisting}

%   By $ \ruleLabelSelect $,
%   \[ \turnsGet (\recordType {name} {String}; {name}) : String \]

%   we have the coercion
%   \[ \lam x {\im {\recordType {name} {String}}} x \]

%   which is just $ \lam x {String} x $ according to type translation.

%   By $ \ruleLabelSelectLeft $,
%   \[ \turnsGet (\recordType {name} {String} \inter \recordType {age} {Int}; {name}) : String \]

%   % we have the coercion
%   % \[ \abs {\rel x {\im {\recordType {name} {String} \inter \recordType
%   %         {age} {Int}}}} \app {(\abs {\rel x {\im {\recordType {name} {String}}}} x)} {(\fst ~ x)} \]
%   % which is just $ \abs {\rel x {(String, Int)}} {\app {(\abs {\rel x {String}} x)} {(\fst ~ x)}} $ by type translation.

%   By typing rules, the translation of the program is
%   \[ ("Isaac", 10) \]. If we apply the coercion to it, we get
%   \[ "Isaac" \]

\begin{restatable}[Translation preserves well-typing]{theorem}{theorempreservation}
  \label{theorem:preservation}
  If $ \hastype \gamma e A \yields E $,
  then $ \judgeTarget {\im \gamma} E {\im A} $.
\end{restatable}
\begin{proof}
(Sketch) By structural induction on the term and the corresponding
inference rule.
\end{proof}

\begin{theorem}[Type safety]
  If $e$ is a well-typed \name term, then $e$ evaluates to some System $F$
  value $v$.
\end{theorem}
\begin{proof}
  Since we define the dynamic semantics of \name in terms of the composition of
  the type-directed translation and the dynamic semantics of System $F$, type safety follows immediately.
\end{proof}
