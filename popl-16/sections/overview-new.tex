\section{Overview}

This section introduces \name and its support for intersection types
and polymorphism. It then shows that, without care, the system
lacks \emph{coherence}. Finally it is shown that by allowing only
disjoint intersection types and extending universal quatification to
disjoint quantification, coherence is possible.

\subsection{Intersection Types}
\bruno{First show what intersection types are and why they are useful}

A number of OO languages, such as Java, C\#, Scala, and
Ceylon\footnote{\url{http://ceylon-lang.org/}}, already support intersection
types to different degrees. In Java, for example,

\begin{lstlisting}
interface AwithB extends A, B {}
\end{lstlisting}

\noindent introduces a new interface \lstinline{AwithB} that satisfies the interfaces of
both \lstinline{A} and \lstinline{B}. Arguably such type can be considered as a nominal
intersection type. Scala takes one step further by eliminating the
need of a nominal type. For example, given two concrete traits, it is possible to
use \emph{mixin composition} to create an object that implements both
traits. Such an object has a (structural) intersection type:

\begin{lstlisting}
trait A
trait B

val newAB : A with B = new A with B
\end{lstlisting}

\noindent Scala also allows intersection of type parameters. For example:
\begin{lstlisting}
def merge[A,B] (x: A) (y: B) : A with B = ...
\end{lstlisting}
uses the annonymous intersection of two type parameters \lstinline{A} and
\lstinline{B}. However, in Scala it is not possible to dynamically
compose two objects. For example, the following code:

\begin{lstlisting}
// Invalid Scala code:
def merge[A,B] (x: A) (y: B) : A with B = x with y
\end{lstlisting}

\noindent is rejected by the Scala compiler. The problem is that the
\lstinline{with} construct for Scala expressions can only be used to
mixin traits or classes, and not arbitrary objects. Note that in the
definition \lstinline{newAB} both \lstinline{A} and \lstinline{B} are
\emph{traits}, whereas in the definition of \lstinline{merge} the variables
\lstinline{x} and \lstinline{y} denote \emph{objects}.

This limitation essentially put intersection types in Scala in a second-class
status. Although \lstinline{merge} returns an intersection type, it is
hard to actually build values with such types. In essense an
object-level introduction contruct for intersection types is missing.
As it turns out using low-level type-unsafe programming features such
as dynamic proxies, reflection or other meta-programming techniques,
it is possible to implement such an introduction
construct in Scala~\cite{oliveira2013feature,rendel14attributes}. However, this
is clearly a hack and it would be better to provide proper language
support for such a feature.

\subsection{Parametric Polymorphism and Intersection Types}
\bruno{Then talk about parametric polymorphism}

Both universal quantification and intersection types provide a kind of
polymorphism. While the former provides parametric polymorphism, the latter
provides ad-hoc polymorphism. In some systems, parametric polymorphism is
considered the infinite analog of intersection polymorphism. But in our system
we do not consider this relationship. \george{Need to argue that why their
coexistence might be a good thing.} \george{May use the merge example}

\bruno{Some more examples in following subsections?}

\subsection{Intersection Types and Noncoherence}

What is an intersection type? The intersection of types $A$ and $B$ contains
exactly those values which can be used as either of type $A$ or of type $B$.
Just as not all intersection of sets are nonempty, not all intersections of types are inhabited.
For example, the intersection of a base type $\tyint$ and a function type $\tyint \to \tyint$ is not inhabited.
The merge operator combines two terms, of type $A$ and $B$
respectively, to form a term of type $A \inter B$. For example, $1 \mergeop
`c'$ is of type $\tyint \inter \tychar$. In this case, no matter $1 \mergeop
`c'$ is used as $\tyint$ or $\tychar$, the result of evaluation is always
clear. However, with overlapping types, it is not straightforward anymore to
see the result. For example, what should be the result of this program, which
asks for an integer out of a merge of two integers:
\[ \app {(\lam x \tyint x)} {1 \mergeop 2} \]
Should the result be $1$ or $2$?

The following shows the naive subtyping rules for intersection types:
\begin{mathpar}
\inferrule* [right=$\ruleLabelsubinter$]
  {A_1 \subtype A_2 \yields {C_1} \\ A_1 \subtype A_3 \yields {C_2}}
  {A_1 \subtype A_2 \inter A_3
    \yields
      {\lam x {\im {A_1}}
        {\pair {\app {C_1} x} {\app {C_2} x}}}}

\inferrule* [right=$\ruleLabelsubinterLeft$]
  {A_1 \subtype A_3 \yields C}
  {A_1 \inter A_2 \subtype A_3
    \yields
      {\lam x {\im {A_1 \inter A_2}}
        {\app C {(\proj 1 x)}}}}

\inferrule* [right=$\ruleLabelsubinterRight$]
  {A_2 \subtype A_3 \yields C}
  {A_1 \inter A_2 \subtype A_3
    \yields
      {\lam x {\im {A_1 \inter A_2}}
        {\app C {(\proj 2 x)}}}}
\end{mathpar}
The crutial challenge lies in the generation of coercions that are derived by the subtyping rules.
Since a program can typecheck via multiple derivations, and different derivation builds up multiple derivations,
noncoherent arises.

If this situation occurs, we say that the semantics is \emph{noncoherent}.
More precisely, coherence is a property about the function that give meaning to valid programs.
A system is coherent if any valid program has exactly one meaning.

Therefore, at this point two candidates of solutions occur:
\begin{itemize}
    \item To forbid overlapping intersection types in a desired type
    system;
    \item To enforce an order of lookup. For example, the right
    item of a merge will take precedence so that it can ``override'' the left
    item.
\end{itemize}

With the second approach, the program above can only evaluate to $2$.
Unfortunately, although it is more liberal than the first,
it makes equational reasoning broken in systems with parametric polymorphism.

Obviously the difficulty above is due to the fact that the type of $1 \mergeop
2$, which is $\tyint \inter \tyint$ is an overlapping intersection.
Generally, if both terms can be assigned some type $C$,
both of them can be chosen as the meaning of the merge,
which leads to multiple meaning of a term.

Therefore the challenge of coherence lies in ensuring that, for any given
types A and B, the result of $A <: B$ always leads to the same coercions.

\subsection{Equational Reasoning}

We can define a \code{fst} function that extracts the first item of a merged value:
\[
\code{fst} \ \alpha \ \beta \ (x : \alpha \inter \beta) = \app {(\lam y \alpha y)} x
\]
What should be the result of this program?
\begin{lstlisting}
fst Int Int (1,,2)
\end{lstlisting}

Then we have the following equational reasoning:
\begin{lstlisting}
fst Int Int (1,,2) => (\(y : Int). y) (1,,2)
\end{lstlisting}
If we favour the second item, the program seems to evaluate to $2$. But in
reality, the result is $2$. No matter we favour the first or the second item,
we can always construct a program such that for that program, equational
reasoning is broken.

Therefore, we require that the two types of an intersection must be not
overlapping, or \emph{disjoint}, and add this requirement to the well-formedness of types.

A well-formed type is such that given any query type,
it is always clear which subpart the query is referring to.
In terms of rules, this notion of well-formedness is almost the same as the one in System $F$
except for intersection types we require the two components to be disjoint.

With parametric polymorphism, disjointness is harder to determine due to type variables.
Consider this program:
\[
\blam \alpha {\lam x {\alpha \inter \tyint} x}
\]
$x$ in the body is of type $\alpha \inter \tyint$ and if $\alpha$ and $\tyint$ are
disjoint depends on the instantiation of $\alpha$.

\subsection{Intuition of Disjoint Quantification}

Inspired by bounded quantification where a type variable is constrained by a type bound,
we introduce the idea of disjoint quantification where a type variable is constrained
to be disjoint with a given type.

There is a nice symmetry between bounded quantification and disjoint quantification.
In systems with bounded quantification,
the usual unconstrained quantifier $\for {\alpha} \ldots$
is a syntactic sugar for $\for {\alpha \subtype \top} \ldots$, and
$\blam \alpha \ldots$ for $\blam {\alpha \subtype \top} \ldots$.
In parellel, in our system with disjoint quantification,
the usual unconstrained quantifier $\for {\alpha} \ldots$
is a syntactic sugar for $\for {\alpha \disjoint \bot} \ldots$, and
$\blam \alpha \ldots$ for $\blam {\alpha \disjoint \top} \ldots$.
The intuition is that since the bottom type is akin to the empty set,
no other type overlaps with it.

With this tool in hand, we can rewrite the program above to:
\[
\blam {\alpha \disjoint \tyint} {\lam x {\alpha \inter \tyint} x}
\]

This program typechecks because while $x$ is of type $\alpha \inter \tyint$,
and $\alpha$ is disjoint with $\tyint$. Similarly, in the new system,
the original program no longer typechecks, thus preventing overlapping types.
