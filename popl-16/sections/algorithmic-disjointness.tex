\section{Algorithmic Disjointness} \label{sec:alg-dis}

As promised, in this section we present the set of rules for determining whether
two types are disjoint, and more importantly, how to derive them from the
definition we have in the previous sections. The derived set of rules for
disjointness is proved to be sound and complete with respect to the definition.

\subsection{Derivation}

In this subsection, we illustrate how to derive the algorithmic disjointness
rules by showing a detailed example for functions. For the ease of discussion,
first we introduce some notations.

\begin{definition}[Set of common supertypes]
  For any two types $A$ and $B$, we can denote the \emph{set of their common
  supertypes} by \[ \commonsuper(A,B) \] In other words, a type $C \in \;
  \commonsuper(A,B)$ if and only  if $A \subtype C$ and $B \subtype C$.
\end{definition}

\begin{example}
  $\commonsuper(\tyint,\tychar)$ is empty, since $\tyint$ and $\tychar$
  share no common supertype.
\end{example}

Parellel to the notion of the set of common supertypes is the notion of the set
of common subtypes.

\begin{definition}[Set of common subtypes]
  For any two types $A$ and $B$, we can denote the \emph{set of their common
  subtypes} by \[ \commonsub(A,B) \] In other words, a type $C \in \; \commonsub(A,B)$
  if and only  if $C \subtype A$ and $C \subtype B$.
\end{definition}

\begin{example}
  $\commonsub(\tyint,\tychar)$ is an infinite set which contains $\tyint \inter
  \tychar$, $\tychar \inter \tyint$, $(\tyint \inter \tybool) \inter \tychar$
  and so on. But the type $\tybool$ is not inside, since it is not a subtype of
  $\tyint$.
\end{example}

\paragraph{Shorthand notation} For brevity, we will use \[ \mathcal{A} \to
\mathcal{B} \] as a shorthand for the \emph{set} of types of the form $A \to B$,
where $A \in \mathcal{A}$ and $B \in \mathcal{B}$. The same shorthand applies to
all other constructors of types, in addition to functions, as well. As a simple
example,  \[ \{ \tyint, \tystring \} \to \{ \tyint, \tychar \} \] is a shorthand for \[ \{
\tyint \to \tyint, \tyint \to \tychar, \tystring \to \tyint, \tystring \to \tychar \} \]

Recall that we say two types $A$ and $B$ are disjoint if they do not share a
common supertype. Therefore, determining if two types $A$ and $B$ are disjoint
is the same as determining if $\commonsuper(A,B)$ is empty.

\paragraph{Determining disjointness of functions} Now let's dive into the case
where both $A$ and $B$ are functions and consider how to compute
$\commonsuper(A_1 \to A_2, B_1 \to B_2)$. By the subtyping rules, the supertype
of a function must also be a function.\george{Nah... only after normalizatoin.
If not, it can also be $\inter$} Let $C_1 \to C_2$ be a common supertype
of $A_1 \to A_2$ and $B_1 \to B_2$. Then it must satisfy the following:
\begin{mathpar}
  \inferrule* [right=SubFun]
    {C_1 \subtype A_1 \\ A_2 \subtype C_2}
    {A_1 \to A_2 \subtype C_1 \to C_2}

  \inferrule* [right=SubFun]
    {C_1 \subtype B_1 \\ B_2 \subtype C_2}
    {B_1 \to B_2 \subtype C_1 \to C_2}
\end{mathpar}
From which we see that $C_1$ is a common subtype of $A_1$ and $B_1$ and that
$C_2$ is a common supertype of $A_2$ and $B_2$. Therefore, we can wrtie:
\[ \commonsuper(A_1 \to A_2, B_1 \to B_2) \ = \ \commonsub(A_1,B_1) \to \commonsuper(A_2,B_2) \]
By definition, $\commonsub(A_1,B_1) \to \commonsuper(A_2,B_2)$ is not empty if and only if both
$\commonsub(A_1,B_1)$ and $\commonsuper(A_2,B_2)$ is nonempty. However, note
that with intersection types, $\commonsub(A_1,B_1)$ is always nonempty because
$A_1 \inter B_1$ belongs to $\commonsub(A_1,B_1)$. Therefore, the problem of
determining if $\commonsuper(A_1 \to A_2, B_1 \to B_2)$ is empty reduces to the
problem of determining if $\commonsuper(B_1 \to B_2)$ is empty, which is, by
definition, if $B_1$ and $B_2$ are disjoint. Finally, we have derived a rule for
functions:
\begin{mathpar}
  \inferrule* [right=DisjointFun]
    {\jdis \Gamma {B_1} {B_2}}
    {\jdis \Gamma {A_1 \to A_2} {B_1 \to B_2}}
\end{mathpar}

The analysis needed for determining if types with other constructors are
disjoint is similar. Below are the major results of the recursive definitions for
$\commonsuper$:
\begin{align*}
  \commonsuper(A_1 \to A_2, B_1 \to B_2) &= \commonsub(A_1,B_1) \to \commonsuper(A_2,B_2) \\
  \commonsuper({A_1 \inter A_2, B})      &= \commonsuper(A_1, B) \cup \commonsuper(A_1,B) \\
  \commonsuper({A, B_1 \inter B_2})      &= \commonsuper(A, B_1) \cup \commonsuper(A,B_2)
\end{align*}\george{Missing the forall case}

\subsection{Rules}

The rules for the disjointness judgement are shown in
Figure~\ref{fig:disjointness}, which consists of two judgements.

\paragraph{Main judgement} The judgement $\jdis \Gamma A B$ says two types
$A$ and $B$ are disjoint in a context $\Gamma$. \label{DisjointVar} is the
base rule and \label{DisjointVar} is its twin rule.
\label{DisjointInter1} and \label{DisjointInter2} inductively distribute
the relation itself over the intersection constructor ($\inter$). We have
derived \label{DisjointFun} in the previous subsection, which says two
function types are disjoint as long as their return types are disjoint
(regardless of their parameter types). \label{DisjointAtomic} says two types
are considered disjoint if they are judged to be disjoint by the $A \not \sim B$
rules, which is going to be explained below.

\paragraph{Axioms} The judgement $A \not \sim B$ says two atomic types $A$ and
$B$ are of different constructors. Note that intersection types are excluded
here, since for example, $\tyint \inter (\tychar \to \tychar)$ and $\tychar \to
\tychar$ have different constructors and yet are not disjoint.

\begin{figure}
  \begin{mathpar}
    \formdis \\
    \ruledisvar \and \ruledissym \and \ruledisinterl \and \ruledisinterr
    \ruledisfun \and \ruledisforall \and \ruledisatomic
  \end{mathpar}

  \begin{mathpar}
    \framebox{$ A \not\sim B$} \\

    \inferrule* [right=NotSimBot1]
      {}
      {\bot \not\sim A \to B}

    \inferrule* [right=NotSimBot2]
      {}
      {\bot \not\sim \fordis \alpha B A}

    \inferrule* [right=NotSimFunForall]
      {}
      {A \to B \not\sim \fordis \alpha B A}

    \inferrule* [right=NotSimFunForall]
      {B \not\sim A}
      {A \not\sim B}
  \end{mathpar}

  \caption{Algorithmic Disjointness.}
  \label{fig:disjointness}
\end{figure}

\subsection{Metatheory}

The algorithmic rules for disjointness is sound and complete.

\begin{lemma}{Symmetry of disjointness} \label{symmetry-of-disjointness}
  If $\jdis \Gamma A B$, then $\jdis \Gamma B A$.
\end{lemma}

\begin{theorem} \label{disjoint-intersect}
  If $\jdis \Gamma A C$ and $\jdis \Gamma B C$,
  then $\jdis \Gamma {A \inter B} {C}$.
\end{theorem}

\begin{lemma} \label{common-supertype}
  If $A_1 \to A_2 \subtype D$ and $B_1 \to B_2 \subtype D$,
  then there exists a $C$ such that $A_2 \subtype C$ and $B_2 \subtype C$.
\end{lemma}

\begin{theorem}[Soundness]
  For any two types $A$, $B$, $\jdisimpl \Gamma A B$ implies $\jdis \Gamma A B$.
\end{theorem}

\begin{theorem}[Completeness]
  For any two types $A$, $B$, $\jdis \Gamma A B$ implies $\jdisimpl \Gamma A B$.
\end{theorem}
