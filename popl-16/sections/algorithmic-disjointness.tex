\section{Algorithmic Disjointness}

\subsection{Motivation}

The rules for the disjointness judgement are shown in
Figure~\ref{fig:disjointness}. The judgement says two types $A$ and $B$ are
disjoint in a context $\Gamma$. Two atomic types with different shapes (except
for the variable) are considered disjoint, which is factored out to the atomic
disjointness rules. The \rulelabel{DisjointInter1} and
\rulelabel{DisjointInter2} inductively distribute the relation itself over the
intersection constructor ($\inter$). \rulelabel{DisjointFun} is quite
interesting, because it says two function types are disjoint as long as their
return types are disjoint (regardless of their parameter types).

\begin{figure}
  \begin{mathpar}
    \framebox{$ \isdisjoint \Gamma A B$} \\

    \disjointvar

% \inferrule* [right=DisjointSym]
%       {\alpha * A \in \Gamma}
%       {\isdisjoint \Gamma A \alpha}

    \disjointinterleft

    \disjointinterright

    \disjointfun

    \disjointforall

    \disjointatomic

\framebox{$ A \not\sim B$} \\

\inferrule* [right=NotSimBot1]
      {}
      {\bot \not\sim A \to B}

\inferrule* [right=NotSimBot2]
      {}
      {\bot \not\sim \for {\alpha * B} A}

\inferrule* [right=NotSimFunForall]
      {}
      {A \to B \not\sim \for {\alpha * B} A}

\inferrule* [right=NotSimFunForall]
      {B \not\sim A}
      {A \not\sim B}

  \end{mathpar}

  \label{fig:disjointness}
  \caption{Algorithmic Disjointness.}
\end{figure}

Although the system in the previous section shows a formal system of
disjoint intersection types, it relies on a non-algorithmic
specification of disjointness. This section shows an algorithmic
specification of disjointness that is proved to be sound and complete.

The problem with the definition of disjointness is that it is a search problem. In this section, we are going to convert it that into an algorithm.

Let $\universe_0$ be the universe of $A$ types. Let $\universe$ be the quotient set of $\universe_0$ by $\approx$, where $\approx$ is defined by \ldots.

Let $\commonsuper$ be the ``common supertype'' function, and $\commonsub$ be the ``common subtype'' function. For example, assume $\integer$ and $\character$ share no common supertype. Then the fact can be expressed by $\commonsuper(\integer,\character)=\emptyset$. Formally,
\begin{align*}
  \commonsuper &: \universe \times \universe \to \powerset {\universe} \\
  \commonsub   &: \universe \times \universe \to \powerset {\universe}
\end{align*}
which, given two types, computes the set of their common supertypes. ($\powerset S$ denotes the power set of $S$, that is, the set of all subsets of $S$.)

\begin{align*}
  \commonsuper(\alpha,\alpha) &= \{ \alpha \} \\
  \commonsuper(\bot,\bot) &= \{ \bot \} \\
  \commonsuper(A_1 \to A_2, A_3 \to A_4) &= \commonsub(A_1,A_3) \to \commonsuper(A_2,A_4) \\
  % \commonsuper({A_1 \inter A_2, A_3}) &= \commonsuper(A_1, A_3) \cup \commonsuper(A_1,A_3) \\
  % \commonsuper({A_1, A_2 \inter A_3}) &= \commonsuper(A_1, A_2) \cup \commonsuper(A_1,A_3)
\end{align*}

Notation. We use $\commonsub(A_1,A_3) \to \commonsuper(A_2,A_4)$ as a shorthand for $\{ s \to t ~|~ s \in \commonsub(A_1 \to A_2), t \in \commonsuper(A_2,A_4) \}$. Therefore, the problem of determining if $\commonsub(A_1,A_3) \to \commonsuper(A_2,A_4)$ is empty reduces to the problem of determining if $\commonsuper(A_2,A_4)$ is empty.

Note that there always exists a common subtype of any two given types (case disjoint / case nondisjoint).

\subsection{Formal system}

Explain the rules and intuitions.

The algorithmic rules for disjointness is sound and complete.

\begin{lemma}{Symmetry of disjointness} \label{symmetry-of-disjointness}
  If $\isdisjoint \Gamma A B$, then $\isdisjoint \Gamma B A$.
\end{lemma}

\begin{proof}
  Trivial by the definition of disjointness.
\end{proof}

\begin{theorem} \label{disjoint-intersect}
  If $\isdisjoint \Gamma A C$ and $\isdisjoint \Gamma B C$,
  then $\isdisjoint \Gamma {A \inter B} {C}$.
\end{theorem}

\begin{lemma} \label{common-supertype}
  If $A_1 \to A_2 \subtype D$ and $B_1 \to B_2 \subtype D$,
  then there exists a $C$ such that $A_2 \subtype C$ and $B_2 \subtype C$.
\end{lemma}

\begin{proof}
  By induction on $D$.
\end{proof}

\begin{theorem}{Soundness}
  For any two types $A$, $B$, $\isdisjointi \Gamma A B$ implies $\isdisjoint \Gamma A B$.
\end{theorem}

\begin{proof}
  By induction on $*_I$.

  \begin{itemize}
    \item Case \[ \disjointfun \]

    Lemma~\ref{common-supertype}

    \george{May need an extracted lemma here}

    \item Case \[ \disjointinterleft \]

    By Lemma~\ref{disjoint-intersect} and the i.h.

    \item Case \[ \disjointinterright \]

    By Lemma~\ref{disjoint-intersect}, Lemma~\ref{symmetry-of-disjointness}, and the i.h.

    \item Case \[ \disjointatomic \]
    Need to show ...
    By unfolding the definition of disjointness
    Need to show there does not exists $C$ such that...
    By induction on $C$.
    Atomic cases...
    If $C = C_1 \inter C_2$
    By inversion and the i.h. we arrive at a contradiction.

  \end{itemize}
\end{proof}

\begin{theorem}{Completeness}
  For any two type $A$, $B$, $\isdisjoint \Gamma A B$ implies $\isdisjointi \Gamma A B$.
\end{theorem}

\begin{proof}
  Induction on $A$.

  \begin{itemize}
    \item Case $\bot$

    Induction on $B$.
      \begin{itemize}
        \item Case $B = \bot$

          Need to show $\Gamma \turns \bot * \bot$ implies $\Gamma \turns \bot *_I \bot$. Take $C = \bot$. Clearly the premise is false by definition. Then the whole statement is true. \george{???}

        \item Case $B = B_1 \to B_2$
        The conclusion is true by the disjoint axioms.

        \item Case $B = B_1 \inter B_2$.
        Need to show $\Gamma \turns \bot * B_1 \inter B_2$ implies $\Gamma \turns \bot *_I B_1 \inter B_2$. Apply \rulelabel{DisjointInter2} and the resulting conditions can be proved by the i.h.

        \item Case
      \end{itemize}

      \item $A = A_1 \to A_2$
        \begin{itemize}
          \item Case $B = \bot$
          The conclusion is true by the disjoint axioms.

          \item Case $B = B_1 \to B_2$
          Need to show $\Gamma \turns  A_1 \to A_2 * B_1 \to B_2$ implies $\Gamma \turns  A_1 \to A_2 *_I B_1 \to B_2$. Apply \rulelabel{DisjointFun} and the result, $\Gamma \turns A_2 *_I B_2$, can be proved by the i.h.

          \item Case $B = B_1 \inter B_2$.
          Need to show $\Gamma \turns A_1 \to A_2 * B_1 \inter B_2$ implies $\Gamma \turns A_1 \to A_2 *_I B_1 \inter B_2$. Apply \rulelabel{DisjointInter2} and the resulting conditions can be proved by the i.h.
        \end{itemize}

        \item $A = A_1 \inter A_2$
        By \rulelabel{DisjointInter1} and by the i.h.

      \end{itemize}
\end{proof}
