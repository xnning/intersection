\subsection{Soundness and Completeness of Algorithmic Disjointness}

\begin{proof}
By induction on the derivation of $\jtype \Gamma e A$. The case of \reflabel{TyTApp} needs special attention
\begin{mathpar}
  \rulettapp
\end{mathpar}
because we need to show that the result of substitution ($\subst A \alpha C$) is well-formed, which is evident by Lemma~\ref{instantiation}.
\end{proof}

\begin{proof}
The typing rules are syntax-directed. The case of \reflabel{TyApp} needs special attention since we still need to show that the generated coercion $C$ is unique.
\begin{mathpar}
  \rulettapp
\end{mathpar}
By Lemma~\ref{wf-typing}, we have $\jwf \Gamma {A_1}$ and $\jwf \Gamma {A_3}$. Therefore we are able to apply Lemma~\ref{unique-coercion} and conclude that $C$ is unique.
\end{proof}

\begin{theorem}
  If $A \subtype C$, then $A \inter B \subtype C$.
  If $B \subtype C$, then $A \inter B \subtype C$.
\end{theorem}

\george{Add interpretation of the theorem}

\begin{proof}
  By induction on $C$.
  If $C \neq E_1 \inter E_2$, trivial.
  If $C = E_1 \inter E_2$,
  Need to show $A \subtype E_1 \inter E_2$ implies $A \inter B \subtype E_1 \inter E_2$.
  By inversion $A \subtype E_1$ and $A \subtype E_2$.
  By the i.h., $A \inter B \subtype E_1$ and $A \inter B \subtype E_2$.
  By \reflabel{subinter}, $A \inter B \subtype E_1 \inter E_2$.
\end{proof}

\begin{lemma}{Symmetry of disjointness} \label{lemma:symmetry-of-disjointness}
  If $\jdis \Gamma A B$, then $\jdis \Gamma B A$.
\end{lemma}

\begin{proof}
  Trivial by the definition of disjointness.
\end{proof}

\begin{theorem} \label{theorem:disjoint-intersect}
  If $\jdis \Gamma A C$ and $\jdis \Gamma B C$,
  then $\jdis \Gamma {A \inter B} {C}$.
\end{theorem}

\begin{lemma} \label{lemma:common-supertype}
  If $A_1 \to A_2 \subtype D$ and $B_1 \to B_2 \subtype D$,
  then there exists a $C$ such that $A_2 \subtype C$ and $B_2 \subtype C$.
\end{lemma}

\begin{proof}
  By induction on $D$.
\end{proof}

\soundness*

\begin{proof}
  By induction on $*_I$.

  \begin{itemize}
    \item Case
    \begin{flalign*}
      & \ruledisfun &
    \end{flalign*}

    Lemma~\ref{common-supertype}

    \george{May need an extracted lemma here} \\

    \item Case
    \begin{flalign*}
      & \ruledisinterl &
    \end{flalign*}

    By Lemma~\ref{disjoint-intersect} and the i.h. \\

    \item Case
    \begin{flalign*}
      & \ruledisinterr &
    \end{flalign*}

    By Lemma~\ref{disjoint-intersect}, Lemma~\ref{lemma:symmetry-of-disjointness}, and the i.h. \\

    \item Case
    \begin{flalign*}
      & \ruledisatomic &
    \end{flalign*}

    Need to show ...
    By unfolding the definition of disjointness
    Need to show there does not exists $C$ such that...
    By induction on $C$.
    Atomic cases...
    If $C = E_1 \inter E_2$
    By inversion and the i.h. we arrive at a contradiction.

  \end{itemize}
\end{proof}


\completeness*

\begin{proof}
  Induction on $A$.

  \begin{itemize}
    \item Case $\bot$

    Induction on $B$.
      \begin{itemize}
        \item Case $B = \bot$

          Need to show $\Gamma \turns \bot * \bot$ implies $\Gamma \turns \bot *_I \bot$. Take $C = \bot$. Clearly the premise is false by definition. Then the whole statement is true. \george{???} \\

        \item Case $B = B_1 \to B_2$
        The conclusion is true by the disjoint axioms. \\

        \item Case $B = B_1 \inter B_2$.
        Need to show $\Gamma \turns \bot * B_1 \inter B_2$ implies $\Gamma \turns \bot *_I B_1 \inter B_2$. Apply \reflabel{\labeldisinterr} and the resulting conditions can be proved by the i.h. \\

        \item Case
      \end{itemize}

      \item $A = A_1 \to A_2$
        \begin{itemize}
          \item Case $B = \bot$
          The conclusion is true by the disjoint axioms. \\

          \item Case $B = B_1 \to B_2$
          Need to show $\Gamma \turns  A_1 \to A_2 * B_1 \to B_2$ implies $\Gamma \turns  A_1 \to A_2 *_I B_1 \to B_2$. Apply \reflabel{\labeldisfun} and the result, $\Gamma \turns A_2 *_I B_2$, can be proved by the i.h. \\

          \item Case $B = B_1 \inter B_2$.
          Need to show $\Gamma \turns A_1 \to A_2 * B_1 \inter B_2$ implies $\Gamma \turns A_1 \to A_2 *_I B_1 \inter B_2$. Apply \reflabel{\labeldisinterr} and the resulting conditions can be proved by the i.h.
        \end{itemize}

        \item $A = A_1 \inter A_2$
        By \reflabel{\labeldisinterl} and by the i.h.

      \end{itemize}
\end{proof}
