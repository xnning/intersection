% \begin{definition}{Type variable constraint}
%   We say the \emph{constraint} of a type variable $\alpha$ inside the context
%   $\Gamma$ is $A$ if $\alpha \disjoint A \in \Gamma$.
% \end{definition}

% \begin{lemma}
% If $A \subtype B$ where both $A$ and $B$ are well-formed, then $A$ and $B$ cannot be disjoint.
% \end{lemma}
%
% \begin{proof}
% $A \subtype B$ implies $B$ is a common supertype of $A$ and $B$. As a result, $A$ and $B$ are not disjoint by definition.
% \end{proof}


% \begin{lemma}[Free type variables of disjoint bounds] \label{free-var-disjoint-bounds}
%   If $\isdisjoint \Gamma \alpha A$, then $\alpha \not \in \ftv A$.
% \end{lemma}

% \begin{proof}
% Since $A \inter B$ is well-formed, $A \disjoint B$ by the formation rule of intersection types \wfinterlabel. Then by the definition of disjointness, there does not exist a type $C$ such that $A \subtype C$ and $B \subtype C$. It follows that $A \subtype C$ and $B \subtype C$ cannot hold simultaneously.
% \end{proof}

% \begin{proof}
% The set of rules for generating coercions is syntax-directed except for the three rules that involve intersection types in the conclusion. Therefore it suffices to show that if well-formed types $A$ and $B$ satisfy $A \subtype B$, where $A$ or $B$ is an intersection type, then at most one of the three rules applies. In the following, we do a case analysis on the shape of $A$ and $B$:
%
% \begin{itemize}
%   \item \textbf{Case} $A \neq A_1 \inter A_2$ and $B = B_1 \inter B_2$: Clearly only \textsc{SubAnd} can apply.
%   \item \textbf{Case} $A = A_1 \inter A_2$ and $B \neq B_1 \inter B_2$: Only two rules can apply, \textsc{SubAnd1} and \textsc{SubAnd2}. Further, by the unique subtype contributor lemma, it is not possible that $A_1 \subtype B$ and that $A_2 \subtype B$. Thus we are certain that at most one rule of \textsc{SubAnd1} and \textsc{SubAnd2} will apply.
%   \item \textbf{Case} $A = A_1 \inter A_2$ and $B = B_1 \inter B_2$\footnote{An example of this case is:
%     \[ (\integer \inter \bool) \inter \character \subtype \bool \inter \character \]}: Since $B$ is not atomic, only \rulelabel{SubAnd} apply.
%
%   %   Suppose the contrary, that is, more than one of the three rules apply. Since it is not possible that both \textsc{SubAnd1} and \textsc{SubAnd2} apply by the unique subtype contributor lemma, only one of \textsc{SubAnd1} and \textsc{SubAnd2} apply. Therefore \textsc{SubAnd} has to hold. Without the loss of generality, assume \textsc{SubAnd1} apply. Then we have:
%   % \[ A_1 \subtype B_1 \inter B_2 \]
%   % \[ A_1 \inter A_2 \subtype B_1 \]
%   % \[ A_1 \inter A_2 \subtype B_2 \]
% \end{itemize}
% \end{proof}

% \begin{lemma}{Invariance of disjointness}
%   \label{invariance-of-disjointness}
%
%   If $\isdisjoint \Gamma A B$ and $R$ respects the constraints of $\beta$, then
%   $\isdisjoint \Gamma {\subst R \beta A} {\subst R \beta B}$.
%
% \end{lemma}
%
% This lemma says that substitution for free type variables preserves disjointness
% of types if the combination of the replacement type and the type variable is
% proven disjoint.

% \begin{proof}
% By induction on the derivation of $\isdisjoint \Gamma A B$.
% \begin{itemize}
%   \item Case \[ \disjointvar \]
%   We need to show \[ \isdisjoint \Gamma {\subst R \beta \alpha} {\subst R \beta B} \]
%   If $\beta$ is not equivalent to $\alpha$ and is not free in $B$, then the above trivially holds by the def. of the substitution function. Otherwise, if $\beta$ is equivalent to $\alpha$, then we need to show
%   \[ \isdisjoint \Gamma R {\subst R \beta B} \]
%
%   % Note that $\beta \not \in \ftv B$. Thus $B$ is equivalent to $\subst R \beta B$.
%   %
%   % If $\beta$ is not equivalent to $\alpha$, $\subst R \beta \alpha$ is equivalent to $\alpha$. Therefore $\isdisjoint \Gamma {\subst R \beta \alpha} {\subst R \beta B}$ is true.
%   % If $\beta$ is equivalent to $\alpha$, then $\subst R \beta \alpha$ is equivalent to $R$ by the def. of the substitution function. It now remains to show \[ \isdisjoint \Gamma R B \].
%
%   \item Case \[ \disjointinterleft \]
%   By applying the i.h. and the def. of the substitution function.
%
%   \item Case \[ \disjointinterright \]
%   Similar.
%
%   \item Case \[ \disjointfun \]
%   By applying the i.h. and the def. of the substitution function.
%
%   \item Case \[ \disjointforall \]
%   By applying the i.h. and the def. of the substitution function. Note that $\alpha$ is fresh.
%
%   \item Case \[ \disjointatomic \]
%   Substitution does not change the shape of types when the variable case is excluded. Therefore, the relation in the premise of the rule continue to hold and hence the conclusion.
%
% \end{itemize}
% \end{proof}

% \begin{lemma}{Substitution} \label{substitution}
%
%   If $\istype \Gamma R$, $\istype \Gamma S$, and $R$ respects the constraints of
%   $\beta$, then $\istype \Gamma {\subst R \beta S}$.
%
% \end{lemma}

% \begin{proof}
% By induction on the derivation of $\istype \Gamma {\subst R \beta S}$.
%
% \begin{itemize}
%   \item Case \[ \wfvar \]
%   If $\alpha$ happens to be the same as $\beta$, then by the def. of substitution $\subst R \beta \alpha = R$. Since $\istype \Gamma R$, we have $\istype \Gamma {\subst R \beta \alpha}$; On the other hand, if not, then by the def. of substitution $\subst R \beta S = S$. Since $\istype \Gamma S$, we also have $\istype \Gamma {\subst R \beta \alpha}$.
%
%   \item Case
%   \begin{mathpar}
%     \wfbot
%   \end{mathpar}
%   Trivial.
%
%   \item Case
%   \begin{mathpar}
%     \wffun
%   \end{mathpar}
%   By i.h., $\istype \Gamma {\subst R \beta A}$ and $\istype \Gamma {\subst R \beta B}$. By the def. of substitution, $\istype \Gamma {\subst R \beta {A \to B}}$.
%
%   \item Case
%   \begin{mathpar}
%     \wfforall
%   \end{mathpar}
%   By the premise and the i.h.,
%   \[ \istype {\Gamma} {\subst R \beta A} \]
%   \[ \istype {\Gamma, \alpha \disjoint A} {\subst R \beta B} \]
%   which by \rulelabel{WFForall} implies
%   \[ \istype \Gamma {\for {\alpha \disjoint A} {\subst R \beta B}} \]
%   By the def. of substitution, $\istype \Gamma {\subst R \beta {\for {\alpha \disjoint A} B}}$~
% \]{Subst. of $A$}.
%
%   \item Case
%   \begin{mathpar}
%     \wfinter
%   \end{mathpar}
%   By i.h., $\istype \Gamma {\subst R \beta A}$ and $\istype \Gamma {\subst R \beta B}$. By Lemma~\ref{invariance-of-disjointness}, we also have $\isdisjoint \Gamma {\subst R \beta A} {\subst R \beta B}$. Therefore by \rulelabel{WFInter}, $\istype \Gamma {\subst R \beta {A \inter B}}$.
% \end{itemize}
% \end{proof}



% \begin{proof}
% By induction.
%
% \begin{itemize}
%   \item Case \[ \wfvar \]
%   If $C = \alpha$, then $\subst A \alpha \alpha = A$. Since $\istype \Gamma A$, it follows that $\istype \Gamma {\subst A \alpha \alpha}$; otherwise, let $C = \beta$, where $\beta$ is a type variable distinct from $\alpha$. Since $\istype {\Gamma, \alpha \disjoint B} \beta$ and $\alpha$ and $\beta$ are distinct, $\beta$ must be in $\Gamma$ and therefore $\istype {\Gamma} \beta$, which is equivalent to $\istype {\Gamma} {\subst A \alpha \beta}$.
%
%   \item Case \[ \wffun \]
%   By straightforwardly applying the i.h and the rule itself.
%
%   \item Case \[ \wfbot \]
%   Trivial.
%
%   \item Case \[ \wfforall \]
%   By straightforwardly applying the i.h and the rule itself.
%
%   \item Case \[ \wfinter \]
%   Let $C$ in the statement of this lemma be $C_1 \inter C_2$.
%   By the condition we know
%   \[ \istype {\Gamma, \alpha \disjoint B} {C_1 \inter C_2} \]
%   Thus we must have,
%   \[ \istype {\Gamma, \alpha \disjoint B} {C_1} \]
%   By the i.h., $\istype \Gamma {\subst A \alpha {C_1}}$ and similarly $\istype \Gamma {\subst A \alpha {C_2}}$. By \rulelabel{WFInter}\todo{Show disjointness},
%   \[ \istype \Gamma {\subst A \alpha {C_1} \inter \subst A \alpha {C_2}} \]
%   and hence
%   \[ \istype \Gamma {\subst A \alpha {(C_1 \inter C_2)}} \]
%
% \end{itemize}
%
% \end{proof}

% \begin{proof}
% By induction on the derivation of $\hastype \Gamma e A$. The case of \rulelabel{TyTApp} needs special attention
% \begin{mathpar}
%   \tytapp
% \end{mathpar}
% because we need to show that the result of substitution ($\subst A \alpha C$) is well-formed, which is evident by Lemma~\ref{instantiation}.
% \end{proof}

% \begin{proof}
% The typing rules are syntax-directed. The case of \rulelabel{TyApp} needs special attention since we still need to show that the generated coercion $C$ is unique.
% \begin{mathpar}
%   \tyapp
% \end{mathpar}
% By Lemma~\ref{wf-typing}, we have $\istype \Gamma {A_1}$ and $\istype \Gamma {A_3}$. Therefore we are able to apply Lemma~\ref{unique-coercion} and conclude that $C$ is unique.
% \end{proof}

% \begin{figure}
%   \begin{mathpar}
%     \framebox{$\isatomic A$} \\
%
%     \inferrule*
%       {}
%       {\isatomic \bot}
%
%     \inferrule*
%       {}
%       {\isatomic {A \to B}}
%
%     \inferrule*
%       {}
%       {\isatomic {\for {\alpha \disjoint B} A}}
%
%   \end{mathpar}
%   \caption{Atomic types.}
% \end{figure}


% \begin{theorem}
%   If $A \subtype C$, then $A \inter B \subtype C$.
%   If $B \subtype C$, then $A \inter B \subtype C$.
% \end{theorem}
%
% \george{Add interpretation of the theorem}

% \begin{proof}
%   By induction on $C$.
%   If $C \neq C_1 \inter C_2$, trivial.
%   If $C = C_1 \inter C_2$,
%   Need to show $A \subtype C_1 \inter C_2$ implies $A \inter B \subtype C_1 \inter C_2$.
%   By inversion $A \subtype C_1$ and $A \subtype C_2$.
%   By the i.h., $A \inter B \subtype C_1$ and $A \inter B \subtype C_2$.
%   By \rulelabel{SubAnd}, $A \inter B \subtype C_1 \inter C_2$.
% \end{proof}
