\section{Proofs}

\paragraph{Notation.} We sketch most of our proofs in two-column style: on the
left are the intermediate results and on the right are the justification (for
the previous intermediate result to reach the corresponding left-hand side).

% \subsection{Subtyping}

% \begin{theorem}[Subtyping is reflexive]
%   If $A$ is a well-formed type, then $A \subtype A$.
% \end{theorem}

% \begin{proof}
%   \george{TODO}
% \end{proof}

% \begin{theorem}[Subtyping is transitive]
%   If $A_1$, $A_2$, $A_3$ are well-formed types, and
%   $A_1 \subtype A_2$ and $A_2 \subtype A_3$, then
%   $A_1 \subtype A_3$.
% \end{theorem}

% \begin{proof}
%   \george{TODO}
% \end{proof}

% \bruno{fix numbering of lemmas}
% \bruno{reflexitivity and transitivity missing. You can do a proof
%   sketch instead of a full proof. Just say in 1 or 2 sentences what is
% the main idea. You can mention that we have a full proof in Coq.}
% \bruno{target type system is missing 3 cases: Tunit; Tproj1; TProj2}

% \subsection{Elaboration}

\lemmasub*
\begin{proof}
  By structural induction of the derivation.

  \begin{itemize}

  \item \textbf{Case}
    \begin{flalign*}
      & \rulesubvar &
    \end{flalign*}

    \begin{tabular}{rll}
      & $ \jtype {\epsilon} {\lam x {\im \alpha} x} {\alpha \to \alpha} $ & By $ \rulelabel{targetvar} $ and $ \george{rulelabeltgtlam} $
    \end{tabular} \\

  \item \textbf{Case}
    \begin{flalign*}
      & \rulesubfun &
    \end{flalign*}

    \begin{tabular}{rll}
      & $ \jtype \epsilon {C_1} {\im {A_3} \to \im {A_1}} $ & By i.h. \\
      & $ \jtype \epsilon {C_2} {\im {A_2} \to \im {A_4}} $ & By i.h. \\
      & $ \jtype {\epsilon, x \oftype {\im {A_3}}} x {\im {A_3}} $ & By $\rulelabel{targetvar}$ \\
      & $ A_3 \subtype A_1 \yields {C_1} $ & Premise \\
      & $ \jtype {\epsilon, x \oftype {\im {A_3}}} {\app {C_1} x} {\im {A_1}} $ & By $\rulelabel{targetApp}$ \\
      & $ \jtype {\epsilon, f \oftype {\im {A_1 \to A_2}}} f {\im {A_1 \to A_2}} $ & By $\rulelabel{targetvar}$ \\
      & $ \jtype {\epsilon, f \oftype {\im {A_1 \to A_2}}} f {\im {A_1} \to \im {A_2}} $ & By the definition of $ \im \cdot $ \\
      & $ \jtype {\epsilon, f \oftype {\im {A_1 \to A_2}}, x \oftype {\im {A_3}}} {\app f {(\app {C_1} x})} {\im {A_2}} $ & By $\rulelabel{targetApp}$ \\
      & $ \jtype {\epsilon, f \oftype {\im {A_1 \to A_2}}, x \oftype {\im {A_3}}} {\app {C_2} {(\app f {(\app {C_1} x})})} {\im {A_4}} $ & By $\rulelabel{targetApp}$ \\
      & $ \jtype \epsilon {\lam f {\im {A_1 \to A_2}} {\lam x {\im {A_3}} {\app {C_2} {(\app f {(\app {C_1} x)})}}}} {\im {A_1 \to A_2} \to \im {A_3} \to \im {A_4}} $ & By applying $\rulelabel{targetlam}$ twice \\
      & $ \jtype \epsilon {\lam f {\im {A_1 \to A_2}} {\lam x {\im {A_3}} {\app {C_2} {(\app f {(\app {C_1} x)})}}}} {\im {A_1 \to A_2} \to \im {A_3 \to A_4}} $ & By the definition of $\im \cdot$
    \end{tabular} \\

  \item \textbf{Case}
    \begin{flalign*}
      & \rulesubforall &
    \end{flalign*}

    \begin{tabular}{rll}
      & $ \jtype \epsilon C {\im {A_1} \to {\im {\subst {\alpha_1} {\alpha_2} A_2}}} $ & By i.h. \\
      & $ \jtype {\epsilon, f \oftype {\im {\for {\alpha_1} {A_1}}}} f {\im {\for {\alpha_1} {A_1}}} $ & By $\rulelabel{targetvar}$ \\
      & $ \jtype {\epsilon, f \oftype {\im {\for {\alpha_1} {A_1}}}} f {\for {\alpha_1} {\im {A_1}}} $ & By the definition of $ \im \cdot $ \\
      & $ \jtype {\epsilon, f \oftype {\im {\for {\alpha_1} {A_1}}}, \alpha} {\tapp f \alpha} {\subst \alpha {\alpha_1} {\im {A_1}}} $ & By $\rulelabel{targetvar}$ and $\rulelabel{targetTApp}$ \\
      & $ \jtype {\epsilon, f \oftype {\im {\for {\alpha_1} {A_1}}}, \alpha} {\app C {(\tapp f \alpha)}} {\subst \alpha {\alpha_1} {\im {\subst {\alpha_1} {\alpha_2} {A_2}}}} $ & By $\george{rulelabeltgtApp}$ \\
      & $ \jtype {\epsilon, f \oftype {\im {\for {\alpha_1} {A_1}}}} {\blam \alpha {\app C {(\tapp f \alpha)}}} {\for {\alpha_2} {\im {A_2}}} $ & By $\george{rulelabeltgtBLam}$ and substitution \george{Substitution is problematic} \\
      & $ \jtype \epsilon {\lam f {\im {\for {\alpha_1} {A_1}}} {\blam \alpha {\app C {(\tapp f \alpha)}}}} {\im {\for {\alpha_1} {A_1}} \to \for {\alpha_2} {\im {A_2}}} $ & By $\george{rulelabeltgtlam}$ \\
      & $ \jtype \epsilon {\lam f {\im {\for {\alpha_1} {A_1}}} {\blam \alpha {\app C {(\tapp f \alpha)}}}} {\im {\for {\alpha_1} {A_1}} \to \im {\for {\alpha_2} {A_2}}} $ & By the definition of $\im \cdot$
    \end{tabular} \\

  \item \textbf{Case}
    \begin{flalign*}
      & \rulesubinter &
    \end{flalign*}

    \begin{tabular}{rll}
      & $\jtype {\epsilon, x \oftype \im {A_1}} x {\im {A_1}}$ & By $\george{rulelabeltgtvar}$ \\
      & $\jtype \epsilon {C_1} {\im {A_1} \to \im {A_2}}$ & By i.h. \\
      & $\jtype {\epsilon, x \oftype \im {A_1}} {\app {C_1} x} {\im {A_2}}$ & By $\george{rulelabeltgtApp}$ and weakening \\
      & $\jtype {\epsilon, x \oftype \im {A_1}} {\app {C_2} x} {\im {A_3}}$ & Similar \\
      & $\jtype {\epsilon, x \oftype \im {A_1}} {\pair {\app {C_1} x} {\app {C_2} x}} {\pair {\im {A_2}} {\im {A_3}}}$ & By $\george{rulelabeltgtPair}$ \\
      & $\jtype {\epsilon, x \oftype \im {A_1}} {\pair {\app {C_1} x} {\app {C_2} x}} {\im {A_2 \inter A_3}}$ & By the definition of $\im \cdot$ \\
      & $\jtype \epsilon {\lam x {\im {A_1}} {\pair {\app {C_1} x} {\app {C_2} x}}} {\im {A_1} \to {\im {A_2 \inter A_3}}}$ & By $\george{rulelabeltgtlam}$
    \end{tabular} \\

  \item \textbf{Case}
    \begin{flalign*}
      & \rulesubinterl &
    \end{flalign*}

    \begin{tabular}{rll}
      & $ \jtype {\epsilon, x \oftype \im {A_1 \inter A_2}} x {\im {A_1 \inter A_2}} $ & By $\george{rulelabeltgtvar}$ \\
      & $ \jtype {\epsilon, x \oftype \im {A_1 \inter A_2}} x {\pair {\im {A_1}} {\im {A_2}}} $ & By the definition of $\im \cdot$ \\
      & $ \jtype {\epsilon, x \oftype \im {A_1 \inter A_2}} {\proj 1 x} {\im {A_1}} $ & By $\george{rulelabeltgtprojl}$ \\
      & $ \jtype \epsilon C {\im {A_1} \to \im {A_3}} $ & By i.h. \\
      & $ \jtype {\epsilon, x \oftype \im {A_1 \inter A_2}} C {\im {A_1} \to \im {A_3}} $ & By weakening \\
      & $ \jtype {\epsilon, x \oftype \im {A_1 \inter A_2}} {\app C {(\proj 1 x)}} {\im {A_3}} $ & By $\george{rulelabeltgtApp}$ \\
      & $ \jtype \epsilon {\lam x {\im {A_1 \inter A_2}} {\app C {(\proj 1 x)}}} {\im {A_1 \inter A_2} \to \im {A_3}} $ & By $\george{rulelabeltgtlam}$
    \end{tabular} \\

  \item \textbf{Case}
    \begin{flalign*}
      & \rulesubinterr &
    \end{flalign*}

    By symmetry with the above case. \\

  \end{itemize}

\end{proof}

\begin{lemma}[Preservation of well-formedness under type translation] \rulelabel{lemma:preserve-wf}
  If $ \jwf \Gamma A $, then $ \jwf {\im \Gamma} {\im A} $.
\end{lemma}

\begin{proof}
  By structural induction of the derivation. The only case to consider is $ \rulelabel{WF} $.

  \begin{itemize}

  \item \textbf{Case}

    \begin{flalign*}
      & \ruleWF &
    \end{flalign*}

    \begin{tabular}{rll}
      & $ \ftv A \in \Gamma $ & Premise \\
      & $ \ftv {\im A} \in \im \Gamma $ & By the definition of $ \im \cdot $ \\
      & $ \jwf {\im \Gamma} {\im A} $ & By $ \rulelabel{targetWF} $
    \end{tabular} \\

  \end{itemize}
\end{proof}

\theorempreservation*
\begin{proof}
  By structural induction of the derivation.

  \begin{itemize}

  \item \textbf{Case}
    \begin{flalign*}
      & \ruletvar &
    \end{flalign*}

    \begin{tabular}{rll}
     & $ (x,A) \in \Gamma $ & Premise \\
     & $ (x,\im A) \in \im \Gamma $ & By the definition of $ \im \cdot $ \\
     & $ \jtype {\im \Gamma} x {\im A} $ & By $ \rulelabel{targetvar} $
    \end{tabular} \\

  \item \textbf{Case}
    \begin{flalign*}
      & \ruletlam &
    \end{flalign*}

    \begin{tabular}{rll}
      & $ \jtype {\Gamma, x \oftype A} e {A_1} \yields E $ & Premise \\
      & $ \jtype {\im {\Gamma, x \oftype A}} E {\im {A_1}} $ & By i.h. \\
      & $ \jtype {\im \Gamma, x \oftype \im A} E {\im {A_1}} $ & By the definition of $ \im \cdot $ \\
      & $ \jtype {\im \Gamma} {\lam x {\im A} E} {\im A \to \im {A_1}} $ & By $ \rulelabel{targetlam} $ \\
      & $ \jtype {\im \Gamma} {\lam x {\im A} E} {\im {A \to A_1}} $ & By the definition of $ \im \cdot $
    \end{tabular} \\

  \item \textbf{Case}
    \begin{flalign*}
      & \rulettapp &
    \end{flalign*}

    \begin{tabular}{rll}
     & $ \jtype \Gamma {e_1} {A_1 \to A_2} \yields {E_1} $  & Premise \\
     & $ \jtype {\im \Gamma} {E_1} {\im {A_1 \to A_2}} $ & By i.h. \\
     & $ \jtype \Gamma {e_2} {A_3} \yields {E_2} $ & Premise \\
     & $ \jtype {\im \Gamma} {E_2} {\im {A_3}} $ & By i.h. \\
     & $ A_3 \subtype A_1 \yields C $ & Premise \\
     & $ \jtype \epsilon C {\im {A_3} \to \im {A_1}} $ & By Lemma~\ref{lemma:sub} \\
     & $ \jtype {\im \Gamma} {\app {E_1} {(\app C E_2)}} {\im {A_2}} $ & By $ \rulelabel{targetApp} $ and the definition of $ \im \cdot $
    \end{tabular} \\

  \item \textbf{Case}
    \begin{flalign*}
      & \ruletblam &
    \end{flalign*}

    \begin{tabular}{rll}
      & $ \jtype {\Gamma, \alpha} e A \yields E $ & Premise \\
      & $ \jtype {\im {\Gamma, \alpha}} E {\im A} $ & By i.h. \\
      & $ \jtype {\im \Gamma, \alpha} E {\im A} $ & By the definition of $ \im \cdot $ \\
      & $ \jtype {\im \Gamma} {\blam \alpha E} {\for \alpha {\im A}} $ & By $ \rulelabel{targetBLam} $ \\
      & $ \jtype {\im \Gamma} {\blam \alpha E} {\im {\for \alpha A}} $ & By the definition of $ \im \cdot $
    \end{tabular} \\

  \item \textbf{Case}
    \begin{flalign*}
      & \rulettapp &
    \end{flalign*}

    \begin{tabular}{rll}
     & $ \jtype \Gamma e {\for \alpha A_1} \yields E $ & Premise \\
     & $ \jtype {\im \Gamma} E {\im {\for \alpha A_1}} $ & By i.h. \\
     & $ \jtype {\im \Gamma} E {\for \alpha \im {A_1}} $ & By the definition of $ \im \cdot $ \\
     & $ \jwf \Gamma A $ & Premise \\
     & $ \jwf {\im \Gamma} {\im A} $ & By Lemma~\ref{lemma:preserve-wf} \\
     & $ \jtype \Gamma {\tapp E {\im A}} {\subst {\im A} \alpha {\im {A_1}}} $ & By $ \rulelabel{targetTApp} $ \\
     & $ \jtype \Gamma {\tapp E {\im A}} {\im {\subst A \alpha {A_1}}} $ & By substitution lemma
    \end{tabular} \\

  \item \textbf{Case}
    \begin{flalign*}
      & \ruletmerge &
    \end{flalign*}

    \begin{tabular}{rll}
      & $ \jtype \Gamma {e_1} {A_1} \yields {E_1} $ & Premise \\
      & $ \jtype {\im \Gamma} {E_1} {\im {A_1}} $ & By i.h. \\
      & $ \jtype {\im \Gamma} {E_2} {\im {A_2}} $ & Similar to the above \\
      & $ \jtype {\im \Gamma} {\pair {E_1} {E_2}} {\pair {\im {A_1}} {\im {A_2}}} $ & By $ \rulelabel{targetPair} $ \\
      & $ \jtype {\im \Gamma} {\pair {E_1} {E_2}} {\im {A_1 \inter A_2}} $ & By the definition of $ \im \cdot $
    \end{tabular} \\

  \end{itemize}
\end{proof}

\begin{definition}{Type variable constraint}
  We say the \emph{constraint} of a type variable $\alpha$ inside the context
  $\Gamma$ is $A$ if $\alpha \disjoint A \in \Gamma$.
\end{definition}

\begin{lemma}
If $A \subtype B$ where both $A$ and $B$ are well-formed, then $A$ and $B$ cannot be disjoint.
\end{lemma}

\begin{proof}
$A \subtype B$ implies $B$ is a common supertype of $A$ and $B$. As a result, $A$ and $B$ are not disjoint by definition.
\end{proof}


\begin{lemma}[Free type variables of disjoint bounds] \label{free-var-disjoint-bounds}
  If $\jdis \Gamma \alpha A$, then $\alpha \not \in \ftv A$.
\end{lemma}

\begin{proof}
Since $A \inter B$ is well-formed, $A \disjoint B$ by the formation rule of intersection types \rulelabel{WF\_Inter}. Then by the definition of disjointness, there does not exist a type $C$ such that $A \subtype C$ and $B \subtype C$. It follows that $A \subtype C$ and $B \subtype C$ cannot hold simultaneously.
\end{proof}

\begin{proof}
The set of rules for generating coercions is syntax-directed except for the three rules that involve intersection types in the conclusion. Therefore it suffices to show that if well-formed types $A$ and $B$ satisfy $A \subtype B$, where $A$ or $B$ is an intersection type, then at most one of the three rules applies. In the following, we do a case analysis on the shape of $A$ and $B$:

\begin{itemize}
  \item \textbf{Case} $A \neq A_1 \inter A_2$ and $B = B_1 \inter B_2$: Clearly only \textsc{subinter} can apply.
  \item \textbf{Case} $A = A_1 \inter A_2$ and $B \neq B_1 \inter B_2$: Only two rules can apply, \textsc{subinter1} and \textsc{subinter2}. Further, by the unique subtype contributor lemma, it is not possible that $A_1 \subtype B$ and that $A_2 \subtype B$. Thus we are certain that at most one rule of \textsc{subinter1} and \textsc{subinter2} will apply.
  \item \textbf{Case} $A = A_1 \inter A_2$ and $B = B_1 \inter B_2$\footnote{An example of this case is:
    \[ (\integer \inter \bool) \inter \character \subtype \bool \inter \character \]}: Since $B$ is not atomic, only \rulelabel{subinter} apply.

  %   Suppose the contrary, that is, more than one of the three rules apply. Since it is not possible that both \textsc{subinter1} and \textsc{subinter2} apply by the unique subtype contributor lemma, only one of \textsc{subinter1} and \textsc{subinter2} apply. Therefore \textsc{subinter} has to hold. Without the loss of generality, assume \textsc{subinter1} apply. Then we have:
  % \[ A_1 \subtype B_1 \inter B_2 \]
  % \[ A_1 \inter A_2 \subtype B_1 \]
  % \[ A_1 \inter A_2 \subtype B_2 \]
\end{itemize}
\end{proof}

\begin{lemma}{Invariance of disjointness}
  \label{invariance-of-disjointness}

  If $\jdis \Gamma A B$ and $R$ respects the constraints of $\beta$, then
  $\jdis \Gamma {\subst R \beta A} {\subst R \beta B}$.

\end{lemma}

This lemma says that substitution for free type variables preserves disjointness
of types if the combination of the replacement type and the type variable is
proven disjoint.

\begin{proof}
By induction on the derivation of $\jdis \Gamma A B$.
\begin{itemize}
  \item Case \[ \ruledisvar \]
  We need to show \[ \jdis \Gamma {\subst R \beta \alpha} {\subst R \beta B} \]
  If $\beta$ is not equivalent to $\alpha$ and is not free in $B$, then the above trivially holds by the def. of the substitution function. Otherwise, if $\beta$ is equivalent to $\alpha$, then we need to show
  \[ \jdis \Gamma R {\subst R \beta B} \]

  % Note that $\beta \not \in \ftv B$. Thus $B$ is equivalent to $\subst R \beta B$.
  %
  % If $\beta$ is not equivalent to $\alpha$, $\subst R \beta \alpha$ is equivalent to $\alpha$. Therefore $\jdis \Gamma {\subst R \beta \alpha} {\subst R \beta B}$ is true.
  % If $\beta$ is equivalent to $\alpha$, then $\subst R \beta \alpha$ is equivalent to $R$ by the def. of the substitution function. It now remains to show \[ \jdis \Gamma R B \].

  \item Case \[ \ruledisinterl \]
  By applying the i.h. and the def. of the substitution function.

  \item Case \[ \ruledisinterr \]
  Similar.

  \item Case \[ \ruledisfun \]
  By applying the i.h. and the def. of the substitution function.

  \item Case \[ \ruledisforall \]
  By applying the i.h. and the def. of the substitution function. Note that $\alpha$ is fresh.

  \item Case \[ \ruledisatomic \]
  Substitution does not change the shape of types when the variable case is excluded. Therefore, the relation in the premise of the rule continue to hold and hence the conclusion.

\end{itemize}
\end{proof}

\begin{lemma}{Substitution} \label{substitution}

  If $\jwf \Gamma R$, $\jwf \Gamma S$, and $R$ respects the constraints of
  $\beta$, then $\jwf \Gamma {\subst R \beta S}$.

\end{lemma}

\begin{proof}
By induction on the derivation of $\jwf \Gamma {\subst R \beta S}$.

\begin{itemize}
  \item Case \[ \rulewfvar \]
  If $\alpha$ happens to be the same as $\beta$, then by the def. of substitution $\subst R \beta \alpha = R$. Since $\jwf \Gamma R$, we have $\jwf \Gamma {\subst R \beta \alpha}$; On the other hand, if not, then by the def. of substitution $\subst R \beta S = S$. Since $\jwf \Gamma S$, we also have $\jwf \Gamma {\subst R \beta \alpha}$.

  \item Case
  \begin{mathpar}
    \rulewfbot
  \end{mathpar}
  Trivial.

  \item Case
  \begin{mathpar}
    \rulewffun
  \end{mathpar}
  By i.h., $\jwf \Gamma {\subst R \beta A}$ and $\jwf \Gamma {\subst R \beta B}$. By the def. of substitution, $\jwf \Gamma {\subst R \beta {A \to B}}$.

  \item Case
  \begin{mathpar}
    \rulewfforall
  \end{mathpar}
  By the premise and the i.h.,
  \[ \jwf {\Gamma} {\subst R \beta A} \]
  \[ \jwf {\Gamma, \alpha \disjoint A} {\subst R \beta B} \]
  which by \rulelabel{WFForall} implies
  \[ \jwf \Gamma {\for {\alpha \disjoint A} {\subst R \beta B}} \]
  By the def. of substitution, $\jwf \Gamma {\subst R \beta {\for {\alpha \disjoint A} B}}$~

  \item Case
  \begin{mathpar}
    \rulewfinter
  \end{mathpar}
  By i.h., $\jwf \Gamma {\subst R \beta A}$ and $\jwf \Gamma {\subst R \beta B}$. By Lemma~\ref{invariance-of-disjointness}, we also have $\jdis \Gamma {\subst R \beta A} {\subst R \beta B}$. Therefore by \rulelabel{WFInter}, $\jwf \Gamma {\subst R \beta {A \inter B}}$.
\end{itemize}
\end{proof}



\begin{proof}
By induction.

\begin{itemize}
  \item Case \[ \rulewfvar \]
  If $C = \alpha$, then $\subst A \alpha \alpha = A$. Since $\jwf \Gamma A$, it follows that $\jwf \Gamma {\subst A \alpha \alpha}$; otherwise, let $C = \beta$, where $\beta$ is a type variable distinct from $\alpha$. Since $\jwf {\Gamma, \alpha \disjoint B} \beta$ and $\alpha$ and $\beta$ are distinct, $\beta$ must be in $\Gamma$ and therefore $\jwf {\Gamma} \beta$, which is equivalent to $\jwf {\Gamma} {\subst A \alpha \beta}$.

  \item Case \[ \rulewffun \]
  By straightforwardly applying the i.h and the rule itself.

  \item Case \[ \rulewfbot \]
  Trivial.

  \item Case \[ \rulewfforall \]
  By straightforwardly applying the i.h and the rule itself.

  \item Case \[ \rulewfinter \]
  Let $C$ in the statement of this lemma be $C_1 \inter C_2$.
  By the condition we know
  \[ \jwf {\Gamma, \alpha \disjoint B} {C_1 \inter C_2} \]
  Thus we must have,
  \[ \jwf {\Gamma, \alpha \disjoint B} {C_1} \]
  By the i.h., $\jwf \Gamma {\subst A \alpha {C_1}}$ and similarly $\jwf \Gamma {\subst A \alpha {C_2}}$. By \rulelabel{WFInter}\george{Show disjointness},
  \[ \jwf \Gamma {\subst A \alpha {C_1} \inter \subst A \alpha {C_2}} \]
  and hence
  \[ \jwf \Gamma {\subst A \alpha {(C_1 \inter C_2)}} \]

\end{itemize}

\end{proof}

\begin{proof}
By induction on the derivation of $\jtype \Gamma e A$. The case of \rulelabel{TyTApp} needs special attention
\begin{mathpar}
  \rulettapp
\end{mathpar}
because we need to show that the result of substitution ($\subst A \alpha C$) is well-formed, which is evident by Lemma~\ref{instantiation}.
\end{proof}

\begin{proof}
The typing rules are syntax-directed. The case of \rulelabel{TyApp} needs special attention since we still need to show that the generated coercion $C$ is unique.
\begin{mathpar}
  \rulettapp
\end{mathpar}
By Lemma~\ref{wf-typing}, we have $\jwf \Gamma {A_1}$ and $\jwf \Gamma {A_3}$. Therefore we are able to apply Lemma~\ref{unique-coercion} and conclude that $C$ is unique.
\end{proof}

\begin{figure}
  \begin{mathpar}
    \framebox{$\isatomic A$} \\

    \inferrule*
      {}
      {\isatomic \bot}

    \inferrule*
      {}
      {\isatomic {A \to B}}

    \inferrule*
      {}
      {\isatomic {\for {\alpha \disjoint B} A}}

  \end{mathpar}
  \caption{Atomic types.}
\end{figure}


\begin{theorem}
  If $A \subtype C$, then $A \inter B \subtype C$.
  If $B \subtype C$, then $A \inter B \subtype C$.
\end{theorem}

\george{Add interpretation of the theorem}

\begin{proof}
  By induction on $C$.
  If $C \neq C_1 \inter C_2$, trivial.
  If $C = C_1 \inter C_2$,
  Need to show $A \subtype C_1 \inter C_2$ implies $A \inter B \subtype C_1 \inter C_2$.
  By inversion $A \subtype C_1$ and $A \subtype C_2$.
  By the i.h., $A \inter B \subtype C_1$ and $A \inter B \subtype C_2$.
  By \rulelabel{subinter}, $A \inter B \subtype C_1 \inter C_2$.
\end{proof}

\begin{lemma}{Symmetry of disjointness} \label{symmetry-of-disjointness}
  If $\jdis \Gamma A B$, then $\jdis \Gamma B A$.
\end{lemma}

\begin{proof}
  Trivial by the definition of disjointness.
\end{proof}

\begin{theorem} \label{disjoint-intersect}
  If $\jdis \Gamma A C$ and $\jdis \Gamma B C$,
  then $\jdis \Gamma {A \inter B} {C}$.
\end{theorem}

\begin{lemma} \label{common-supertype}
  If $A_1 \to A_2 \subtype D$ and $B_1 \to B_2 \subtype D$,
  then there exists a $C$ such that $A_2 \subtype C$ and $B_2 \subtype C$.
\end{lemma}

\begin{proof}
  By induction on $D$.
\end{proof}

\begin{theorem}{Soundness}
  For any two types $A$, $B$, $\jdisimpl \Gamma A B$ implies $\jdis \Gamma A B$.
\end{theorem}

\begin{proof}
  By induction on $*_I$.

  \begin{itemize}
    \item Case \[ \ruledisfun \]

    Lemma~\ref{common-supertype}

    \george{May need an extracted lemma here}

    \item Case \[ \ruledisinterl \]

    By Lemma~\ref{disjoint-intersect} and the i.h.

    \item Case \[ \ruledisinterr \]

    By Lemma~\ref{disjoint-intersect}, Lemma~\ref{symmetry-of-disjointness}, and the i.h.

    \item Case \[ \ruledisatomic \]
    Need to show ...
    By unfolding the definition of disjointness
    Need to show there does not exists $C$ such that...
    By induction on $C$.
    Atomic cases...
    If $C = C_1 \inter C_2$
    By inversion and the i.h. we arrive at a contradiction.

  \end{itemize}
\end{proof}

\begin{theorem}{Completeness}
  For any two type $A$, $B$, $\jdis \Gamma A B$ implies $\jdisimpl \Gamma A B$.
\end{theorem}

\begin{proof}
  Induction on $A$.

  \begin{itemize}
    \item Case $\bot$

    Induction on $B$.
      \begin{itemize}
        \item Case $B = \bot$

          Need to show $\Gamma \turns \bot * \bot$ implies $\Gamma \turns \bot *_I \bot$. Take $C = \bot$. Clearly the premise is false by definition. Then the whole statement is true. \george{???}

        \item Case $B = B_1 \to B_2$
        The conclusion is true by the disjoint axioms.

        \item Case $B = B_1 \inter B_2$.
        Need to show $\Gamma \turns \bot * B_1 \inter B_2$ implies $\Gamma \turns \bot *_I B_1 \inter B_2$. Apply \rulelabel{DisjointInter2} and the resulting conditions can be proved by the i.h.

        \item Case
      \end{itemize}

      \item $A = A_1 \to A_2$
        \begin{itemize}
          \item Case $B = \bot$
          The conclusion is true by the disjoint axioms.

          \item Case $B = B_1 \to B_2$
          Need to show $\Gamma \turns  A_1 \to A_2 * B_1 \to B_2$ implies $\Gamma \turns  A_1 \to A_2 *_I B_1 \to B_2$. Apply \rulelabel{DisjointFun} and the result, $\Gamma \turns A_2 *_I B_2$, can be proved by the i.h.

          \item Case $B = B_1 \inter B_2$.
          Need to show $\Gamma \turns A_1 \to A_2 * B_1 \inter B_2$ implies $\Gamma \turns A_1 \to A_2 *_I B_1 \inter B_2$. Apply \rulelabel{DisjointInter2} and the resulting conditions can be proved by the i.h.
        \end{itemize}

        \item $A = A_1 \inter A_2$
        By \rulelabel{DisjointInter1} and by the i.h.

      \end{itemize}
\end{proof}
