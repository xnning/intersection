%*******************************************************************************
\section{Overview} \label{sec:overview}
%*******************************************************************************

This section introduces \name and its support for intersection types and the
merge operator. It then discusses the challenges of coherence and shows how the
notion of disjoint intersection types achieve a coherent semantics.

Note that this section uses some syntactic sugar, as well as standard
programming language features, to illustrate the various concepts in
\name. Although the minimal core language that we formalize in
Section~\ref{sec:fi} does not present all such features, our implementation
supports them.

\subsection{Intersection Types and the Merge Operator}
%%\subsection{Intersection Types, Merge and Polymorphism in \name}

%However, as we shall see in
%Section~\ref{subsec:incoherence}, it also introduces difficulties. In what follows
%intersection types and the merge operator are informally introduced.

\paragraph{Intersection Types.}
The intersection of type $A$ and $B$ (denoted as \lstinline{A & B} in
\name) contains exactly those values
which can be used as values of type $A$ and of type $B$. For instance,
consider the following program in \name:

\begin{lstlisting}
let x : Int & Bool = (*$ \ldots $*) in -- definition omitted
let succ (y : Int) : Int = y+1 in
let not (y : Bool) : Bool = if y then False else True in
(succ x, not x)
\end{lstlisting}

\noindent If a value \lstinline{x} has type \lstinline{Int & Bool} then
\lstinline{x} can be used as an integer and as a boolean. Therefore,
\lstinline{x} can be used as an argument to any function that takes
an integer as an argument, and any
function that take a boolean as an argument. In the program above
the functions \lstinline{succ} and \lstinline{not} are 
simple functions on integers and characters, respectively.
Passing \lstinline{x} as an argument to either one (or both) of the
functions is valid.

\paragraph{Merge Operator.}
In the previous program we deliberately did not show how to introduce
values of an intersection type. There are many variants of
intersection types in the literature. Our work follows a particular
formulation, where intersection types are introduced by a \emph{merge
  operator}~\cite{reynolds1991coherence,Castagna92calculus,dunfield2014elaborating}. 
As Dunfield~\cite{dunfield2014elaborating} has argued a
merge operator adds considerable expressiveness to a calculus. The
merge operator allows two values to be merged in a single intersection
type. In \name (following Dunfield's notation), the merge of two
values $v_1$ and $v_2$ is denoted as $v_1 ,, v_2$.  For example, an
implementation of \lstinline{x} is constructed in \name as follows:

\begin{lstlisting}
let x : Int & Boolean = 1,,True in (*$ \ldots $*)
\end{lstlisting}

\paragraph{Merge Operator and Pairs.}
The merge operator is similar to the introduction construct on pairs.
An analogous implementation of \lstinline{x} with pairs would be:

\begin{lstlisting}
let xPair : (Int, Bool) = (1, True) in (*$ \ldots $*)
\end{lstlisting}

\noindent The significant difference between intersection types with a
merge operator and pairs is in the elimination construct. With pairs
there are explicit eliminators (\lstinline{fst} and
\lstinline{snd}). These eliminators must be used to extract the
components of the right type. For example, in order to use
\lstinline{succ} and \lstinline{not} with pairs, we would need to
write a program such as:

\begin{lstlisting}
(succ (fst xPair), not (snd xPair))
\end{lstlisting}

\noindent In contrast the elimination of intersection types is done
implicitly, by following a type-directed process. For example,
when a value of type \lstinline{Int} is needed, but an intersection
of type \lstinline{Int & Bool} is found, the compiler generates code 
that uses \lstinline{fst} to extract the corresponding value at runtime.

\subsection{(In)Coherence} \label{subsec:incoherence}

Coherence is a desirable property for the semantics of a programming
language. A semantics is said to be coherent if any \emph{valid
  program} has exactly one meaning~\cite{reynolds1991coherence} (that
is, the semantics is not ambiguous). In contrast a semantics is said
to be \emph{incoherent} if there are multiple possible meanings for
the same valid program. 

\paragraph{Incoherence in Dunfield's calculus}
A problem with Dunfield's calculus~\cite{dunfield2014elaborating} is
that it is incoherent.  Unfortunately the implicit nature of
elimination for intersection types built with a merge operator can
lead to incoherence.  The merge operator combines two terms, of type
$A$ and $B$ respectively, to form a term of type $A \inter B$. For
example, $1 \mergeop True$ is of type $\code{Int} \inter
\code{Bool}$. In this case, no matter whether $1 \mergeop True$ is used as
$\code{Int}$ or $\code{Bool}$, the result of evaluation is always
clear. However, with overlapping types, it is not clear 
anymore to see the intended result. For example, what should be the result of
the following program, which asks for an integer (using a type annotation) out
of a merge of two integers:
\begin{lstlisting}
(1,,2) : Int
\end{lstlisting}
Should the result be \lstinline$1$ or \lstinline$2$?

Dunfield's calculus~\cite{dunfield2014elaborating} accepts the program
above, and it allows that program to result in \lstinline$1$ or \lstinline$2$.
In other words the results of the program are incoherent.

\paragraph{Getting Around Incoherence}
In a real implementation of Dunfield calculus a choice has to be made
on which value to compute. For example, one potential option is to
always take the left-most value matching the type in the
merge. Similarly, one could always take the right-most value matching
the type in the merge. Either way, the meaning of a program will
depend on a biased implementation choice, which is clearly
unsatisfying from the theoretical point of view.  Dunfield suggests
some other possibilities, such as the possibility of restricting typing
of merges so that a merge has type \lstinline$A$ only if exactly one
branch has type \lstinline$A$. He also suggested another possibility,
which is to allow only for \emph{disjoint} types in an intersection.
This is the starting point for us and the approach that we 
investigate in this paper. 

\subsection{Disjoint Intersection Types and their Challenges}\label{subsec:challenges}
\name requires that the
two types in an intersection to be \emph{disjoint}.
Informally saying that two types are disjoint means that the set of
values of both types are disjoint. Disjoint intersection types are
potentially useful for coherence, since they can rule out ambiguity
when looking up a value of a certain type in an intersection. However
there are several issues that need to be addressed first in order to
design a calculus with disjoint intersection types and that ensures
coherence. The key issues and the solutions provided by our work are
discussed next. We emphasize that even though Dunfield
has mentioned disjointness as an option to restore coherence, he
has not studied the approach further or addressed the issues discussed next.

\paragraph{Simple disjoint intersection types}
Looking back at the expression $1 \mergeop 2$ in
Section~\ref{subsec:incoherence}, we can see that the reason for
incoherence is that there are multiple, overlapping, integers in the
merge. Generally speaking, if both terms can be assigned some type
$C$, both of them can be chosen as the meaning of the merge, which
leads to multiple meanings of a term. A natural option is to try
to forbid such overlapping values of the same type in a merge.
Thus, for simple types such as \lstinline$Int$ and \lstinline$Bool$, it is
easy to see that disjointness holds when the two types are
different. Intersections such as \lstinline$Int & Bool$ and 
\lstinline$String & Bool$ are clearly disjoint.
While an informal, intuitive notion of disjointness is sufficient to
see what happens with simple types, it is less clear of what
disjointness means in general.

\paragraph{Formalizing disjointness} 
Clearly a formal notion of disjointness is needed to design a
calculus with disjoint intersection types, and to clarify what
disjointness means in general.
%Another challenge is how to formalize 
%the notion of disjointness? So far all discussion about disjointness has been 
%based on an informal understanding of what disjointness means. Clearly a formal 
%definition of disjointness is desirable to design a calculus with disjoint 
%intersection types. 
As we shall see the particular notion of disjointness is quite
sensitive to the features that are allowed in a language.
Nevertheless, the different notions of disjointness follow the same
principle: they are defined in terms of the subtyping relation; 
%\joao{should we (briefly) introduce this subtyping relation beforehand?}
and they describe which common supertypes are allowed in order for 
two types to be considered disjoint.

A first attempt at a definition for disjointness is to require that,
given two types $A$ and $B$, both types are not subtypes of each
other. Thus, denoting disjointness as $A * B$, we would have:
\[A * B \equiv A \not<: B~\text{and}~B \not<: A\]
At first sight this seems a reasonable definition and it does prevent
merges such as \lstinline{1,,2}. However some moments of thought are enough to realize that
such definition does not ensure disjointness. For example, consider
the following merge:

\begin{lstlisting}
(1,,'c') ,, (2,,True)
\end{lstlisting}

\noindent This merge has two components which are also merges. 
The first component \lstinline{(1,,'c')} has type $\code{Int} \inter
\code{Char}$, whereas the second component \lstinline{(2 ,, True)} has type
$\code{Int} \inter \code{Bool}$. Clearly,
\[ \code{Int} \inter \code{Char} \not \subtype \code{Int} \inter \code{Bool} \wedge \code{Int} \inter \code{Bool} \not \subtype \code{Int} \inter \code{Char} \]
Nevertheless the following program still leads to
incoherence:
\begin{lstlisting}
succ ((1,,'c'),,(2,,True))
\end{lstlisting}
as both \lstinline{2} or \lstinline{3} are possible outcomes
of the program. Although this attempt to define disjointness failed,
it did bring us some additional insight: although the types of the two
components of the merge are not subtypes of each other, they share
some types in common.

In order for two types to be truly disjoint, they must not have any
subcomponents sharing the same type. In a simply typed calculus with intersection
types (and without a $\top$ type) this can be ensured by requiring the two types 
not to share a common supertype: 

\begin{definition}[Simple disjointness]
  Two types $A$ and $B$ are disjoint
  (written $A \disjoint B$) if there is no type $C$ such that both $A$ and $B$ are
  subtypes of $C$:
  \[A \disjoint B \equiv \not\exists C.~A \subtype C~\text{and}~B\subtype C\]
\end{definition}\label{def:simple_dis}

\noindent This definition of disjointness prevents the problematic merge
\lstinline$(1,,'c'),,(2,,True)$. Since $\code{Int}$ is a
common supertype of both $\code{Int} \& \code{Char}$ and $\code{Int}
\& \code{Bool}$, those two types are not disjoint, according to this
simple notion of disjointness.

The simple definition of disjointness is the basis for the first calculus 
presented in this paper: a simply typed lambda calculus with intersection 
(but without a $\top$ type). 
This variant of \name is useful to study many 
important issues arizing from disjoint intersections, without the additional 
complications of $\top$. As shown in Section~\ref{sec:disjoint}, this
definition of disjointness is sufficient to ensure coherence in \name.

\paragraph{Disjointness of other types} Equipped with a formal
notion of disjointness, we are now ready to see how disjointness works 
for other types. 
For example, consider the following intersection types of functions:

\begin{enumerate}

\item $(\code{Int} \to \code{Int})~\inter~(\code{String} \to \code{String})$
\item $(\code{String} \to \code{Int})~\inter~(\code{String} \to \code{String})$
\item $(\code{Int} \to \code{String})~\inter~(\code{String} \to \code{String})$

\end{enumerate}

\noindent Which of those intersection types are disjoint? 
It seems reasonable to expect that the first intersection type is
disjoint: both the domain and codomain of the two functions in the
intersection are different. However, it is less clear whether the two
other intersection types are disjoint or not. Looking at the 
definition of simple disjointness for further guidance, and the subtyping rule for functions in
\name (which is standard~\cite{cardelli88semantics}):

\begin{mathpar}
    \inferrule* [right=\labelsubfun]
    {{B_1} \subtype {A_1} \\
     {A_2} \subtype {B_2} }
    {{A_1 \to A_2} \subtype {B_1 \to B_2}}
\end{mathpar}

\noindent we can see that the types in the 
second intersection do not share any common supertypes. Since the
target types of the two function types
(\lstinline$Int$ and \lstinline$String$) do not share a common
supertype, 
it is not possible to find a type \lstinline$C$ that is both a common
supertype of $(\code{String} \to \code{Int})$ and $(\code{String} \to \code{String})$.
In contrast, for the third intersection type, it is possible to find a 
common supertype: $\code{String}~\inter~\code{Int} \to
\code{String}$. The contravariance of argument types
in \reflabel{\labelsubfun}~is important here. All that we need in order to find a common supertype 
between $(\code{Int} \to \code{String})$ and $(\code{String} \to
\code{String})$ is to find a common subtype between $\code{Int}$ and 
$\code{String}$. One such common subtype is
$\code{String}~\inter~\code{Int}$. Preventing the third intersection
type ensures that type-based lookups are not ambiguous (and cannot
lead to incoherence). If the third intersection type was allowed then
the following program:

\begin{lstlisting}
f,,g : String & Int (*$\to$*) String
\end{lstlisting}

\noindent where \lstinline$f$ is of type
$\code{Int}~\to~\code{String}$  and \lstinline$g$ is of type
$\code{String}~\to~\code{String}$, would be problematic. In this 
case both \lstinline$f$ or \lstinline$g$ could be selected,
potentially leading to very different (and incoherent) results when
applied to some argument. 

\begin{comment}
According to the rule the first 
two intersection types are disjoint, whereas the last one is not. This 
rule is justified in Section~\ref{}.\bruno{Should we stop here; or briefly argue 
about some justification} 

This paper proposes the 
following rule for determining the disjointness of two function types:

\begin{mathpar}
    \ruledisfun
\end{mathpar}

\noindent Here $*_i$ denotes disjointness. 
\end{comment}

\paragraph {Is disjointness sufficient to ensure coherence?} Another question
is whether disjoint intersection types are sufficient to ensure coherence. 
Consider the following example:

\begin{lstlisting}
(succ,,not) (3,,True)
\end{lstlisting}

\noindent Here there are two merges, with the first merge being applied to the second. 
The first merge contains two functions 
(\lstinline$succ$ and \lstinline$not$). The second merge contains two values 
that can serve as input to the functions. The two merges are disjoint. 
However what should be the result of this program? Should it be 
\lstinline$4$ or \lstinline$False$? %\joao{why not (4,,False)?} 

There is semantic ambiguity in this program, even though it only uses
disjoint merges. However a close look reveals 
a subtly different problem from the previous programs. There are two possible 
types for this program: \lstinline{Int} or \lstinline{Bool}. Once the type 
of the program is fixed, there is only one possible result: if the type is 
\lstinline$Int$ the result of the program is \lstinline$4$; if the type 
is \lstinline$Bool$ then the result of the program is \lstinline$False$. 
Like other programming language features (for example type classes~\cite{Wadler89ad-hoc}), 
types play a fundamental role in determining the result of a program, and the 
semantics of the language is not completely independent from types. 
In essence there is some \emph{type ambiguity}, which happens 
when some expressions can 
be typed in multiple ways. Depending on which type is chosen 
for the subexpressions the program may lead to different results. 
This fenomenon is common in type-directed mechanisms, and it also 
affects type-directed mechanisms such as type classes or, more
generally, \emph{qualified types}~\cite{Mark93coherence}.

The typical solution to remove type ambiguity is to add type
annotations that chose a particular type for a subexpression when
multiple options exist. Our approach to address this problem is to 
use bi-directional type-checking techniques~\cite{Pierce00local,Dunfield04tri}. We show that 
with a fairly simple and standard bi-directional type system, we can guarantee 
that every subexpression (possibly with annotations) has a unique
type. For example:

\begin{lstlisting}
((succ,,not) : Int (*$\to$*) Int) (3,,True)
\end{lstlisting}

\noindent is a valid \name program, which has a unique determined type, and results in
\lstinline$4$. In this case the bi-directional type-system 
forces users to provide an annotation of the expression being applied.
Without the annotation the program would be rejected. With the
bi-directional type system, disjointness is indeed sufficient to ensure coherence. 

\subsection{Disjoint Intersection Types with $\top$} 
In the presence of a $\top$ type the simple definition of disjointness
is useless: $\top$ is always a common supertype of any two types.
Therefore, with the previous definition of disjointness no disjoint
intersections can ever be well-formed in the presence of $\top$!
Moreover, since $\top$ is not disjoint to any type, it does not make
sense to allow its presence in a disjoint intersection type.  Adding a
$\top$ type requires some adaptations on the notion of disjointness.
This paper studies two additional variants of \name with $\top$ types.
In both variants the definition of disjointness is revised as follows:

\begin{definition}[$\top$-Disjointness]
  Two types $A$ and $B$ are disjoint
  (written $A \disjoint B$) if the following two conditions are satisfied:
\begin{enumerate}
  \item $\neg \toplike{A}~\text{and}~\neg \toplike{B}$
  \item $\forall C.~\text{if}~ A \subtype C~\text{and}~B \subtype C~\text{then}~\toplike{C}$
\end{enumerate}
\end{definition}

In the presence of $\top$, instead of requiring that two types do not
share any common supertype, we require that the only allowed common supertypes are
\emph{top-like} (condition \#2). Additionally, it is also required that
the two types $A$ and $B$ are not themselves
top-like (condition \#1). The unary relation $\toplike{\cdot}$ denotes such top-like
types. Top-like types obviously include the $\top$ type. However
top-like types also include other types which are \emph{syntactically different}
from $\top$, but behave like a $\top$ type. For example, $\code{$\top$}~\inter~\code{$\top$}$
is syntactically different from $\top$, but it is still a supertype of
every other type (including $\top$ itself). The standard subtyping relation for
intersection types includes a rule:

\begin{mathpar}
      \inferrule* [right=\labelsubinter]
    {{A_1} \subtype {A_2} \\
     {A_1} \subtype {A_3} }
    {{A_1} \subtype {A_2 \inter A_3}}
\end{mathpar}

\noindent The presence of  \reflabel{\labelsubinter} means that not only  
${\top\inter \top} \!\!\subtype\!\! {\top}$ but also 
${\top} \!\!\subtype\!\! {\top \inter \top}$ are derivable.
In other words, in a calculus like Dunfield's there are infinitely
many syntactically different types that behave like a $\top$ type: 
$\top$, $\top \inter \top$, $\top \inter \top \inter \top$, \ldots

The notion of $\top$-Disjointness has two benefits. Firstly, and more
importantly, $\top$-Disjointness is sufficient to ensure
coherence. For example, the following program is valid, and coherent:

\begin{lstlisting}
3,,True : (*$\top$*) 
\end{lstlisting}

\noindent Even though the types of both components of the merge are a
subtype of the type of the program ($\top$), the result of the
program is always the \emph{unique} $\top$ value. Secondly,
$\top$-Disjointness has the side-effect of excluding other top-like
types from the system: the type $\top \inter \top$ is not
a \emph{well-formed} disjoint intersection type. In contrast to Dunfield's
calculus, \name has a unique syntactic $\top$ type.

\paragraph{Functional intersections and top-like types}
The two variants of \name with $\top$ differ slightly on the
definition of top-like types. The concrete definition of top-like 
types is important because it affects what types are allowed in intersections.
In the simpler version of \name with $\top$, multiple
functions cannot coexist in intersections. This is obviously a drawback and 
reduces the expressiveness of the system. The essential
problem is that a simple notion of top-like types is too restrictive. 
For example consider again the intersection type: \\

$(\code{String} \to \code{Int})~\inter~(\code{String} \to  \code{String})$
\vspace{5pt}

\noindent According to the simple definition of disjointness, this disjoint
intersection type is valid. However according $\top$-Disjointness and a simple definition
of top-like types, which accounts only for proper top types, this disjoint intersection type is not valid. 
The two types have a common supertype which is not a supertype of
every type: $\code{String} \to \top$. In this case $(\code{String} \to
\top)\!\!\subtype\!\!\top$, but $\top\!\!\not\subtype\!\!(\code{String} \to
\top)$. Therefore the two types do not meet the second condition of
$\top$-Disjointness as there exists a common supertype, which is not top-like.

\paragraph{Generalizing top-like types} In the second variant of \name 
top-likes are defined more liberally and they allow function types
whose co-domain is a top type to be considered as a top-like type.
For example, the type $\code{String}\to\top$ is considered a top-like
type. The benefit of allowing this more liberal notion of top-like
types is that, similarly to the first variant of \name without $\top$, 
disjoint intersections can have multiple functions types. 

%\name's type system only accepts programs that use disjoint intersection
%types. As shown in Section~\ref{sec:disjoint} disjoint intersection types will
%play a crucial rule in guaranteeing that the semantics is coherent.




 

% \subsection{Parametric Polymorphism and Intersection Types}\label{subsec:polymorphism}
% Before we show how \name extends the idea of disjointness to parametric
% polymorphism, we discuss some non-trivial issues that arise from
% the interaction between parametric polymorphism and intersection types.
%The interaction between parametric polymorphism and
%intersection types when coherence is a goal is non-trivial.
%In particular biased choice .
%The key challenge is to have a type
%system that still ensures coherence, but at the same time is not too
%restrictive in the programs that can be accepted.
% Dunfield~\cite{} provides a
% good illustrative example of the issues that arise when combining
% disjoint intersection types and parametric polymorphism:
% \[\lambda x. {\bf let}~y = 0 \mergeop x~{\bf in}~x\]
% Consider the attempt to write
% the following polymorphic function in \name (we use
% uppercase Latin letters to denote type variables):
% \begin{lstlisting}
% let fst A B (x: A & B) = (fun (z:A) (*$ \to $*) z) x in (*$ \ldots $*)
% \end{lstlisting}
% The
% \code{fst} function is supposed to extract a value of type
% (\lstinline{A}) from the merge value $x$ (of type \lstinline{A&B}). However
% this function is problematic.  The reason is that when
% \lstinline{A} and \lstinline{B} are instantiated to non-disjoint
% types, then uses of \lstinline{fst} may lead to incoherence.
% For example, consider the following use of \lstinline{fst}:
% \begin{lstlisting}
% fst Int Int (1,,2)
% \end{lstlisting}
% \noindent This program is clearly incoherent as both
% $1$ and $2$ can be extracted from the merge and still match the type
% of the first argument of \lstinline{fst}.

% \paragraph{Biased choice breaks equational reasoning.} At first sight, one option
% to workaround the issue incoherence would be to bias the type-based merge lookup
% to the left or to the right (as discussed in
% Section~\ref{subsec:incoherence}). Unfortunately, biased choice is
% very problematic when parametric polymorphism is present in the language.
% To see the issue, suppose we chose to always pick the
% rightmost value in a merge when multiple values of same type exist.
% Intuitively, it would appear that the result of the use of
% \lstinline{fst} above is $2$. Indeed simple equational reasoning
% seems to validate such result:
% \begin{lstlisting}
%    fst Int Int (1,,2)
% (*$ \rightsquigarrow $*) (fun (z: Int) (*$ \to $*) z) (1,,2) -- (* \textnormal{By the definition of \code{fst}} *)
% (*$ \rightsquigarrow $*) (fun (z: Int) (*$ \to $*) z) 2      -- (* \textnormal{Right-biased coercion} *)
% (*$ \rightsquigarrow $*) 2                          -- (* \textnormal{By $\beta$-reduction} *)
% \end{lstlisting}
%
% However (assumming a straightforward implementation of right-biased
% choice) the result of the program would be 1! The reason for this has
% todo with \emph{when} the type-based lookup on the merge happens. In
% the case of \lstinline{fst}, lookup is triggered by a coercion
% function inserted in the definition of \lstinline{fst} at
% compile-time.
% In the definition of \lstinline$fst$ all it is known is that a
% value of type $A$ should be returned from a merge with an intersection
% type $A\&B$.  Clearly the only type-safe choice to coerce the value of
% type $A\&B$ into $A$ is to
% take the left component of the merge. This works perfectly for merges
% such as \lstinline$(1,,'c')$, where the types of the first and second components
% of the merge are disjoint. For the merge \lstinline$(1,,'c')$, if a integer lookup
% is needed, then \lstinline$1$ is the rightmost integer, which is consistent with the
% biased choice. Unfortunately, when given the merge \lstinline$(1,,2)$ the
% left-component (\lstinline$1$) is also picked up, even though in this case \lstinline$2$
% is the rightmost integer in the merge. Clearly this is inconsistent
% with the biased choice!
%
% Unfortunately this subtle interaction of polymorphism and type-based lookup
%  means that equational reasoning is broken!
% In the equational reasoning steps above, doing apparently correct
% substitutions lead us to a wrong result. This is a major problem for
% biased choice and a reason to dismiss it as a possible implementation
% choice for \name.
