\section{Semantics, Disjointness and Coherence}
\label{sec:disjoint}
% \bruno{This section still needs to be cleaned up to remove stuff
%   related to polymorphism.}
%The previous section introduced the type system of \name. However it 
%did not discuss the semantics or how the calculus ensures coherence. 
This section discusses the elaboration semantics of \name, and shows a
bidirectional type system that guarantees coherence and type
soundness. Moreover the bidirectional type system is shown to be sound
and complete with respect to the type system presented in
Section~\ref{sec:fi} (provided that terms have some additional type
annotations). 
%The new type system is needed because additional type
%annotations are necessary for both avoiding type ambiguity (as well as
%ensuring decidability of type-checking).  
The coherence theorem presented in this section, relies on two key
aspects of the calculus:

\begin{itemize}

\item {\bf Uniqueness of subtyping coercions:} the notion of ordinary
  types and well-formed disjoint intersection types ensures that
  coercions produced by subtyping relation are unique.

\item {\bf No type ambiguity:} the bi-directional type system does not
  have type-ambiguity, and it infers a
  unique type for every well-typed program. 

%Removing type ambiguity 
%  is essential to ensure that the dynamic 
%Uniqueness of
%  types makes it easier to prove the coherence.  This property is not
%  true for the declarative type system, making a direct coherence
%  proof in that system more complicated.

\end{itemize}

%From the theoretical point-of-view, the end goal of this section is to show that the resulting system has
%a coherent (or unique) elaboration semantics:
%\begin{restatable}[Unique elaboration]{theorem}{uniqueelaboration}
%  \label{theorem:unique-elaboration}
%
%  If $\jtype \Gamma e {A_1} \yields {E_1}$ and $\jtype \Gamma e {A_2} \yields
%  {E_2}$, then $E_1 \equiv E_2$. (``$\equiv$'' means syntactical equality, up to
%  $\alpha$-equality.)
%
%\end{restatable}
%
%\noindent In other words, given a source term $e$, elaboration always produces
%the same target term $E$. The most important hurdle we need to overcome is that
%if $A \inter B \subtype C$, then either $A$ or $B$ contributes to that subtyping
%relation, resulting in two possible coercions.


%TODO move this part somewhere else

% \bruno{There are no changes in syntax, right? So the whole section
%   (and figure) can be dropped! Also, without polymorphism, we don't
%   need a bottom type.}
% \subsection{Syntax}
%
% \begin{figure}
%   \[
%     \begin{array}{l}
%       \begin{array}{llrll}
%         \text{Types}
%         & A, B & \Coloneqq & A \to B                 & \\
%         &      & \mid & A \inter B                   & \\
%
%         \\
%         \text{Terms}
%         & e & \Coloneqq & x                        & \\
%         &   & \mid & \lam {(x \oftype A)} e          & \\
%         &   & \mid & \app {e_1} {e_2}              & \\
%         &   & \mid & e_1 \mergeop e_2              & \\
%
%         \\
%         \text{Contexts}
%         & \Gamma & \Coloneqq & \cdot
%                    \mid \Gamma, x \oftype A  & \\
%       \end{array}
%     \end{array}
%   \]
%
%   \caption{Amendments of the rules.}
%   \label{fig:fi-syntax-dis}
% \end{figure}

% \george{May also note on the scoping of type variables inside contexts.}

% Figure~\ref{fig:fi-syntax-dis} shows the updated syntax with the
% changes highlighted.

% First, type
% variables are now always associated with their disjointness
% constraints (like $\alpha \disjoint A$) in types, terms, and
% contexts. Second, the bottom type ($\bot$) is introduced so that
% universal quantification becomes a special case of disjoint
% quantification: $\blam \alpha e$ is really a syntactic sugar for
% $\blamdis \alpha \bot e$. The underlying idea is that any type is
% disjoint with the bottom type.  Note the analogy with bounded
% quantification, where the top type is the trivial upper bound in
% bounded quantification, while the bottom type is the trivial
% disjointness constraint in disjoint quantification.

%Indeed, \bruno{unfinidhed sentence}\george{Mabe show a diagram here to contrast
%with bounded polymorphism.}

\subsection{Target of Elaboration}
The dynamic semantics of the call-by-value \name is defined via a
type-directed translation into the simply typed $\lambda$-calculus
with pair and unit types. 
%We define the dynamic semantics of the call-by-value \name by means of
%a type-directed translation to an extension of the simply typed 
%$\lambda$-calculus with pair and unit types.
%The type-directed translation is also shown in
%Figure~\ref{}, where the resulting System F terms are highlighted in
%gray.
The syntax and typing of our target language is unsurprising. The syntax of the
target language is shown in Figure~\ref{fig:f-syntax}. The highlighted part
shows its difference with the $\lambda$-calculus. 
The typing rules can be found in our Coq development.

\begin{figure}[t]
  \[
    \begin{array}{llrl}
      \text{Types}    & T & \Coloneqq & \code{Int} \\
                      &   & \mid      & \unit \\
                      &   & \mid      & {T_1} \to {T_2} \\
                      &   & \mid      & \highlight {$ \pair {T_1} {T_2} $} \\
      \text{Terms}    & E & \Coloneqq & x \\
                      &   & \mid      & i \\
                      &   & \mid      & \unit \\
                      &   & \mid      & \lamty x T E \\
                      &   & \mid      & \app {E_1} {E_2} \\
                      &   & \mid      & \highlight {$ \pair {E_1} {E_2} $} \\
                      &   & \mid      & \highlight {$ \proj k E $} \quad k \in \{ 1, 2 \} \\
      \text{Contexts} & G & \Coloneqq & \cdot \mid G, x \oftype T \\
    \end{array}
  \]
  \caption{Target language syntax.}
  \label{fig:f-syntax}
\end{figure}

\paragraph{Type and Context Translation.}

Figure~\ref{fig:type-and-context-translation} defines the type translation
function $\im \cdot$ from \name types $A$ to target language types $T$. The
notation $\im \cdot$ is also overloaded for context translation from \name
contexts $\Gamma$ to target language contexts $G$.

\begin{figure}[t]
  \framebox{$\im A = T$}

  \begin{align*}
    \im {\code{Int}}        &= \code{Int} \\
    \im {A_1 \to A_2}       &= \im {A_1} \to \im {A_2} \\
    \im {\pair {A_1} {A_2}} &= \pair {\im {A_1}} {\im {A_2}} \\
    \im {A_1 \inter A_2}    &= \pair {\im {A_1}} {\im {A_2}} \\
%    \im \top             &= \unit \\
  \end{align*}

  \framebox{$\im \Gamma = G$}

  \begin{align*}
    \im \cdot                      &= \cdot \\
    \im {\Gamma, \alpha \oftype A} &= \im \Gamma, \alpha \oftype \im A
  \end{align*}

  \caption{Type and context translation.}
  \label{fig:type-and-context-translation}
\end{figure}

% The rules given in this section are identical with those in
% Section~\ref{sec:fi}, except for the light blue part. The translation consists
% of four sets of rules, which are explained below:


\subsection{Coercive Subtyping and Coherence} 
The \name calculus uses coercive subtyping, where subtyping
derivations produce a coercion that is used to transform values of one
type to another type. 
Our calculus ensures that the coercions produced
by subtyping are unique. Unique coercions are fundamental for proving 
our coherence result of the semantics of \name.


\paragraph{Coercive subtyping.}

The judgment
\[
A_1 \subtype A_2 \yields E
\]
extends the subtyping judgment in Figure~\ref{fig:fi-type} with a coercion
on the right hand side of $ \yields {} $. A coercion $ E $ is just a term
in the target language and is ensured to have type
$ \im {A_1} \to \im {A_2} $ (by Lemma~\ref{lemma:sub}). For example,
\[
\code{Int} \inter \code{Bool} \subtype \code{Bool} \yields {\lamty x {\im {\code{Int} \inter \code{Bool}}} {\proj 2 x}}
\]
generates a coercion function with type: $\code{Int} \inter \code{Bool} \to \code{Bool}$.
Note that, in contrast to Dunfield's elaboration approach, where subtyping produces
coercions that are source language terms, in \name, coercions are
produced directly on the target language.

In \reflabel{\labelsubfun}, we elaborate the subtyping of
parameter and return types by $\eta$-expanding $f$ to $\lamty x {\im {A_3}}
{\app f x}$, applying $E_1$ to the argument and $E_2$ to the result. Rules
\reflabel{\labelsubinterl}, \reflabel{\labelsubinterr}, and
\reflabel{\labelsubinter} elaborate intersection types.
\reflabel{\labelsubinter} uses both coercions to form a pair. Rules
\reflabel{\labelsubinterl} and \reflabel{\labelsubinterr} reuse the coercion
from the premises and create new ones that cater to the changes of the argument
type in the conclusions. Note that the two rules are overlapping and
hence a program can be elaborated differently, depending on which rule
is used. Finally, all rules produce type-correct coercions:

%But in the implementation one usually applies the rules sequentially with
%pattern matching, essentially defining a deterministic order of lookup.

% if we know $A_1$ is a subtype of $A_3$ and $C$ is a coercion from $A_1$
% to $A_3$, then we can conclude that $A_1 \inter A_2$ is also a subtype
% of $A_3$ and the new coercion is a function that takes a value $ x $ of type
% $A_1\inter A_2$, project $x$ on the first item, and apply $ C $ to it.

\begin{restatable}[Subtyping rules produce type-correct coercions]{lemma}{lemmasub}
  \label{lemma:sub}
  If $ A_1 \subtype A_2 \yields E $, then $ \jtype \cdot E {\im {A_1} \to \im {A_2}} $.
\end{restatable}

\begin{proof}
  By a straightforward induction on the derivation.
\end{proof}

\paragraph{Overlapping subtyping rules} The key problem with the
subtyping rules in Figure~\ref{fig:fi-type} is that all three rules dealing
with intersection types (\reflabel{\labelsubinterl} and
\reflabel{\labelsubinterr} and \reflabel{\labelsubinter}) overlap.
Unfortunately, this means that different coercions may be given when checking
the subtyping between two types, depending on which derivation is chosen. This
is the ultimate reason for incoherence. There are two important types of
overlap:

\begin{enumerate}

\item The left decomposition rules for intersections (\reflabel{\labelsubinterl}
and \reflabel{\labelsubinterr}) overlap with each other.

\item The left decomposition rules for intersections (\reflabel{\labelsubinterl}
and \reflabel{\labelsubinterr}) overlap with the right decomposition rules for
intersections \reflabel{\labelsubinter}.

\end{enumerate}

\paragraph{Well-formedness and disjointness} 
The fact that in \name all intersection types are disjoint is useful to deal
with problem 1). Recall the definition of disjointness:

\begin{definition}[Simple disjointness]\label{def:simple_dis}
  Two types $A$ and $B$ are disjoint
  (written $A \disjoint B$) if there is no type $C$ such that both $A$ and $B$ are
  subtypes of $C$:
  \[A \disjoint B \equiv \not\exists C.~A \subtype C~\text{and}~B \subtype C\]
\end{definition}

Disjoint intersections are enforced by well-formedness of types.
Since the two types in an intersection are disjoint, it is impossible
that both of the preconditions of the left decompositions are
satisfied at the same time. Therefore, only one of the two left
decomposition rules can be chosen for a disjoint intersection
type. More formally, with disjoint intersections, we have the
following theorem:

\begin{lemma}[Unique subtype contributor]
  \label{lemma:unique-subtype-contributor}

  If $A_1 \inter A_2 \subtype B$, where $A_1 \inter A_2$ and $B$ are well-formed types,
  then it is not possible that the following holds at the same time:
  \begin{enumerate}
    \item $A_1 \subtype B$
    \item $A_2 \subtype B$
  \end{enumerate}
\end{lemma}

Unfortunately, disjoint intersections alone are insufficient to deal with
problem 2). In order to deal with problem 2), we introduce a distinction between
types, and ordinary types.

\paragraph{Ordinary types.} Ordinary types are just those which are not intersection
types, and are asserted by the judgment \[ \jatomic A \]

\noindent Since types in \name are simple, the only ordinary types are 
function type, integers and pairs.
But in richer systems, it can also include, for example, record types.
In the left decomposition rules for intersections we introduce a requirement
that $A_3$ is ordinary. The consequence of this requirement is that when $A_3$ is
an intersection type, then the only rule that can be applied is
\reflabel{\labelsubinter}. 

\paragraph{Unique coercions.}
Well-formedeness and ordinary types guarantee that at
any moment during the derivation of a subtyping relation, at most one rule can
be used. Consequently, the coercion of a subtyping relation $A \subtype B$ is
uniquely determined. This fact is captured by the following lemma:

\begin{restatable}[Unique coercion]{lemma}{uniquecoercion}
  \label{lemma:unique-coercion}

  If $A \subtype B \yields {E_1}$ and $A \subtype B \yields {E_2}$, where $A$
  and $B$ are well-formed types, then $E_1 \equiv E_2$.
\end{restatable}

\begin{comment}
\paragraph{No Loss of Expressiveness.}
Interestingly, our restrictions on subtyping do not sacrifice the expressiveness of
subtyping since we have the following two theorems:
\begin{theorem}
  If $A_1 \subtype A_3$, then $A_1 \inter A_2 \subtype A_3$.
\end{theorem}
\begin{theorem}
If $A_2 \subtype A_3$, then $A_1 \inter A_2 \subtype A_3$.
\end{theorem}

\noindent The interpretation of the two theorems is that: even though the
premise is made stricter by the atomic condition, we can still derive every
subtyping relation which is valid in the unrestricted system.


Throughout the paper we already presented an intuitive
definition for disjointness. 
% \bruno{This definition is wrong and too complicated. Since there is no
% polymorphism, we do not need $\Gamma$.}
We recall the definition of disjointness next.

\begin{definition}[Simple disjointness]\label{def:simple_dis}
  Two types $A$ and $B$ are disjoint
  (written $A \disjoint B$) if there is no type $C$ such that both $A$ and $B$ are
  subtypes of $C$:
  \[A \disjoint B \equiv \not\exists C.~A \subtype C~\text{and}~B \subtype C\]
\end{definition}

For example, $\code{Int}$ and $\code{Char}$ are disjoint,
since the two types do not have a common supertype. On the other hand, $\code{Int}$ is not
disjoint with itself, because $\code{Int} \subtype \code{Int}$. This implies that
disjointness is not reflexive as subtyping is. Two types with different ``shapes''
are always disjoint, unless one of them is an intersection type.
For example, a function type and an intersection
type may not be disjoint. Consider:
\[ \code{Int} \to \code{Int} \quad \text{and} \quad (\code{Int} \to \code{Int}) \inter (\code{String} \to \code{String}) \]
Those two types are not disjoint since $\code{Int} \to \code{Int}$ is their common supertype.
%Also, $\code{Int} \to \code{Char}$ and $\code{Int} \to \code{String}$ are disjoint, 
%since their supertypes are all types with the form $A \to \top$ 
%(where A \emph{contains} \code{Int}, i.e. $\code{Int}$, $\code{Int} \inter \code{Char}$) and $\top$.
\end{comment}


% Note that $A$ \emph{exclusive} or $B$ is true if and only if their truth value
% differ. Next, we are going to investigate the minimal requirement (necessary and
% sufficient conditions) such that the theorem holds.
%
% If $A_1$ and $A_2$ in this setting are the same, for example,
% $\code{Int} \inter \code{Int} \subtype \code{Int}$, obviously the theorem will
% not hold since both the left $\code{Int}$ and the right $\code{Int}$ are a
% subtype of $\code{Int}$.
%
% We can try to rule out such possibilities by making the requirement of
% well-formedness stronger. This suggests that the two types on the sides of
% $\inter$ should not ``overlap''. In other words, they should be ``disjoint''. It
% is easy to determine if two base types are disjoint. For example, $\code{Int}$
% and $\code{Int}$ are not disjoint. Neither do $\code{Int}$ and $\code{Nat}$.
% Also, types built with different constructors are disjoint. For example,
% $\code{Int}$ and $\code{Int} \to \code{Int}$. For function types, disjointness
% is harder to visualize. But bear in the mind that disjointness can defined by
% the very requirement that the theorem holds.
%
%
% This result is captured more formally by the following lemma:

% \george{Note that $\bot$ does not participate in subtyping and why (because the
% empty set intersecting the empty set is still empty).}

% \george{What's the variance of the disjoint constraint? C.f. bounded
% polymorphism.}

% \george{Two points are being made here: 1) nondisjoint intersections, 2) atomic
% types. Show an offending example for each?}

\subsection{Bidirectional Type System with Elaboration}

In order to prove the coherence result we first introduce a bidirectional
type system, which is closely related to the type system presented in
Section~\ref{sec:fi}. The bidirectional type system is elaborating, producing a term 
in the target language while performing the typing derivation.

The bidirectional type system is useful for two different
reasons. Firstly, the presence of the subsumption rule
(\reflabel{\labeltsub}) makes the type system not syntax directed,
which presents a challenge for an implementation. Bidirectional
type-checking makes the rules syntax directed again.  Secondly, and
more importantly, the subsumption rule also creates type ambiguity:
the same term can have multiple types. This is problematic because
there can be different semantics for a term, depending on the type of
the term. Bidirectional type-checking comes to the rescue again, by
ensuring that with the additional type annotations only one type is
inferred for a term.

\paragraph{Key Idea of the elaboration.}
The key idea in the elaboration is to turn merges into usual pairs, similar to Dunfield's
elaboration approach~\cite{dunfield2014elaborating}.
For example, \[ 1 \mergeop \code{"one"} \] becomes \pair 1
{\code{"one"}}. In usage, the pair will be coerced according to type
information. For example, consider the function application: 
\[ \app {(\lamty x {\code{String}} x : \code{String} \to \code{String})} {(1 \mergeop \code{"one"})} \] 
This expression will be translated to \[ \app
{(\lamty x {\code{String}} x)} {(\app {(\lamty x {\pair {\code{Int}} {\code{String}}} {\proj 2~x})}
{\pair 1 {\code{"one"}}})} \] The coercion in this case is $(\lamty x {\pair
{\code{Int}} {\code{String}}} {\proj 2~x})$. It extracts the second item from the pair, since
the function expects a $\code{String}$ but the translated argument is of type $\pair
{\code{Int}} {\code{String}}$.

\paragraph{The elaboration judgments and rules.}
Figure~\ref{fig:fi-typebd} presents the elaborating bidirectional type
system. The bidirectional type system itself is rather standard.  The
key differences to the type system in Figure~\ref{fig:fi-type} are
that all $:$'s are replaced with $\Rightarrow$ or $\Leftarrow$. There
are two check-mode rules: \reflabel{\labeltlam} and
\reflabel{\labeltsub}. The remaining rules are all in the synthesis
mode. Moreover there is one additional rule for annotation expressions
(\reflabel{\labeltann}). The syntax of source terms also needs to be
extended with annotation expressions $e : A$. 

The two elaboration judgments $\jinfer \Gamma e A \yields E$ and
$\jcheck \Gamma e A \yields E$ extend the usual typing judgments with
an elaborated term on the right hand side of the arrows. The
elaboration ensures that $E$ has type $\im A$. 
Noteworthy are the \reflabel{\labeltmerge} 
and \reflabel{\labeltsub} rules. The \reflabel{\labeltmerge}
straightforwardly translates merges into pairs. The
\reflabel{\labeltsub} accounts for the type coercions arizing 
from subtyping. The additional coercions are necessary to ensure that
the target terms are correctly typed, since the target language lacks
subtyping.

\begin{figure}
  \begin{mathpar}
    \formbi \\
    \bruletvar \and \bruletint \and
    \bruletapp \and \brulettprod \and
    \brulettproj \and
    \bruletmergedis \and \bruletann \\
   \formbc\\
   \bruletlam \and \bruletsub
  \end{mathpar}
  \caption{Bidirectional type system of \name.}
  \label{fig:fi-typebd}
\end{figure}

\paragraph{Type-safety}
The type-directed elaboration is type-safe. This property is captured
by the following two theorems.

\begin{restatable}[Type preservation]{theorem}{typepreservation}
  \label{theorem:type-preservation}

  If $ \jtype \Gamma e A \yields E $,
  then $ \jtype {\im \Gamma} E {\im A} $.
\end{restatable}\bruno{Joao: the statement is incorrect! we need to
  use the bi-directional judgments (and probably two of them?)!}

\begin{proof}
  (Sketch) By structural induction on the term and the corresponding
  inference rule.
\end{proof}

\begin{theorem}[Type safety]
  If $e$ is a well-typed \name term, then $e$ evaluates to some $\lambda$-calculus
  value $v$.
\end{theorem}

\begin{proof}
  Since we define the dynamic semantics of \name in terms of the composition of
  the type-directed translation and the dynamic semantics of $\lambda$-calculus, type safety follows immediately.
\end{proof}

\begin{figure}[t]
  \framebox{$\erase A = T$}
  \begin{align*}
    \erase {i}                   &= i \\
    \erase {x}                   &= x \\
    \erase {\lam x E}            &= \lam x {\erase E} \\
    \erase {\app {e_1} {e_2}}    &= {\erase {e_1}} {\erase {e_2}} \\
    \erase {\pair {e_1} {e_2}}   &= \pair {\erase {e_1}} {\erase {e_2}} \\
    \erase {\proj k e}           &= \proj k {\erase e} \quad (k \in {1,2}) \\ 
    \erase {{e_1} ,, {e_2}}      &= {\erase {e_1}} \, ,, \, {\erase {e_2}} \\
    \erase {e : A}               &= {\erase e} 
  \end{align*}
  \caption{Type annotation erasure.}
  \label{fig:type-ann-erasure}
\end{figure}

\paragraph{Soundness and Compleness} The declarative type system presented in
Figure~\ref{fig:fi-type} is closely related to the bidirectional type system 
in Figure~\ref{fig:fi-typebd}. We can prove that the bidirectional type system 
is sound and complete with respect to the declarative specification, 
modulo some additional type-annotations. To relate terms which are
typeable in both systems, we use the definition of erasure shown in Figure~\ref{fig:type-ann-erasure}.

\begin{theorem}[Soundness of bidirectional type-checking]
If $ \jinfer \Gamma e A \yields E$, then $ \jtype \Gamma {\erase e} A \yields {E'} $.
\end{theorem}

\begin{proof}
  (Sketch) By structural induction on the term and the corresponding
  inference rule.
\end{proof}

\begin{theorem}[Completeness of bidirectional type-checking]
If $ \jtype \Gamma e A \yields {E'}$, then $ \jinfer \Gamma {e'} A \yields E$, where $\erase{e'} = e$.
\end{theorem}

\begin{proof}
  (Sketch) By structural induction on the term and the corresponding
  inference rule.
\end{proof}
\bruno{Joao: is this correct? I think soundness or complenteness
  require theorems in two modes (checking or inference). Please double-check!}

\paragraph{Uniqueness of type-inference} An important property of the
bidirectional type-checking in Figure~\ref{fig:fi-typebd} is that, given an expression $e$, if it is
possible to infer a type for $e$, then $e$ has a unique type.

\begin{theorem}[Uniqueness of type-inference]
\label{theorem:uniqueness-type-inference}
If $\jinfer \Gamma {e_1} A \yields {E_1}$ and $\jinfer \Gamma {e_2} A \yields {E_2}$ then 
${e_1} = {e_2}$.
\end{theorem}

\begin{proof}
  (Sketch) By structural induction on the term and the corresponding
  inference rule.
\end{proof}

In contrast, as illustrated in Section~\ref{subsec:challenges}, in the declarative type
system some terms may have multiple, incompatible types. Therefore
there is no uniqueness of types for the declarative type system, and
the same term can have different semantics depending on its type.

\subsection{Coherency of Elaboration}

Combining the previous results, we are able to show the central theorem:

\begin{restatable}[Unique elaboration]{theorem}{uniqueelaboration}
  \label{theorem:unique-elaboration}

  If $\jtype \Gamma e {A_1} \yields {E_1}$ and $\jtype \Gamma e {A_2} \yields
  {E_2}$, then $E_1 \equiv E_2$. (``$\equiv$'' means syntactical equality, up to
  $\alpha$-equality.)

\end{restatable}
\bruno{Wrong theorem again! you need to use bi-directional
  judgements. }

\begin{proof}
  By induction on the first derivation.
  Note that two cases need special attention: \reflabel{\labeltsub} and \reflabel{\labeltapp}.
  In the former, we know that $A$ is unique by Theorem~\ref{theorem:uniqueness-type-inference} and 
  that $C$ is also unique, by Lemma~\ref{lemma:unique-coercion}.
  The latter requires inference of a function type, \bruno{forcing the
    use of a type annotation}\bruno{I don't think this is a good explanation}.
  Again, by Theorem~\ref{theorem:uniqueness-type-inference} we can use the induction hypotheses.
%  Note that the typing rules are already syntax-directed but the case of
%  \reflabel{\labeltapp} (copied below) still needs special attention since we
%  need to show that the generated coercion $E$ is unique.
%  \begin{mathpar}
%    \ruletapp
%  \end{mathpar}
%  Luckily we know that typing
%  judgments give well-formed types, and thus $\jwf \Gamma {A_1}$ and $\jwf
%  \Gamma {A_3}$. Therefore we are able to apply
%  Lemma~\ref{lemma:unique-coercion} and conclude that $E$ is
%  unique.
\end{proof}

\begin{comment}
\subsection{Typing}

Figure~\ref{fig:fi-type-patch} shows modifications to Figure~\ref{fig:fi-type}
in order to support disjoint intersection types.
Only new rules or rules that different are shown. Importantly, the disjointness
judgment appears in the well-formedness rule for intersection types and the
typing rule for merges.

\begin{figure}[t]
  \begin{mathpar}
    \framebox{$\jatomic A$} \\

    \inferrule*
    {}
    {\jatomic {A \to B}}

    \inferrule*
    {}
    {\jatomic {\code{Int}}}

    \inferrule*
    {}
    {\jatomic \top}
  \end{mathpar}

  \begin{mathpar}
    \formsub \\ \rulesubinterldis \and \rulesubinterrdis
  \end{mathpar}

  \begin{mathpar}
    \formwf \\ \rulewfinterdis
  \end{mathpar}

  \begin{mathpar}
    \formt \\ \ruletmergedis
  \end{mathpar}

  \caption{Affected rules.}
  \label{fig:fi-type-patch}
\end{figure}

%\paragraph{Atomic Types.} The new system introduces atomic types. Essentially a type
%is atomic if it is any type, which is not an intersection type.
%The notion of atomic
%type will be helpful

\paragraph{Well-formedness.}
We require that the two types of an intersection must be disjoint. Under the new rules, intersection types such as $\code{Int} \inter \code{Int}$
are no longer well-formed because the two types are not disjoint.

\paragraph{Metatheory.}
Since in this section we only restrict the type system
in the previous section, it is easy to see that type preservation and
type-safety still holds.

% Additionally, we can see that typing always produces a
% well-formed type. \bruno{Revise text below, as I uncommented it.}

%%\begin{restatable}[Instantiation]{lemma}{instantiation}
%%  \label{lemma:instantiation}

%%   If $\jwf {\Gamma, \alpha \disjoint B} C$, $\jwf \Gamma A$, $\jdis \Gamma A B$
%%   then $\jwf \Gamma {\subst A \alpha C}$.
%% \end{restatable}

% \begin{restatable}[Well-formed typing]{lemma}{wellformedtyping}
%    \label{lemma:wellformed-typing}
%
%    If $\jtype \Gamma e A$, then $\jwf \Gamma A$.
% \end{restatable}

% \begin{proof}
%   By induction on the derivation that leads to $\jtype \Gamma e A$ and applying
%   Lemma~\ref{lemma:instantiation} in the case of \reflabel{\labelttapp}.
% \end{proof}

% This lemma will play an important role in the proof of coherence.
%
% \bruno{Tell the reader why the typing produces a well-formed type is important.
% Probably you want to say something like: ``it will be important to prove coherence''.}

% With our new definition of well-formed types, this result is nontrivial.

% In general, disjointness judgments are not invariant with respect to
% free-variable substitution. In other words, a careless substitution can violate
% the disjoint constraint in the context. For example, in the context $\alpha
% \disjoint \code{Int}$, $\alpha$ and $\code{Int}$ are disjoint:
% \[ \jdis {\alpha \disjoint \code{Int}} \alpha \code{Int} \]
% But after the substitution of $\code{Int}$ for $\alpha$ on the two types, the sentence
% \[ \jdis {\alpha \disjoint \code{Int}} \code{Int} \code{Int} \]
% is longer true since $\code{Int}$ is clearly not disjoint with itself.

\end{comment}
