\section{The \name Calculus and its Type System}
\label{sec:fi}

This section presents the syntax, subtyping and type assignment of \name: 
a calculus with intersection types and a merge
operator.  This calculus is inspired by Dunfield's
calculus~\cite{dunfield2014elaborating}, without 
considering union types. Moreover, since we are only
interested in disjoint intersections, \name also has different
typing rules for intersections from those in Dunfield's calculus. Sections~\ref{sec:disjoint}
and \ref{sec:alg-dis} will present the more fundamental contributions
of this paper by showing other required changes for supporting
disjoint intersection types and ensuring coherence. The calculus in
this section does not include the $\top$ type, since it brings 
additional complications.  Section~\ref{sec:top} presents two variants of
\name with a $\top$ type, while preserving coherence.

\subsection{Syntax}

Figure~\ref{fig:fi-syntax} shows the syntax of \name. The difference to the
$\lambda$-calculus (with pairs), 
highlighted in gray, are intersection types 
($A \inter B$) at the type-level, and merges ($e_1 \mergeop e_2$) at the term level.

\begin{figure}[t]
  \[
    \begin{array}{l}
      \begin{array}{llrll}
        \text{Types}
        & A, B, C & \Coloneqq & \code{Int} & \\
        &         & \mid      & A \to B   & \\
        &         & \mid      & \tpair A B & \\
        &         & \mid      & \highlight{$A \inter B$}  & \\
%        &         & \mid      & \highlight{$\top$}  & \\

        \\
        \text{Terms}
        & e & \Coloneqq & i                      & \\
        &   & \mid & x                           & \\
        &   & \mid & \lamty x A e                & \\
        &   & \mid & \app {e_1} {e_2}            & \\
        &   & \mid & \pair {e_1} {e_2}           & \\
        &   & \mid & \proj k e \quad k \in \{ 1 , 2 \} & \\
        &   & \mid & \highlight{$e_1 \mergeop e_2$}  & \\
%        &   & \mid & \highlight{$\top$} & \\

        \\
        \text{Contexts}
        & \Gamma & \Coloneqq & \cdot
                   \mid \Gamma, x \oftype A  & \\
      \end{array}
    \end{array}
  \]

  \caption{\name syntax.}
  \label{fig:fi-syntax}
\end{figure}

\paragraph{Types.} Metavariables $A$, $B$ range over types. Types include
function types $A \to B$ and product types $\tpair A B$. 
$A \inter B$ denotes the intersection of types $A$ and
$B$. We also include integer types $\code{Int}$.

\paragraph{Terms.} Metavariables $e$ range over terms.  Terms include standard
constructs: variables $x$; abstraction of terms over variables
$\lamty x A e$; application of terms $e_1$ to terms $e_2$, written $\app
{e_1} {e_2}$; pairing of two terms (denoted as $\pair {e_1} {e_2}$);
and both projections of a pair $e$, written $\proj k e$ (with $k \in \{1,2\}$).
The expression $e_1 \mergeop e_2$ is the \emph{merge} of two terms
$e_1$ and $e_2$. Merges of terms correspond to intersections of types $A \inter
B$. 
%The term $\top$ corresponds to the only inhabitant of the $\top$ type.
In addition, we also include integer literals $i$.

\paragraph{Contexts.} Typing contexts are standard: $ \Gamma $ tracks bound variables $x$ with their type $A$.

% We use $\subst {A} \alpha {B}$
% to denote the capture-avoiding substitution of $A$ for $\alpha$ inside $B$ and
% $\ftv \cdot$ for sets of free type variables.

In order to focus on the key features that make this language interesting, we do
not include other forms such as type constants and fixpoints here. However they
can be included in the formalization in standard ways. 
%The formalization of nonessential features such as
%records and record operations can be found in the full version of the paper.

% \paragraph{Discussion.} A natural question the reader might ask is that why we
% have excluded union types from the language. The answer is we found that
% intersection types alone are enough support extensible designs.

\subsection{Subtyping}

% In some calculi, the subtyping relation is external to the language: those
% calculi are indifferent to how the subtyping relation is defined. In \name, we
% take a syntactic approach, that is, subtyping is due to the syntax of types.
% However, this approach does not preclude integrating other forms of subtyping
% into our system. One is ``primitive'' subtyping relations such as natural
% numbers being a subtype of integers. The other is nominal subtyping relations
% that are explicitly declared by the programmer.

\begin{figure}
%\begin{comment}
\begin{mathpar}
    \framebox{$\jatomic A$} \\

    \inferrule*
    {}
    {\jatomic {A \to B}}

    \inferrule*
    {}
    {\jatomic {\tpair A B}}

    \inferrule*
    {}
    {\jatomic {\code{Int}}}

    %\inferrule*
    %{}
    %{\jatomic \top}
  \end{mathpar}
%\end{comment}

  \begin{mathpar}
    \formsub \\
    \rulesubint \and \rulesubprod \\ %\and \rulesubtop 
    \rulesubfun \and \rulesubinter 
    \and \rulesubinterldis \and \rulesubinterrdis 
  \end{mathpar}

  \begin{mathpar}
    \formwf \\
    \rulewfint \and \rulewffun \and \rulewfprod \and \rulewfinterdis %\and \rulewftop
  \end{mathpar}

{\renewcommand{\yields}[1]{}
  \begin{mathpar}
    \formt \\
    \ruletvar \and \ruletlam \and
    \ruletint \and \ruletprod \and \ruletapp \and
    \ruletmergedis \\
    \ruletproj \and \ruletsub %\and \rulettop
  \end{mathpar}
}

  \caption{Declarative type system of \name.}
  \label{fig:fi-type}
\end{figure}

% Intersection types introduce natural subtyping relations among types. For
% example, $ \code{Int} \inter \code{Bool} $ should be a subtype of $ \code{Int} $, since the former
% can be viewed as either $ \code{Int} $ or $ \code{Bool} $. To summarize, the subtyping rules
% are standard except for three points listed below:
% \begin{enumerate}
% \item $ A_1 \inter A_2 $ is a subtype of $ A_3 $, if \emph{either} $ A_1 $ or
%   $ A_2 $ are subtypes of $ A_3 $,

% \item $ A_1 $ is a subtype of $ A_2 \inter A_3 $, if $ A_1 $ is a subtype of
%   both $ A_2 $ and $ A_3 $.

% \item $ \recordType {l_1} {A_1} $ is a subtype of $ \recordType {l_2} {A_2} $, if
%   $ l_1 $ and $ l_2 $ are identical and $ A_1 $ is a subtype of $ A_2 $.
% \end{enumerate}
% The first point is captured by two rules $ \reflabelsubinterl $ and
% $ \reflabelsubinterr $, whereas the second point by $ \reflabelsubinter $.
% Note that the last point means that record types are covariant in the type of
% the fields.

The subtyping rules of the form $A \subtype B$ are shown in the top part of
Figure~\ref{fig:fi-type}. At the moment, the reader is advised to ignore the
gray-shaded part in the rules, which will be explained later. The rule
\reflabel{\labelsubfun} says that a function is contravariant in its parameter
type and covariant in its return type. The three rules dealing with
intersection types are just what one would expect when interpreting types as
sets. Under this interpretation, for example, the rule \reflabel{\labelsubinter}
says that if $A_1$ is both the subset of $A_2$ and the subset of $A_3$, then
$A_1$ is also the subset of the intersection of $A_2$ and $A_3$.

Note that the notion of ordinary types, which is used in rules
\reflabel{\labelsubinter1} and \reflabel{\labelsubinter2}, was
introduced by Davies and Pfenning~\cite{davies2000intersection} to provide an algorithmic version of
subtyping. In our system ordinary types are used for a different 
purpose as well: they play a fundamantal role in ensuring that 
subtyping produces unique coercions. Section~\ref{sec:alg-dis} will present 
a detailed discussion on this. The subtyping relation, is known to be
\emph{reflexive} and \emph{transitive}~\cite{davies2000intersection}. 

\begin{comment}
\begin{lemma}[Subtyping is reflexive] \label{lemma:sub-refl}
  For all types $ A $, $ A \subtype A $.
\end{lemma}

\begin{lemma}[Subtyping is transitive] \label{lemma:sub-trans}
  If $ A_1 \subtype A_2 $ and $ A_2 \subtype A_3 $,
  then $ A_1 \subtype A_3 $.
\end{lemma}
\end{comment}

%For the corresponding mechanized proofs in Coq, we refer the reader to the
%repository associated with the paper\footnote{\url{https://github.com/zhiyuanshi/intersection/tree/master/esop-16}}.

\subsection{Declarative Type System}

The well-formdness of types and typing relation are shown in the
middle and bottom of Figure~\ref{fig:fi-type}, respectively.
Importantly, the disjointness judgment, which is highlighted using a
box, appears in the well-formedness rule for intersection types (\reflabel{\labelwfinter}) and 
the typing rule for merges (\reflabel{\labeltmerge}). The presence of
the disjointness judgement, as well as the use of ordinary types in
the subtyping relation, are the most essential differences between our type system and the
original type system by Dunfield. 
%The role of the disjointness
%judgement in ensuring coherence is discussed later, in
%Section~\ref{}. 

%The well-formedness rules are shown in the middle part of
%Figure~\ref{fig:fi-type}. In addition to the standard rules,
%\reflabel{\labelwfinter} is also not surprising, except for the
%presence of the additional disjointness conditional.

Apart from \reflabel{\labelwfinter}, the remaining rules for 
well-formedness are standard. The typing judgment is of the form:
\[ \jtype \Gamma e A \]
It reads: ``in the typing context $\Gamma$, the term $e$ is of type
$A$''. The standard rules are those for 
variables \reflabel{\labeltvar}; lambda abstractions \reflabel{\labeltlam}; 
application \reflabel{\labeltapp}; integer literals \reflabel{\labeltint};
products \reflabel{\labeltprod}; and projections \reflabel{\labeltproj}. 
\reflabel{\labeltmerge} means that a merge
$e_1 \mergeop e_2$, is assigned an intersection type composed of the
resulting types of $e_1$ and $e_2$, as long as the types of the two
expressions are disjoint.
Finally, \reflabel{\labeltsub} states that for any types $A$ and $B$, if
$A \subtype B$ then any expression $e$ with assigned type $A$ can also be assigned 
the type $B$.

\paragraph{Typing disjoint intersections}
Dunfield's calculus has different typing rules for intersections.
%
%The extra rules have an important benefit: they
%allow eliminating any uses of the subsumption rule. This is
%accomplished by transforming all expressions that make use of
%subtyping into expressions without any uses of subtyping. Our
%type-system does not have those rules. 
%One reason for having different rules in \name is that 
However, his rules
make less sense in a system with disjoint intersections. For example, 
they include the following typing rule, for
introducing intersection types:

\begin{mathpar}
\inferrule* 
  {\jtype \Gamma {e} A \\
   \jtype \Gamma {e} B }
  {\jtype \Gamma {e} {A \inter B}}
\end{mathpar}

\noindent A first reason why such rule would not work in \name is
that it does not restrict $A \inter B$ to be disjoint. Therefore,
in the presence of such unrestricted rule it would be possible to 
create non-disjoint intersections types. It is easy
enough to have an additional disjointness restriction, which would 
ensure the disjointness between $A$ and $B$. However, then the rule 
would be pointless, because no derivations of programs could ever be
created with such a rule. If $A$ and $B$ are disjoint then, by
definition, no programs should ever have types $A$ and $B$ at 
the same time. In contrast, in \name, the rule \reflabel{\labeltmerge}
captures the fact that two disjoint pieces of evidence are needed to
create a (disjoint) intersection type.

\begin{comment}
\paragraph{Typing Applications}
The rule that deserves more attention is the application rule
\reflabel{\labeltapp}.  This is where the most important difference between
\name and Dunfield calculus is. In Dunfield's calculus, if
\lstinline$f$ and \lstinline$g$ are both functions expecting an
integer, the following program typechecks:

\begin{lstlisting}
(f ,, g) 1
\end{lstlisting}

\noindent In our calculus, however, one needs manually extract the desired
function by passing \lstinline$f ,, g$ to a function. For example, if
\lstinline$f : Int$ $\to$  \lstinline$Bool$ and \lstinline$g : Int$ $\to$ \lstinline$Char$, we could select  \lstinline$f$
from the intersection as follows:

\begin{lstlisting}
((fun (h : Int (*$\to$*) Bool) (*$\to$*) h) (f ,, g)) 1
\end{lstlisting}

The reason for this design choice is that otherwise we could have
another source of semantic ambiguity. Consider the following program:

\begin{lstlisting}
(idInt ,, idChar) (1,,'c')
\end{lstlisting}

\noindent where \lstinline$idInt$ and \lstinline$idChar$ are,
respectively, identity functions on integers and on characters. Clearly both
merges in the program (\lstinline$(f ,, g)$ and \lstinline$(1,,'c')$)
are disjoint. Yet there are two possible outcomes for this program:
\lstinline$1$ or \lstinline$'c'$. Thus application requires a
different treatment, if we wish to have coherence.
Our design choice is simply to only employ subtyping in the argument
of the application, but not on the expression being applied.
It should be possible to design a ``smarter''
application rule, that still avoids the ambiguity problems that would
be present in Dunfield's calculus, while at the same time avoiding
explicitly extracting functions from intersections. However such rule
would necessarily be complicated, and rather ad-hoc.
So we opted for simplicity. In any case there is no expressiveness
loss (only loss of convinience), as it is always possibly to extract the function component from
a merge and then apply it.
\end{comment}

