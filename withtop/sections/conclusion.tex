\section{Conclusion and Future Work}
\label{sec:conclusion}

This paper described \name: a coherent and type-safe core calculus
that combines intersection types and a merge operator. We investigated
three different variants of \name: two variants with a $\top$ type;
and another one without. To ensure coherence the type system accepts
only disjoint intersections. For each variant of \name there is a
different definition of disjointness. Nevertheless all definitions of
disjointness follow the same principle: they are defined in terms of
the subtyping relation; and they describe which common supertypes are
allowed in order for two types to be considered disjoint.

\begin{comment}
We implemented the core functionalities of the \name as part of a JVM-based
compiler. Based on the type system of \name, we have built an ML-like
source language compiler that offers interoperability with Java (such as object
creation and method calls). The source language is loosely based on the more
general System $F_{\omega}$ and supports a
number of other features, including records, polymorphism, mutually recursive
\code{let} bindings, type aliases, algebraic data types, pattern matching, and
first-class modules that are encoded using \code{letrec} and records.
\end{comment}

For the future, we would like to study the
addition of union types. This will also require changes in our
notion of disjointness, since with union types there always exists
a type $A \union B$, which is the common supertype of two
types $A$ and $B$, and that is not a top-like type.
Another interesting challenge is to address the combination between 
disjoint intersection types and polymorphism. A naive combination 
does not seem to be difficult. Since an expression with a polymorphic
type can be instantiated to \emph{any} type, a simple option is simply 
to forbid polymorphic variables in intersections. However this has 
limited expressiveness, and would prevent many useful programs. 
More thought is needed to achieve more expressiveness. 

% Some immediate topics for
% further improvement of the results in this paper are discussed next.
%
% \paragraph{Union types.}
%
% If a type system ever contains union types (the counterpart of intersection
% types), with the following standard subtyping rules,
% \begin{mathpar}
%   \inferrule* [right=Sub\_Union\_1]
%     { }
%     {A \subtype A \union B}
%
%   \inferrule* [right=Sub\_Union\_2]
%     { }
%     {B \subtype A \union B}
% \end{mathpar}
% then no two types $A$ and $B$ can ever be disjoint, since there always exists
% the type $A \union B$, which is their common supertype. So it is reasonable to
% conjecture that such system cannot be coherent.
% \bruno{I wouldn't say this is a motivation: it sounds like we caould
%   not support union types, when I think this is not true. For example
% we could say something like: there does not exist an \emph{atomic} C ...}
%
%
% \paragraph{Implementation.}
%
% We implemented the core functionalities of the \name as part of a JVM-based
% compiler. Based on the type system of \name, we built an ML-like
% source language compiler that offers interoperability with Java (such as object
% creation and method calls). The source language is loosely based on the more
% general System $F_{\omega}$ and supports a
% number of other features, including records, mutually recursive
% \code{let} bindings, type aliases, algebraic data types, pattern matching, and
% first-class modules that are encoded using \code{letrec} and records.
%
% Relevant to this paper are the three phases in the compiler, which
% collectively turn source programs into System $F$:
%
% \begin{enumerate}
% \item A \emph{typechecking} phase that checks the usage of \name features and
%   other source language features against an abstract syntax tree that follows
%   the source syntax.
%
% \item A \emph{desugaring} phase that translates well-typed source terms into
%   \name terms. Source-level features such as multi-field records, type aliases
%   are removed at this phase. The resulting program is just an \name term
%   extended with some other constructs necessary for code generation.
%
% \item A \emph{translation} phase that turns well-typed \name terms into System
%   $F$ ones.
% \end{enumerate}
%
% Phase 3 is what we have formalized in this paper.
%
%
% \paragraph{Reduce the number of coercions.}
%
% Our translation inserts a coercion (many of them are identity functions)
% whenever subtyping occurs during a function application, which could mean
% notable run-time overhead. In the current implementation, we introduced a
% partial evaluator with three simple rewriting rules to eliminate the redundant
% identity functions as another compiler phase after the translation. In another
% version of our implementation, partial evaluation is weaved into the process of
% translation so that the unwanted identity functions are not introduced during
% the translation. Besides, since the order of the two types in a binary
% intersection does not matter, we may normalize them to avoid unnecessary
% coercions.
