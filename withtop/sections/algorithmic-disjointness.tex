\section{Algorithmic Disjointness} \label{sec:alg-dis}

Section~\ref{sec:disjoint} presented a type system with disjoint
intersection types that is both type-safe
and coherent. Unfortunately the type system is not yet algorithmic
because the specification of disjointness does not lend itself to an
implementation directly. This is a problem, because we need an
algorithm for checking whether two types are disjoint or not in order
to implement the type-system.

% Now that we have obtained a specification for disjointness, but the definition
% involves an existence problem. How can we implement it? One possibility is
% bidirectional subtyping, that is, we say two types, $A$ and $B$, are disjoint if
% neither $A \subtype B$ nor $B \subtype A$. However, this implementation is
% wrong. For example, $\code{Int} \inter \code{String}$ and $\code{String} \inter \code{Char}$ are
% not disjoint by specification since $\code{String}$ is their common supertype. Yet
% by the implementation they are, since neither of them is a subtype of
% the other. \bruno{You need a concrete code example to make this point.}
% Hence the algorithmic rules are more nuanced. For now, it is enough to treat the
% disjoint judgment $\jdis \Gamma A B$ as oracle and we will come back to
% this topic in the next section.

This section presents the set of rules for determining whether two
types are disjoint. The set of rules is algorithmic and an
implementation is easily derived from them. The derived
set of rules for disjointness is proved to be sound and complete with
respect to the definition of disjointness in
Section~\ref{sec:disjoint}.

% \subsection{Derivation of the Algorithmic Rules}
%
% In this subsection, we illustrate how to derive the algorithmic disjointness
% rules by showing a detailed example for functions. For the ease of discussion,
% first we introduce some notation.
%
% \begin{definition}[Set of common supertypes]
%   For any two types $A$ and $B$, we can denote the \emph{set of their common
%   supertypes} by \[ \commonsuper(A,B) \] In other words, a type $C \in \;
%   \commonsuper(A,B)$ if and only  if $A \subtype C$ and $B \subtype C$.
% \end{definition}
%
% \begin{example}
%   $\commonsuper(\code{Int},\code{Char})$ is empty, since $\code{Int}$ and $\code{Char}$
%   share no common supertype.
% \end{example}
%
% Parellel to the notion of the set of common supertypes is the notion of the set
% of common subtypes.
%
% \begin{definition}[Set of common subtypes]
%   For any two types $A$ and $B$, we can denote the \emph{set of their common
%   subtypes} by \[ \commonsub(A,B) \] In other words, a type $C \in \; \commonsub(A,B)$
%   if and only  if $C \subtype A$ and $C \subtype B$.
% \end{definition}
%
% \begin{example}
%   $\commonsub(\code{Int},\code{Char})$ is an infinite set which contains $\code{Int} \inter
%   \code{Char}$, $\code{Char} \inter \code{Int}$, $(\code{Int} \inter \code{Bool}) \inter \code{Char}$
%   and so on. But the type $\code{Bool}$ is not inside, since it is not a subtype of
%   $\code{Int}$ or $\code{Char}$.
% \end{example}
%
%
% \paragraph{Shorthand Notation.} For brevity, we will use \[ \mathcal{A} \to
% \mathcal{B} \] as a shorthand for the \emph{set} of types of the form $A \to B$,
% where $A \in \mathcal{A}$ and $B \in \mathcal{B}$. The same shorthand applies to
% all other constructors of types, in addition to functions, as well. As a simple
% example,  \[ \{ \code{Int}, \code{String} \} \to \{ \code{Int}, \code{Char} \} \] is a shorthand for \[ \{
% \code{Int} \to \code{Int}, \code{Int} \to \code{Char}, \code{String} \to \code{Int}, \code{String} \to \code{Char} \} \]
%
%
% Recall that we say two types $A$ and $B$ are disjoint if they do not share a
% common supertype. Therefore, determining if two types $A$ and $B$ are disjoint
% is the same as determining if $\commonsuper(A,B)$ is empty.
%
% \paragraph{Determining Disjointness of Functions.} Now let's dive into the case
% where both $A$ and $B$ are functions and consider how to compute
% $\commonsuper(A_1 \to A_2, B_1 \to B_2)$. By the subtyping rules, the supertype
% of a function must also be a function.\george{Nah... only after normalization.
% If not, it can also be $\inter$} Let $C_1 \to C_2$ be a common supertype
% of $A_1 \to A_2$ and $B_1 \to B_2$. Then it must satisfy the following:
% \begin{mathpar}
%   \inferrule
%     {C_1 \subtype A_1 \\ A_2 \subtype C_2}
%     {A_1 \to A_2 \subtype C_1 \to C_2}
%
%   \inferrule
%     {C_1 \subtype B_1 \\ B_2 \subtype C_2}
%     {B_1 \to B_2 \subtype C_1 \to C_2}
% \end{mathpar}
% From which we see that $C_1$ is a common subtype of $A_1$ and $B_1$ and that
% $C_2$ is a common supertype of $A_2$ and $B_2$. Therefore, we can write:
% \[ \commonsuper(A_1 \to A_2, B_1 \to B_2) \ = \ \commonsub(A_1,B_1) \to \commonsuper(A_2,B_2) \]
% By definition, $\commonsub(A_1,B_1) \to \commonsuper(A_2,B_2)$ is not empty if and only if both
% $\commonsub(A_1,B_1)$ and $\commonsuper(A_2,B_2)$ is nonempty. However, note
% that with intersection types, $\commonsub(A_1,B_1)$ is always nonempty because
% $A_1 \inter B_1$ belongs to $\commonsub(A_1,B_1)$. Therefore, the problem of
% determining if $\commonsuper(A_1 \to A_2, B_1 \to B_2)$ is empty reduces to the
% problem of determining if $\commonsuper(B_1 \to B_2)$ is empty, which is, by
% definition, if $B_1$ and $B_2$ are disjoint. Finally, we have derived a rule for
% functions:
% \begin{mathpar}
%   \ruledisfun
% \end{mathpar}
%
% The analysis needed for determining if types with other constructors are
% disjoint is similar. Below are the major results of the recursive definitions for
% $\commonsuper$:
% \begin{align*}
%   \commonsuper(A_1 \to A_2, B_1 \to B_2) &= \commonsub(A_1,B_1) \to \commonsuper(A_2,B_2) \\
%   \commonsuper({A_1 \inter A_2, B})      &= \commonsuper(A_1, B) \cup \commonsuper(A_1,B) \\
%   \commonsuper({A, B_1 \inter B_2})      &= \commonsuper(A, B_1) \cup \commonsuper(A,B_2)
% \end{align*}

\subsection{Algorithmic Rules}

\begin{figure}[t]
  \begin{mathpar}
    \formdis \\
    \ruledisinterl \and \ruledisinterr \\ 
    \ruledisprojl \and \ruledisprojr \\ 
    \ruledisfun \and \ruledisatomic
  \end{mathpar}

  \begin{mathpar}
    \formax \\
    %\ruledisaxintfun \and \ruledisaxtop \and \ruleaxsym
    \ruledisaxintfun \and \ruledisaxintprod \and \ruledisaxprodfun \and \ruleaxsym
  \end{mathpar}

  \caption{Algorithmic Disjointness.}
  \label{fig:disjointness}
\end{figure}

%\begin{figure}[h]
%  \begin{mathpar}
%    \formdis \\
%    \ruledisfunimp \\ 
%    \ruledisfunimpl \\ 
%    \ruledisfunimpr \\
%    \ruledisinterl \\ 
%    \ruledisinterr \\
%    \ruledisatomic
%  \end{mathpar}
%
%  \begin{mathpar}
%    \formax \\
%    \ruledisaxintfun \and \ruledisaxtopint \and \ruledisaxtopfun \and \ruleaxsym
%  \end{mathpar}
%
%  \caption{Impure Algorithmic Disjointness.}
%  \label{fig:disjointness}
%\end{figure}
%

The rules for the disjointness judgment are shown in
Figure~\ref{fig:disjointness}, which consists of two judgments.

\paragraph{Main Judgment.} The judgment $\jdisimpl \Gamma A B$ says
two types $A$ and $B$ are disjoint.
The rules dealing with intersection types (\reflabel{\labeldisinterl} and
\reflabel{\labeldisinterr}) are quite intuitive. The intuition is that if two
types $A$ and $B$ are disjoint to some type $C$, then their intersection
($A\&B$) is also clearly disjoint to $C$.  The rules capture this intuition by
inductively distributing the relation itself over the intersection constructor
($\inter$). Although those two rules overlap, the order of applying them in an
implementation does not matter as applying either of them will eventually leads
to the same conclusion, that is, if two types are disjoint or not.

The rule for functions (\reflabel{\labeldisfun}) is more interesting. It says that two function
types are disjoint if and only if their return types are disjoint (regardless of
their parameter types!). At first this rule may look surprising
because the parameter types play no role in the definition of
disjointness. To see the reason for this consider the two function types:
\[ \code{Int} \to \code{String} \qquad \code{Bool} \to \code{String} \]
Even though their parameter types are disjoint, we are still able to think of a
type which is a supertype for both of them. For example, $ \code{Int} \inter \code{Bool}
\to \code{String} $. Therefore, two function types with
the same return type are not
disjoint. Essentially, due to the contravariance of function types,
functions of the form $A \to C$ and $B \to C$ always have a common
supertype (for example $A \& B \to C$).
The lesson from this example is that the parameter types of two
function types do not have any influence in determining whether those two function
types are disjoint or not: only the return types matter.

Finally, the rules for pairs (\reflabel{\labeldisprojl} and \reflabel{\labeldisprojr}) say that
either the first components or the second components must be disjoint.
This is due to product subtyping being covariant: disjointness of either
component is enough to make it impossible to construct a supertype.
For instance, take the following two product types:
\[\pair {\code{Int}} {\code{String}} \qquad \pair {\code{Bool}} {\code{String}}\]
We can never come up with a common supertype for these because it does not exist a type $C$ which is 
both a supertype of $\code{Int}$ and $\code{Bool}$.

% The rule for disjoint quantification (\reflabel{\labeldisforall}) . Consider the following two types:
% \[ (\forall (\alpha * \code{Int}).~\code{Int} \& \alpha) \qquad (\forall (\alpha * \code{Char}). ~\code{Char} \& \alpha) \]
% The question is under which conditions are those two types disjoint.
% In the first type $\alpha$ cannot be instantiated with $\code{Int}$ and in
% the second case $\alpha$ cannot be instantiated with $\code{Char}$.
% Therefore we can see that for the two bodies to be considered
% disjoint, $\alpha$ cannot be instantiated with either $\code{Int}$ or
% $\code{Char}$. The rule for disjoint quantification captures this fact by
% requiring the bodies of disjoint quantification to be checked for
% disjointness under both constraints.

\paragraph{Axioms.} Up till now, the rules of $ \jdisimpl \Gamma A B $ have only
taken care of two types with the same language constructs. But how can be the
fact that $\code{Int}$ and $\code{Int} \to \code{Int}$ are disjoint be decided?
That is exactly the place where the judgment $ A \disjointax B $ comes in handy.
It provides the axioms for disjointness. What is captured by the set of rules is
that $ A \disjointax B $ holds for all two types of different constructs unless
any of them is an intersection type.
%\joao{mention the "hd" alternative described by Dunfield? (even though we had to excl the
%And case)}

\subsection{Meta-theory}

The following two theorems together say that the algorithmic
judgment and the definition of disjointness are equivalent. 
%For detailed
%proofs, we refer to the Coq scripts in the supplementary material of
%the submission.

\begin{restatable}[Soundness of algorithmic disjointness]{theorem}{algodissoundness}
  \label{theorem:soundness}

  For any two types $A$ and $B$, $\jdisimpl \Gamma A B$ implies $\jdis \Gamma A B$.
\end{restatable}

\begin{proof}
  By induction on the derivation of $\jdisimpl \Gamma A B$.
\end{proof}

\begin{restatable}[Completeness of algorithmic disjointness]{theorem}{algodiscompleteness}
  \label{theorem:completeness}

  For any two types $A$ and $B$, $\jdis \Gamma A B$ implies $\jdisimpl \Gamma A B$.
\end{restatable}

\begin{proof}
  By a case analysis on the shape of $A$ and $B$.
\end{proof}
