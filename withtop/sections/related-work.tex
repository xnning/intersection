\section{Related Work}
\label{sec:related-work}

%*******************************************************************************
\paragraph{Merge Operator.}
%*******************************************************************************
Reynolds invented Forsythe~\cite{reynolds1997design} in the
1980s. Forsythe has a merge operator operator $p_1, p_2 $ and a
coherent semantics. The result was proved formally by
Reynolds~\cite{reynolds1991coherence} in a lambda calculus with
intersection types and a merge operator. He has four different typing
rules for the merge operator, each accounting for various
possibilities of what the types of the first and second components
are. With those four rules, a merge can only be constructed when the
second component of the merge is either a function or a (one-field)
record. The set of rules is restrictive and it forbids, for instance,
the merge of two functions. In \name a merge can contain, 
for example two primitive values (which is disallowed in Forsythe). 
Moreover, except for the variant presented in Section~\ref{subsec:simpletop}, multiple 
functions can co-exist in a merge as long as they are provably disjoint.
The treatment of disjointness of functions is particularly
challenging, specially in combination with a $\top$ type, and
supporting multiple functions (as well as other types) in a merge is 
a significant innovation over Reynolds approach. 

Castagna et al.~\cite{Castagna92calculus} proposed $\lamint$ to study
the overloading problem for functions. Their calculus does not have a
$\top$ type, and contains a special merge operator that works only for
functions. The calculus is coherent. Similarly to us they impose
well-formedness conditions on the formation of a (functional)
merge. However, their well-formedness conditions cannot be ported to a
system with arbitrary intersections like \name, since those conditions
assume function types only.  They also show how to encode records, but
it seems that encoding arbitrary merges and intersections is not
possible.  For the special case of functional
merges, the conditions that are used in $\lamint$ are incomparable in
expressive power to those in \name. That is, $\lamint$ accepts some
functional merges that \name rejects, and \name accepts some
functional merges that $\lamint$ rejects. For example, the functional
merge
\[(\code{Int}\to\code{Char})\inter(\code{Int}\to\code{Bool})\] 
\noindent is accepted in \name but not in $\lamint$. One reason 
why \name also rejects some functional intersections, which are 
accepted in $\lamint$, seems to be related to the presence of arbitrary 
merges. As discussed in Section~\ref{sec:alg-dis}, the combination of
contravariance and the presence of arbitrary merges means that we can
always find a common supertype of two functions that have non-disjoint
co-domains. In $\lamint$ the non-existence of arbitrary merges means
that it is harder to find common supertypes of functions, allowing for
a more liberal notion of coherent functional merges.  

Our work is largely inspired by
Dunfield~\cite{dunfield2014elaborating}, and throughout the paper we
have already made extensive comparisons with his work. He
described a similar approach to ours: compiling a system with intersection types
and a merge operator into ordinary $ \lambda $-calculus terms with pairs. One
major difference is that our system does not include unions. As
acknowledged by Dunfield, his calculus lacks of coherence. Dunfield
also mechanically formalized his proofs, using the Twelf proof
assistant~\cite{Pfenning99cade}. However he did not prove any results about
coherence, so his meta-theoretical results were not immediately
useful to us. Since we were also not familiar with Twelf, we decided to
start a new formalization in Coq (which we are familiar with), while
proving many new results related to coherence.


Pierce~\cite{pierce1991programming2} made a comprehensive review
of coherence, especially on Curien and Ghelli~\cite{curienl1990coherence} and
Reynolds' methods of proving coherence; but he was not able to prove coherence
for his $F_\wedge$ calculus. He introduced a primitive $\code{glue}$ function as
a language extension which corresponds to our merge operator. However, in his
system users can ``glue'' two arbitrary values, which can lead to incoherence.

\begin{comment}
 the way coherence is ensured is not general
enough. He has four different typing rules for the merge operator, each
accounting for various possibilities of what the types of the first and second
components are. In some cases the meaning of the second component takes
precedence (that is, is biased) over the first component. The set of rules is
restrictive and it forbids, for instance, the merge of two functions (even when
they a provably disjoint). Therefore, Forsythe treatment of coherence
is rather ad-hoc. In contrast, disjointness in \name has a simple, well-defined
specification and it is quite flexible.
\end{comment}

\paragraph{Coherence without Merge.}
Recently, Castagna~\textit{et al.}~\cite{Castagna:2014} studied a very interesting
and coherent calculus that has polymorphism and set-theoretic type connectives (such as
intersections, unions, and negations). Unfortunately their calculus
does not include a merge operator like ours, which is our major source
of difficulty for achieving coherence.
%As a result, in their calculus one is
%also able to express a type variable that can be instantiated to any type other
%than $\code{Int}$ as $\alpha \setminus \code{Int}$, which is syntactic sugar for
%$\alpha \wedge \neg \code{Int}$.

Going in the direction of higher
kinds, Compagnoni and Pierce~\cite{compagnoni1996higher} added
intersection types to System $ F_{\omega} $ and used a new calculus,
$ F^{\omega}_{\wedge} $, to model multiple inheritance. In their
system, types include the construct of intersection of types of the
same kind $ K $. Davies and Pfenning
\cite{davies2000intersection} studied the interactions between
intersection types and effects in call-by-value languages. And they
proposed a ``value restriction'' for intersection types, similar to
value restriction on parametric polymorphism.
We borrowed the notion of ordinary types from Davies and Pfenning's.
Ordinary types play a fundamental role in ensuring coherence in \name.
In contrast to \name, none of those calculi include a merge operator.

There have been attempts to provide a foundational calculus
for Scala that incorporates intersection
types~\cite{amin2014foundations,amin2012dependent}.
%Although the minimal Scala-like calculus does not natively support
%parametric polymorphism, it is possible to encode parametric
%polymorphism with abstract type members. Thus it can be argued that
%this calculus also supports intersection types and parametric
%polymorphism.
However, the type-soundness of a minimal Scala-like
calculus with intersection types and parametric polymorphism is not
yet proven. Recently, some form of intersection
types has been adopted in object-oriented languages such as Scala,
Ceylon, and Grace. Generally speaking,
the most significant difference to \name is that in all those languages
there is no explicit introduction construct like our merge operator.

\begin{comment}
only allow intersections of concrete types (classes),
whereas our language allows intersections of type variables, such as
\texttt{A \& B}. Without that vehicle, we would not be able to define
the generic \texttt{merge} function (below) for all interpretations of
a given algebra, and would incur boilerplate code:

\begin{lstlisting}
let merge [A, B] (f: ExpAlg A) (g: ExpAlg B) = {
  lit (x : Int) = f.lit x ,, g.lit x,
  add (x : A & B) (y : A & B) =
    f.add x y ,, g.add x y
}
\end{lstlisting}
\end{comment}

%*******************************************************************************
\paragraph{Other Type Systems with Intersection Types.}
%*******************************************************************************

% Although similar in spirit,
% our elaboration typing is simpler: we require subtyping in the case of
% applications, thus avoiding the subsumption rule. Besides, our treatment
% combines the merge rules ($ k $ ranges over $ \{1, 2\} $)
% \inferrule
% {\Gamma \turns e_k : A}
% {\Gamma \turns e_1 \mergeop e_2 : A}
% and the standard intersection-introduction rule
% \inferrule
% {\Gamma \turns e : A_1 \andalso \Gamma \turns e : A_2}
% {\Gamma \turns e : A_1 \inter A_2}
% into one rule:
% \inferrule [Merge]
% {\Gamma \turns e_1 : A_1 \andalso \Gamma \turns e_2 : A_2}
% {\Gamma \turns e_1 \mergeop e_2 : A_1 \inter A_2}
%Castagna, and Dunfield describe
%elaborating multi-fields records into merge of single-field records.
% Reynolds and Castagna do not consider elaboration and Dunfield do not
% formalize elaborating records.
% Both Pierce and Dunfield's system include a subsumption rule, which states that
% if an term has been inferred of type $ A $, then it is also of any
% supertype of $ A $. Our system does not have this rule.
Refinement
intersection~\cite{dunfield2007refined,davies2005practical,freeman1991refinement}
is the more conservative approach of adopting intersection types. 
Refinement intersections usually have a restriction on the formation
of an intersection type $A \inter B$. In refinement intersections
$A \inter B$ is a well-formed type if $A$ and $B$ are
refinements of the same simple (unrefined) type. However this is a
different restriction from disjointness. Refinement intersection increases
only the expressiveness of types but not terms. But without a term-level
construct like ``merge'', it is not possible to encode various language
features. 

\begin{comment}
As an alternative to syntactic subtyping described in this paper,
Frisch~\textit{et al.}~\cite{frisch2008semantic} studied semantic subtyping. Semantic
subtyping attempts to increase the expressive power of the 

seems to have important advantages over syntactic subtyping. One
worthy avenue for future work is to study languages with intersection types
and merge operator in a semantic subtyping setting.
\end{comment}

\paragraph{Coherence in other Type-Directed Mechanisms} Other
type-directed mechanisms such as type classes~\cite{Wadler89ad-hoc} and, more generally, 
qualified types~\cite{jones94} also require special care to ensure coherence.
For example, in Haskell, a well-know example of ambiguity~\cite{Mark93coherence} that could
lead to incoherence is:

\begin{verbatim}
f :: String -> String
f s = show (read s)
\end{verbatim}

\noindent Here the functions \verb|read| and \verb|show| have,
respectively the types \verb|Read a => String -> a| and 
\verb|Show a => a -> String|. The constraints \verb|Read a| 
and \verb|Show a| represent requirements that the compiler 
should implicitly infer. That is the compiler should find a suitable 
implementation of the read and show functions for the particular type 
\verb|a|. The problem is that there is not enough type information 
to determine what the type \verb|a| should be. 
An arbitrary choice 
of \verb|a| could lead to different code being inferred depending on
the particular choice of \verb|a|. In this case the behavior of the program would be dependent on the particular
choice made by the compiler. As a result Haskell compilers, reject 
programs like the above. Jones~\cite{Mark93coherence} provided a
formal treatment of coherence for qualified types and type classes,
designing suitable restrictions that ensure that incoherent programs 
are not accepted. In recent versions of Haskell compilers, it is possible to use 
type annotations to remove type ambiguity:

\begin{verbatim}
f :: String -> String
f s = (show :: Int -> String) (read s)
\end{verbatim}

\noindent Here the type annotation instantiates \verb|a| 
to \verb|Int|, removing the ambiguity and allowing the program to 
be accepted. In \name we also use annotations to remove type
ambiguity. The bidirectional type system ensures that the additional 
annotations required in terms are sufficient to remove any type
ambiguity in \name programs.

\begin{comment}
%*******************************************************************************
\paragraph{Traits and Trait Calculi.}
%*******************************************************************************

The seminal paper by Schärli~\textit{et al.} introduced the ideas behind
traits. In their original paper, they documented an implementation of
the trait mechanism in a dynamically typed version of Smalltalk.
Fisher and Reppy~\cite{fisher2004typed} presented a statically typed
calculus that models traits. \name is not dedicated to traits; but rather, it
supports a source language that models traits. Compared to Fisher and Reppy's
calculus, \name is more lightweight. For example, self references (as
well as other OO-specific constructs) are
not built-in \name. One reason for the difference is that Fisher and Reppy's
calculus supports \emph{classes} in addition to traits, and considers the
interaction between them, whereas our object oriented source language is
\emph{prototype} (or delegation) based---the mechanism for code reuse is purely traits. Of
course, there have been many other formalizations of traits, such
as~\cite{scharli2003traitsformal}. But most of them are heavyweight and specific
to modeling traits and typical class-based models of OOP, and therefore differ from our approach.

Bettini~\textit{et al.}'s prototype language,
SWRTJ~\cite{bettini2010prototypical} distinguishes, in their terminology,
``records'' and ``traits''---the former contain fields and the latter contain
methods. Since we try to model a pure object-oriented language, we have excluded
fields, which provide state reuse. In SWRTJ, traits themselves are not meant to
be the generator of instances. Instead, another construct, called ``classes''
are, and make use of traits.

The Scala language also has a notion of ``traits''. However, unlike what its
name suggests,
the semantics of trait composition in Scala is more similar to
mixins~\cite{bracha1990mixin}. That is, like traditional mixin
semantics, when two traits are composed, Scala attempts to do
\emph{implicit resolution of conflicts}. In comparison, the traits modeled in
\name are
intended to model the original trait idea closely, and conflicts must
be resolved explicitly. Schärli~\textit{et al.} document well the trade-offs
between mixins and traits. Aside from that, Scala's
traits and our source language's traits have four major differences:

\begin{enumerate}

  \item Scala's traits cannot be instantiated but only mixed into a class (which
  can be anonymous), whereas traits in our language can be instantiated
  directly.

  \item Scala's traits cannot take constructor parameters whereas ours can. As in
  the point example below, our trait is itself a constructor and takes the x- and
  y-coordinates as parameters:

  \begin{lstlisting}
  trait Point(x: Int, y: Int) { self: Point (*$ \to $*)
    x() = x
    y() = y
  } in (* \ldots *)
  \end{lstlisting}

%  \item Scala allows traits to contain methods whose body is not implemented.
%  But our current language does not support this feature.

   \item Dynamic instantiation is supported in \name, but not in
     Scala. In Scala instantiating an object from a class or traits
     requires that all classes or traits are statically known.

   \item Our model of traits is purely functional, but Scala's traits
     also support fields, mutable state and abstract types.

\end{enumerate}
\end{comment}
