\section{Design Space}

This section discusses some alternatives in the design-space.

\subsection{Disjointness of Functions}

Talk about the option of not allowing subtyping of function arguments. 
This should allow for a more flexible rule for disjointness of functions.
Maybe a good option for OO type systems, where methods are invariant 
with respect to subtyping of arguments. It would allow for static overloading, 
similar to what is present in conventional OO languages.

\begin{mathpar}
  \inferrule* [right=\labelsubfun]
    {{A_2} \subtype {B_2} }
    {{A_1 \to A_2} \subtype {A_1 \to B_2}}
\end{mathpar}

\begin{mathpar}
  \inferrule* [right=\labeldisfun]
    {\jdisimpl \Gamma {A_1} {B_1}}
    {\jdisimpl \Gamma {A_1 \to A_2} {B_1 \to B_2}}
\end{mathpar}

not exists E . A -> B <: E /\ C -> D <: E ->  

$Int -> Int \& Char -> Int$ (disjoint according to the spec and algorithmic rules)



Are those 2 functions disjoint? 

$ f,,g : Int -> Int \& Char -> Int$

Well, two things to consider:

1) what happens if they are applied:  

$(f,,g) (3,'c')$ 

well, a type-annotation will then select one of the functions. So this seems to be ok.

2) what happens if the functions are selected. I have two choices:

f,,g : Int -> Int

f,,g : Char -> Int 

$f,,g : Char\&Int -> Int$ (fails because subtyping of functions is invariant).



\subsection{Union Types}

\begin{lstlisting}
case 3,,'c' of
   Int -> 1
   Char -> 2 : Int
\end{lstlisting}

Here we have $Int \& Char <: Int | Char$, but this leads to ambiguity. The program can either 
be $1$ or $2$. 

Possible solution: require atomic constraints in or-rules, similar to the and-rules. 
Big Problem: subtyping is no longer transitive. Minor problem, type-system is incomplete.

\subsection{Parametric Polymorphism?}

In principle it should be easy to extend disjointness to parametric polymorphism. 
bruno{Show rules for parametric polymorphism}

However, such rules would be quite restrictive. Future work includes how to integrate 
parametric polymorphism is a more flexible way.  
