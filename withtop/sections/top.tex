\section{Disjoint Intersection Types with $\top$}

Discuss the two variants of \name with $\top$. 

\begin{figure}[t]
  \[
    \begin{array}{l}
      \begin{array}{llrll}
        \text{Types}
        & A, B, C & \Coloneqq & \ldots \mid \highlight{$\top$}  & \\

        \\
        \text{Terms}
        & e & \Coloneqq & \ldots \mid \highlight{$\top$} & \\
      \end{array}
    \end{array}
  \]

  \begin{mathpar}
    \formsub \\
    \rulesubtop 
  \end{mathpar}

  \begin{mathpar}
    \formwf \\
    \rulewftop
  \end{mathpar}

  \begin{mathpar}
    \formt \\
    \brulettop
  \end{mathpar}

  \caption{Extending \name with $\top$.}
  \label{fig:fi-syntax-top}
\end{figure}\bruno{rule form needs to be fixed in typing.}

\subsection{Disjointness} Show the new definition of disjointness

\subsection{A Naive Calculus with $\top$}

\paragraph{Top-Like Types}

\paragraph{Algorithmic disjointness rules}

\subsection{An Improved Calculus with $\top$}

motivation ...

\paragraph{Top-Like Types}

\paragraph{Coercive Subtyping} Discuss the changes in 
coercive subtyping rules. Namely the more flexible way to 
generate coercions depending on whether a type is top-like 
or not. 

\paragraph{Algorithmic disjointness rules}