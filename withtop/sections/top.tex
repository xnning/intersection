\section{Disjoint Intersection Types with $\top$}

Discuss the two variants of \name with $\top$. 

\begin{figure}[t]
  \[
    \begin{array}{l}
      \begin{array}{llrll}
        \text{Types}
        & A, B, C & \Coloneqq & \ldots \mid \highlight{$\top$}  & \\

        \\
        \text{Terms}
        & e & \Coloneqq & \ldots \mid \highlight{$\top$} & \\
      \end{array}
    \end{array}
  \]

  \begin{mathpar}
    \formsub \\
    \rulesubtop 
  \end{mathpar}

  \begin{mathpar}
    \formwf \\
    \rulewftop
  \end{mathpar}

  \begin{mathpar}
    \formt \\
    \brulettop
  \end{mathpar}

  \caption{Extending \name with $\top$.}
  \label{fig:fi-syntax-top}
\end{figure}\bruno{rule form needs to be fixed in typing.}

\subsection{Disjointness} Show the new definition of disjointness

\subsection{A Naive Calculus with $\top$}

\paragraph{Top-Like Types}
Before we discuss that definition, let us introduce first the notion of a top-like type 

%Here such definition is made a bit more precise, and
%well-suited to \name.
\begin{figure}[t]
  \begin{mathpar}
    \formtoplike \\ %\framebox{$\jatomic A$} \\

    \ruletopltop \and \ruletoplfun \and \ruletoplinterl \and \ruletoplinterr

  \end{mathpar}
  \caption{Top-like types.}
  \label{fig:fi-toplike}
\end{figure}

\begin{definition}[Top-like types]
  
  One type $A$ is a top-like type, denoted as $\toplike{A}$, if it has the form $A_k \to \top$, where $k \in {0,1,..}$.
  That is, any type with arity $k$ can be a top-like type, as long as $\top$ is the result type. 

\end{definition}
Thus a top-like type is a unary relation at the type level, which can be formalized according to \ref{fig:fi-toplike}.

Also, $\code{Int} \to \code{Char}$ and $\code{Int} \to \code{String}$ are disjoint, 
since their supertypes are all types with the form $A \to \top$ 
(where A \emph{contains} \code{Int}, i.e. $\code{Int}$, $\code{Int} \inter \code{Char}$) and $\top$.


At last, take as an example ($\top \inter \top$).
In a pure language (i.e. no side-effects), this type can be safely allowed since both components of the merge
will evaluate to the same value. 
However, in an effectful language, the evaluation of either the right or left component might lead to distinct results. 
Following the same reasoning, we might want to allow or restrict, for instance $String \to Int \inter String \to \top$,  

\paragraph{Algorithmic disjointness rules}

\subsection{An Improved Calculus with $\top$}

motivation ...

\paragraph{Top-Like Types}
\joao{do we want to allow this?
 f: String $\to$ Int \\ 
 g: String $\to$ T \\
 ($\lam$ h : String $\to$ T. h "") (f ,, g) }

\paragraph{Coercive Subtyping} Discuss the changes in 
coercive subtyping rules. Namely the more flexible way to 
generate coercions depending on whether a type is top-like 
or not. 

\paragraph{Algorithmic disjointness rules}