\section{Disjoint Intersection Types with $\top$}\label{sec:top}

This section shows how to add a $\top$ type to \name.
%So far we have discussed a system with intersection types, but we have (intentionally) left out $\top$.
%The $\top$ type is not only a supertype of all types, but also the type for 0-ary intersection. 
%Unfortunately, introducing $\top$ into our system leads to several
%drastic consequences:
Introducing $\top$ poses some important challenges. Most prominently,
the simple definition of disjointness is useless in the presence of
$\top$. Since all types now have a common supertype, it is impossible
for any two types to satisfy a simple notion of disjointness. To
address this problem a notion of $\top$-disjointess is proposed.  The
definition of $\top$-disjointess depends on a notion of a top-like
type. We formalise two different variants of \name, based on two different
definitions of a top-like type, while discussing their expressiveness. Both variants retain coherence, and all other key
properties of \name.  For simplicity and space reasons we omit pairs
from the two variants of \name presented in this section.
Mechanized Coq proofs for both variants are available as part of the 
supplementary materials for the paper.


%\begin{itemize}
%\item The definition of disjointness becomes useless:
%since every type has now a common supertype, it does not exist any pair of two types for which the disjointness relation
%holds. Thus, we can no longer build an intersection type.
%\item The subtyping relation becomes incoherent: its rules allow us to derive, for example, both 
%$\top \subtype \top \inter \top$ and $\top \inter \top \subtype \top$.
%\end{itemize}

\begin{comment}
However, by adjusting the definition of disjointness, we can obtain more useful results, while still preserving the
coherence of the subtyping relation.
Even though these are desirable properties, a new definition of disjointess can be formulated in many different ways. 
\joao{expand}

We will first show how we can introduce $\top$ into our system.
Next, a new definition of disjointness will be presented - which we will refer to as $\top$-disjointness - and
we will briefly discuss how it can make the resulting system more useful, while maintaining the correctness
of all properties proven for the original system.
The definition of $\top$-disjointess will depend on a notion of a top-like type, for which at first we will just provide 
some intuition for it.
We will then formalise two different systems, based on two different definitions of a top-like type, while discussing their
usability and limitations.
The first top-like definition will solve the problem of incoherence, but it will still forbid the inclusion of
function types within intersection types.
To face this limitation, we proceed by extending the top-like definition, and showing how the resulting system will become 
more expressive.
We also show, for both systems, how to adjust the algorithmic disjointness rules such that they become consistent with the 
respective disjointess specification.
\end{comment}

\subsection{Introducing $\top$}

\begin{figure}[t]
  \[
    \begin{array}{l}
      \begin{array}{llrll}
        \text{Types}
        & A, B, C & \Coloneqq & \ldots \mid \highlight{$\top$}  & \\

        \\
        \text{Terms}
        & e & \Coloneqq & \ldots \mid \highlight{$\top$} & \\
      \end{array}
    \end{array}
  \]

  \begin{mathpar}
    \formsub \\
    \rulesubtop 
  \end{mathpar}

  \begin{mathpar}
    \formwf \\
    \rulewftop
  \end{mathpar}

  \begin{mathpar}
    \formbi \\
    \brulettop
  \end{mathpar}

  \begin{mathpar}
    \framebox{$\im A = T$} \\
    \im \top = \unit \\
  \end{mathpar}

  \caption{Extending \name with $\top$.}
  \label{fig:fi-syntax-top}
\end{figure}

Introducing the $\top$ type in the \name calculus is not difficult, as
shown in Figure \ref{fig:fi-syntax-top}.  Existing types are extended
with $\top$ and, correspondingly, we add the canonical inhabitant of type
$\top$: the term $\top$.  The subtyping relation is extended with
\reflabel{\labelsubtop}, declaring that any type is a sub-type of
$\top$.  The coercion in the target language, is a function that
always returns the term $\unit$, regardless of its argument.  We also
add $\top$ to the set of well-formed types by extending the
well-formedness relation with \reflabel{\labelwftop}.  Finally, the
typing rule \reflabel{\labelttop} states that, under type inference,
the term $\top$ has type $\top$ and generates the term $\unit$ in the
target language.

%However, as discussed in Section \ref{sec:overview}, the definition of disjointness used so far turns intersection types to
%be useless within this new system.
%Since every type now has at least one common supertype (namely, $\top$), it is impossible to create an intersection type.
%Thus, we need to reformulate that definition, as presented next. 

\subsection{Disjointness} 
As discussed in Section \ref{sec:overview}, the definition of
simple disjointness is useless when \name is extended with $\top$.
%There are several reasons supporting statement: 
%\begin{itemize}
%\item There exists more than one \emph{syntactical} $\top$: any n-ary intersection composed by $\top$'s, 
%is a super-type of all other types;
%\item Intersection types can no longer be built, according to the disjointness specification.
%\end{itemize}
For these reasons, we differentiate \emph{top-like} types from the rest of the types,
so that restrictions may be imposed based on the former.
For now, the formal definition of a \emph{top-like} type is omitted,
and we informally define it as a type that resembles $\top$ in some way. 
Having this in mind, $\top$-disjointness is defined as follows:
%Even though we will show two different definitions of $\top$-disjointness, they share the same top-level definition, 
%presented next.
\begin{definition}[$\top$-Disjointness]
Given two types $A$ and $B$ we have that:

%\[A * B \equiv \not \toplike{A} \and \not \toplike{A} \and A \subtype B\]
\[A \disjoint_{\top} B \equiv \neg \toplike{A} \; \wedge \; \neg \toplike{B} \; \wedge \; 
(\forall_C \; ( A \subtype C \wedge B \subtype C ) \rightarrow \toplike{C}) \]

\noindent where $\toplike{C}$ means that $C$ is a top-like type.
\end{definition}

\noindent Informally, given two
types $A$ and $B$:
\begin{itemize}
\item $A$ and $B$ cannot be top-like types (i.e. preventing types such as $T \inter T$ to be well-formed).
%This deals with problem 1 (described above).
\item If there is any common supertype of $A$ and $B$, that is not top-like, then 
intersection of these types is forbidden, as there might be an overlap between them.
%This deals with problem 2.
\end{itemize}

Another problem arising from the introduction of $\top$ is that Lemma~\ref{lemma:unique-subtype-contributor}
no longer holds.
For example, given any type $A_1$ and $A_2$, and considering $B = \top$; then we can prove that both of 
them can be coerced to $B$ (because any type is a subtype of $\top$).
Thus, in this case, the type ${A_1} \inter {A_2}$ has two contributors which can produce $\top$.
In spite of this Lemma being a core property of our system, not all hope is lost:
we can still prove a weaker version of it.
Namely:

\begin{lemma}[Unique subtype contributor (with $\top$)]
  \label{lemma:unique-subtype-contributor-with-top}

  If $A_1 \inter A_2 \subtype B$, where $A_1 \inter A_2$ and $B$ are well-formed types,
  \textbf{and $\text{B}$ is not toplike};
  then it is not possible that the following holds at the same time:
  \begin{enumerate}
    \item $A_1 \subtype B$
    \item $A_2 \subtype B$
  \end{enumerate}
\end{lemma}

In other words, if $B$ is not top-like, then the previous Lemma still holds!
This suggests that we will need to introduce restrictions concerning top-like types, if we want to achieve
coherence.
Therefore this will be the main challenge we address during the upcoming sections,
by showing two suitable definitions of top-like types and discussing 
their consequences in a system with $\top$-disjointess.
%Next we will present a simple top-like type definition and show how resulting system
%becomes more expressive while achieving coherence.
%However, we still consider that system not expressive enough, 
%so an extended definition of the top-like type relation will be presented (also the resulting system).
%We will refer to the first system as \emph{simple} and to the second as \emph{improved}. 

\subsection{A Simple Calculus with $\top$}\label{subsec:simpletop}

In the first variant of \name with $\top$ the basic idea is to have a 
definition of top-like types, which captures all syntactically
distinct top types:  $\top$, $\top \inter \top$, $\top \inter \top
\inter \top$. The resulting system has only one \emph{syntactic}
$\top$, namely $\top$ itself. Moreover, coherence is preserved.

\paragraph{Top-Like Types}

%\begin{definition}[Top-like types]
%  Any type consisting of a $n$-ary intersection composed of $n$ $\top$'s, is a top-like type. 
%\end{definition}

A top-like type can be formalised as an unary relation on a type $A$, denoted as $\toplike{A}$, as show in 
Figure \ref{fig:tltypesdis}.
The rule \reflabel{\labeltopltop} states that $\top$ is a top-like type; 
the rule \reflabel{\labeltoplinter} indicates that any 
intersection composed of just top-like types is also a top-like type.

\begin{figure}[t]
  \begin{mathpar}
    \formtoplike \\ %\framebox{$\jatomic A$} \\

    \ruletopltop \and \ruletoplinter

  \end{mathpar}
  \begin{mathpar}
    \formdis \\
    \ruledisinterl \and \ruledisinterr \\ 
    \ruledisatomic
  \end{mathpar}

  \begin{mathpar}
    \formax \\
    %\ruledisaxintfun \and \ruledisaxtop \and \ruleaxsym
    \ruledisaxintfun \and \ruleaxsym
  \end{mathpar}

  \caption{Top-like types and Algorithmic Disjointness.}
  \label{fig:tltypesdis}
\end{figure}



%However, the current system still has a substantial limitation: 
%we cannot construct an intersection out of two function types. 
%This will become evident while explaining the algorithmic disjointness rules, in the following section.
%Let us consider a term $t$ with type $Char \inter Int$ and a function $f$ with type $\top \rightarrow Char$.
%The application of $f$ to $x$ could lead to a choice between $Char$ and $Int$ and $Char \inter Int$.
%However, since for any type $A$, $\top \rightarrow A$ is isomorphic to $A$, the choice of the argument should
%not influence coherence. 
%Indeed, our system translates 
%Now, lets change the type of $f$ to $String \rightarrow \top$, and consider a second function $\g$, with type 
%$String \rightarrow Int$.

%Here such definition is made a bit more precise, and
%well-suited to \name.


%since their supertypes are all types with the form $A \to \top$ 
%(where A \emph{contains} \code{Int}, i.e. $\code{Int}$, $\code{Int} \inter \code{Char}$) and $\top$.


%At last, take as an example ($\top \inter \top$).
%In a pure language (i.e. no side-effects), this type can be safely allowed since both components of the merge
%will evaluate to the same value. 
%However, in an effectful language, the evaluation of either the right or left component might lead to distinct results. 
%Following the same reasoning, we might want to allow or restrict, for instance $(String \to Int) \inter (String \to \top)$,  


\paragraph{Algorithmic disjointness rules}

Similarly to the original system, the definition $\top$-disjointness
does not lead to an implementation.  Fortunately, the algorithmic
disjoitness rules remain the almost same as described in Section
\ref{sec:alg-dis}. The only significant difference is the absence
of the \reflabel{\labeldisfun} rule. The reason for this is that, in
this variant of \name, two functions \emph{always} have \emph{non top-like} common 
supertypes. For example, consider the function types:
\[ \code{Bool} \to \code{Int} \qquad \code{String} \to \code{String} \]
Although both the domains and co-domains of the functions seem 
to be unrelated, there are still non top-like common supertypes in the presence of 
$\top$. For example, $\code{Bool}\inter\code{String} \to \code{$\top$}$
is a common supertype of the previous function types. In general, in
this variant of \name, any two function types are never disjoint.

Finally, note that this variant of \name excludes all types of the form
$A \inter B$, where $A$  and $B$ are top-like. Thus,
the system is left with only one well-formed syntactic $\top$ type.


%This is clearly a limitation imposed by our definition of $\top$-disjointness - or rather, definition of top-like type - 
%for which we will present a solution in the next section.

\subsection{An Improved Calculus with $\top$}

The definition of top-like types in Section~\ref{subsec:simpletop} is,
unfortunately, quite restrictive: multiple function types are not allowed
within intersection types. This is in contrast with the original \name
calculus, where multiple function types can co-exist in an
intersection. To address this limitation, we generalize the definition
of top-like types. The generalization introduces a new ambiguity in
our subtyping rules, which requires some changes on the generation of target
language coercions.

\paragraph{Top-Like Types}
The extended top-like definitions and resulting system are formalised
in Figure \ref{fig:tltypesextdis}.  Note how we just added
\reflabel{\labeltoplfun} to the top-like relation, by stating that a
function is top-like whenever its return type is also
top-like. Although function types that return a top-like types are not
strictly speaking top-types, they can be viewed as local top-types for function
types. Such function types represent functions that produce $\top$, no
matter what arguments they are given.  In general, any type of the
form $A_k \to \top$ (with $k \in \mathbb{N}$), is considered a
top-like type in the new variant of \name. The consequence of allowing 
this more liberal notion of top-like types is that now disjoint
intersections such as \[ (\code{Bool} \to \code{Int}) \inter (\code{String} \to \code{String}) \]
are well-formed: both functions types are not top-like types according
to the new definition; and the only common supertypes that they share
are top-like types (for example: $\code{Bool}\inter\code{String} \to \code{$\top$}$).

\begin{figure}[t]
  \begin{mathpar}
    \formtoplike \\ %\framebox{$\jatomic A$} \\

    \ruletopltop \and \ruletoplinter \and \ruletoplfun
  \end{mathpar}

  \begin{mathpar}
    \formsub \\
    \rulesubinterlcoerce \and \rulesubinterrcoerce
  \end{mathpar}

  \begin{mathpar}
    \formdis \\
    \ruledisinterl \and \ruledisinterr \\ 
    \ruledisfun \and \ruledisatomic
  \end{mathpar}

  \begin{mathpar}
    \formax \\
    %\ruledisaxintfun \and \ruledisaxtop \and \ruleaxsym
    \ruledisaxintfunext \and \ruleaxsym
  \end{mathpar}

  \caption{Top-like types, Subtyping (changed rules only) and Algorithmic Disjointness for the improved calculus.}
  \label{fig:tltypesextdis}
\end{figure}

\begin{figure}[t]
  \framebox{$\andcoerce{A}_{C} = t$}
  \[
  \andcoerce{A}_{C} = 
  \begin{cases} 
        \toplike{A} & \andcoerce{A} \\ 
        %A = \top & () \\
        %A = A_1 \to A_2 \; \wedge \; \toplike{A_2} & \lam x \andcoerce{A_2}_{C} \\
        \text{otherwise} & {C} 
  \end{cases}
  \]

  \framebox{$\andcoerce{A} = t$}
  \[
  \andcoerce{A} = 
  \begin{cases} 
        A = \top & () \\
        A = A_1 \to A_2 & \lam x \andcoerce{A_2} \\
  \end{cases}
  \]


  %\begin{align*}
  %  \erase {i}                  &= i \\
  %  \erase {x}                  &= x \\
  %  \erase {\lam x E}           &= \lam x {\erase E} \\
  %  \erase {\app {e_1} {e_2}}   &= {\erase {e_1}} ({\erase {e_2}}) \\
  %  \erase {{x_1} ,, {x_2}}     &= {\erase {x_1}} \, ,, \, {\erase {x_2}} \\
  %  \erase {e : A}              &= {\erase e} \\
  %\end{align*}
  \caption{Coercion generation considering Top-like types.}
  \label{fig:andcoercion}
\end{figure}

%\joao{do we want to allow this?
% f: String $\to$ Int \\ 
% g: String $\to$ T \\
% ($\lam$ h : String $\to$ T. h "") (f ,, g) }

\paragraph{Coercive Subtyping}
The new definition of top-like types introduces a new problem when generating coercions. 
Introducing functions within intersection types leads to ambiguity between subtype contributors 
under intersection types.
%In other words, Lemma ... no longer holds because, under some contexts, 
%\reflabel{\labelsubinterl} and \reflabel{\labelsubinterr} now overlap. 
Let us demonstrate this using an example.
Suppose that we want to build a derivation for:  
\[(\code{Int} \to \code{Int}) \inter (\code{Char} \to \code{Char}) \subtype (\code{Int} \inter \code{Char}) \to \top\]
There are two possible derivations: 

\begin{itemize}

\item one using $\code{Int} \to \code{Int} \subtype (\code{Int} \inter \code{Char}) \to \top$ 
(via \reflabel{\labelsubinterl});

\item another using $\code{Char} \to \code{Char} \subtype (\code{Int} \inter \code{Char}) \to \top$
(via \reflabel{\labelsubinterr}).

\end{itemize}

\noindent Unfortunatelly the two derivations generate \emph{different}
coercions.  Thus, without further changes in \name, this would be a
source of incoherence.
%We could solve this problem at least in two distinct ways:

%\begin{itemize}
%\item Forbid intersection types that include more than one function type; or
%\item Adjust the subtyping relation to relax its contraints. 
%\end{itemize}

%Certainly, the first option would be an easy path to take, but the system would not be as expressive as we desired.
%On the other hand, the second option would require changing the existing rules.

%However, taking a closer look to \reflabel{\labeldisfun} (the next step in one of the two derivations), reveals that both coercions generated will $\beta$-reduce to the same term. 
%To illustrate this, suppose we are deriving $(A_k \to A) \inter (B_k \to B) \subtype C_k \to \top$.
%The coercions generated will be either $\lambda y. E_1 (proj_1 y)$ and $\lambda y. E_2 (proj_2 y)$, depending which intersection rule we choose.
%However, $E_1$ and $E_2$ will both be of the form $\lambda f_1. \lambda x_1. ... (\lambda f_k. \lambda x_k (\lambda e. \unit)...) (...)$ - a term that $\beta$-reduces to $\lambda f_1. \lambda x_1. ... \lambda x_k. ()$.
%Performing substitution and one $\beta$ reduction step in either of the former 
%coercions lead to the following term  
%$\lambda y. \lambda x_1... \lambda x_k. \unit$.
To address this new source of incoherence, we change the way coercions
are generated in \reflabel{\labelsubinterl} and
\reflabel{\labelsubinterr}. The changes are shown in Figure \ref{fig:tltypesextdis}.
%Namely, \reflabel{\labelsubinterl} and \reflabel{\labelsubinterr}, include a new premisse, ensuring that 
%the coercion can be produced the original way, as long as $A_3$ is not top-like. 
%On the other hand, in case that $A_3$ is top-like (and well-formed) then $A_3$ must be of the form $A_k \to \top$. 
%With this assumption over $A_3$, we know that the coercion generated should be a nested number of $\lambda$ equal to $k+1$,
%returning $\unit$ as the result, regardless of the parameters.
%These leads to two new rules, \reflabel{\labelsubintertopl} and
%\reflabel{\labelsubintertopr}.
The basic idea is to look at the form of $A_3$ in both rules:
\begin{itemize}
\item when $A_3$ is top-like ($\toplike{A_3}$), the coercion is a function with the same arity of $A_3$, returning $\unit$;  
\item when $A_3$ is not top-like: the same coercion (as in the previous
  systems) is generated.
\end{itemize}
This behaviour is formalised with a meta-function, denoted as $\andcoerce{A}_{C}$, and described in
Figure~\ref{fig:andcoercion}.

Finally, the reader may notice how the modified rules still overlap, in case $\toplike{A_3}$. 
However, in this case, both rules can be used interchangeably as they both lead to the \emph{same coercion}.
With this change in the subtyping rules we are able to retain coherence.
%Next, we will modify the algorithmic disjointess rules, so that they match the improved top-like specification.

\paragraph{Algorithmic disjointness rules}

The algorithmic disjointness rules are, again, similar to the ones presented in the original system.
In relation to the simple system with $\top$ we placed back $\reflabel{\labeldisfun}$, since we lifted the restriction
of intersections with function types.
We also had to modify $\reflabel{\labeldisaxintfun}$ to include the premise $\neg \toplike{B}$.
This is due to the specification of $\top$-disjointness, which requires two types not to be top-like in order for them to be disjoint.

\joao{add a paragraph mentioning the possibility of using two different definitions of top-like in T-disjointness, so we can get a DisAx-Int-Fun with no premise?}


%\bruno{
%Namely, \reflabel{\labelsubinterl} and \reflabel{\labelsubinterr} should be specialised into four rules, depending
%whether $A_3$ is toplike or not.
%We argue that this would not be very elegant, especially for formalizing the properties of the resulting system.
%Instead, we observed that, in case $A_3$ is a top-like type, the same coercion could be generated regardless of 
%the rule that is chosen.
%Indeed, there is only one way of generating a $\unit$ term, regardless of the form of the argument(s).
%}
