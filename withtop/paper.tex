\documentclass[numbers]{sigplanconf}

% For pdflatex, replaced by fontspec:
\usepackage[T1]{fontenc}
\usepackage[utf8]{inputenc}

\usepackage{fixltx2e}
\usepackage[usenames,dvipsnames,svgnames,table]{xcolor}
\usepackage{url}
\usepackage{fancyvrb}
\usepackage{mdwlist}  % Miscellaneous list-related commands
\usepackage{xspace}   % Smart spacing
\usepackage{ucs}
\usepackage{comment}

\usepackage{tikz}
\usetikzlibrary{positioning}

\usepackage{amsmath,amssymb,amsthm}
\usepackage{bm}         % Bold symbols in maths mode
\usepackage{dsfont}
\usepackage{stmaryrd}
\usepackage{mathtools}  % For "::=" ( \Coloneqq )

% http://tex.stackexchange.com/questions/114151/how-do-i-reference-in-appendix-a-theorem-given-in-the-body
\usepackage{thmtools, thm-restate}

\theoremstyle{definition}
\newtheorem{definition}{Definition}
\newtheorem{example}{Example}[section]

\theoremstyle{plain}
\newtheorem{theorem}{Theorem}
\newtheorem{lemma}{Lemma}
\newtheorem*{remark}{Remark}

\usepackage{cleveref}  % Never do "Theorem~\ref{thm1}" agin!
\crefname{lemma}{Lemma}{Lemmas}

% Font
\usepackage[euler-digits,euler-hat-accent]{eulervm}

% Figures with borders
% http://en.wikibooks.org/wiki/LaTeX/Floats,_Figures_and_Captions
% \usepackage{float}
% \floatstyle{boxed}
% \restylefloat{figure}

% Typesetting inference rules
\usepackage{styles/bcprules}    % by Benjamin C. Pierce
\usepackage{styles/cmll}
\usepackage{styles/mathpartir}  % by Didier Rémy

\usepackage{listings}   % For code listings

\lstdefinelanguage{scala}{
  morekeywords={abstract,case,catch,class,def,%
    do,else,extends,false,final,finally,%
    for,if,implicit,import,match,mixin,%
    new,null,object,override,package,%
    private,protected,requires,return,sealed,%
    super,this,throw,trait,true,try,%
    type,val,var,while,with,yield},
  otherkeywords={=>,<-,<\%,<:,>:,\#,@},
  sensitive=true,
  morecomment=[l]{//},
  morecomment=[n]{/*}{*/},
  morestring=[b]",
  morestring=[b]',
  morestring=[b]"""
}

\lstdefinelanguage{F2J}{
  morekeywords={let,rec,type,in},
  otherkeywords={->},
  sensitive=true,
  morecomment=[l]{--},
  morestring=[b]", % 'b' means inside a string delimiters are escaped by a backslash.
  morestring=[b]'
}

\lstset{ %
  % language=F2J,                % choose the language of the code
  columns=flexible,
  lineskip=-1pt,
  basicstyle=\ttfamily\small,       % the size of the fonts that are used for the code
  numbers=none,                   % where to put the line-numbers
  stepnumber=1,                   % the step between two line-numbers. If it's 1 each line will be numbered
  numbersep=5pt,                  % how far the line-numbers are from the code
  backgroundcolor=\color{white},  % choose the background color. You must add \usepackage{color}
  showspaces=false,               % show spaces adding particular underscores
  showstringspaces=false,         % underline spaces within strings
  showtabs=false,                 % show tabs within strings adding particular underscores
  tabsize=2,                  % sets default tabsize to 2 spaces
  captionpos=none,                   % sets the caption-position to bottom
  breaklines=true,                % sets automatic line breaking
  breakatwhitespace=false,        % sets if automatic breaks should only happen at whitespace
  title=\lstname,                 % show the filename of files included with \lstinputlisting; also try caption instead of title
  escapeinside={(*}{*)},          % if you want to add a comment within your code
  keywordstyle=\ttfamily\bfseries,
% commentstyle=\color{Gray}
% stringstyle=\color{Green}
}

%% Code listings
\usepackage{listings}

\lstdefinestyle{f2j}{
    basicstyle=\sffamily\small,
    keywordstyle=\sffamily\bfseries,
    tabsize=2,
    keepspaces=true,
    showstringspaces=false,
    escapeinside={(*}{*)},
    morekeywords={let,in,type,trait,def,interface}
}

\lstset{style=f2j}

% Copied from the FCore paper:
\usepackage[colorlinks=true,allcolors=black,breaklinks,draft=false]{hyperref}   % hyperlinks, including DOIs and URLs in bibliography
% known bug: http://tex.stackexchange.com/questions/1522/pdfendlink-ended-up-in-different-nesting-level-than-pdfstartlink

% math-mode versions of \rlap, etc
% from Alexander Perlis, "A complement to \smash, \llap, and lap"
%   http://math.arizona.edu/~aprl/publications/mathclap/
\def\clap#1{\hbox to 0pt{\hss#1\hss}}
\def\mathllap{\mathpalette\mathllapinternal}
\def\mathrlap{\mathpalette\mathrlapinternal}
\def\mathclap{\mathpalette\mathclapinternal}
\def\mathllapinternal#1#2{\llap{$\mathsurround=0pt#1{#2}$}}
\def\mathrlapinternal#1#2{\rlap{$\mathsurround=0pt#1{#2}$}}
\def\mathclapinternal#1#2{\clap{$\mathsurround=0pt#1{#2}$}}

% math-mode versions of \rlap, etc
% from Alexander Perlis, "A complement to \smash, \llap, and lap"
%   http://math.arizona.edu/~aprl/publications/mathclap/
\def\clap#1{\hbox to 0pt{\hss#1\hss}}
\def\mathllap{\mathpalette\mathllapinternal}
\def\mathrlap{\mathpalette\mathrlapinternal}
\def\mathclap{\mathpalette\mathclapinternal}
\def\mathllapinternal#1#2{\llap{$\mathsurround=0pt#1{#2}$}}
\def\mathrlapinternal#1#2{\rlap{$\mathsurround=0pt#1{#2}$}}
\def\mathclapinternal#1#2{\clap{$\mathsurround=0pt#1{#2}$}}

\newcommand{\horizontalrule}{
  \begin{center}
    \line(1,0){250}
  \end{center}
}

\newcommand{\code}[1]{\texttt{#1}}
\definecolor{light-gray}{gray}{0.9}
\newcommand{\highlight}[1]{\colorbox{light-gray}{#1}}

\newcommand{\im}[1]{\lvert #1 \rvert}
\newcommand{\powerset}[1]{\mathcal{P}(#1)}
\newcommand{\universe}{\mathbb{U}}

\newcommand{\turns}{\vdash}
\newcommand{\oftype}{\!:\!}
\newcommand{\subtype}{<:}
\newcommand{\commonsuper}{\Uparrow}
\newcommand{\commonsub}{\Downarrow}
\newcommand{\disjoint}{*}
\newcommand{\disjointimpl}{*_\textnormal{i}}
\newcommand{\disjointax}{*_\textnormal{ax}}
\newcommand{\subst}[2]{\lbrack #2 := #1 \rbrack~}

\newcommand{\yields}[1]{\highlight{$\; \hookrightarrow #1$}}
% \newcommand{\yields}[1]{}

\newcommand{\ftv}[1]{\textsf{ftv}(#1)}

\newcommand{\binderspace}{\,}
\newcommand{\appspace}{\;}

\newcommand{\inter}{\&}
\newcommand{\union}{|}
\newcommand{\for}[2]{\forall #1.\binderspace #2}
\newcommand{\fordis}[3]{\for {(#1 \disjoint #2)} {#3}}
\newcommand{\lam}[2]{\lambda #1.\binderspace #2}
\newcommand{\lamty}[3]{\lam {(#1 \oftype #2)} #3}
\newcommand{\blam}[2]{\Lambda #1.\binderspace #2}
\newcommand{\blamdis}[3]{\blam {(#1 \disjoint #2)} #3}
\newcommand{\mergeop}{,,}
\newcommand{\app}[2]{#1 \; #2}
\newcommand{\tapp}[2]{#1 \appspace #2}
\newcommand{\pair}[2]{(#1, #2)}
\newcommand{\proj}[2]{{\code{proj}}_{#1} #2}
\newcommand{\fst}[1]{\app {\code{fst}} {#1}}
\newcommand{\snd}[1]{\app {\code{snd}} {#1}}
\newcommand{\recordType}[2]{\{ #1 : #2 \}}
\newcommand{\recordCon}[2]{\{ #1 = #2 \}}

\newcommand{\true}{\code{True}}
\newcommand{\tyint}{\code{Int}}
\newcommand{\tybool}{\code{Bool}}
\newcommand{\tychar}{\code{Char}}
\newcommand{\tystring}{\code{String}}

% Judgements
\newcommand{\jwf}[2]{#1 \turns #2}
\newcommand{\jatomic}[1]{#1 \ \textnormal{atomic}}
\newcommand{\jtype}[3]{\turns #2 \ \oftype \ #3}
\newcommand{\jdis}[3]{\turns #2 \disjoint #3}
\newcommand{\jdisimpl}[3]{\turns #2 \disjointimpl #3}

\newcommand{\reflabel}[1]{(\textsc{#1})}

% \newcommand{\name}{$ \lambda_{\&} $\xspace}
\newcommand{\name}{\xspace[name]\xspace}

\newcommand{\authornote}[3]{\textcolor{#2}{\textsc{#1}: #3}}
\newcommand\bruno[1]{\authornote{bruno}{Red}{#1}}
\newcommand\george[1]{\authornote{george}{Blue}{#1}}


\newcommand{\formwf}{\framebox{$ \jwf \Gamma A $}}

\newcommand{\makelabelwf}[1]{F\_WF\_#1}

\newcommand{\labelwfvar}{\makelabelwf Var}
\newcommand{\rulewfvar}{
  \inferrule* [right=\labelwfvar]
    {\alpha \in \Gamma}
    {\jwf \Gamma \alpha}
}

\newcommand{\labelwfbot}{\makelabelwf Bot}
\newcommand{\rulewfbot}{
  \inferrule* [right=\labelwfbot]
    { }
    {\jwf \Gamma \bot}
}

\newcommand{\labelwffun}{\makelabelwf Fun}
\newcommand{\rulewffun}{
  \inferrule* [right=\labelwffun]
    {\jwf \Gamma A \\ \jwf \Gamma B}
    {\jwf \Gamma {A \to B}}
}

\newcommand{\labelwfforall}{\makelabelwf Forall}
\newcommand{\rulewfforall}{
  \inferrule* [right=\labelwfforall]
    {\jwf {\Gamma, \alpha} A}
    {\jwf \Gamma {\for \alpha A}}
}

\newcommand{\rulewfforalldis}{
  \inferrule* [right=\labelwfforall]
    {\jwf \Gamma A \\ \jwf {\Gamma, \alpha \disjoint A} B}
    {\jwf \Gamma {\fordis \alpha A B}}
}

\newcommand{\labelwfinter}{\makelabelwf Inter}
\newcommand{\rulewfinter}{
  \inferrule* [right=\labelwfinter]
    {\jwf \Gamma A \\ \jwf \Gamma B}
    {\jwf \Gamma {A \inter B}}
}

\newcommand{\rulewfinterdis}{
  \inferrule* [right=\labelwfinter]
    {\jwf \Gamma A \\
     \jwf \Gamma B \\
     \highlight{$\jdis \Gamma A B$}}
    {\jwf \Gamma {A \inter B}}
}

\newcommand{\formdis}{\framebox{$ \jdisimpl \Gamma A B $}}

\newcommand{\makelabeldis}[1]{Dis\_#1}

\newcommand{\labeldisvar}{\makelabeldis Var}
\newcommand{\ruledisvar}{
  \inferrule* [right=\labeldisvar]
    {\alpha * B \in \Gamma}
    {\jdisimpl \Gamma \alpha B}
}

\newcommand{\labeldissym}{\makelabeldis Sym}
\newcommand{\ruledissym}{
  \inferrule* [right=\labeldissym]
    {\alpha \disjoint A \in \Gamma}
    {\jdis \Gamma A \alpha}
}

\newcommand{\labeldisinterl}{\makelabeldis {Inter\_1}}
\newcommand{\ruledisinterl}{
  \inferrule* [right=\labeldisinterl]
    {\jdisimpl \Gamma A C \\ \jdisimpl \Gamma B C}
    {\jdisimpl \Gamma {A \inter B} {C}}
}

\newcommand{\labeldisinterr}{\makelabeldis {Inter\_2}}
\newcommand{\ruledisinterr}{
  \inferrule* [right=\labeldisinterr]
    {\jdisimpl \Gamma A B \\ \jdisimpl \Gamma A C}
    {\jdisimpl \Gamma {A} {B \inter C}}
}

\newcommand{\labeldisfun}{\makelabeldis Fun}
\newcommand{\ruledisfun}{
  \inferrule* [right=\labeldisfun]
    {\jdisimpl \Gamma {B_1} {B_2}}
    {\jdisimpl \Gamma {A_1 \to A_2} {B_1 \to B_2}}
}

\newcommand{\labeldisforall}{\makelabeldis Forall}
\newcommand{\ruledisforall}{
  \inferrule* [right=\labeldisforall]
    {\jdisimpl \Gamma A C}
    {\jdisimpl \Gamma {\fordis \alpha B A} {\fordis \alpha B C} \george{Subst???}}
}

\newcommand{\labeldisatom}{\makelabeldis Atom}
\newcommand{\ruledisatomic}{
  \inferrule* [right=\labeldisatom]
    {A \not \sim B}
    {\jdisimpl \Gamma A B}
}

\newcommand{\formsub}{\framebox{$ A \subtype B \yields E $}}

\newcommand{\makelabelsub}[1]{Sub\_#1}

\newcommand{\labelsubint}{\makelabelsub Int}
\newcommand{\rulesubint}{
  \inferrule* [right=\labelsubint]
    { }
    {\tyint \subtype \tyint \yields {\lamty x {\im \alpha} x}}
}

\newcommand{\labelsubtop}{\makelabelsub Top}
\newcommand{\rulesubtop}{
  \inferrule* [right=\labelsubtop]
    { }
    {A \subtype \top \yields {\lamty x {\im A} \unit}}
}

\newcommand{\labelsubvar}{\makelabelsub Var}
\newcommand{\rulesubvar}{
  \inferrule* [right=\labelsubvar]
    { }
    {\alpha \subtype \alpha \yields {\lamty x {\im \alpha} x}}
}

\newcommand{\labelsubfun}{\makelabelsub Fun}
\newcommand{\rulesubfun}{
  \inferrule* [right=\labelsubfun]
    {{B_1} \subtype {A_1} \yields {E_1} \\
     {A_2} \subtype {B_2} \yields {E_2}}
    {{A_1 \to A_2} \subtype {B_1 \to B_2}
    \yields
        {\lamty f {\im {A_1 \to A_2}}
        {\lamty x {\im {B_1}}
            {\app {E_2} {(\app f {(\app {E_1} x)})}}}}}
}

\newcommand{\labelsubforall}{\makelabelsub Forall}
\newcommand{\rulesubforall}{
  \inferrule* [right=\labelsubforall]
    {{A_1} \subtype {A_2} \yields E}
    {\for {\alpha} {A_1} \subtype \for {\alpha} {A_2}
      \yields
        {\lamty f {\im {\for {\alpha} {A_1}}}
          {\blam \alpha {\app E {(\app f \alpha)}}}}}
}

\newcommand{\rulesubforalldis}{
  \inferrule* [right=\labelsubforall]
    {{B_1} \subtype {B_2} \yields E}
    {\fordis {\alpha} A {B_1} \subtype \fordis {\alpha} A {B_2}
      \yields
        {\lamty f {\im {\for {\alpha} {B_1}}}
          {\blam \alpha {\app E {(\app f \alpha)}}}}}
}

\newcommand{\rulesubforallext}{
  \inferrule* [right=\labelsubforall]
    {{B_1} \subtype {B_2} \yields E \\
     {A_2} \subtype {A_1} \yields {E_d}}
    {\fordis {\alpha} {A_1} {B_1} \subtype \fordis {\alpha} {A_2} {B_2}
      \yields
        {\lamty f {\im {\for {\alpha} {B_1}}}
          {\blam \alpha {\app E {(\app f \alpha)}}}}}
}

\newcommand{\labelsubinter}{\makelabelsub Inter}
\newcommand{\rulesubinter}{
  \inferrule* [right=\labelsubinter]
    {{A_1} \subtype {A_2} \yields {E_1} \\
     {A_1} \subtype {A_3} \yields {E_2}}
    {{A_1} \subtype {A_2 \inter A_3}
      \yields
        {\lamty x {\im {A_1}}
          {\pair {\app {E_1} x} {\app {E_2} x}}}}
}

\newcommand{\labelsubinterl}{\makelabelsub {Inter\_1}}
\newcommand{\rulesubinterl}{
  \inferrule* [right=\labelsubinterl]
    {{A_1} \subtype {A_3} \yields E}
    {{A_1 \inter A_2} \subtype {A_3}
      \yields
        {\lamty x {\im {A_1 \inter A_2}}
          {\app E {(\proj 1 x)}}}}
}

\newcommand{\rulesubinterldis}{
\inferrule* [right=\labelsubinterl]
    {{A_1} \subtype {A_3} \yields E \\ \jatomic {A_3}}
    {{A_1 \inter A_2} \subtype {A_3}
      \yields
        {\lamty x {\im {A_1 \inter A_2}}
          {\app E {(\proj 1 x)}}}}
}

\newcommand{\labelsubinterr}{\makelabelsub {Inter\_2}}
\newcommand{\rulesubinterr}{
  \inferrule* [right=\labelsubinterr]
    {{A_2} \subtype {A_3} \yields E}
    {{A_1 \inter A_2} \subtype {A_3}
      \yields
        {\lamty x {\im {A_1 \inter A_2}}
          {\app E {(\proj 2 x)}}}}
}

\newcommand{\rulesubinterrdis}{
  \inferrule* [right=\labelsubinterr]
    {{A_2} \subtype {A_3} \yields E \\ \jatomic {A_3}}
    {{A_1 \inter A_2} \subtype {A_3}
      \yields
        {\lamty x {\im {A_1 \inter A_2}}
          {\app E {(\proj 2 x)}}}}
}

\newcommand{\rulesubinterlcoerce}{
  \inferrule* [right=\labelsubinterl]
    {{A_1} \subtype {A_3} \yields E \\ \jatomic {A_3} }
    {{A_1 \inter A_2} \subtype {A_3}
      \yields
        { \lamty x {\im {A_1 \inter A_2}} {\andcoerce{A_3}_{{(
          {\app E {(\proj 1 x)}})}}}}}
}

\newcommand{\rulesubinterrcoerce}{
  \inferrule* [right=\labelsubinterr]
    {{A_2} \subtype {A_3} \yields E \\ \jatomic {A_3} }
    {{A_1 \inter A_2} \subtype {A_3}
      \yields 
        { \lamty x {\im {A_1 \inter A_2}} {\andcoerce{A_3}_{{(
          {\app E {(\proj 2 x)}})}}}}}
}

\newcommand{\rulelabel}{\text{Ty}}
\newcommand{\rulelabelSelect}{\text{Sel}}
\newcommand{\rulelabelRestrict}{\text{Res}}

% Var
\newcommand{\rulelabelVar}{\rulelabel\text{Var}}
\newcommand{\ruleVar} {
\inferrule* [right=$\rulelabelVar$]
  {x \hast A \in \Gamma}
  {\hastype \Gamma x A \yields x}
}

% Top
\newcommand{\rulelabelTop}{\rulelabel\text{Top}}
\newcommand{\ruleTop} {
\inferrule* [right=$\rulelabelTop$]
  { }
  {\hastype \Gamma \top \top \yields {()}}
}

% Lam
\newcommand{\rulelabelLam}{\rulelabel\text{Lam}}
\newcommand{\ruleLam} {
\inferrule* [right=$\rulelabelLam$]
  {\istype \Gamma A \\ \hastype {\Gamma, x \hast A} e B \yields E}
  {\hastype \Gamma {\lam x A e} {A \to B} \yields {\lam x {\im A} E}}
}

% App
\newcommand{\rulelabelApp}{\rulelabel\text{App}}
\newcommand{\ruleApp}{
\inferrule* [right=$\rulelabelApp$]
  {\hastype \Gamma {e_1} {A_1 \to A_2} \yields {E_1} \\
   \hastype \Gamma {e_2} {A_3} \yields {E_2} \\
   A_3 \subtype A_1 \yields C}
  {\hastype \Gamma {\app {e_1} {e_2}} {A_2} \yields {\app {E_1} {(\app C E_2)}}}
}

% BLam
\newcommand{\rulelabelBLam}{\rulelabel\text{BLam}}
\newcommand{\ruleBLam}{
\inferrule* [right=$\rulelabelBLam$]
  {\hastype {\Gamma, \alpha * B} e A \yields E}
  {\hastype \Gamma {\blam {\alpha * B} e} {\for {\alpha * B} A} \yields {\blam \alpha E}}
}

% TApp
\newcommand{\rulelabelTApp}{\rulelabel\text{TApp}}
\newcommand{\ruleTApp}{
\inferrule* [right=$\rulelabelTApp$]
  {\hastype \Gamma e {\for {\alpha * C} B} \yields E \\ \isdisjoint
  \Gamma A C \\\istype \Gamma A}
  {\hastype \Gamma {\tapp e A} {\subst A \alpha B} \yields {\tapp E {\im A}}}
}

% Merge
\newcommand{\rulelabelMerge}{\rulelabel\text{Merge}}
\newcommand{\ruleMerge}{
\inferrule* [right=$\rulelabelMerge$]
  {\hastype \Gamma {e_1} A \yields {E_1} \\
   \hastype \Gamma {e_2} B \yields {E_2} \\
   % A \bot B}
   \isdisjoint \Gamma A B}
  {\hastype \Gamma {e_1 \mergeOp e_2} {A \intersect B} \yields {\pair {E_1} {E_2}}}
}

% ConstraintIntro
\newcommand{\rulelabelConstraintIntro}{\rulelabel\text{ConstraintIntro}}
\newcommand{\ruleConstraintIntro}{
  \inferrule* [right=$\rulelabelConstraintIntro$]
    {\istype \Gamma {A_1} \\ \istype \Gamma {A_2} \\
     \hastype {\Gamma, A_1 \disjoint A_2} e B \yields E}
    {\hastype \Gamma {\assume {(A_1 \disjoint A_1)} e} {\constraints {A_1   \disjoint A_2} B} \yields E}
}

% ConstraintElim
\newcommand{\rulelabelConstraintElim}{\rulelabel\text{ConstraintElim}}
\newcommand{\ruleConstraintElim}{
\inferrule* [right=$\rulelabelConstraintElim$]
  {\hastype \Gamma e {\constraints {A_1 \disjoint A_2} B} \yields E \\
  \isdisjoint \Gamma {A_1} {A_2}}
  {\hastype \Gamma {\app e {\_}} B \yields E}
}

% rec-con
\newcommand{\rulelabelRecConstruct}{\rulelabel\text{rec-construct}}
\newcommand{\rulerecordConstruct}{
\inferrule* [right=$\rulelabelRecConstruct$]
  {\hastype \Gamma e A \yields E}
  {\hastype \Gamma {\recordCon l e} {\recordType l A} \yields E}
}

% rec-select
\newcommand{\rulelabelRecSelect}{\rulelabel\text{rec-select}}
\newcommand{\ruleRecSelect}{
\inferrule* [right=$\rulelabelRecSelect$]
  {\hastype \Gamma e A \yields E \\
   \judgeSelect A l {A_1} \yields C}
  {\hastype \Gamma {e.l} {A_1} \yields {\app C E}}
}

% rec-restrict
\newcommand{\rulelabelRecRestrict}{\rulelabel\text{rec-restrict}}
\newcommand{\ruleRecRestrict}{
\inferrule* [right=$\rulelabelRecRestrict$]
  {\hastype \Gamma e A \yields E \\
   \judgeRestrict A l {A_1} \yields C}
  {\hastype \Gamma {e \restrictOp l} {A_1} \yields {\app C E}}
}

\newcommand{\judgeSelect}[3]{#1 \bullet #2 = #3}

% select
\newcommand{\ruleGet}{
  \inferrule* [right=$\rulelabelSelect$]
  { }
  {\judgeSelect {\recordType l A} l A \yields {\lam x {\im {\recordType l A}} x}}
}

% select1
\newcommand{\rulelabelSelectLeft}{{\rulelabelSelect}_1}
\newcommand{\ruleGetLeft}{
  \inferrule* [right=$\rulelabelSelectLeft$]
  {\judgeSelect {A_1} l A \yields C}
  {\judgeSelect {A_1 \intersect A_2} l A \yields {\lam x {\im {A_1
          \intersect A_2}} {\app C {(\proj 1 x)}}}}
}

% select2
\newcommand{\rulelabelSelectRight}{{\rulelabelSelect}_2}
\newcommand{\ruleGetRight}{
  \inferrule* [right=$\rulelabelSelectRight$]
  {\judgeSelect {A_2} l A \yields C}
  {\judgeSelect {A_1 \intersect A_2} l A \yields {\lam x {\im {A_1
          \intersect A_2}} {\app C {(\proj 2 x)}}}}
}

\newcommand{\judgeRestrict}[3]{#1 \bm{\restrictOp} #2 = #3}

% restrict
\newcommand{\ruleRestrict}{
  \inferrule* [right=$\rulelabelRestrict$]
  { }
  {\judgeRestrict {\recordType l A} l \top \yields {\lam x {\im {\recordType l A}} {()}}}
}

% restrict1
\newcommand{\rulelabelRestrictleft}{{\rulelabelRestrict}_1}
\newcommand{\ruleRestrictLeft}{
  \inferrule* [right=$\rulelabelRestrictleft$]
  {\judgeRestrict {A_1} l A \yields C}
  {\judgeRestrict {A_1 \intersect A_2} l {A \intersect A_2} \yields {\lam x {\im {A_1
          \intersect A_2}} {\pair {\app C {(\proj 1 x)}} {\proj 2 x}}}}
}

% restrict2
\newcommand{\rulelabelRestrictRight}{{\rulelabelRestrict}_2}
\newcommand{\ruleRestrictRight}{
  \inferrule* [right=$\rulelabelRestrictRight$]
  {\judgeRestrict {A_2} l A \yields C}
  {\judgeRestrict {A_1 \intersect A_2} l {A_1 \intersect A} \yields {\lam x {\im {A_1
          \intersect A_2}} {\pair {\proj 1 x} {\app C {(\proj 2 x)}}}}}
}

% \renewcommand{\yields}[1]{}

\newcommand{\formbi}{
\framebox{
$ \jinfer \Gamma e A \yields E\\e$ synthesizes type $A$}
}

\newcommand{\formbc}{
\framebox{
$ \jcheck \Gamma e A \yields E\\e$ checks against given type $A$
}
}

%$ \jtype \Gamma e A \Leftarrow E~~~~~
%e$ checks againts known type $A$}}

\newcommand{\bruletvar} {
\inferrule* [right=\labeltvar]
  {x \oftype A \in \Gamma}
  {\jinfer \Gamma x A \yields x}
}

\newcommand{\bruletint} {
\inferrule* [right=\labeltint]
  { }
  {\jinfer \Gamma i {\code{Int}} \yields i}
}

\newcommand{\bruletlam} {
\inferrule* [right=\labeltlam]
  {\jwf \Gamma A \\ \jcheck {\Gamma, x \oftype A} e B \yields E}
  {\jcheck \Gamma {\lamty x A e} {A \to B} \yields {\lamty x {\im A} E}}
}

\newcommand{\bruletapp}{
\inferrule* [right=\labeltapp]
  {\jinfer \Gamma {e_1} {A_{1} \to A_{2}} \yields {E_1} \\
   \jcheck \Gamma {e_2} {A_{1}} \yields {E_2}}
%%   {A_3} \subtype {A_1} \yields {E}}
  {\jinfer \Gamma {\app {e_1} {e_2}} {A_{2}} \yields {\app {E_1} E_2}}
}

\newcommand{\bruletblam}{
\inferrule* [right=\labeltblam]
  {\jtype {\Gamma, \alpha} e A \yields E \\
   \jwf \Gamma B \\
   \alpha \not \in \ftv \Gamma}
  {\jtype \Gamma {\blam {\alpha} e} {\for {\alpha} A} \yields {\blam \alpha E}}
}

\newcommand{\bruletblamdis}{
\inferrule* [right=\labeltblam]
  {\jwf \Gamma A \\
   \jtype {\Gamma,\alpha \disjoint A} e B \yields E \\
   \alpha \not \in \ftv \Gamma}
  {\jtype \Gamma {\blamdis \alpha A e} {\fordis \alpha A B}
    \yields {\blam \alpha E}}
}

\newcommand{\brulettapp}{
\inferrule* [right=\labelttapp]
  {\jtype \Gamma e {\for \alpha B} \yields E \\
   \jwf \Gamma A}
  {\jtype \Gamma {\tapp e A} {\subst A \alpha B} \yields {\tapp E {\im A}}}
}

\newcommand{\brulettappdis}{
\inferrule* [right=\labelttapp]
  {\jtype \Gamma e {\for {\alpha \disjoint B} C} \yields E \\
   \jwf \Gamma A \\
   \jdis \Gamma A B}
  {\jtype \Gamma {\tapp e A} {\subst A \alpha C} \yields {\tapp E {\im A}}}
}

\newcommand{\bruletmerge}{
\inferrule* [right=\labeltmerge]
  {\jinfer \Gamma {e_1} A \yields {E_1} \\
   \jinfer \Gamma {e_2} B \yields {E_2}}
  {\jinfer \Gamma {e_1 \mergeop e_2} {A \inter B} \yields {\pair {E_1} {E_2}}}
}

\newcommand{\bruletsub}{
\inferrule* [right=\labeltsub]
  {\jinfer \Gamma {e} A \yields {E} \\ {A} \subtype {B} \yields {C}}
  {\jcheck \Gamma {e} {B} \yields {\tapp C E}}
}

\newcommand{\bruletmergedis}{
\inferrule* [right=\labeltmerge]
  {\jinfer \Gamma {e_1} A \yields {E_1} \\
   \jinfer \Gamma {e_2} B \yields {E_2} \\
   \framebox{$\jdis \Gamma A B$}}
  {\jinfer \Gamma {e_1 \mergeop e_2} {A \inter B} \yields {\pair {E_1} {E_2}}}
}

\newcommand{\brulettop}{
\inferrule* [right=\labelttop]
  { }
  {\jinfer \Gamma \top \top \yields \unit}
}

\newcommand{\labeltann}{\makelabelt Ann}
\newcommand{\bruletann}{
\inferrule* [right=\labeltann]
  {\jcheck \Gamma {e} A \yields {E}}
  {\jinfer \Gamma {e : A} {A} \yields {E}}
}
\newcommand{\makelabeltgt}[1]{Tgt\_#1}

% Target WF
\newcommand{\formtgtwf}{\framebox{$ \jwf G T $}}

\newcommand{\makelabeltgtwf}[1]{\makelabeltgt {WF\_#1}}

\newcommand{\labeltgtwffv}{\makelabeltgtwf FV}
\newcommand{\ruletgtwffv} {
\inferrule* [right=\labeltgtwffv]
  {\ftv T \in G}
  {\jwf G T}
}

% Target typing
\newcommand{\formtgt}{\framebox{$ \jtype G E T $}}

\newcommand{\makelabeltgtty}[1]{\makelabeltgt {Ty\_#1}}

\newcommand{\labeltgtvar}{\makelabeltgtty Var}
\newcommand{\ruletgtvar} {
\inferrule* [right=\labeltgtvar]
  {(x,T) \in \Gamma}
  {\jtype \Gamma x T}
}

\newcommand{\labeltgtlam}{\makelabeltgtty Lam}
\newcommand{\ruletgtlam} {
\inferrule* [right=\labeltgtlam]
  {\jtype {\Gamma, x \oftype T} E {T_1} \andalso \jwf \Gamma T}
  {\jtype \Gamma {\lamty x T E} {T \to T_1}}
}

\newcommand{\labeltgtapp}{\makelabeltgtty App}
\newcommand{\ruletgtapp}{
\inferrule* [right=\labeltgtapp]
  {\jtype \Gamma {E_1} {T_1 \to T_2} \andalso \jtype \Gamma {E_2} {T_1}}
  {\jtype \Gamma {\app {E_1} {E_2}} {T_2}}
}

\newcommand{\labeltgtblam}{\makelabeltgtty BLam}
\newcommand{\ruletgtblam}{
\inferrule* [right=\labeltgtblam]
  {\jtype {\Gamma, \alpha} E T}
  {\jtype \Gamma {\blam \alpha E} {\for \alpha T}}
}

\newcommand{\labeltgttapp}{\makelabeltgtty TApp}
\newcommand{\ruletgttapp}{
\inferrule* [right=\labeltgttapp]
  {\jtype \Gamma E {\for \alpha {T_1}} \andalso \jwf \Gamma T}
  {\jtype \Gamma {\tapp E T} {\subst T \alpha T_1}}
}

\newcommand{\labeltgtpair}{\makelabeltgtty Pair}
\newcommand{\ruletgtpair}{
\inferrule* [right=\labeltgtpair]
  {\jtype \Gamma {E_1} {T_1} \andalso \jtype \Gamma {E_2} {T_2}}
  {\jtype \Gamma {\pair {E_1} {E_2}} {\pair {T_1} {T_2}}}
}

\newcommand{\labeltgtprojl}{\makelabeltgtty {Proj\_1}}
\newcommand{\ruletgtprojl}{
\inferrule* [right=\labeltgtprojl]
  {\jtype \Gamma E {\pair {T_1} {T_2}}}
  {\jtype \Gamma {\proj 1 E} {T_1}}
}

\newcommand{\labeltgtprojr}{\makelabeltgtty {Proj\_2}}
\newcommand{\ruletgtprojr}{
\inferrule* [right=\labeltgtprojr]
  {\jtype \Gamma E {\pair {T_1} {T_2}}}
  {\jtype \Gamma {\proj 2 E} {T_2}}
}

\newcommand{\formtoplike}{\framebox{$ \toplike{A} $}}

\newcommand{\makelabeltopl}[1]{$TL#1$}

\newcommand{\labeltopltop}{\makelabeltopl \top}
\newcommand{\ruletopltop}{
  \inferrule* [right=\labeltopltop]
    { }
    { \toplike{\top} }
}

\newcommand{\labeltoplfun}{\makelabeltopl \rightarrow}
\newcommand{\ruletoplfun}{
  \inferrule* [right=\labeltoplfun]
    { \toplike{B} }
    { \toplike{A \to B} }
}

\newcommand{\labeltoplinterl}{\makelabeltopl Inter-1}
\newcommand{\ruletoplinterl}{
  \inferrule* [right=\labeltoplinterl]
    { \toplike{A} }
    { \toplike{A \inter B} }
}

\newcommand{\labeltoplinterr}{\makelabeltopl Inter-2}
\newcommand{\ruletoplinterr}{
  \inferrule* [right=\labeltoplinterr]
    { \toplike{B} }
    { \toplike{A \inter B} }
}

\newcommand{\labeltoplinter}{\makelabeltopl \&}
\newcommand{\ruletoplinter}{
  \inferrule* [right=\labeltoplinter]
    { \toplike{A} \\ \toplike{B}}
    { \toplike{A \inter B} }
}

\newcommand{\labeltoplforall}{\makelabeltopl \forall}
\newcommand{\ruletoplforall}{
  \inferrule* [right=\labeltoplforall]
    { \toplike{A} }
    { \toplike{\fordis \alpha B A} }
}

\newcommand{\labeltoplrec}{\makelabeltopl Rec}
\newcommand{\ruletoplrec}{
  \inferrule* [right=\labeltoplrec]
    { \toplike{A} }
    { \toplike{\recordType l A} }
}


\begin{document}
\toappear{}
%\special{papersize=8.5in,11in}
%\setlength{\pdfpageheight}{\paperheight}
%\setlength{\pdfpagewidth}{\paperwidth}

\conferenceinfo{CONF 'yy}{Month d--d, 20yy, City, ST, Country}
\copyrightyear{20yy}
\copyrightdata{978-1-nnnn-nnnn-n/yy/mm}
\doi{nnnnnnn.nnnnnnn}

\titlebanner{banner above paper title}        % These are ignored unless
\preprintfooter{\name}                        % 'preprint' option specified.

\title{Disjoint Intersection Types} 
% \subtitle{Full Version with Appendix}

\authorinfo{Bruno C. d. S. Oliveira\and Zhiyuan Shi\and João Alpuim}
           {The University of Hong Kong}
           {\{bruno,zyshi,alpuim\}@cs.hku.hk}
%\authorinfo{Name2\and Name3}
%           {Affiliation2/3}
%           {Email2/3}

\maketitle

\begin{abstract}
  Dunfield has shown that a simply typed core calculus with intersection types and
a merge operator is able to capture various programming language features. While
his calculus is type-safe, it is known that it is not \emph{coherent}:
different derivations for the same expression can lead to different results. The
lack of coherence is an important disadvantage for adoption of his core calculus
in implementations of programming languages, as the semantics of the programming
language becomes implementation-dependent.

This paper presents \name: a core calculus with a variant of \emph{intersection
types} and a \emph{merge operator}. The semantics \name is both type-safe and
coherent. Coherence is achieved by ensuring that intersection types are
\emph{disjoint}. Formally, two types are disjoint if they do not share a common
supertype. We present a type system that prevents intersection types that are
not disjoint, as well as an algorithmic specification to determine whether two
types are disjoint. Moreover, we show the applicability of this calculus to
express a simple, yet powerful form of dynamically composable traits, paving the way for new
designs of object-oriented programming languages.

\end{abstract}

\category{D.3.2}{Language Classifications}{Applicative (functional) languages}
\category{F.3.3}{Studies of Program Constructs}{Functional constructs}

% general terms are not compulsory anymore,
% you may leave them out
\terms Design, Languages, Theory

\keywords Intersection Types, Type System

\section{Introduction}

There has been a remarkable number of works aimed at improving support
for extensibility in programming languages. These works include:
visions of new programming models~\cite{}; new programming languages or
language extensions~\cite{}, and \emph{design patterns} that can be
used with existing mainstream languages~\cite{}.

%\cite{family polymorphism and virtual
%classes.}. Another line of work are proposals for precise formal models or new 
%programming languages. Yet another line are \emph{design patterns}
%that can be used  with existing mainstream languages. 
%%Part of the motivation behind 

Some of the more recent work on extensibility is focused on various
proposals for design patterns.  Examples include \emph{Object
  Algebras}~\cite{}, \emph{Modular Visitors}~\cite{} or
Torgersen's~\cite{} four design patterns using generics. In those
approaches the idea is to use some advanced (but already available)
features, such as \emph{generics}, in combination with conventional
OOP features to model more extensible designs.  Those designs work in
modern OOP languages such as Java, C\# or Scala.

Although such design patterns give practical benefits in terms of
extensibility, they also expose limitations in existing mainstream OOP
languages. In particular there are three pressing limitations: 
1) lack of good mechanisms for
  \emph{object-level} composition; 2) \emph{conflation of 
    (type) inheritance with subtyping}; 3) \emph{heavy reliance on generics}.

  The first limitation shows up, for example, in Oliveira et
  al.~\cite{} encodings of Feature-Oriented Programming using Object
  Algebras~\cite{}. These programs are best expressed using a form of
  \emph{type-safe}, \emph{dynamic}, \emph{delegation}-based
  composition. Although such form of composition can be encoded in
  languages like Scala, it requires the use of low-level reflection
  techniques, such as dynamic proxies, reflection or other forms of
  meta-programming~\cite{}. It is clear that better language support
  would be desirable.

  The second limitation shows up in designs for modelling
  modular or extensible visitors~\cite{}.  The vast majority of modern
  OOP languages combines type inheritance and subtyping. 
  That is a type extension induces a subtype. However
  as Cook et al.~\cite{} famously argued there are programs where
  ``\emph{subtyping is not inheritance}''. Interestingly previously
  not many practical programs have been reported in the literature
  where the distinction between subtyping and inheritance is
  relevant. However, as shown in this paper, it turns out that this
  difference does show up in practice when designing modular
  (extensible) visitors.  We believe that modular visitors provide a
  compeling practical example where inheritance and subtyping should
  not be conflated!

  Finally, the third limitation is prevalent in many extensible
  designs~\cite{}. Such designs rely on advanced features of generics,
  such as \emph{f-bounded polymorphism}~\cite{}, \emph{variance
    annotations}~\cite{}, \emph{wildcards}~\cite{} and/or \emph{higher-kinded
    types}~\cite{} to achieve type-safety. Sadly, the amount of
  type-annotations, combined with the lack of understanding of these
  features, usually deters programmers from using such designs.

\begin{comment}
Motivated by the insights gained in previous work, this paper presents 
a minimal core calculus that addresses current limitations and
provides a better foundational model for statically typed
delegation-based OOP? We show that Object Algebras fit nicely in this
model. 
\end{comment}

This paper presents System \name: an extension of System F~\cite{}
with intersection types and a merge operator~\cite{}.  
The goal of System \name is to study the \emph{minimal} foundational language
constructs that are needed to support various extensible designs,
while at the same time addressing the limitations of existing OOP
languages. To address the lack of good object-level composition
mechanisms, System \name uses the merge operator to allow dynamic
composition of values/objects. Moreover, in System \name (type-level)
extension is independent of subtyping, and it is possible for an
extension to be a supertype of a base object type.


Technically speaking System \name is mainly inspired by the work of
Dundfield~\cite{}.  Dundfield shows how to model a simply typed
calculus with intersection types and a merge operator. The presence of
a merge operator adds significant expressiveness to the language,
allowing encodings for many other language constructs as syntactic
sugar. System \name differs from Dundfield's work in a few
ways. Firstly it adds parametric polymorphism and formalizes a
extension for records to support a basic form of objects. Secondly,
the elaboration semantics into System F is done directly from the
source calculus with subtyping. In contrast Dunfield has an additional
step which eliminates subtyping.  Finally a non-technical difference
is that System \name is aimed at studying issues of OOP languages and
extensibility, whereas Dunfield's work was aimed at Functional
Programming and he did not consider applications to extensibility.
Like many other foundational formal models for OOP (for
example~\cite{}), System \name is purely functional and it uses
structural typing.

System \name is
formalized and implemented. Furthermore the paper illustrates how
various extensible designs can be encoded in System \name.

\begin{comment}
We present a polymorphic calculus containing intersection types and records, and show
how this language can be used to solve various common tasks in functional
programming in a nicer way.Intersection types provides a power mechanism for functional programming, in
particular for extensibility and allowing new forms of composition.

Prototype-based programming is one of the two major styles of object-oriented
programming, the other being class-based programming which is featured in
languages such as Java and C\#. It has gained increasing popularity recently
with the prominence of JavaScript in web applications. Prototype-based
programming supports highly dynamic behaviors at run time that are not possible
with traditional class-based programming. However, despite its flexibility,
prototype-based programming is often criticized over concerns of correctness and
safety. Furthermore, almost all prototype-based systems rely on the fact that
the language is dynamically typed and interpreted.
\end{comment}

In summary, the contributions of this paper are:

\begin{itemize}

\item {\bf A Minimal Core Language for Extensibility:} This paper
  identifies a minimal core language, System \name, capable of
  expressing various extensibility designs in the literature.
  System \name also addresses limitations of existing OOP
  languages that complicate extensible designs. 
  
\item {\bf Formalization of System \name:} An elaboration semantics of
  System \name into System F is given, and type-soundness is proved.

\item {\bf Encodings of Extensible Designs:} Various encodings of
  extensible designs into System \name, including \emph{Object
    Algebras} and \emph{Modular Visitors}. 

\item {\bf Implementation and Examples:} An implementation of an
  extension of System \name, as well as the examples presented in the
  paper, are publicly available. 

\begin{comment}

\item{elaboration typing rules which given a source expression with intersection
    types, typecheck and translate it into an ordinary F term. Prove a type
    preservation result: if a term $ e $ has type $ \tau $ in the source language,
    then the translated term $ \image e $ is well-typed and has type $ \image \tau $ in the
    target language.}

\item{present an algorithm for detecting incoherence which can be very important
    in practice.}

\item{explores the connection between intersection types and object algebra by
    showing various examples of encoding object algebra with intersection
    types.}

\end{comment}

\end{itemize}

\begin{comment}
\subsection{Other Notes}

finitary overloading: yes
but have other merits of intersection been explored?

-- Compare Scala:
-- merge[A,B] = new A with B

-- type IEval  = { eval :  Int }
-- type IPrint = { print : String }

-- F[\_]
\end{comment}
\section{Overview} \label{sec:overview}

\joao{review this paragraph once the section is finished}
This section introduces \name and its support for intersection types,
parametric polymorphism and the merge operator. 
It then discusses the issue of coherence and shows how the notion of disjoint
intersection types and disjoint quantification achieve a coherent semantics.

Note that this section uses some syntactic sugar, as well as standard
programming language features, to illustrate the various concepts in \name. 
Although the minimal core language that we formalize in
Section~\ref{sec:fi} does not present all such features, our implementation
supports them.

\subsection{Intersection Types and the Merge Operator}
%%\subsection{Intersection Types, Merge and Polymorphism in \namedis}
%\bruno{We should't simply copy and paste the text from the previous
%  paper. We should try to at least rephrase some things.}

Intersection types date back as early as Coppo et al.'s work~\cite{coppo1981functional}. 
Since then various researchers have studied intersection types, and some languages have 
adopted them in one form or another.
%However, as we shall see in
%Section~\ref{subsec:incoherence}, it also introduces difficulties. In what follows
%intersection types and the merge operator are informally introduced.

\paragraph{Intersection types.}
The intersection of type $A$ and $B$ (denoted as \lstinline{A & B} in
\name) contains exactly those values
which can be used as either values of type $A$ or of type $B$. 
For instance, consider the following program in \name:

\begin{lstlisting}
let x : Int & Char = (*$ \ldots $*) in -- definition omitted
let idInt (y : Int) : Int = y in
let idChar (y : Char) : Char = y in
(idInt x, idChar x)
\end{lstlisting}
\bruno{I had comment about not having the same text. I should have
  perhaps have been a bit more clear and say that having the same code
is ok, it is just the text that needs a little rewriting. In fact now
the code examples are worst: identity functions are pretty boring and
they don't really illustrate the point.}

\noindent If a value \lstinline{x} has type \lstinline{Int & Char} then
\lstinline{x} can be used anywhere where either values of type \lstinline{Int} or a 
\lstinline{Char} are expected. 
This means that, in the program above
the functions \lstinline{idInt} and \lstinline{idChar} -- the
identity functions on integers and characters, respectively -- 
both accept \lstinline{x} as an argument. 

\paragraph{Merge operator.}
The previous program deliberately ommitted the introduction of values of an
intersection type. 
There are many variants of intersection types in the literature. 
Our work follows a particular formulation, where
intersection types are introduced by a \emph{merge operator}. \joao{add citations?}
As Dunfield~\cite{dunfield2014elaborating} has argued a merge operator adds considerable
expressiveness to a calculus. 
The merge operator allows two values to be merged in a single intersection type. 
For example, an implementation of \lstinline{x} is constructed in \name as follows:

\begin{lstlisting}
let x : Int & Char = 1,,'c' in (*$ \ldots $*)
\end{lstlisting}
\noindent In \name (following Dunfield's notation), the
merge of two values $v_1$ and $v_2$ is denoted as $v_1 ,, v_2$.

\paragraph{Merge vs Pairs}
The significant difference between intersection types with a
merge operator and regular pairs is in the elimination construct. 
With pairs there are explicit eliminators (\lstinline{fst} and
\lstinline{snd}), and these eliminators must be used to extract the
components of the right type.
With intersection types and a merge operator, elminators are implicit in the language,
meaning no uses of projection functions are necessary.

%\paragraph{Merge operator and pairs.}
%The merge operator is similar to the introduction construct on pairs.
%An analogous implementation of \lstinline{x} with pairs would be:

%\begin{lstlisting}
%let xPair : (Int, Char) = (1, 'c') in (*$ \ldots $*)
%\end{lstlisting}

% For example, in order to use
%\lstinline{idInt} and \lstinline{idChar} with pairs, we would need to
%write a program such as:

%\begin{lstlisting}
%(idInt (fst xPair), idChar (snd xPair))
%\end{lstlisting}

%\noindent In contrast the elimination of intersection types is done
%implicitly, by following a type-directed process. For example,
%when a value of type \lstinline{Int} is needed, but an intersection
%of type \lstinline{Int & Char} is found, the compiler uses the
%type system to extract the corresponding value.

\subsection{Coherence and Disjointness}
\label{subsec:coherence}
Coherence is a desirable property for a semantics. 
A semantics is said to be coherent if any \emph{valid program} has exactly one
meaning~\cite{reynolds1991coherence} (that is, the semantics is not ambiguous).
Unfortunately the implicit nature of elimination for intersection
types built with a merge operator can lead to incoherence,
This is due to intersections with overlapping types, as in
$\tyint \inter \tyint$.
For example, the result of the program (\lstinline$(1,,2) : Int$)
can be either \lstinline$1$ or \lstinline$2$, depending on the implementation 
of the language.

One option to restore coherence is to reject programs which may have
multiple meanings.
The \oldname \joao{add reference} system -- a calculus with
a simply-typed calculus with intersection types and 
a merge operator, in the same flavour of Dunfield's -- solves this problem
by using the concept of disjoint intersections. 

\begin{comment}
If both results are accepted, we say that the semantics is
\emph{incoherent}: there are multiple possible meanings for the same
valid program. 
Dunfield's calculus~\cite{dunfield2014elaborating} is incoherent and accepts the
program above.\bruno{Well this text needs a significant revision
  because now our ICFP paper largely solves this problem. So you want,
at some point, to summarise the key result of the ICFP paper: we know
how todo coherent intersection types for a simply typed calculus. 
Then you want to setup the problem for this paper: but how about
polymorphism? what are the challenges, why is it hard?}
\end{comment}

\paragraph{Disjoint intersections}
%Of course, when rejecting programs it is important
%not to be too conservative, and reject too many programs which are
%actually coherent.
The incoherence problem with the expression $1 \mergeop 2$
happens because there are two overlapping integers in the merge. 
Generally speaking, if both terms can be assigned some type $C$
then both of them can be chosen as the meaning of the merge,
which in its turn leads to multiple meanings of a term.
Thus a natural option is to forbid such overlapping
values of the same type in a merge.

This is precisely the approach taken in \oldname: a merge can only be composed of
two values as long as their types are \emph{disjoint}.  
Disjointness is a binary relation between two types, defined for any types which
do not contain any overlapping types.
It can be specified as: given any two types, they are disjoint if there does not 
exist any common super-type.

\begin{comment}
More formally, the notion of disjointness can be specified as follows:
\bruno{I think we should avoid presenting the specification, since we
  do not have one for this paper. We can refer to this in the related
  work.}

\begin{definition}[Disjointness]
  Given two types $A$ and $B$, two types are disjoint
  (written $A \disjoint B$) if there is no type $C$ such that both $A$ and $B$ are
  subtypes of $C$:
  \[A \disjoint B \equiv \not\exists C.~A \subtype C \wedge B \subtype C\]
\end{definition}
\end{comment}

With this concept of disjointness in mind, it is easy to verify that the previous example 
will no longer be accepted since \lstinline$((1,,2) : Int)$ is no longer well-typed!
The merge operator requires two types to be disjoint in order to type-check it into
an intersection.
In the example \lstinline$(1,,2)$ is rejected by the compiler, since \lstinline$Int$ is not
disjoint with \lstinline$Int$.
In other words, there exists a super-type of both \lstinline$Int$ and \lstinline$Int$, 
namely \lstinline$Int$ itself.
This result can be generalized and \oldname has shown that it can lead to a coherent calculus. 
Although this is a promising result, the question remains: is it possible to incorporate 
parametric polymorphism in such calculus, while retaining coherence?

%\oldname not only provided a specification for disjointness but also an algorithmic version
%of it, and proved that both versions are equivalent.
%This turned disjointness as a concept which is both easy to understand and easy to implement. 

\begin{comment}
\subsection{Top-like types}\bruno{You are spending too much time to
  get to the point. When writting a paper you want to get to the point
  (what's the problem) 
as fast as possible. So you should have enough background to understand
the paper, but keep that background minimal. Look at the ICFP paper: 
we got to Section 2.2 (the problem) in less than a column.
I think we don't need to
cover top-like types here. They are not essential. }
The \oldname calculus also showed how to extend the type system with a type $\top$, the supertype 
of all types.
Since introducing $\top$ leads to a useless disjointness specification (i.e. no type is disjoint to
any other type) and introduces some ambiguity because $\top \subtype \top \inter \top$ and
$\top \inter \top \subtype \top$.
Therefore, the specification was changed to the following:

\begin{definition}[$\top$-disjointness]
  Two types $A$ and $B$ are disjoint
  (written $A \disjoint B$) if the following two conditions are satisfied:
\begin{enumerate}
  \item $(\text{not}~\toplike{A})~\text{and}~(\text{not}~\toplike{B}) $
  \item $\forall C.~\text{if}~ A \subtype C~\text{and}~B \subtype C~\text{then}~\toplike{C}$
\end{enumerate}
\end{definition}
The unary relation $\toplike{\cdot}$ represents the so-called top-like types, which are types that resemble 
$\top$.
This set of types includes $\top$ itself, intersections composed of other 
top-like types (i.e. $\top \inter \top$), and pre-top-types, which are functions that have
$\top$ as their co-domain (i.e. $\tyint \to \tychar \to \top$).
\end{comment}

\subsection{Parametric Polymorphism}
Unfortunately, adding parametric polymorphism is non-trivial.
A naive attempt to add polymorphism would consist in introducing a forall type, type variables, and a 
big lambda at term level.
Type variables can be assumed as disjoint to any other type, as a starting point.
Now consider the attempt to write the following polymorphic function in such system (we will use
uppercase Latin letters to denote type variables): 
\begin{lstlisting}
let fst A B (x: A & B) : A = (x : A) in (*$ \ldots $*)
\end{lstlisting}
The \code{fst} function is supposed to extract a value of type
\lstinline{A} from the merge value $x$ (of type \lstinline{A&B}). 
This function is problematic: when
\lstinline{A} and \lstinline{B} are instantiated to non-disjoint
types, then uses of \lstinline{fst} may lead to incoherence.
For example, consider the following use of \lstinline{fst}:
\begin{lstlisting}
fst Int Int (1,,2)
\end{lstlisting}
\noindent This program is clearly incoherent as both
$1$ and $2$ can be extracted from the merge and still match the type
of the first argument of \lstinline{fst}.

\paragraph{Biased choice breaks equational reasoning.} 
At first sight, one option
to workaround the incoherence issue would be to bias the type-based merge lookup
to the left or to the right. %(as discussed in Section~\ref{subsec:incoherence}). 
However, biased choice is
very problematic when parametric polymorphism is present in the language.
To see the issue, suppose we chose to always pick the
rightmost value in a merge when multiple values of same type exist.
Intuitively, it would appear that the result of the use of
\lstinline{fst} above is $2$. 
Indeed simple equational reasoning seems to validate such result:
\begin{lstlisting}
   fst Int Int (1,,2)
(*$ \rightsquigarrow $*) ((fun z (*$ \to $*) z) : Int (*$ \to $*) Int) (1,,2) -- (* \textnormal{By the definition of \code{fst}} *)
(*$ \rightsquigarrow $*) ((fun z (*$ \to $*) z) : Int (*$ \to $*) Int) 2      -- (* \textnormal{Right-biased coercion} *)
(*$ \rightsquigarrow $*) 2                                  -- (* \textnormal{By $\beta$-reduction} *)
\end{lstlisting}

However (assuming a straightforward implementation of right-biased
choice) the result of the program would be 1! The reason for this has
todo with \emph{when} the type-based lookup on the merge happens. In
the case of \lstinline{fst}, lookup is triggered by a coercion
function inserted in the definition of \lstinline{fst} at
compile-time.
In the definition of \lstinline$fst$ all it is known is that a
value of type $A$ should be returned from a merge with an intersection
type $A\&B$.  
Clearly the only type-safe choice to coerce the value of type $A\&B$ into $A$ is to
take the left component of the merge. 
This works perfectly for merges
such as \lstinline$(1,,'c')$, where the types of the first and second components
of the merge are disjoint. 
For the merge \lstinline$(1,,'c')$, if a integer lookup
is needed, then \lstinline$1$ is the rightmost integer, which is consistent with the
biased choice. 
Unfortunately, when given the merge \lstinline$(1,,2)$ the
left-component \lstinline$1$ is also picked up, even though in this case \lstinline$2$
is the rightmost integer in the merge. 

The subtle interaction of polymorphism and type-based lookup
means that equational reasoning is broken.
In the equational reasoning steps above, doing apparently correct
substitutions lead us to a wrong result. 
This is a major problem for biased choice and a reason to dismiss it as a possible implementation
choice for \name.

\paragraph{A more conservative attempt.}
Another attempt at restoring coherence can be to forbid type variables inside
intersections (i.e. type variables are not disjoint to any type).
This conservative approach would solve the problem of coherence, but it would also greatly 
restrict the expressiveness of the resulting language.
For example, the function $fst$ defined above, would no longer be accepted by the system.
In fact, parametric polymorphism and intersection types could only be mixed in a very limited manner -
as long as variables do not reside under intersections - and this is arguably a useful improvement
in respect to other standard type systems, such as System $F$.

\begin{comment}
\subsection{Disjoint Polymorphism}
A more liberal solution,which enables the combination of type
variables and intersection types, is disjoint polymorphism.
Disjoint polymorphism assign constraints to each type variable, which
allows delaying the check for disjointness until type application/instantiation. 
\end{comment}

\subsection{Disjoint Quantification}
To avoid being overly conservative, while still retaining coherence in the
presence of parametric polymorphism and intersection types, \name uses
an extension to universal quantification called \emph{disjoint quantification}.
Inspired by bounded quantification~\cite{Cardelli:1994},
where a type variable is constrained by a type bound, disjoint quantification 
allows a type variable to be constrained so that it is disjoint with a given type. 
With disjoint quantification a variant of the program $fst$, which
is accepted by \name, would be written as:
\begin{lstlisting}
let fst A ((*$ \highlight {B *~A} $*)) (x: A & B) = (x : A)
in (*$ \ldots $*)
\end{lstlisting}
The small change is in the declaration of the type parameter $B$. The notation
$B*A$ means that in this program the type variable $B$ is constrained so that
it can only be instantiated with any type disjoint to $A$.
This ensures that the
merge denoted by $x$ is disjoint for all valid instantiations of $A$ and $B$.
In other words, only coherent uses of \lstinline$fst$ will be accepted.
For example, the following use of \lstinline$fst$:
\begin{lstlisting}
fst Int Char (1,,'c')
\end{lstlisting}
is accepted since \lstinline$Int$ and \lstinline$Char$ are disjoint, thus satisfying the constraint
on the second type parameter of \lstinline$fst$.
Furthermore, problematic uses of \lstinline$fst$ are rejected. 
However, the following use of \lstinline$fst$:
\begin{lstlisting}
fst Int Int (1,,2)
\end{lstlisting}
is rejected because \lstinline$Int$ is not disjoint with \lstinline$Int$, thus failing to satisfy the
disjointness constraint on the second type parameter of \lstinline$fst$.

\paragraph{Empty constraint}
Even though disjoint quantification solves the problem of coherence, there is still one detail 
that needs further justification.
The reader might have noticed how we ommitted the disjointness constraint of 
the type variable $A$ in the \lstinline$fst$ function.
This actually means that $A$ should be associated with the empty contraint,
which raises the question: which type should be used to represent such empty constraint?
Or, in other words, which type is disjoint to every other type? 
It is obvious that this type should be one of the bounds of the subtyping lattice: either $\bot$ or
$\top$.

The essential intuition here is that the more specific a type in the subtyping relation is, the less types
exist that are disjoint to it.
For example, $\tyint$ is disjoint with all types except the intersection that contain $\tyint$, $\tyint$
itself, and $\bot$; while $\tyint \inter \tychar$ is disjoint to all types that $\tyint$ is, plus the
types disjoint to $\tychar$.
Thus, the more specific a type variable constraint is, the less options we have to instantiate it with.
This reasoning implies that $\top$ should be treated as the empty constraint.
Indeed, in \name, a single type variable $A$ is only syntactic sugar for $A * \top$.
\joao{should we say anything here about this going against our
  previous top-disjointness formulation?}
\bruno{yes, but not here. We can discuss this in later sections when
  discussing the technical details.}
For instance, the type of the identity function in System $F$ that reads $\forall A. A \to A$ is 
equivalent to the \name's type $\forall (A * \top). A \to A$.

%Let us assume two distinct interpretations for the subtyping relation.
%Given two types $A$ and $B$, we can say relation is either:
%\begin{enumerate}
%\item Subset relation ($A \subseteq B$): every element of type $A$ is also of type $B$; or
%\item Coercion ($\exists t. t : A \to B$): every element of type $A$ can be coerced into type $B$.
%\end{enumerate}

\begin{comment}
First, if we consider our subtyping lattice as unbounded (i.e. no $\top$ and no $\bot$), then we have that
disjointness is covariant with respect to the subtyping relation.
More formally:
\[ \inferrule {\jwf \Gamma {A \disjoint B} \\ B \subtype C }
              {\jwf \Gamma {A \disjoint C}} \]

To illustrate this, take $A$ as $\tyint$ and $B$ as $\tybool \inter \tychar$.
This lemma states that every supertype of $\tybool \inter \tychar$, namely $\tybool$, $\tychar$ and 
$\tybool \inter \tychar$ itself are also disjoint with $\tyint$.
Coming back to a bounded subtyping lattice, let us now consider both bounds. 
If some type $A$ were to be disjoint with $\bot$, then by the lemma above $A$ will be disjoint to
virtually any type.
This means that, if $A$ is a type variable, then the possible types that it can be instantiated with
are the ones which are disjoint with every other type (otherwise the lemma above will no longer hold).
Clearly $\bot$ is not a suitable option,  
In other words, we can think of $\bot$ as the type as specific as the infinite intersection.
Conversely, $\top$ can be thought as specific - or rather, as general - as the 0-ary intersection.
\end{comment}

%\joao{do we want to say this here?:}
%However, the previous specification of $\top$-disjointness does not reflect this, since it states that $\top$ 
%is not disjoint to any other type, not even to itself.
%Thus, in this paper, we slightly changed the notion of $\top$-disjointness to read instead:

%\begin{definition}[$\top$-disjointness]
%  Two types $A$ and $B$ are disjoint (written $A \disjoint B$) if:
%  \[\forall C.~\text{if}~ A \subtype C~\text{and}~B \subtype C~\text{then}~\toplike{C}\]
%\end{definition}
%\joao{say that we manage to retain coherence wrt to the simply typed version?}

\subsection{Stability of Substitutions}
From the technical point of view, the main challenge in the design of
System \name is that, in general, types are not stable under
substitution. This contrasts, for example, with System $F$ where types
are stable under substitution. That is in System $F$ the following
property holds: 
\bruno{state usual property here}

In \name if a type variable $A$ is substituted in a type $T_1$, for a type $T_2$ 
(written $\subst {T_2} A {T_1}$), where $T_1$ and $T_2$ are well-formed, the resulting type might be ill-formed. 
To understand why, recall the previous example: 
\begin{lstlisting}
fst Int Int (1,,2)
\end{lstlisting}
The type signature of \lstinline$fst$ may be read as $\forall A. (B * A) . (A \inter B) \to A$.
An application to the type $\tyint$ will lead to instantiation of the variable $A$, leading to the type
$\forall (B * \tyint). (\tyint \inter B) \to \tyint$.
Now, the second $\tyint$ application is problematic, since instantiating $B$ with $\tyint$ will lead to
the ill-formed type $(\tyint \inter \tyint) \to \tyint$.
However, from this example it is easy to see that all types which are not problematic are exactly the
the ones disjoint with $A$.
This paper shows how a weaker version of the usual type substituition stability still holds, 
namely by requiring that the type varible's disjointness constraint is compatible
with the type as target of the instantiation. 

\begin{comment}
This section introduces \namedis and its support for intersection types,
parametric polymorphism and the merge operator. It then discusses
the issue of coherence and shows how the notion of disjoint
intersection types and disjoint quantification achieve a coherent semantics.
%Finally we illustrate the expressive power of \namedis by encoding
%extensible type-theoretic encodings of datatypes.

Note that this section uses some syntactic sugar, as well as standard
programming language features, to illustrate the various concepts in
\namedis. Although the minimal core language that we formalize in
Section~\ref{sec:fi} does not present all such features, our implementation
supports them.

%\bruno{Need to type-check the programs!}

%\begin{comment}
%It then shows that,
%with unrestricted intersection types, the system
%lacks \emph{coherence}. This motivates the introduction of
%disjoint intersection types and extending universal quantification to
%disjoint quantification, which is enough to ensure coherence.
%\end{comment}

\subsection{Intersection Types and the Merge Operator}
%%\subsection{Intersection Types, Merge and Polymorphism in \namedis}

Intersection types date back as early as Coppo et
al.'s work~\cite{coppo1981functional}. Since then various researchers have
studied intersection types, and some languages have adopted them in one
form or another.
%However, as we shall see in
%Section~\ref{subsec:incoherence}, it also introduces difficulties. In what follows
%intersection types and the merge operator are informally introduced.

\paragraph{Intersection types.}
The intersection of type $A$ and $B$ (denoted as \lstinline{A & B} in
\namedis) contains exactly those values
which can be used as either values of type $A$ or of type $B$. For instance,
consider the following program in \namedis:

\begin{lstlisting}
let x : Int & Char = (*$ \ldots $*) in -- definition omitted
let idInt (y : Int) : Int = y in
let idChar (y : Char) : Char = y in
(idInt x, idChar x)
\end{lstlisting}

\noindent If a value \lstinline{x} has type \lstinline{Int & Char} then
\lstinline{x} can be used as an integer or as a character. Therefore,
\lstinline{x} can be used as an argument to any function that takes
an integer as an argument, or any
function that take a character as an argument. In the program above
the functions \lstinline{idInt} and \lstinline{idChar} are the
identity functions on integers and characters, respectively.
Passing \lstinline{x} as an argument to either one (or both) of the
functions is valid.

\paragraph{Merge operator.}
In the previous program we deliberately did not show how to introduce values of an
intersection type. There are many variants of intersection types
in the literature. Our work follows a particular formulation, where
intersection types are introduced by a \emph{merge operator}.
As Dunfield~\cite{dunfield2014elaborating} has argued a merge operator adds considerable
expressiveness to a calculus. The merge operator allows
two values to be merged in a single intersection type. For example, an
implementation of \lstinline{x} is constructed in \namedis as follows:

\begin{lstlisting}
let x : Int & Char = 1,,'c' in (*$ \ldots $*)
\end{lstlisting}

\noindent In \namedis (following Dunfield's notation), the
merge of two values $v_1$ and $v_2$ is denoted as $v_1 ,, v_2$.

\paragraph{Merge operator and pairs.}
The merge operator is similar to the introduction construct on pairs.
An analogous implementation of \lstinline{x} with pairs would be:

\begin{lstlisting}
let xPair : (Int, Char) = (1, 'c') in (*$ \ldots $*)
\end{lstlisting}

\noindent The significant difference between intersection types with a
merge operator and pairs is in the elimination construct. With pairs
there are explicit eliminators (\lstinline{fst} and
\lstinline{snd}). These eliminators must be used to extract the
components of the right type. For example, in order to use
\lstinline{idInt} and \lstinline{idChar} with pairs, we would need to
write a program such as:

\begin{lstlisting}
(idInt (fst xPair), idChar (snd xPair))
\end{lstlisting}

\noindent In contrast the elimination of intersection types is done
implicitly, by following a type-directed process. For example,
when a value of type \lstinline{Int} is needed, but an intersection
of type \lstinline{Int & Char} is found, the compiler uses the
type system to extract the corresponding value.

\subsection{Incoherence}\label{subsec:incoherence}
Unfortunatelly the implicit nature of elimination for intersection
types built with a merge operator can lead to incoherence.
The merge operator combines two terms, of type $A$ and $B$
respectively, to form a term of type $A \inter B$. For example,
$1 \mergeop `c'$ is of type $\tyint \inter \tychar$. In this case, no
matter if $1 \mergeop `c'$ is used as $\tyint$ or $\tychar$, the result
of evaluation is always clear. However, with overlapping types, it is
not straightforward anymore to see the result. For example, what
should be the result of this program, which asks for an integer out of
a merge of two integers:
\begin{lstlisting}
(fun (x: Int) (*$ \to $*) x) (1,,2)
\end{lstlisting}
Should the result be \lstinline$1$ or \lstinline$2$?

If both results are accepted, we say that the semantics is
\emph{incoherent}: there are multiple possible meanings for the same
valid program. Dunfield's calculus~\cite{dunfield2014elaborating} is incoherent and accepts the
program above.

\paragraph{Getting around incoherence: biased choice.}
In a real implementation of Dunfield calculus a choice has to be made
on which value to compute. For example, one potential option is to
always take the left-most value matching the type in the
merge. Similarly, one could always take the right-most
value matching the type in the merge. Either way, the meaning
of a program will depend on a biased implementation choice,
which is clearly unsatisfying from the theoretical point of view
(although perhaps acceptable in practice).
Moreover, even if we accept a particular biased choice as
being good enough, the approach cannot be easily
extended to systems with parametric polymorphism, as we illustrate
in Section~\ref{subsec:polymorphism}.

\subsection{Restoring Coherence: Disjoint Intersection Types}\label{sec:restoring}
Coherence is a desirable property for a semantics. A semantics is said
to be coherent if any \emph{valid program} has exactly one
meaning~\cite{reynolds1991coherence} (that is, the semantics is not ambiguous).
One option to restore coherence is to reject programs which may have
multiple meanings.
%Of course, when rejecting programs it is important
%not to be too conservative, and reject too many programs which are
%actually coherent.
Analyzing the expression $1 \mergeop 2$, we can see that the reason
for incoherence is that there are multiple, overlapping, integers in the
merge. Generally speaking, if both terms can be assigned some type $C$,
both of them can be chosen as the meaning of the merge,
which leads to multiple meanings of a term.
Thus a natural option is to try to forbid such overlapping
values of the same type in a merge.

This is precisely the approach taken in \namedis. \namedis requires that the
two types of in intersection must be \emph{disjoint}.  However,
although disjointness seems a natural restriction to impose on
intersection types, it is not obvious to formalize it. Indeed Dunfield
has mentioned disjointness as an option to restore coherence, but he
left it for future work due to the non-triviality of the approach.

\paragraph{Searching for a definition of disjointness.}
The first step towards disjoint intersection types is to come up
with a definition of disjointness. A first attempt at such definition would
be to require that, given two types $A$ and $B$, both types are not
subtypes of each other. Thus, denoting disjointness as $A * B$, we would have:
\[A * B \equiv A \not<: B \wedge B \not<: A\]
At first sight this seems a reasonable definition and it does prevent
merges such as \lstinline{1,,2}. However some moments of thought are enough to realize that
such definition does not ensure disjointness. For example, consider
the following merge:

\begin{lstlisting}
((1,,'c') ,, (2,,True))
\end{lstlisting}

\noindent This merge has two components which are also intersection
types. The first component (\lstinline{(1,,'c')}) has type $\tyint \inter
\tychar$, whereas the second component (\lstinline{(2 ,, True)}) has type
$\tyint \inter \tybool$. Clearly,
\[ \tyint \inter \tychar \not \subtype \tyint \inter \tybool \wedge \tyint \inter \tybool \not \subtype \tyint \inter \tychar \]
Nevertheless the following program still leads to
incoherence:
\begin{lstlisting}
(fun (x: Int) (*$ \to $*) x) ((1,,'c'),,(2,,True))
\end{lstlisting}
as both \lstinline{1} or \lstinline{2} are possible outcomes
of the program. Although this attempt to define disjointness failed,
it did bring us some additional insight: although the types of the two
components of the merge are not subtypes of each other, they share
some types in common.

\paragraph{A proper definition of disjointness.} In order for two types
to be trully disjoint, they must not have any subcomponents sharing
the same type. In a system with intersection types this can be ensured
by requiring the two types do not share a common supertype. The
following definition captures this idea more formally.

\begin{definition}[Disjointness]
  Given two types $A$ and $B$, two types are disjoint
  (written $A \disjoint B$) if there is no type $C$ such that both $A$ and $B$ are
  subtypes of $C$:
  \[A \disjoint B \equiv \not\exists C.~A \subtype C \wedge B \subtype C\]
\end{definition}

\noindent This definition of disjointness prevents the problematic
merge. Since $Int$ is a common supertype of both $Int \& Char$ and
$Int \& Bool$, those two types are not disjoint.

\namedis's type system only accepts programs that use disjoint
intersection types. As shown in Section~\ref{sec:disjoint} disjoint intersection
types will play a crutial rule in guaranteeing that the semantics is coherent.

\subsection{Parametric Polymorphism and Intersection Types}\label{subsec:polymorphism}
Before we show how \namedis extends the idea of disjointness to parametric
polymorphism, we discuss some non-trivial issues that arise from
the interaction between parametric polymorphism and intersection types.
%The interaction between parametric polymorphism and
%intersection types when coherence is a goal is non-trivial.
%In particular biased choice .
%The key challenge is to have a type
%system that still ensures coherence, but at the same time is not too
%restrictive in the programs that can be accepted.
%\begin{comment}
%Dunfield~\cite{} provides a
%good illustrative example of the issues that arise when combining
%disjoint intersection types and parametric polymorphism:
%\[\lambda x. {\bf let}~y = 0 \mergeop x~{\bf in}~x\]
%\end{comment}
Consider the attempt to write
the following polymorphic function in \namedis (we use
uppercase Latin letters to denote type variables):
\begin{lstlisting}
let fst A B (x: A & B) = (fun (z:A) (*$ \to $*) z) x in (*$ \ldots $*)
\end{lstlisting}
The
\code{fst} function is supposed to extract a value of type
(\lstinline{A}) from the merge value $x$ (of type \lstinline{A&B}). However
this function is problematic.  The reason is that when
\lstinline{A} and \lstinline{B} are instantiated to non-disjoint
types, then uses of \lstinline{fst} may lead to incoherence.
For example, consider the following use of \lstinline{fst}:
\begin{lstlisting}
fst Int Int (1,,2)
\end{lstlisting}
\noindent This program is clearly incoherent as both
$1$ and $2$ can be extracted from the merge and still match the type
of the first argument of \lstinline{fst}.

\paragraph{Biased choice breaks equational reasoning.} At first sight, one option
to workaround the issue incoherence would be to bias the type-based merge lookup
to the left or to the right (as discussed in
Section~\ref{subsec:incoherence}). Unfortunately, biased choice is
very problematic when parametric polymorphism is present in the language.
To see the issue, suppose we chose to always pick the
rightmost value in a merge when multiple values of same type exist.
Intuitively, it would appear that the result of the use of
\lstinline{fst} above is $2$. Indeed simple equational reasoning
seems to validate such result:
\begin{lstlisting}
   fst Int Int (1,,2)
(*$ \rightsquigarrow $*) (fun (z: Int) (*$ \to $*) z) (1,,2) -- (* \textnormal{By the definition of \code{fst}} *)
(*$ \rightsquigarrow $*) (fun (z: Int) (*$ \to $*) z) 2      -- (* \textnormal{Right-biased coercion} *)
(*$ \rightsquigarrow $*) 2                          -- (* \textnormal{By $\beta$-reduction} *)
\end{lstlisting}

However (assumming a straightforward implementation of right-biased
choice) the result of the program would be 1! The reason for this has
todo with \emph{when} the type-based lookup on the merge happens. In
the case of \lstinline{fst}, lookup is triggered by a coercion
function inserted in the definition of \lstinline{fst} at
compile-time.
In the definition of \lstinline$fst$ all it is known is that a
value of type $A$ should be returned from a merge with an intersection
type $A\&B$.  Clearly the only type-safe choice to coerce the value of
type $A\&B$ into $A$ is to
take the left component of the merge. This works perfectly for merges
such as \lstinline$(1,,'c')$, where the types of the first and second components
of the merge are disjoint. For the merge \lstinline$(1,,'c')$, if a integer lookup
is needed, then \lstinline$1$ is the rightmost integer, which is consistent with the
biased choice. Unfortunately, when given the merge \lstinline$(1,,2)$ the
left-component (\lstinline$1$) is also picked up, even though in this case \lstinline$2$
is the rightmost integer in the merge. Clearly this is inconsistent
with the biased choice!

Unfortunately this subtle interaction of polymorphism and type-based lookup
 means that equational reasoning is broken!
In the equational reasoning steps above, doing apparently correct
substitutions lead us to a wrong result. This is a major problem for
biased choice and a reason to dismiss it as a possible implementation
choice for \namedis.

\paragraph{Conservatively rejecting intersections.}
To avoid incoherence, and the issues of biased choice, another option
is simply to reject programs where the
instantiations of type variables may lead to incoherent programs.
In this case the definition of \lstinline$fst$ would be rejected, since there
are indeed some cases that may lead to incoherent programs.
Unfortunately this is too restrictive and prevents many useful
programs using both parametric polymorphism and intersection types.
In particular, in the case of \lstinline{fst}, if the two type
parameters are used with two disjoint intersection
types, then the merge will not lead to ambiguity.

In summary, it seems hard to have parametric polymorphism, intersection
types and coherence without being overly conservative.


%\begin{comment}
%\subsection{Intersection Types in Existing Languages}
%
%What is an intersection type? The intersection of types $A$ and $B$
%contains exactly those values which can be used as either of type $A$
%or of type $B$.  Just as not all intersection of sets are nonempty,
%not all intersections of types are inhabited.  For example, the
%intersection of a base type $\tyint$ and a function type
%$\tyint \to \tyint$ is not inhabited.\bruno{put this text somewhere?}
%
%Since then various researchers have
%studied intersection types, and some languages have adopted in one
%form or another. However, while intersection types are already used
%in various languages, the lack of a merge operator removes
%considerable expressiveness.
%
%
%A number of OO languages, such as
%Java, C\#, Scala, and Ceylon\footnote{\url{http://ceylon-lang.org/}},
%already support intersection types to different degrees. Intersection
%types are particularly relevant for OOP as they can be used to model
%multiple interface inheritance. In Java, for example,
%
%\begin{lstlisting}
%interface AwithB extends A, B {}
%\end{lstlisting}
%
%\noindent introduces a new interface \lstinline{AwithB} that satisfies the interfaces of
%both \lstinline{A} and \lstinline{B}. Arguably such type can be considered as a nominal
%intersection type. Scala takes one step further by eliminating the
%need of a nominal type. For example, given two concrete traits, it is possible to
%use \emph{mixin composition} to create an object that implements both
%traits. Such an object has a (structural) intersection type:
%
%\begin{lstlisting}
%trait A
%trait B
%
%val newAB : A with B = new A with B
%\end{lstlisting}
%
%\noindent Scala also allows intersection of type parameters. For example:
%\begin{lstlisting}
%def merge[A,B] (x: A) (y: B) : A with B = ...
%\end{lstlisting}
%uses the anonymous intersection of two type parameters \lstinline{A} and
%\lstinline{B}. However, in Scala it is not possible to dynamically
%compose two objects. For example, the following code:
%
%\begin{lstlisting}
%// Invalid Scala code:
%def merge[A,B] (x: A) (y: B) : A with B = x with y
%\end{lstlisting}
%
%\noindent is rejected by the Scala compiler. The problem is that the
%\lstinline{with} construct for Scala expressions can only be used to
%mixin traits or classes, and not arbitrary objects. Note that in the
%definition \lstinline{newAB} both \lstinline{A} and \lstinline{B} are
%\emph{traits}, whereas in the definition of \lstinline{merge} the variables
%\lstinline{x} and \lstinline{y} denote \emph{objects}.
%
%This limitation essentially put intersection types in Scala in a second-class
%status. Although \lstinline{merge} returns an intersection type, it is
%hard to actually build values with such types. In essence an
%object-level introduction construct for intersection types is missing.
%As it turns out using low-level type-unsafe programming features such
%as dynamic proxies, reflection or other meta-programming techniques,
%it is possible to implement such an introduction
%construct in Scala~\cite{oliveira2013feature,rendel14attributes}. However, this
%is clearly a hack and it would be better to provide proper language
%support for such a feature.
%
%
%
%
%\paragraph{Parametric polymorphism and intersection types.}
%Both universal quantification and intersection types provide a kind of
%polymorphism. While the former provides parametric polymorphism, the latter
%provides ad-hoc polymorphism. In some systems, parametric polymorphism is
%considered the infinite analog of intersection polymorphism. But in our system
%we do not consider this relationship. \george{Need to argue that why their
%coexistence might be a good thing.} \george{May use the merge example}
%\bruno{Some more examples in following subsections?}
%
%
%To address the limitations of intersection types in languages like
%Scala, \namedis allows intersecting any two terms at run time using a
%\emph{merge} operator (denoted by $ \mergeop $)~\cite{dunfield2014elaborating}.  With the merge
%operator it is trivial to implement the \lstinline{merge} function in \namedis:
%
%\begin{lstlisting}
%let merge[A, B * A] (x : A) (y : B) : A & B = x ,, y in (*$ \ldots $*)
%\end{lstlisting}
%
%\noindent In contrast to Scala's term-level \lstinline{with}
%construct, the operator \lstinline{,,} allows two arbitrary values \lstinline{x}
%and \lstinline{y} to be merged. The resulting type is a \emph{disjoint}
%intersection of the types of  \lstinline{x}
%and \lstinline{y} (\lstinline{A & B} in this case).
%
%\paragraph{Incoherence and parametric Polymorphism}
%We can define a \code{fst} function that extracts the first item of a merged value:
%\[
%\code{fst} \ \alpha \ \beta \ (x : \alpha \inter \beta) = \app {(\lam y \alpha y)} x
%\]
%What should be the result of this program?
%\begin{lstlisting}
%fst Int Int (1,,2)
%\end{lstlisting}
%
%Then we have the following equational reasoning:
%\begin{lstlisting}
%fst Int Int (1,,2) => (\(y : Int). y) (1,,2)
%\end{lstlisting}
%If we favor the second item, the program seems to evaluate to $2$. But in
%reality, the result is $2$. No matter we favor the first or the second item,
%we can always construct a program such that for that program, equational
%reasoning is broken.
%
%Therefore, we require that the two types of an intersection must be not
%overlapping, or \emph{disjoint}, and add this requirement to the well-formedness of types.
%
%A well-formed type is such that given any query type,
%it is always clear which subpart the query is referring to.
%In terms of rules, this notion of well-formedness is almost the same as the one in System $F$
%except for intersection types we require the two components to be disjoint.
%
%With parametric polymorphism, disjointness is harder to determine due to type variables.
%Consider this program:
%\[
%\blam \alpha {\lam x {\alpha \inter \tyint} x}
%\]
%$x$ in the body is of type $\alpha \inter \tyint$ and if $\alpha$ and $\tyint$ are
%disjoint depends on the instantiation of $\alpha$.
%\end{comment}

\subsection{Disjoint Quantification}
To avoid being overly conservative, while still retaining coherence in the
presence of parametric polymorphism and intersection types, \namedis uses
an extension to universal quantification called \emph{disjoint quantification}.
Inspired by
bounded quantification~\cite{Cardelli:1994},
where a type variable is constrained by a type
bound, disjoint quantification allows a type variable to be
constrained so that it is disjoint with a
given type. With disjoint quantification a variant of the program $fst$, which
is accepted by \namedis, would be written as:
\begin{lstlisting}
let fst A ((*$ \highlight {B *~A} $*)) (x: A & B) = (fun (z: A) (*$ \to $*) z) x
in (*$ \ldots $*)
\end{lstlisting}
The small change is in the declaration of the type parameter $B$. The notation
$B*A$ means that in this program the type variable $B$ is constrained so that
it can only be instantiated with any type disjoint to $A$.
This ensures that the
merge denoted by $x$ is disjoint for all valid instantiations of $A$ and $B$.

The nice thing about this solution is that many uses of \lstinline$fst$ are accepted.
For example, the following use of \lstinline$fst$:
\begin{lstlisting}
fst Int Char (1,,'c')
\end{lstlisting}
is accepted since \lstinline$Int$ and \lstinline$Char$ are disjoint, thus satisfying the constraint
on the second type parameter of \lstinline$fst$.
However, problematic uses of \lstinline$fst$ are rejected. For example:
\begin{lstlisting}
fst Int Int (1,,2)
\end{lstlisting}
is rejected because \lstinline$Int$ is not disjoint with \lstinline$Int$, thus failing to satisfy the
disjointness constraint on the second type parameter of \lstinline$fst$.

%\begin{comment}
%Note that there is a nice symmetry between bounded quantification and disjoint quantification.
%In systems with bounded quantification,
%the usual unconstrained quantifier $\for {\alpha} \ldots$
%is a syntactic sugar for $\for {\alpha \subtype \top} \ldots$, and
%$\blam \alpha \ldots$ for $\blam {\alpha \subtype \top} \ldots$.
%In parellel, in our system with disjoint quantification,
%the usual unconstrained quantifier $\for {\alpha} \ldots$
%is a syntactic sugar for $\for {\alpha \disjoint \bot} \ldots$, and
%$\blam \alpha \ldots$ for $\blam {\alpha \disjoint \top} \ldots$.
%The intuition is that since the bottom type is akin to the empty set,
%no other type overlaps with it.\george{Format this paragraph better.}
%\end{comment}
%
%
%\begin{comment}
%With this tool in hand, we can rewrite the program above to:
%\[
%\blam {\alpha \disjoint \tyint} {\lam x {\alpha \inter \tyint} x}
%\]
%
%This program typechecks because while $x$ is of type $\alpha \inter \tyint$,
%and $\alpha$ is disjoint with $\tyint$. Similarly, in the new system,
%the original program no longer typechecks, thus preventing overlapping types.
%\end{comment}
\end{comment}

%%*******************************************************************************
\section{Dynamically Composable Traits} \label{sec:trait}
%*******************************************************************************

\bruno{This section suffers from having examples which are different
  to illustrate the various concepts. A better approach would be to
use related examples (say various things todo with points), to
illustrate the approach.}

As an application of disjoint intersection types, we show how to model a
simple, yet expressive form of dynamically composable
traits~\cite{scharli2003traits} in \name. Traits provide a
mechanism of code reuse in object-oriented programming, that
can be used as an alternative to multiple inheritance.
The interesting aspect about traits is the way conflicts that
typically arise in multiple-inheritance~\cite{} are dealt with.
Instead of trying to automatically resolve conflicts, traits
detect those conflicts and require programmers to explicitly resolve
them. This is where the relation to disjoint intersection types comes
in: the mechanism to detect incoherence of disjoint intersection types
provides us with the mechanism to detect conflicts in traits.

We demonstrate various trait
features in a simple OO language, and then show a straightforward
translation from that language to \name. \george{Need to note the difference
between class-based and prototype-based somewhere.}

\paragraph{Basic Traits.}

A trait is a collection of related methods that characterizes only a specific
perspective of the features of an object. Therefore, compared with
programs uses inheritance, programs using traits usually have a number of small
traits rather than fewer but larger classes. Code reuse with traits is easier
with traits than with classes, since traits are usually shorter and traits can
be \emph{composed}. In fact, trait composition offers a variety of
possibilities: two traits can be ``added'' together (which is an symmetric
operation); methods can be removed from a trait; and trait systems provide
conflict detection, etc.

The first example shows basic trait composition. Many social networking sites
allow users to ``upvote'' a comment and the number of upvotes that comment has
received is also displayed. We would like to separate the logic for upvotes from
comments so that it can be reused in other entities such as posts and sharings.
At a high level, the code below defines a trait, \lstinline$Comment$, which
contains a single method \lstinline$content$ and another trait, \lstinline$Up$,
for tracking the number of upvotes. In the end, we create a single object from
those two traits and test its functionalities.

\begin{lstlisting}
type Comment = { content: () (*$ \to $*) String } in
trait Comment(content: String) { self: Comment (*$ \to $*)
  content() = content
} in

type Up = { upvotes: () (*$ \to $*) Int } in
trait Up(upvotes: Int) { self: Up (*$ \to $*)
  upvotes() = upvotes
} in

let comment = new (Comment("Have fun!"), Up(4))
in console.log(comment.content(), comment.upvotes())
-- Output: "Have fun!" 4
\end{lstlisting}
\bruno{Consider mixing explanation with the source code. Instead of
  presenting the full source and then explaining. Alternatively, make
this a figure.}

At this point the reader may wonder why there are duplicate declarations related
to \lstinline$Comment$ and \lstinline$Up$. In mainstream OO languages such as
Java, a class declaration such as \lstinline$class C { ... }$ does two things at
the same time:

\begin{itemize}
\item Declaring a \emph{template} for creating objects;
\item Declaring a new \emph{type}.
\end{itemize}

\noindent In contrast, trait declarations in this source language only does the
former. Back to our example, the purpose of declaring two types is just to use
them for type annotations of the self reference.  In traits literature, a trait
usually ``requires'' some methods and, based on that,  ``provides'' another set
of methods. In our examples, the type of \lstinline$self$ actually denotes what
methods are required.

A trait can expect parameters, which become in scope in the entire trait body.
For example, the \lstinline$Comment$ trait is parametrized by
\lstinline$content$, and the \lstinline$content$ method does nothing more than
returning the eponymous variable. In comparison, traits in Scala do not allow
taking parameters.

The origin of self references is always explicit. The \lstinline$Comment$ trait
requires that \lstinline$self$ be of type \lstinline$Comment$, which is defined
as a type synonym for a record in the first line. But the name ``self'' is
nothing special. In fact, \lstinline$self$ is just another parameter after the
preceding parameter list, and becomes in scope after the arrow. We could have
named it \lstinline$this$ or even \lstinline$s$, but that is generally
discouraged.

Creating an object is via the \lstinline$new$ keyword, similar to many OO
languages, except for one crucial novelty: we can create an object from multiple
traits. More precisely, the object is created from the \emph{composition} of
those traits. Therefore, we are able to call methods from different traits on a
single object.

\paragraph{Traits with Dependencies.} The following example shows that a trait
can depend on another trait. The norm of a point is a generalized notion of
distance of that point to the origin. In the following example, we provide two
definitions of norm via two traits. Later we compose the definition of point and
norm so that the norm have different meanings. Composition of traits is
anonymous in the sense that there is no need to explicitly give a name to the
composition.

\begin{lstlisting}
type Point = {x: () (*$ \to $*) Int, y: () (*$ \to $*) Int} in
trait Point(x: Int, y: Int) { self: Point (*$ \to $*)
  x() = x
  y() = y
} in
trait EuclideanNorm() { self: Point (*$ \to $*)
  norm() = Math.sqrt(self.x() * self.x() + self.y() * self.y())
} in
trait ManhattanNorm() { self: Point (*$ \to $*)
  norm() = Math.abs(self.x()) + Math.abs(self.y())
} in
console.log(new (Point(3,4) & EuclideanNorm()).norm()) -- This prints 5
console.log(new (Point(3,4) & ManhattanNorm()).norm()) -- This prints 7
\end{lstlisting}

\paragraph{Detecting and Resolving Conflicts in Trait Composition.}
Traits usually supports explicit conflict detection and resolution.
In inheritance, one pattern is for the subclass to override methods defined in the parent.
The trait-based approach analog is excluding a method from a trait.
We show how the mechanism can be modeled in \name.
The following example shows a counter object and how we could extend its
behavior so that it supports reset. First we define a \lstinline$Counter$ as a
type synonym for a record that contains a \lstinline$val$ method, which returns
the current counter value. Next we define a trait \lstinline$Counter$ that
contains two methods. The \lstinline$val$ method just returns the value that is
bound at the parameter of the trait, and \lstinline$incr$ returns a new counter.
\begin{lstlisting}
type Counter = { val: () (*$ \to $*) Int } in
trait Counter(val: Int) { self: Counter (*$ \to $*)
  val() = val
  incr() = new Counter(val + 1)
} in
trait Reset() { self: Counter (*$ \to $*)
  reset() = new Counter(0)
} in
let counter = new (Counter(0) & Reset())
in counter.incr()
\end{lstlisting}

In the above code, even though \lstinline$counter$ has a reset method, after we
call the \lstinline$incr$ method, the resulting object no longer has that.
Therefore, naturally we would like to override the \lstinline$incr$ method
inside \lstinline$Reset$.

\begin{lstlisting}
type Counter = { val: () (*$ \to $*) Int } in
trait Counter(val: Int) { self: Counter (*$ \to $*)
  val() = val
  incr() = new Counter(val + 1)
} in
trait Reset() { self: Counter (*$ \to $*)
  incr() = new (Counter(val + 1) & Reset())
  reset() = new Counter(0)
} in
let counter = new (Counter(0) & Reset())
in counter.incr()
\end{lstlisting}

However the modified code should not typecheck according to the specification of
traits, since both \lstinline$Counter$ and \lstinline$Reset$ contains a
conflicting \lstinline$incr$ method. The code also does not typecheck since it
violates the typing rule of \name. The programmer can resolve the conflict by
excluding the \lstinline$incr$ being overridden using the record exclusion
operator.

\begin{lstlisting}
(* \ldots *)
let counter = new (Counter(0) \ incr & Reset())
in counter.incr()
\end{lstlisting}

\paragraph{Mutual Dependencies.}
The next example, although a little bit contrived, illustrates that when two
traits are composed, any two methods in those two traits can refer to each
other via the self reference, just as if they were inside the same class.

\begin{lstlisting}
type EvenOdd = {
  even: Int (*$ \to $*) Bool,
  odd:  Int (*$ \to $*) Bool
} in
trait Even() { self: EvenOdd (*$ \to $*)
  even(n: Int) = if n == 0 then True else self.odd(n - 1)
} in
trait Odd() { self: EvenOdd (*$ \to $*)
  odd(n: Int) = if n == 0 then False else self.even(n - 1)
} in
new (Even() & Odd()).odd(42)
\end{lstlisting}

When the two traits are composed, conceptually it is as if that a new class were
being created on the fly by copying all the definitions inside those two traits.
If there is any unresolved conflict, the program will be rejected by the type
system.

\paragraph{Dynamic Composition.}

One difference with the trait approach and the class approach is that in our
language we are able to compose traits \emph{dynamically} and then instantiate
them, which is impossible in traditional OO languages such as Java since classes
being instantiated must be known statically. Actually, since traits are just
terms, traits are first-class values and can be defined inside a function,
passed around or returned just as normal terms. The following function takes a
trait, \lstinline$pt$, with unknown implementation and instantiate it.\bruno{Bad as it exposes the encoding.}

\begin{lstlisting}
let f (pt: (Int, Int) (*$ \to $*) (() (*$ \to $*) Point) (*$ \to $*) Point) = new [Point] (pt (3,4))
in (* \ldots *)
\end{lstlisting}

%*******************************************************************************
\subsection{Desugaring}
%*******************************************************************************

Of course, this whole section will lose its point if the source language cannot
be translated to \name and checked against the type system of \name. A more
formal description can be found in the appendix. The idea of trait translation
is inspired by the functional mixin semantics~\cite{cook1989denotational}, which was
proposed by Cook in an untyped setting. However, our translation is done context of a
statically-typed programming language, which is what provides the ability to statically
detect conflicts in traits.

\paragraph{Trait Declarations.} A trait in the source language is translated into
nothing but a normal term in \name. For example,

\begin{lstlisting}
trait Point(x: Int, y: Int) { self: Point (*$ \to $*)
  x() = x
  y() = self.z()
}
(* \ldots *)
\end{lstlisting}

becomes

\begin{lstlisting}
let Point (x: Int) (y: Int) (self: () (*$ \to $*) Point) = {
  x = (*$ \lambda $*)(_: ()) (*$ \to $*) x,
  y = (*$ \lambda $*)(_: ()) (*$ \to $*) (self ()).z()
} in
\end{lstlisting}

One difference is that the self reference becomes a thunk and all occurrences of
it have been replaced by \lstinline$self ()$ and the position of the self
reference in the parameter list is adjusted. In fact, \lstinline$self$ is not a
special keyword. It can have any name, but \lstinline$self$ is a
convention.

The body of the trait becomes a record whose labels are the method names.
\lstinline$Point$ has type:

\begin{lstlisting}
Int (*$ \to $*) Int (*$ \to $*) (() (*$ \to $*) Point) (*$ \to $*) Point
\end{lstlisting}

The syntax for construction such as \lstinline$Point(3,4)$ is just function
application in \name. And note that \lstinline$Point(3,4)$ is of type
\begin{lstlisting}
(() (*$ \to $*) Point) (*$ \to $*) Point
\end{lstlisting}

Therefore it is an open recursive term: the recursive call is passed as an argument.

\paragraph{The ``new'' Keyword.} \lstinline$new$ instantiates a trait by taking the
fixpoint of its corresponding open term. In fact, \lstinline$new$ is translated as
an inlined fixpoint. For example,

\begin{lstlisting}
new Point(3,4)
\end{lstlisting}

becomes

\begin{lstlisting}
let rec x : () (*$ \to $*) Point = (*$ \lambda $*)(_: ()) (*$ \to $*) Point 3 4 x in x ()
\end{lstlisting}

The composition of traits in the source language is desugared using the merge
operator. The reason that traits built on \name have conflict detection for free
is that the merge operator is enforcing that the two terms being merged are
disjoint. For example,

\begin{lstlisting}
new (Point(3,4) & Z(5))
\end{lstlisting}

is turned into

\begin{lstlisting}
new ((\self: Point3D) (*$ \to $*) ((Point 3 4 self) ,, (Z 5 self)))
\end{lstlisting}

% \subsection{Intersection Types in Existing Languages}
%
% What is an intersection type? The intersection of types $A$ and $B$
% contains exactly those values which can be used as either of type $A$
% or of type $B$.  Just as not all intersection of sets are nonempty,
% not all intersections of types are inhabited.  For example, the
% intersection of a base type $\tyint$ and a function type
% $\tyint \to \tyint$ is not inhabited.\bruno{put this text somewhere?}
%
% Since then various researchers have
% studied intersection types, and some languages have adopted in one
% form or another. However, while intersection types are already used
% in various languages, the lack of a merge operator removes
% considerable expressiveness.
%
%
% A number of OO languages, such as
% Java, C\#, Scala, and Ceylon\footnote{\url{http://ceylon-lang.org/}},
% already support intersection types to different degrees. Intersection
% types are particularly relevant for OOP as they can be used to model
% multiple interface inheritance. In Java, for example,
%
% \begin{lstlisting}
% interface AwithB extends A, B {}
% \end{lstlisting}
%
% \noindent introduces a new interface \lstinline{AwithB} that satisfies the interfaces of
% both \lstinline{A} and \lstinline{B}. Arguably such type can be considered as a nominal
% intersection type. Scala takes one step further by eliminating the
% need of a nominal type. For example, given two concrete traits, it is possible to
% use \emph{mixin composition} to create an object that implements both
% traits. Such an object has a (structural) intersection type:
%
% \begin{lstlisting}
% trait A
% trait B
%
% val newAB : A with B = new A with B
% \end{lstlisting}
%
% \noindent Scala also allows intersection of type parameters. For example:
% \begin{lstlisting}
% def merge[A,B] (x: A) (y: B) : A with B = ...
% \end{lstlisting}
% uses the anonymous intersection of two type parameters \lstinline{A} and
% \lstinline{B}. However, in Scala it is not possible to dynamically
% compose two objects. For example, the following code:
%
% \begin{lstlisting}
% // Invalid Scala code:
% def merge[A,B] (x: A) (y: B) : A with B = x with y
% \end{lstlisting}
%
% \noindent is rejected by the Scala compiler. The problem is that the
% \lstinline{with} construct for Scala expressions can only be used to
% mixin traits or classes, and not arbitrary objects. Note that in the
% definition \lstinline{newAB} both \lstinline{A} and \lstinline{B} are
% \emph{traits}, whereas in the definition of \lstinline{merge} the variables
% \lstinline{x} and \lstinline{y} denote \emph{objects}.
%
% This limitation essentially put intersection types in Scala in a second-class
% status. Although \lstinline{merge} returns an intersection type, it is
% hard to actually build values with such types. In essence an
% object-level introduction construct for intersection types is missing.
% As it turns out using low-level type-unsafe programming features such
% as dynamic proxies, reflection or other meta-programming techniques,
% it is possible to implement such an introduction
% construct in Scala~\cite{oliveira2013feature,rendel14attributes}. However, this
% is clearly a hack and it would be better to provide proper language
% support for such a feature.
%
% To address the limitations of intersection types in languages like
% Scala, \name allows intersecting any two terms at run time using a
% \emph{merge} operator (denoted by $ \mergeop $)~\cite{dunfield2014elaborating}.  With the merge
% operator it is trivial to implement the \lstinline{merge} function in \name:
%
% \begin{lstlisting}
% let merge[A, B * A] (x : A) (y : B) : A & B = x ,, y in (*$ \ldots $*)
% \end{lstlisting}
%
% \noindent In contrast to Scala's term-level \lstinline{with}
% construct, the operator \lstinline{,,} allows two arbitrary values \lstinline{x}
% and \lstinline{y} to be merged. The resulting type is a \emph{disjoint}
% intersection of the types of  \lstinline{x}
% and \lstinline{y} (\lstinline{A & B} in this case).
%
% A well-formed type is such that given any query type,
% it is always clear which subpart the query is referring to.
% In terms of rules, this notion of well-formedness is almost the same as the one in System $F$
% except for intersection types we require the two components to be disjoint.
%

\section{The \name calculus}

\footnote{Joshua Dunfield}

This section formalizes the syntax, subtyping, and typing of \name. In the next
section, we will go through the type-directed translation from \name to System
F.

% Note the semantics of this language is not defined formally, instead, by a
% translation into the target language, System F.

\subsection{Syntax}

The syntax of the \name calculus extends System F by adding the two features:
intersection types and records. The formalization includes only single records
because single record types as the multi-records can be desugared into the merge
of multiple single records.

\[
\begin{array}{l}
  \begin{array}{llrl}
    \text{Types} 
    & \tau & \Coloneqq & \alpha \mid \highlight{\top} \mid \tau_1 \to \tau_2 \mid \for \alpha \tau \\
    &      & \mid      & \highlight{\tau_1 \andop \tau_2} \mid \highlight{\recty l \tau} \\
    \text{Expressions} 
    & e & \Coloneqq & x \mid \highlight{\top} \mid \lam x \tau e \mid \app e e \mid \blam \alpha e \mid \tapp e \tau \\
    &   & \mid      & \highlight {e \mergeop e} \mid \highlight {\reccon l e} \mid
                      \highlight {e.l} \mid \highlight{e \restrictop l} \\
    \text{Contexts} 
    & \gamma & \Coloneqq & \epsilon \mid \gamma, \alpha \mid \gamma, x \hast \tau \\
    \text{Labels} & l
  \end{array} 
\end{array}
\]


% Types
Types $ t $ have five constructs. The first three are standard (present in
System F): type variable $ \alpha $, function types $ t \to t $, and type
abstraction $ \forall \alpha. t $; while the last two, intersection types
$ t \with t $ and record types $ \recordtype{l}{t} $, are novel in \Name. In
record types, $ l $ is the label and $ t $ the type.

% Expressions - standard ones
First five constructs of expressions are also standard: variables $ x $ and two
abstraction-elimination pairs. $ \abs{\rel{x}{t}}{e} $ abstracts expression
$ e $ over values of type $ t $ and is eliminated by application $ \app{e}{e} $;
$ \Abs{\alpha}{e} $ abstracts expression $ e $ over types and is eliminated by
type application $ \app{e}{t} $.

% Expressions - new ones
The last four constructs are novel. $ e \dcomma e $ is the \emph{merge} of two
terms. $ \recordintro{l}{e} $ introduces a record literal having $ l $ as the
label for field containing expression $ e $ . $ e.l $ access the field with
label $ l $ in $ e $. Finally, $ \recordupdate{e}{l}{e} $ updates the field
labelled $ l $ in expression $ e $. For simplicity, we omit other constructs in
order to focus on the essence of the calculus. For example, fixpoints can be
added in standard ways.

% Fields
The field $ F $ is non-standard and introduced to deal with records. It is an
associative list. Each item is a pair whose first item is either empty or a
label and the second the types.

% The most central construct of our language is ...

% Dunfield has described a language that includes a ``top'' type but it does not appear in our language. Our work differs from Dunfield in that ...

% Remark. The operational semantics of FI is not presented in this paper. However,

\subsection{Subtyping}

\begin{figure*}
\framebox{\( \ty \subtype \ty \yieldsnothing C \)}

\infax[SVar]
{\alpha \subtype \alpha \yieldsnothing {\abs {\rel x {\image \alpha}} x}}

\infrule[SFun]
{\ty_3 \subtype \ty_1 \yieldsnothing {C_1} \andalso \ty_2 \subtype \ty_4 \yieldsnothing {C_2}}
{\ty_1 \to \ty_2 \subtype \ty_3 \to \ty_4
  \yieldsnothing
    {\abs {\rel f {\image {\ty_1 \to \ty_2}}}
      {\abs {\rel x {\image {\ty_3}}}
        {\app {C_2} {(\app {f} {(\app {C_1} {x})})}}}}}

\infrule[SForall]
{\ty_1 \subtype \subst {\alpha_2} {\alpha_1} \ty_2 \yieldsnothing C}
{\forall \alpha_1. \ty_1 \subtype \forall \alpha_2. \ty_2
  \yieldsnothing
    {\abs {\rel{f} {\image {\forall \alpha. \ty_1}}}
      {\Abs \alpha {\app C {(\app f \alpha)}}}}}

\infrule[SAnd1]
{\ty_1 \subtype \ty_3 \yieldsnothing C}
{\ty_1 \with \ty_2 \subtype \ty_3
  \yieldsnothing
    {\abs {\rel x {\image {\ty_1 \with \ty_2}}}
      {\app C {(\app \fst x)}}}}

\infrule[SAnd2]
{\ty_2 \subtype \ty_3 \yieldsnothing C}
{\ty_1 \with \ty_2 \subtype \ty_3
  \yieldsnothing
    {\abs {\rel x {\image {\ty_1 \with \ty_2}}}
      {\app C {(\app \snd x)}}}}

\infrule[SAnd3]
{\ty_1 \subtype \ty_2 \yieldsnothing {C_1} \andalso \ty_1 \subtype \ty_3 \yieldsnothing {C_2}}
{\ty_1 \subtype \ty_2 \with \ty_3
  \yieldsnothing
    {\abs {\rel x {\image {\ty_1}}}
      {\tupled {\app {C_1} x, \app {C_2} x}}}}

\infrule[SRcd]
{\ty_1 \subtype \ty_2 \yieldsnothing C}
{\recordtype l {\ty_1} \subtype \recordtype l {\ty_2}
  \yieldsnothing
    {\abs {\rel x {\image {\recordtype l {\ty_1}}}} {\app C x}}}

\caption{Subtyping}
\end{figure*}

Thanks to intersection types, we have natural subtyping relations among types.
For example, $ Int \with Bool $ should be a subtype of $ Int $, since the former
can be viewed as either $ Int $ or $ Bool $. The subtyping rules are standard
except for three points listed below:
\begin{enumerate}
\item $ t_1 \with t_2 $ is a subtype of $ t_3 $, if \emph{either} $ t_1 $ or
  $ t_2 $ are subtypes of $ t_3 $,

\item $ t_1 $ is a subtype of $ t_2 \with t_3 $, if $ t_1 $ is a subtype of
  both $ t_2 $ and $ t_3 $.

\item $ \recordtype{l_1}{t_1} $ is a subtype of $ \recordtype{l_2}{t_2} $, if
  $ l_1 $ and $ l_2 $ are identical and $ t_1 $ is a subtype of $ t_2 $.
\end{enumerate}
The first point is captured by two rules \texttt{S-And-1} and \texttt{S-And-2},
whereas the second point by \texttt{S-And-3}. Note that the last point means
that record types are covariant in the type of the fields.

\subsection{Typing}

\begin{figure*}
\framebox{\(\Gamma \turns {t}\)}

\infrule[WF-Var]
{\alpha \in \Gamma}
{\Gamma \turns \alpha}

\infrule[WF-Fun]
{\Gamma \turns \ty_1 \andalso \Gamma \turns \ty_2}
{\Gamma \turns \ty_1 \to \ty_2}

\infrule[WF-Forall]
{\Gamma, \alpha \turns \ty}
{\Gamma \turns \forall \alpha. \ty}

\infrule[WF-And]
{\Gamma \turns \ty_1 \andalso \Gamma \turns \ty_2}
{\Gamma \turns \ty_1 \intersects \ty_2}

\infrule[WF-Rcd]
{\Gamma \turns \ty}
{\Gamma \turns \recordtype l \ty}

\caption{Well-formedness}
\end{figure*}

\begin{figure*}
\framebox{\( \Gamma \turns e : \ty \yieldsnothing E \)}

\infrule[Var]
{(x,\ty) \in \Gamma}
{\Gamma \turns x : \ty \yieldsnothing x}

\infrule[Abs]
{\Gamma, \rel x \ty \turns e : \ty_1 \yieldsnothing E \andalso
 \Gamma \turns \ty}
{\Gamma \turns \abs {\rel x \ty} e : \ty \to \ty_1 \yieldsnothing {\abs {\rel x {\image \ty}} E}}

\infrule[TAbs]
{\Gamma, \alpha \turns e : \ty \yieldsnothing E}
{\Gamma \turns \Abs \alpha e : \forall \alpha. \ty \yieldsnothing {\Abs \alpha E}}

\infrule[App]
{\Gamma \turns e_1 : \ty_1 \to \ty_2 \yieldsnothing {E_1} \\
 \Gamma \turns e_2 : \ty_3 \yieldsnothing {E_2} \andalso
 \ty_3 \subtype \ty_1 \yieldsnothing C}
{\Gamma \turns \app {e_1} {e_2} : \ty_2 \yieldsnothing {\app {E_1} {(\app C E_2)}}}

\infrule[TApp]
{\Gamma \turns e : \forall \alpha. \ty_1 \yieldsnothing E \andalso
 \Gamma \turns \ty}
{\Gamma \turns \app e \ty : \subst \alpha \ty \ty_1 \yieldsnothing {\app E {\image \ty}}}

\infrule[Merge]
{\Gamma \turns e_1 : \ty_1 \yieldsnothing {E_1} \andalso
 \Gamma \turns e_2 : \ty_2 \yieldsnothing {E_2}}
{\Gamma \turns e_1 \dcomma e_2 : \ty_1 \intersects \ty_2 \yieldsnothing {\tupled {E_1, E_2}}}

\infrule[RecCon]
{\Gamma \turns e : \ty \yieldsnothing E}
{\Gamma \turns \reccon l e : \recty l \ty \yieldsnothing E}

\infrule[RecProj]
{\Gamma \turns e : \ty \yieldsnothing E \andalso
 \Gamma \turnsget (\ty; l) : \ty_1 \yieldsnothing C}
{\Gamma \turns e.l : \ty_1 \yieldsnothing {\app C E}}

\infrule[RecUpd]
{\Gamma \turns e : \ty \yieldsnothing E \andalso
 \Gamma \turns e_1 : \ty_1 \yieldsnothing {E_1} \\
 \turnsput (t; l; e_1 : \ty_1 \yieldsnothing {E_1}) : (\ty_2, \ty_3) \yieldsnothing C \andalso
 \ty_1 \subtype \ty_2}
{\Gamma \turns \recupd e l {e_1} : \ty_3 \yieldsnothing {\app C E}}

\framebox{\( \vdash_{get} (t; l) : t \yieldsnothing C \)}

\infax[GetBase]
{\vdash_{get} (\recordtype l t; l) : t
  \yieldsnothing {\abs {\rel x {\imageof {\recordtype l t}}} x}}

\infrule[GetLeft]
{\vdash_{get} (t_1; l) : t \yieldsnothing C}
{\vdash_{get} (t_1 \with t_2; l) : t
  \yieldsnothing {\abs {\rel x {\imageof {t_1 \with t_2}}} {C (\app \fst x)}}}

\infrule[GetRight]
{\vdash_{get} (t_2; l) : t \yieldsnothing C}
{\vdash_{get} (t_1 \with t_2; l) : t
  \yieldsnothing {\abs {\rel x {\imageof {t_1 \with t_2}}} {C (\app \snd x)}}}

\begin{mathpar}
\framebox{\( \turnsput (\tau; l; e : \tau \yieldsnothing E) : (\tau, \tau) \yieldsnothing C \)}

\inferrule* [right=Put]
{ }
{\turnsput (\RecTy l \tau; l; e : \tau_1) : (\tau, \RecTy l {\tau_1})}

\inferrule* [right=Put1]
{\turnsput (\tau_1; l; e : \tau) : (\tau_3, \tau_4)}
{\turnsput (\tau_1 \Intersect \tau_2; l; e : \tau) : (\tau_3, \tau_4 \Intersect \tau_2)}

\inferrule* [right=Put2]
{\turnsput (\tau_2; l; e : \tau) : (\tau_3, \tau_4)}
{\turnsput (\tau_1 \Intersect \tau_2; l; e : \tau) : (\tau_3, \tau_1 \Intersect \tau_4)}
\end{mathpar}
\caption{Typing}
\end{figure*}

The typing judgment for \name is of the form: $ \Gamma \vdash e : t $. This
judgment uses the context $ \Gamma $. The typing rules for our core languages
are mostly standard ones for System F. In particular we introduce a
\texttt{T-Merge} rule that applies to \emph{merge} constructs.

The last two rules make use of the $ \texttt{fields} $ function just to make
sure that the field being accessed (\texttt{T-RcdElim}) or updated
(\texttt{T-RcdUpd}) actually exists. The function is defined recursively, in
Haskell pseudocode, as:
\[ \begin{array}{rll}
  \fields{\alpha} & = & \rel{\cdot}{\alpha} \\
  \fields{t_1 \to t_2} & = & \rel{\cdot}{t_1 \to t_2} \\
  \fields{\forall \alpha. t} & = & \rel{\cdot}{\forall \alpha. t} \\
  \fields{t_1 \with t_2} & = & \fields{t_1} \dplus \fields{t_2} \\
  \fields{\recordtype{l}{t}} & = & \rel{l}{t}
\end{array} \]
where $ \cdot $ means empty list, $ \dplus $ list concatenation, and $ : $ is an
infix operator that prepend the first argument to the second. The function
returns an associative list whose domain is field labels and range types.

\textit{dom} reads: ``the domain of''. $ F(l) $ means the result of lookup for
$ l $ inside the associative list $ F $. The order of lookup can be either from
left to right or from right to left but has to be consistent inside one
implementation. We prefer the order from the right to the left because it make
possible record overriding. For example,
$ (\recordintro{count}{1} ,, \recordintro{count}{2}).count $ will evaluate to
$ 2 $ in this case.
\section{Disjointness and Coherence}
\label{sec:disjoint}

Although the system shown in the Section~\ref{sec:fi} is type-safe, it is not
coherent. This section shows how to modify \name so that it guarantees coherence
as well as type soundness. The result is a calculus named \namedis. The keys
aspects are the notion of disjoint intersections, and disjoint quantification
for polymorphic types.

%From the theoretical point-of-view, the end goal of this section is to show that the resulting system has
%a coherent (or unique) elaboration semantics:
%\begin{restatable}[Unique elaboration]{theorem}{uniqueelaboration}
%  \label{theorem:unique-elaboration}
%
%  If $\jtype \Gamma e {A_1} \yields {E_1}$ and $\jtype \Gamma e {A_2} \yields
%  {E_2}$, then $E_1 \equiv E_2$. (``$\equiv$'' means syntactical equality, up to
%  $\alpha$-equality.)
%
%\end{restatable}
%
%\noindent In other words, given a source term $e$, elaboration always produces
%the same target term $E$. The most important hurdle we need to overcome is that
%if $A \inter B \subtype C$, then either $A$ or $B$ contributes to that subtyping
%relation, resulting in two possible coercions.

\subsection{Disjointness} Throughout the paper we already presented an intuitive
definition for disjoiness. Here such definition is made a bit more precise, and
well-suited to \namedis.

\begin{definition}[Disjoint types]

  Given a context $\Gamma$, two types $A$ and $B$ are said to be disjoint
  (written $\jdis \Gamma A B$) if they do not share a common supertype. That is,
  there does not exist a type $C$ such that $A \subtype C$ and that $B \subtype
  C$. Note that we assume that all free type variables in $A$,
    $B$ and $C$ are bound in $\Gamma$.

  \[\jdis \Gamma A B \equiv \not \exists C.~A \subtype C \wedge B \subtype C\]

\end{definition}

To see this definition in action, $\tyint$ and $\tychar$ are disjoint,
because there is
no type that is a supertype of the both. On the other hand, $\tyint$ is not
disjoint with itself, because $\tyint \subtype \tyint$. This implies that
disjointness is not reflexive as subtyping is. Two types with different shapes
are always disjoint, unless one of them is an intersection type. For example, a function
type and a universally quantified type must be disjoint. But a function type and an intersection
type may not be. Consider:
\[ \tyint \to \tyint \quad \text{and} \quad (\tyint \to \tyint) \inter (\tystring \to \tystring) \]
Those two types are not disjoint since $\tyint \to \tyint$ is their common supertype.

\subsection{Syntax}

\begin{figure}
  \[
    \begin{array}{l}
      \begin{array}{llrll}
        \text{Types}
        & A, B & \Coloneqq & \alpha                  & \\
        &      & \mid & \highlight {$\bot$}          & \\
        &      & \mid & A \to B                      & \\
        &      & \mid & \for {(\alpha \highlight {$\disjoint A$})} B  & \\
        &      & \mid & A \inter B                   & \\

        \\
        \text{Terms}
        & e & \Coloneqq & x                        & \\
        &   & \mid & \lam {(x \oftype A)} e          & \\
        &   & \mid & \app {e_1} {e_2}              & \\
        &   & \mid & \blam {(\alpha \highlight {$\disjoint A$})} e  & \\
        &   & \mid & \tapp e A                     & \\
        &   & \mid & e_1 \mergeop e_2              & \\

        \\
        \text{Contexts}
        & \Gamma & \Coloneqq & \cdot
                   \mid \Gamma, \alpha \highlight {$\disjoint A$}
                   \mid \Gamma, x \oftype A  & \\

        \text{Syntactic sugar} & \blam \alpha e & \equiv & \blamdis \alpha \bot e & \\
                               & \forall \alpha.~A & \equiv & \forall (\alpha * \bot)~.~A & \\
      \end{array}
    \end{array}
  \]

  \caption{Amendments of the rules.}
  \label{fig:fi-syntax-dis}
\end{figure}

\begin{figure}
  \framebox{$\im A = T$}

  \begin{align*}
    \im \bot                  &= () \\
    \im {\fordis \alpha A B}  &= \for \alpha \im B \\
  \end{align*}

  \framebox{$\im \Gamma = G$}

  \begin{align*}
    \im {\Gamma, \alpha \disjoint A} &= \im \Gamma, \alpha \\
  \end{align*}

  \caption{Additional type and context translation.}
  \label{fig:additional-type-and-context-translation}
\end{figure}

% \george{May also note on the scoping of type variables inside contexts.}

Figure~\ref{fig:fi-syntax-dis} shows the updated syntax with the
changes highlighted and Figure~\ref{fig:additional-type-and-context-translation}
shows type and context translations for the new constructs.
Note how similar the changes are to those needed
to extend System $F$ with bounded quantification. First, type
variables are now always associated with their disjointness
constraints (like $\alpha \disjoint A$) in types, terms, and
contexts. Second, the bottom type ($\bot$) is introduced so that
universal quantification becomes a special case of disjoint
quantification: $\blam \alpha e$ is really a syntactic sugar for
$\blamdis \alpha \bot e$. The underlying idea is that any type is
disjoint with the bottom type.  Note the analogy with bounded
quantification, where the top type is the trivial upper bound in
bounded quantification, while the bottom type is the trivial
disjointness constraint in disjoint quantification.

%Indeed, \bruno{unfinidhed sentence}\george{Mabe show a diagram here to contrast
%with bounded polymorphism.}

\subsection{Typing}
Figure~\ref{fig:fi-type-patch} shows modifications to Figure~\ref{fig:fi-type} in
order to support disjoint intersection types and disjoint
quantification. Only new rules or rules that different are shown.
Importantly, the disjointness judgement appears in the well-formedness rule for intersection
types and the typing rule for merges.

\begin{figure}
  \begin{mathpar}
    \framebox{$\jatomic A$} \\

    \inferrule*
    {}
    {\jatomic \bot}

    \inferrule*
    {}
    {\jatomic {A \to B}}

    \inferrule*
    {}
    {\jatomic {\fordis \alpha B A}}
  \end{mathpar}

  \begin{mathpar}
    \formsub \\ \rulesubinterldis \and \rulesubinterrdis \and \rulesubforalldis
  \end{mathpar}

  \begin{mathpar}
    \formwf \\ \rulewfforalldis \and \rulewfinterdis
  \end{mathpar}

  \begin{mathpar}
    \formt \\ \ruletblamdis \and \rulettappdis \and \ruletmergedis
  \end{mathpar}

  \caption{Affected rules.}
  \label{fig:fi-type-patch}
\end{figure}

\begin{figure}
  \begin{mathpar}
    \framebox{$\jatomic A$} \\

    \inferrule*{}{\jatomic {A \to B}}

    \inferrule*{}{\jatomic \alpha}

    \inferrule*{}{\jatomic {\for \alpha A}}
  \end{mathpar}

  \begin{mathpar}
    \formsub \\ \rulesubint \and \rulesubvar \and \rulesubfun \and \rulesubforall
    \and \rulesubinter  \and \rulesubinterldis \and \rulesubinterrdis 
  \end{mathpar}

  \begin{mathpar}
    \formwf \\ \rulewfint \and \rulewfvar \and \rulewffun \and \rulewfforall \and \rulewfinter
  \end{mathpar}

%  \begin{mathpar}
%    \formt \\ \ruletvar \and \ruletlam \and \ruletapp \and
%    \ruletblam \and \rulettapp \and \ruletmergedis 
%  \end{mathpar}

  \begin{mathpar}
    \formbi \\ \bruletint \and \bruletvar \and \bruletapp \and
    \brulettapp \and \bruletmergedis \and \bruletann 
  \end{mathpar}

  \begin{mathpar}
    \formbc \\ \bruletlam \and  \bruletblam \and \bruletsub
  \end{mathpar}

  \caption{Rules for Naive system.}
  \label{fig:fi-type-naive}
\end{figure}

\begin{figure}
  \begin{mathpar}
    \framebox{$\jatomic A$} \\

    \inferrule*{}{\jatomic {\fordis \alpha B A}}
  \end{mathpar}

  \begin{mathpar}
    \formsub \\ \rulesubtop \and \rulesubforalldis \and 
    \rulesubinterlcoerce \and \rulesubinterrcoerce
  \end{mathpar}

  \begin{mathpar}
    \formwf \\ \rulewftop \and \rulewfvardis \and \rulewfforalldis 
  \end{mathpar}

  \begin{mathpar}
    \formbi \\ \brulettop \and \brulettappdis 
  \end{mathpar}

  \begin{mathpar}
    \formbc \\ \bruletblamdis 
  \end{mathpar}


  \caption{Changes for Extended systems.}
  \label{fig:fi-type-extended}
\end{figure}

\begin{figure}
  \begin{mathpar}
    \formsub \\ \rulesubforallext
  \end{mathpar}

  \caption{Rules for Extended system with improved ForAll.}
  \label{fig:fi-type-extended_forall}
\end{figure}

\begin{figure}[t]
  \begin{mathpar}
    \formtoplike \\ %\framebox{$\jatomic A$} \\

    \ruletopltop \and \ruletoplinter \and \ruletoplfun \and \ruletoplforall
  \end{mathpar}


  \begin{center}
    \framebox{$\andcoerce{A}_{C} = T$}
  \end{center}

  \[
  \andcoerce{A}_{C} = 
  \begin{cases} 
        \toplike{A} & \andcoerce{A} \\ 
        %A = \top & () \\
        %A = A_1 \to A_2 \; \wedge \; \toplike{A_2} & \lam x \andcoerce{A_2}_{C} \\
        \text{otherwise} & {C} 
  \end{cases}
  \]

  \begin{center}
    \framebox{$\andcoerce{A} = T$}
  \end{center}

  \[
  \andcoerce{A} = 
  \begin{cases} 
        A = \top & () \\
        A = A_1 \to A_2 & \lam x \andcoerce{A_2} \\
        A = A_1 \inter A_2 & \pair {\andcoerce{A_1}} {\andcoerce{A_2}} \\
        A = \fordis \alpha B A & \blam \alpha {\andcoerce{A}}
  \end{cases}
  \]
  \caption{Coercion generation considering Top-like types.}
  \label{fig:andcoercion}
\end{figure}



%\paragraph{Atomic Types.} The new system introduces atomic types. Essentially a type
%is atomic if it is any type, which is not an intersection type.
%The notion of atomic
%type will be helpful

\paragraph{Well-formedness.}
We require that the two types of an intersection must be disjoint in their
context, and that the disjointness constraint in a universal type is well-formed.
Under the new rules, intersection types such as $\tyint \inter \tyint$ are no
longer well-formed because the two types are not disjoint.

\paragraph{Disjoint quantification.} A disjoint quantification is introduced by
the big lambda $\blamdis \alpha A e$ and eliminated by the usual type
application $\tapp e A$. The constraint is added to the context with this rule.
During a type application, the type system makes sure that the type argument
satisfies the disjointness constraint.

\paragraph{Metatheory.}

Since in this section we only restrict the type system in the previous section,
it is easy to see that type preservation and type-safety still holds.
Additionally, we can show that typing always produces a well-formed type by
proving the following results.

\begin{restatable}[Instantiation]{lemma}{instantiation}
  \label{lemma:instantiation}

  If $\jwf {\Gamma, \alpha \disjoint B} C$, $\jwf \Gamma A$, $\jdis \Gamma A B$
  then $\jwf \Gamma {\subst A \alpha C}$.
\end{restatable}

\begin{restatable}[Well-formed typing]{lemma}{wellformedtyping}
  \label{lemma:wellformed-typing}

  If $\jtype \Gamma e A$, then $\jwf \Gamma A$.
\end{restatable}

\begin{proof}
  By induction on the derivation that leads to $\jtype \Gamma e A$ and applying
  Lemma~\ref{lemma:instantiation} in the case of \reflabel{\labelttapp}.
\end{proof}

With our new definition of well-formed types, this result is nontrivial.
In general, disjointness judgements are not invariant with respect to
free-variable substitution. In other words, a careless substitution can violate
the disjoint constraint in the context. For example, in the context $\alpha
\disjoint \tyint$, $\alpha$ and $\tyint$ are disjoint:
\[ \jdis {\alpha \disjoint \tyint} \alpha \tyint \]
But after the substitution of $\tyint$ for $\alpha$ on the two types, the sentence
\[ \jdis {\alpha \disjoint \tyint} \tyint \tyint \]
is longer true since $\tyint$ is clearly not disjoint with itself.

\subsection{Subtyping}

% \george{Explain \reflabel{\labelsubforall} and distinction of kernel and all version.}

The subtyping rules need some adjustment.
Note that the $\bot$ type does not participate in subtyping since it holds no value.
An important
problem with the subtyping rules in Figure~\ref{fig:fi-type} is that the all three rules
dealing with intersection types
(\reflabel{\labelsubinterl} and \reflabel{\labelsubinterr} and \reflabel{\labelsubinter})
overlap. Unfortunatelly,
this means that different coercions may be given when checking the subtyping
between two types, depending on which derivation is chosen. This is ultimately the reason
for incoherence.
There are two important types of overlap:

\begin{enumerate}

\item The left decomposition rules for intersections (\reflabel{\labelsubinterl} and \reflabel{\labelsubinterr}) overlap with each other.

\item The left decomposition rules for intersections (\reflabel{\labelsubinterl} and \reflabel{\labelsubinterr})
overlap with the right decomposition rules for intersections (\reflabel{\labelsubinter}).

\end{enumerate}

\noindent Fortunately, disjoint intersections (which are enforced by well-formedness)
deal with problem 1): only one of the two left decomposition rules
can be chosen for a disjoint intersection type. Since the two types in the intersection
are disjoint it is impossible that both of the preconditions of the left decompositions are satisfied
at the same time. More formally, with disjoint intersections, we have the following theorem:

\begin{lemma}[Unique subtype contributor]
  \label{lemma:unique-subtype-contributor}

  If $A_1 \inter A_2 \subtype B$, where $A_1 \inter A_2$ and $B$ are well-formed types,
  then it is not possible that the following holds at the same time:
  \begin{enumerate}
    \item $A_1 \subtype B$
    \item $A_2 \subtype B$
  \end{enumerate}
\end{lemma}

Unfortunatelly, disjoint intersections alone are insufficient to deal with problem 2).
In order to deal with problem 2), we introduce a distinction between types, and atomic types.

\paragraph{Atomic types.} Atomic types are just those which are not intersection
types, and are asserted by the judgement \[ \jatomic A \]

In the left decomposition rules for intersections we introduce a requirement that
$A_3$ is atomic. The consequence of this requirement is that when $A_3$ is an intersection type, then
the only rule that can be applied is \reflabel{\labelsubinter}.
With the atomic constraint, one can guarantee that at any moment during the
derivation of a subtyping relation, at most one rule can be used.
Consequently, the coercion of a subtyping relation $A \subtype B$ is uniquely determined.
This fact is captured by the following lemma:

\begin{restatable}[Unique coercion]{lemma}{uniquecoercion}
  \label{lemma:unique-coercion}

  If $A \subtype B \yields {E_1}$ and $A \subtype B \yields {E_2}$, where $A$
  and $B$ are well-formed types, then $E_1 \equiv E_2$.
\end{restatable}

\paragraph{Expressiveness.}
Remarkably, our restrictions on subtyping do not sacrifice the expressiveness of
subtyping since we have the following two theorems:
\begin{theorem}
  If $A_1 \subtype A_3$, then $A_1 \inter A_2 \subtype A_3$.
\end{theorem}
\begin{theorem}
If $A_2 \subtype A_3$, then $A_1 \inter A_2 \subtype A_3$.
\end{theorem}

The interpretation of the theorem is that: even though the premise is made more
strict by the atomic condition, we can still derive the every subtyping relation
in the unrestricted system.
% \george{Explain why the proof shows this.}

\begin{comment}
Note that $A$ \emph{exclusive} or $B$ is true if and only if their truth value
differ. Next, we are going to investigate the minimal requirement (necessary and
sufficient conditions) such that the theorem holds.

If $A_1$ and $A_2$ in this setting are the same, for example,
$\tyint \inter \tyint \subtype \tyint$, obviously the theorem will
not hold since both the left $\tyint$ and the right $\tyint$ are a
subtype of $\tyint$.

We can try to rule out such possibilities by making the requirement of
well-formedness stronger. This suggests that the two types on the sides of
$\inter$ should not ``overlap''. In other words, they should be ``disjoint''. It
is easy to determine if two base types are disjoint. For example, $\tyint$
and $\tyint$ are not disjoint. Neither do $\tyint$ and $\code{Nat}$.
Also, types built with different constructors are disjoint. For example,
$\tyint$ and $\tyint \to \tyint$. For function types, disjointness
is harder to visualize. But bear in the mind that disjointness can defined by
the very requirement that the theorem holds.


This result is captured more formally by the following lemma:
\end{comment}

% \george{Note that $\bot$ does not participate in subtyping and why (because the
% empty set intersecting the empty set is still empty).}

% \george{What's the variance of the disjoint constraint? C.f. bounded
% polymorphism.}

% \george{Two points are being made here: 1) nondisjoint intersections, 2) atomic
% types. Show an offending example for each?}

\subsection{Coherence of the Elaboration}
Combining the previous results, we are able to show the central theorem:

\begin{restatable}[Unique elaboration]{theorem}{uniqueelaboration}
  \label{theorem:unique-elaboration}

  If $\jtype \Gamma e {A_1} \yields {E_1}$ and $\jtype \Gamma e {A_2} \yields
  {E_2}$, then $E_1 \equiv E_2$. (``$\equiv$'' means syntactical equality, up to
  $\alpha$-equality.)

\end{restatable}

\begin{proof}
  Note that the typing rules are already syntax-directed but the case of
  \reflabel{\labeltapp} (copied below) still needs special attention since we
  need to show that the generated coercion $E$ is unique.
  \begin{mathpar}
    \ruletapp
  \end{mathpar}
  Luckily, by Lemma~\ref{lemma:wellformed-typing}, we know that typing
  judgements give well-formed types, and thus $\jwf \Gamma {A_1}$ and $\jwf
  \Gamma {A_3}$. Therefore we are able to apply
  Lemma~\ref{lemma:unique-coercion} and conclude that $E$ is unique.

\end{proof}

\section{Algorithmic Disjointness} \label{sec:alg-dis}

Section~\ref{sec:disjoint} presented a type system with disjoint
intersection types that is both type-safe
and coherent. Unfortunately the type system is not algorithmic
because the specification of disjointness does not lend itself to an
implementation directly. This is a problem, because we need an
algorithm for checking whether two types are disjoint or not in order
to implement the type-system.

% Now that we have obtained a specification for disjointness, but the definition
% involves an existence problem. How can we implement it? One possibility is
% bidirectional subtyping, that is, we say two types, $A$ and $B$, are disjoint if
% neither $A \subtype B$ nor $B \subtype A$. However, this implementation is
% wrong. For example, $\code{Int} \inter \tystring$ and $\tystring \inter \tychar$ are
% not disjoint by specification since $\tystring$ is their common supertype. Yet
% by the implementation they are, since neither of them is a subtype of
% the other. \bruno{You need a concrete code example to make this point.}
% Hence the algorithmic rules are more nuanced. For now, it is enough to treat the
% disjoint judgment $\jdis \Gamma A B$ as oracle and we will come back to
% this topic in the next section.

This section presents the set of rules for determining whether two
types are disjoint. The set of rules is algorithmic and an
implementation is easily derived from them. Therefore we solve the
problem of finding an algorithm to compute disjointness. The derived
set of rules for disjointness is proved to be sound and complete with
respect to the definition of disjointness in
Section~\ref{sec:disjoint}.

% \subsection{Derivation of the Algorithmic Rules}
%
% In this subsection, we illustrate how to derive the algorithmic disjointness
% rules by showing a detailed example for functions. For the ease of discussion,
% first we introduce some notation.
%
% \begin{definition}[Set of common supertypes]
%   For any two types $A$ and $B$, we can denote the \emph{set of their common
%   supertypes} by \[ \commonsuper(A,B) \] In other words, a type $C \in \;
%   \commonsuper(A,B)$ if and only  if $A \subtype C$ and $B \subtype C$.
% \end{definition}
%
% \begin{example}
%   $\commonsuper(\code{Int},\tychar)$ is empty, since $\code{Int}$ and $\tychar$
%   share no common supertype.
% \end{example}
%
% Parellel to the notion of the set of common supertypes is the notion of the set
% of common subtypes.
%
% \begin{definition}[Set of common subtypes]
%   For any two types $A$ and $B$, we can denote the \emph{set of their common
%   subtypes} by \[ \commonsub(A,B) \] In other words, a type $C \in \; \commonsub(A,B)$
%   if and only  if $C \subtype A$ and $C \subtype B$.
% \end{definition}
%
% \begin{example}
%   $\commonsub(\code{Int},\tychar)$ is an infinite set which contains $\code{Int} \inter
%   \tychar$, $\tychar \inter \code{Int}$, $(\code{Int} \inter \tybool) \inter \tychar$
%   and so on. But the type $\tybool$ is not inside, since it is not a subtype of
%   $\code{Int}$ or $\tychar$.
% \end{example}
%
%
% \paragraph{Shorthand Notation.} For brevity, we will use \[ \mathcal{A} \to
% \mathcal{B} \] as a shorthand for the \emph{set} of types of the form $A \to B$,
% where $A \in \mathcal{A}$ and $B \in \mathcal{B}$. The same shorthand applies to
% all other constructors of types, in addition to functions, as well. As a simple
% example,  \[ \{ \code{Int}, \tystring \} \to \{ \code{Int}, \tychar \} \] is a shorthand for \[ \{
% \code{Int} \to \code{Int}, \code{Int} \to \tychar, \tystring \to \code{Int}, \tystring \to \tychar \} \]
%
%
% Recall that we say two types $A$ and $B$ are disjoint if they do not share a
% common supertype. Therefore, determining if two types $A$ and $B$ are disjoint
% is the same as determining if $\commonsuper(A,B)$ is empty.
%
% \paragraph{Determining Disjointness of Functions.} Now let's dive into the case
% where both $A$ and $B$ are functions and consider how to compute
% $\commonsuper(A_1 \to A_2, B_1 \to B_2)$. By the subtyping rules, the supertype
% of a function must also be a function.\george{Nah... only after normalization.
% If not, it can also be $\inter$} Let $C_1 \to C_2$ be a common supertype
% of $A_1 \to A_2$ and $B_1 \to B_2$. Then it must satisfy the following:
% \begin{mathpar}
%   \inferrule
%     {C_1 \subtype A_1 \\ A_2 \subtype C_2}
%     {A_1 \to A_2 \subtype C_1 \to C_2}
%
%   \inferrule
%     {C_1 \subtype B_1 \\ B_2 \subtype C_2}
%     {B_1 \to B_2 \subtype C_1 \to C_2}
% \end{mathpar}
% From which we see that $C_1$ is a common subtype of $A_1$ and $B_1$ and that
% $C_2$ is a common supertype of $A_2$ and $B_2$. Therefore, we can write:
% \[ \commonsuper(A_1 \to A_2, B_1 \to B_2) \ = \ \commonsub(A_1,B_1) \to \commonsuper(A_2,B_2) \]
% By definition, $\commonsub(A_1,B_1) \to \commonsuper(A_2,B_2)$ is not empty if and only if both
% $\commonsub(A_1,B_1)$ and $\commonsuper(A_2,B_2)$ is nonempty. However, note
% that with intersection types, $\commonsub(A_1,B_1)$ is always nonempty because
% $A_1 \inter B_1$ belongs to $\commonsub(A_1,B_1)$. Therefore, the problem of
% determining if $\commonsuper(A_1 \to A_2, B_1 \to B_2)$ is empty reduces to the
% problem of determining if $\commonsuper(B_1 \to B_2)$ is empty, which is, by
% definition, if $B_1$ and $B_2$ are disjoint. Finally, we have derived a rule for
% functions:
% \begin{mathpar}
%   \ruledisfun
% \end{mathpar}
%
% The analysis needed for determining if types with other constructors are
% disjoint is similar. Below are the major results of the recursive definitions for
% $\commonsuper$:
% \begin{align*}
%   \commonsuper(A_1 \to A_2, B_1 \to B_2) &= \commonsub(A_1,B_1) \to \commonsuper(A_2,B_2) \\
%   \commonsuper({A_1 \inter A_2, B})      &= \commonsuper(A_1, B) \cup \commonsuper(A_1,B) \\
%   \commonsuper({A, B_1 \inter B_2})      &= \commonsuper(A, B_1) \cup \commonsuper(A,B_2)
% \end{align*}

\subsection{Algorithmic Rules}

\begin{figure}[h]
  \begin{mathpar}
    \formdis \\
    \ruledisfun \and
    \ruledisinterl \and \ruledisinterr \and \ruledisatomic
  \end{mathpar}

  \begin{mathpar}
    \formdis \\
    \ruledisfun \and
    \ruledisinterl \and \ruledisinterr \and \ruledisatomic
  \end{mathpar}

  \caption{Algorithmic Disjointness.}
  \label{fig:disjointness}
\end{figure}

The rules for the disjointness judgment are shown in
Figure~\ref{fig:disjointness}, which consists of two judgments.

\paragraph{Main Judgment.} The judgment $\jdisimpl \Gamma A B$ says
two types $A$ and $B$ are disjoint in a context
$\Gamma$.

The rules dealing with intersection types (\reflabel{\labeldisinterl}
and \reflabel{\labeldisinterr}) are quite intuitive. The intuition is
that if two types $A$ and $B$ are disjoint to some type $C$, then
their intersection ($A\&B$) is also clearly disjoint to $C$.  The
rules capture this intuition by inductively distributing the relation
itself over the intersection constructor ($\inter$).

The rule for functions \reflabel{\labeldisfun} is more interesting. It says that two function
types are disjoint if and only if their return types are disjoint (regardless of
their parameter types!). At first this rule may look surprising
because the parameter types play no role in the definition of
disjointness. To see the reason for this consider the two function types:
\[ \code{Int} \to \tystring \qquad \tybool \to \tystring \]
Even though their parameter types are disjoint, we are still able to think of a
type which is a supertype for both of them. For example, $ \code{Int} \inter \tybool
\to \tystring $. Therefore, by definition, the two function types with
the same return type are not
disjoint. Essentially, due to the contravariance of function types,
functions of the form $A \to C$ and $B \to C$ always have a common
supertype (for example $A \& B \to C$).
The lesson from this example is that the parameter types of two
function types do not have any influence in determining whether those two function
types are disjoint or not: only the return types matter.

% The rule for disjoint quantification (\reflabel{\labeldisforall}) . Consider the following two types:
% \[ (\forall (\alpha * \code{Int}).~\code{Int} \& \alpha) \qquad (\forall (\alpha * \tychar). ~\tychar \& \alpha) \]
% The question is under which conditions are those two types disjoint.
% In the first type $\alpha$ cannot be instantiated with $\code{Int}$ and in
% the second case $\alpha$ cannot be instantiated with $\tychar$.
% Therefore we can see that for the two bodies to be considered
% disjoint, $\alpha$ cannot be instantiated with either $\code{Int}$ or
% $\tychar$. The rule for disjoint quantification captures this fact by
% requiring the bodies of disjoint quantification to be checked for
% disjointness under both constraints.

\paragraph{Axioms.} Up till now, the rules of $ \jdisimpl \Gamma A B $ have only
taken care of two types with the same language constructs. But how can be the
fact that $\code{Int}$ and $\code{Int} \to \code{Int}$ are disjoint be decided?
That is exactly the place where the judgment $ A \disjointax B $ comes in handy.
It provides the axioms for disjointness. What is captured by the set of rules is
that $ A \disjointax B $ holds for all two types of different constructs unless
any of them is an intersection type. That is because for example, $ \code{Int}
\inter (\tychar \to \tychar) $ and $ \tychar \to \tychar $ use different
constructs and yet are not disjoint.\george{Recover axiom rules.}

\subsection{Metatheory}

The following two theorems together say that the algorithmic disjointness
judgment and the definition of disjointness are ``equivalent''. For detailed
proofs, we refer to the Coq code in our repository.

\begin{restatable}[Soundness of algorithmic disjointness]{theorem}{algodissoundness}
  \label{theorem:soundness}

  For any two types $A$ and $B$, $\jdisimpl \Gamma A B$ implies $\jdis \Gamma A B$.
\end{restatable}

\begin{proof}
  By induction on the derivation of $\jdisimpl \Gamma A B$.
\end{proof}

\begin{restatable}[Completeness of algorithmic disjointness]{theorem}{algodiscompleteness}
  \label{theorem:completeness}

  For any two types $A$, $B$, $\jdis \Gamma A B$ implies $\jdisimpl \Gamma A B$.
\end{restatable}

\begin{proof}
  By a case analysis on the shape of $A$ and $B$.
\end{proof}

\section{Disjoint Intersection Types with $\top$}

This section shows how to add a $\top$ type to \name.
%So far we have discussed a system with intersection types, but we have (intentionally) left out $\top$.
%The $\top$ type is not only a supertype of all types, but also the type for 0-ary intersection. 
%Unfortunately, introducing $\top$ into our system leads to several
%drastic consequences:
Introducing $\top$ poses some important challenges. Most prominently,
the simple definition of disjointness is useless in the presence of
$\top$. Since all types now have a common supertype, it is impossible
for any two types to satisfy a simple notion of disjointness. To
address this problem a notion of $\top$-disjointess is proposed.  The
definition of $\top$-disjointess depends on a notion of a top-like
type. We formalise two different variants of \name, based on two different
definitions of a top-like type, while discussing their usability and
limitations. Both variants retain coherence, and all other key
properties of \name. 
Mechanized Coq proofs for both variants are available as part of the 
supplementary materials for the paper.


%\begin{itemize}
%\item The definition of disjointness becomes useless:
%since every type has now a common supertype, it does not exist any pair of two types for which the disjointness relation
%holds. Thus, we can no longer build an intersection type.
%\item The subtyping relation becomes incoherent: its rules allow us to derive, for example, both 
%$\top \subtype \top \inter \top$ and $\top \inter \top \subtype \top$.
%\end{itemize}

\begin{comment}
However, by adjusting the definition of disjointness, we can obtain more useful results, while still preserving the
coherence of the subtyping relation.
Even though these are desirable properties, a new definition of disjointess can be formulated in many different ways. 
\joao{expand}

We will first show how we can introduce $\top$ into our system.
Next, a new definition of disjointness will be presented - which we will refer to as $\top$-disjointness - and
we will briefly discuss how it can make the resulting system more useful, while maintaining the correctness
of all properties proven for the original system.
The definition of $\top$-disjointess will depend on a notion of a top-like type, for which at first we will just provide 
some intuition for it.
We will then formalise two different systems, based on two different definitions of a top-like type, while discussing their
usability and limitations.
The first top-like definition will solve the problem of incoherence, but it will still forbid the inclusion of
function types within intersection types.
To face this limitation, we proceed by extending the top-like definition, and showing how the resulting system will become 
more expressive.
We also show, for both systems, how to adjust the algorithmic disjointness rules such that they become consistent with the 
respective disjointess specification.
\end{comment}

\subsection{Introducing $\top$}

\begin{figure}[t]
  \[
    \begin{array}{l}
      \begin{array}{llrll}
        \text{Types}
        & A, B, C & \Coloneqq & \ldots \mid \highlight{$\top$}  & \\

        \\
        \text{Terms}
        & e & \Coloneqq & \ldots \mid \highlight{$\top$} & \\
      \end{array}
    \end{array}
  \]

  \begin{mathpar}
    \formsub \\
    \rulesubtop 
  \end{mathpar}

  \begin{mathpar}
    \formwf \\
    \rulewftop
  \end{mathpar}

  \begin{mathpar}
    \formbi \\
    \rulettop
  \end{mathpar}

  \begin{mathpar}
    \framebox{$\im A = T$} \\
    \im \top = \unit \\
  \end{mathpar}

  \caption{Extending \name with $\top$.}
  \label{fig:fi-syntax-top}
\end{figure}

Introducing the $\top$ type in \name is a straightforward process, as shown in Figure \ref{fig:fi-syntax-top}.
We start by adding $\top$ to the existing types. 
Similarly, we add the only inhabitant of type $\top$: the term $\top$.
We extend the subtyping relation with \reflabel{\labelsubtop}, declaring that any type is a sub-type of $\top$.
The coercion in the target language, is a function that always returns the term $\top$, regardless of its argument.
We also add $\top$ to the set of well-formed types by extending the well-formedness relation with \reflabel{\labelwftop}. 
Finally, the typing rule \reflabel{\labelttop} states that, under type inference, the term $\top$ has type $\top$ 
and generates the term $\unit$ in the target language.

%However, as discussed in Section \ref{sec:overview}, the definition of disjointness used so far turns intersection types to
%be useless within this new system.
%Since every type now has at least one common supertype (namely, $\top$), it is impossible to create an intersection type.
%Thus, we need to reformulate that definition, as presented next. 

\subsection{Disjointness} 
As discussed in Section \ref{sec:overview}, the definition of
simple disjointness is useless when \name is extended with $\top$.
We opted to differentiate \emph{top-like} types from the rest of the types,
so that we may impose restrictions related to the former, since they are the 
root of the problems exposed.
For now, we will omit the formal definition of a \emph{top-like} type, and informally
define it as a type that resembles $\top$ in some way. 
Having this in mind, we defined $\top$-disjointness as follows:
%Even though we will show two different definitions of $\top$-disjointness, they share the same top-level definition, 
%presented next.
\begin{definition}[$\top$-Disjointness]
Given two types $A$ and $B$ we have that:

%\[A * B \equiv \not \toplike{A} \and \not \toplike{A} \and A \subtype B\]
\[A \disjoint_{\top} B \equiv \neg \toplike{A} \; \wedge \; \neg \toplike{B} \; \wedge \; 
\forall_C \; ( A \subtype C \wedge B \subtype C ) \rightarrow \toplike{C} \]

\end{definition}
\noindent where $\toplike{C}$ means that $C$ is a top-like type.
A top-like type is thus a unary relation at the type level, which we will further specialise into two different definitions. 
However, we can first provide an intuition about this general $\top$-disjointness definition, in two steps:
\begin{enumerate}
\item The concept of disjointness is inherently connected to the way we wish to form intersection types thus,
we explicitly state that both $A$ and $B$ cannot be top-like types (i.e. preventing types such as $T \inter T$ to be well-formed).
\item Additionaly, if there is any common supertype of $A$ and $B$, that is not top-like, then we want to forbid the 
intersection of these types, since there might be an overlap between them. 
\end{enumerate}

Next we will present a simple top-like type definition and show how resulting system
becomes more expressive while achieving coherence.
However, we still consider that system not to be expressive enough, 
so we will introduce later an extended definition of the top-like type relation.
We will refer to the first system as \emph{naive} and to the second as \emph{improved}. 

\subsection{A Naive Calculus with $\top$}

As a first definition of top-like, we aim to solve the aforementioned coherence problem of the subtyping relation.
With the original disjointness specification, we could prove that $\top \subtype \top \inter \top$ and $\top \inter \top \subtype \top$, leading to incoherence in our system. 
Also, this hints that the type system has more than one \emph{syntactical} $\top$,
meaning that any n-ary intersection composed by $\top$'s, is a super-type of all other
types.
This is obviously undesirable and we show how we could tackle this problem, using a
suitable top-like definition together with the $\top$-disjointness definition.

We first present the top-like definition, formalise the resulting system and  
show how the coherence problem is solved.
Then we will show how the original algorithmic disjointness rules can be modified to match the new $\top$-disjointness
definition.
Finally we will discuss the limitations of resulting system and motivate the need for a refined definition of the top-like 
relation. 

\paragraph{Top-Like Types}

A simple way to achieve coherence in the presented system with $\top$ is to forbid any intersection enclosing 
more than one $\top$.
This will ensure that types such as $\top \inter \top$ are not well-formed and thus ending with the inconsistency of
$\top \subtype \top \inter \top$ and $\top \inter \top \subtype \top$.
In other words, our system contains only one \emph{syntactic} $\top$. 
\joao{explain this better?}
For a similar reason, our system should not allow types of the form $A \inter \top$, where $A \neq \top$.  
Should our system accept these kind of intersections, then both $A \subtype A \inter \top$ and 
$A \inter \top \subtype A$ would hold (by \reflabel{\labelsubinter} and \reflabel{\labelsubinterl}, respectively).
Therefore we can describe the notion of a top-like type that, as a consequence, these kind of intersections become no longer well-formed.

\begin{figure}[h]
  \begin{mathpar}
    \formtoplike \\ %\framebox{$\jatomic A$} \\

    \ruletopltop \and \ruletoplinter

  \end{mathpar}
  \begin{mathpar}
    \formdis \\
    \ruledisinterl \and \ruledisinterr \\ 
    \ruledisatomic
  \end{mathpar}

  \begin{mathpar}
    \formax \\
    %\ruledisaxintfun \and \ruledisaxtop \and \ruleaxsym
    \ruledisaxintfun \and \ruleaxsym
  \end{mathpar}

  \caption{Top-like types and Algorithmic Disjointness.}
  \label{fig:tltypesdis}
\end{figure}

\begin{definition}[Top-like types]
  Any type consisting of a $n$-ary intersection composed of $n$ $\top$'s, is a top-like type. 
\end{definition}

A top-like type can be formalised as an unary relation on a type $A$, denoted as $\toplike{A}$, as show in 
Figure \ref{fig:tltypesdis}.
This definition resembles the previous description with two rules.
The rule \reflabel{\labeltopltop} states that $\top$ is a top-like type; 
the rule \reflabel{\labeltoplinter} indicates that any n-ary intersection composed of just $\top$'s is also a top-like type.

%Let us consider a term $t$ with type $Char \inter Int$ and a function $f$ with type $\top \rightarrow Char$.
%The application of $f$ to $x$ could lead to a choice between $Char$ and $Int$ and $Char \inter Int$.
%However, since for any type $A$, $\top \rightarrow A$ is isomorphic to $A$, the choice of the argument should
%not influence coherence. 
%Indeed, our system translates 
%Now, lets change the type of $f$ to $String \rightarrow \top$, and consider a second function $\g$, with type 
%$String \rightarrow Int$.

%Here such definition is made a bit more precise, and
%well-suited to \name.


%since their supertypes are all types with the form $A \to \top$ 
%(where A \emph{contains} \code{Int}, i.e. $\code{Int}$, $\code{Int} \inter \code{Char}$) and $\top$.


%At last, take as an example ($\top \inter \top$).
%In a pure language (i.e. no side-effects), this type can be safely allowed since both components of the merge
%will evaluate to the same value. 
%However, in an effectful language, the evaluation of either the right or left component might lead to distinct results. 
%Following the same reasoning, we might want to allow or restrict, for instance $(String \to Int) \inter (String \to \top)$,  

\paragraph{Algorithmic disjointness rules}

Again, the newly presented definition of $\top$-disjointness does not lead to an implementation. 
Fortunately, the algorithmic disjoitness rules remain the same as described in Section \ref{sec:sec:alg-dis}, 
except for \reflabel{\labeldisfun}.
The intuition for this becomes clear with an example.
Take both function types used as an example in Section \ref{sec:overview}: 
$\code{String} \to \code{Int}$ and $\code{String} \to \code{String}$. 
Should we have \reflabel{\labeldisfun} as a rule, then these two types would be disjoint. 
However, introducing $\top$ into into our system has the consequence of also introducing a new supertype for these 
two functions: $\code{String} \to \top$ (due to function type co-variance). 
This \emph{new} type is not a top-like type - according to the presented definitions - and thus violates one of the
conditions of $\top$-disjointness.

This is clearly a limitation imposed by our definition of $\top$-disjointness - or rather, definition of top-like type - 
for which we will present a solution in the next section.

\subsection{An Improved Calculus with $\top$}

The former definition of top-like types is, unfortunately, too restrictive.
Namely, function types are disallowed within intersection types, which is clearly a limitation from an expressivity
angle.
Let us consider, again, both function types $\code{String} \to \code{Int}$ and $\code{String} \to \code{String}$. 
In a system without $\top$, these two types would be disjoint.
However, according to our $\top$-disjointness definition, these two types are not disjoint in the naive system, since they have a 
common super-type which is not top-like:
%The intersection of these two types could be useful, for instance ... \joao{finish this example}
$\code{String} \to \top$.
This new supertype is a direct consequence of introducing $\top$ intro our system,
and we argue that it is acting also as a top-like type: it represents a function that produces
$\top$, no matter what argument it is given.
More generally, any type of the form $A_k \to \top$ (with $k \in \mathbb{N}$), should also be considered a top-like type.
Therefore, we will extend the definition of top-like type to include types of that form. 
This will introduce a new ambiguity in our subtyping rules, which will lead us to changing the coercions produced by some
of them.
Similarly to the naive system, we will also need to re-adjust the algorithmic disjointness rules to match the 
extended $\top$-disjointness definition. 

\paragraph{Top-Like Types}

\begin{figure}[h]
  \begin{mathpar}
    \formtoplike \\ %\framebox{$\jatomic A$} \\

    \ruletopltop \and \ruletoplinter \and \ruletoplfun
  \end{mathpar}

  \begin{mathpar}
    \formsub \\
    \rulesubinterlcoerce \and \rulesubinterrcoerce
  \end{mathpar}

  \begin{mathpar}
    \formdis \\
    \ruledisinterl \and \ruledisinterr \\ 
    \ruledisfun \and \ruledisatomic
  \end{mathpar}

  \begin{mathpar}
    \formax \\
    %\ruledisaxintfun \and \ruledisaxtop \and \ruleaxsym
    \ruledisaxintfunext \and \ruleaxsym
  \end{mathpar}

  \caption{Top-like types, Subtyping (changed rules only) and Algorithmic Disjointness for the improved calculus.}
  \label{fig:tltypesextdis}
\end{figure}

\begin{figure}[t]
  \framebox{$\andcoerce{A}_{C} = t$}
  \[
  \andcoerce{A}_{C} = 
  \begin{cases} 
        \toplike{A} & \andcoerce{A} \\ 
        %A = \top & \top \\
        %A = A_1 \to A_2 \; \wedge \; \toplike{A_2} & \lam x \andcoerce{A_2}_{C} \\
        \text{otherwise} & {C} 
  \end{cases}
  \]

  \framebox{$\andcoerce{A} = t$}
  \[
  \andcoerce{A} = 
  \begin{cases} 
        A = \top & \top \\
        A = A_1 \to A_2 & \lam x \andcoerce{A_2} \\
  \end{cases}
  \]


  %\begin{align*}
  %  \erase {i}                  &= i \\
  %  \erase {x}                  &= x \\
  %  \erase {\lam x E}           &= \lam x {\erase E} \\
  %  \erase {\app {e_1} {e_2}}   &= {\erase {e_1}} ({\erase {e_2}}) \\
  %  \erase {{x_1} ,, {x_2}}     &= {\erase {x_1}} \, ,, \, {\erase {x_2}} \\
  %  \erase {e : A}              &= {\erase e} \\
  %\end{align*}
  \caption{Coercion considering intersection of top-like types.}
  \label{fig:andcoercion}
\end{figure}

In relation to the previous definition of top-like, we extend it as follows:
\begin{definition}[Top-like types]
  
  A type $A$ is (also) a top-like type, if it has the form $A_k \to \top$, where $k \in \mathbb{N}$. 
  That is, any type with arity $k$ can be a top-like type, as long as $\top$ is the result type. 

\end{definition}

Now, according to our $\top$-disjointess definition,
$\code{String} \to \code{Int}$ and $\code{String} \to \code{String}$ are disjoint and their intersection is a 
well-formed type.
The extended top-like definitions and resulting system are formalised in Figure \ref{fig:tltypesextdis}.
Note how we just added \reflabel{\labeltoplfun} to the top-like relation, by stating that a function is top-like whenever
its return type is also top-like.
The rest of the changes will be discussed in the following sections.

%\joao{do we want to allow this?
% f: String $\to$ Int \\ 
% g: String $\to$ T \\
% ($\lam$ h : String $\to$ T. h "") (f ,, g) }

\paragraph{Coercive Subtyping}

In this improved calculus, a new problem arises when generating coercions. 
Introducing functions within intersection types leads to ambiguity between subtype contributors 
under intersection types.
In other words, Lemma ... no longer holds because, under some contexts, 
\reflabel{\labelsubinterl} and \reflabel{\labelsubinterr} now overlap. 
Let us demonstrate this using an example:
suppose that we want to build a derivation for  
$\code{Int} \to \code{Int} \inter \code{Char} \to \code{Char} \subtype (\code{Int} \inter \code{Char}) \to \top$
Then, we can either derive it to $\code{Int} \to \code{Int} \subtype (\code{Int} \inter \code{Char}) \to \top$ 
(using \reflabel{\labelsubinterl}), or to $\code{Char} \to \code{Char} \subtype (\code{Int} \inter \code{Char}) \to \top$
(using \reflabel{\labelsubinterr}), and thus introducing ambiguity in our system.
We could solve this problem at least in two distinct ways:

\begin{itemize}
\item Forbid intersection types that include more than one function type; or
\item Modify the subtyping relation rules. 
\end{itemize}

We could opt for the easy way and go with the former option. \joao{expand: modifying rules would be complicated}
However, taking a closer look to \reflabel{\labeldisfun} (the next step in one of the two derivations), reveals that both coercions generated will $\beta$-reduce to the same term. 
To illustrate this, suppose we are deriving $(A_k \to A) \inter (B_k \to B) \subtype C_k \to \top$.
The coercions generated will be either $\lambda y. E_1 (proj_1 y)$ and $\lambda y. E_2 (proj_2 y)$, depending which intersection rule we choose.
However, $E_1$ and $E_2$ will both be of the form $\lambda f_1. \lambda x_1. ... (\lambda f_k. \lambda x_k (\lambda e. \unit)...) (...)$ - a term that $\beta$-reduces to $\lambda f_1. \lambda x_1. ... \lambda x_k. ()$.
Performing substitution and one $\beta$ reduction step in either of the former 
coercions lead to the following term  
$\lambda y. \lambda x_1... \lambda x_k. \unit$.
%This is $\beta$-equivalent to a coercion function that accepts the source type 
%plus $C_k$ as arguments and returns $\top$. 
Intuitively, this means there is only one way of generating a $\unit$ term, regardless of the form of the argument(s).

So we opted to modify existing rules to reflect this observation, as shown in Figure \ref{fig:tltypesextdis}.
Namely, \reflabel{\labelsubinterl} and \reflabel{\labelsubinterr}, include a new premisse, ensuring that 
the coercion can be produced the original way, as long as $A_3$ is not top-like. 
On the other hand, in case that $A_3$ is top-like (and well-formed) then $A_3$ must be of the form $A_k \to \top$. 
With this assumption over $A_3$, we know that the coercion generated should be a nested number of $\lambda$ equal to $k+1$,
returning $\unit$ as the result, regardless of the parameters.
These leads to two new rules, \reflabel{\labelsubintertopl} and \reflabel{\labelsubintertopr}.
The reader might notice how they still overlap. 
If that is the case, then both rules can be used interchangeably as they both lead to the same coercion.
The rules of the subtyping relation therefore only suffered slight changes while retaining coherence.
%Next, we will modify the algorithmic disjointess rules, so that they match the improved top-like specification.

\paragraph{Algorithmic disjointness rules}

Fortunately, the algorithmic disjointness rules are, again, similar to the ones presented in the original system.
In relation to the naive system with $\top$ we placed back $\reflabel{\labeldisfun}$, since we lifted the restriction
of intersections with function types.
We also had to modify $\reflabel{\labeldisaxintfun}$ to include the premise $\neg \toplike{B}$.
This is due to the specification of $\top$-disjointness, which requires two types not to be top-like in order for them to be disjoint.

\joao{add a paragraph mentioning the possibility of using two different definitions of top-like in T-disjointness, so we can get a DisAx-Int-Fun with no premise?}

%\section{Design Space}

This section discusses some alternatives in the design-space.

\subsection{Disjointness of Functions}

Talk about the option of not allowing subtyping of function arguments. 
This should allow for a more flexible rule for disjointness of functions.
Maybe a good option for OO type systems, where methods are invariant 
with respect to subtyping of arguments. It would allow for static overloading, 
similar to what is present in conventional OO languages.

\begin{mathpar}
  \inferrule* [right=\labelsubfun]
    {{A_2} \subtype {B_2} }
    {{A_1 \to A_2} \subtype {A_1 \to B_2}}
\end{mathpar}

\begin{mathpar}
  \inferrule* [right=\labeldisfun]
    {\jdisimpl \Gamma {A_1} {B_1}}
    {\jdisimpl \Gamma {A_1 \to A_2} {B_1 \to B_2}}
\end{mathpar}

not exists E . A -> B <: E /\ C -> D <: E ->  

$Int -> Int \& Char -> Int$ (disjoint according to the spec and algorithmic rules)



Are those 2 functions disjoint? 

$ f,,g : Int -> Int \& Char -> Int$

Well, two things to consider:

1) what happens if they are applied:  

$(f,,g) (3,'c')$ 

well, a type-annotation will then select one of the functions. So this seems to be ok.

2) what happens if the functions are selected. I have two choices:

f,,g : Int -> Int

f,,g : Char -> Int 

$f,,g : Char\&Int -> Int$ (fails because subtyping of functions is invariant).



\subsection{Union Types}

\begin{lstlisting}
case 3,,'c' of
   Int -> 1
   Char -> 2 : Int
\end{lstlisting}

Here we have $Int \& Char <: Int | Char$, but this leads to ambiguity. The program can either 
be $1$ or $2$. 

Possible solution: require atomic constraints in or-rules, similar to the and-rules. 
Big Problem: subtyping is no longer transitive. Minor problem, type-system is incomplete.

\subsection{Parametric Polymorphism?}

In principle it should be easy to extend disjointness to parametric polymorphism. 
bruno{Show rules for parametric polymorphism}

However, such rules would be quite restrictive. Future work includes how to integrate 
parametric polymorphism is a more flexible way.  

\section{Related Work} \label{sec:related-work}

% \url{http://homepages.inf.ed.ac.uk/gdp/publications/Sub_Par.pdf}

% \cite{plotkin1994subtyping}

% Also discussed intersection types!~\cite{malayeri2008integrating}.

% Pierce Ph.D thesis: F<: + /|
%        technical report: F + /|, closer to ours

% \cite{barbanera1995intersection}

\paragraph{Intersection types with polymorphism.}
Our type system combines intersection types and parametric polymorphism. Closest
to us is Pierce's work~\cite{pierce1991programming1} on a prototype
compiler for a language with both intersection types, union types, and
parametric polymorphism. Similarly to \name in his system universal
quantifiers do not support bounded quantification. However Pierce did not try to prove any
meta-theoretical results and his calculus does not have a merge
operator.  Pierce also studied a system where both intersection
types and bounded polymorphism are present in his Ph.D.
dissertation~\cite{pierce1991programming2} and a 1997
report~\cite{pierce1997intersection}. Going in the direction of higher
kinds, Compagnoni and Pierce~\cite{compagnoni1996higher} added
intersection types to System $ F_{\omega} $ and used the new calculus,
$ F^{\omega}_{\wedge} $, to model multiple inheritance. In their
system, types include the construct of intersection of types of the
same kind $ K $. Davies and Pfenning
\cite{davies2000intersection} studied the interactions between
intersection types and effects in call-by-value languages. And they
proposed a ``value restriction'' for intersection types, similar to
value restriction on parametric polymorphism. Although they proposed a system with
parametric polymorphism, their subtyping rules are significantly different from ours,
since they consider parametric polymorphism
as the ``infinit analog'' of intersection polymorphism.
There have been attempts to provide a foundational calculus
for Scala that incorporates intersection
types~\cite{amin2014foundations,amin2012dependent}.
Although the minimal Scala-like calculus does not natively support
parametric polymorphism, it is possible to encode parametric
polymorphism with abstract type members. Thus it can be argued that
this calculus also supports intersection types and parametric
polymorphism. However, the type-soundness of a minimal Scala-like
calculus with intersection types and parametric polymorphism is not
yet proven. Recently, some form of intersection
types has been adopted in object-oriented languages such as Scala,
Ceylon, and Grace. Generally speaking,
the most significant difference to \name is that in all previous systems
there is no explicit introduction construct like our merge operator. As shown in
Section~\ref{sec:overview}, this feature is pivotal in supporting modularity
and extensibility because it allows dynamic composition of values.

\begin{comment}
only allow intersections of concrete types (classes),
whereas our language allows intersections of type variables, such as
\texttt{A \& B}. Without that vehicle, we would not be able to define
the generic \texttt{merge} function (below) for all interpretations of
a given algebra, and would incur boilerplate code:

\begin{lstlisting}{language=haskell}
let merge [A, B] (f: ExpAlg A) (g: ExpAlg B) = {
  lit (x : Int) = f.lit x ,, g.lit x,
  add (x : A & B) (y : A & B) =
    f.add x y ,, g.add x y
}
\end{lstlisting}
\end{comment}


\paragraph{Other type systems with intersection types.}
Intersection types date back to as early as Coppo et
al.~\cite{coppo1981functional}. As emphasized throughout the paper our
work is inspired by Dunfield~\cite{dunfield2014elaborating}. He described a similar approach to ours:
compiling a system with intersection types into ordinary $ \lambda $-calculus
terms. The major difference is that his system does not include parametric
polymorphism, while ours does not include unions. Besides, our rules are
algorithmic and we formalize a record system.
% Although similar in spirit,
% our elaboration typing is simpler: we require subtyping in the case of
% applications, thus avoiding the subsumption rule. Besides, our treatment
% combines the merge rules ($ k $ ranges over $ \{1, 2\} $)
% \inferrule
% {\Gamma \turns e_k : A}
% {\Gamma \turns e_1 \mergeop e_2 : A}
% and the standard intersection-introduction rule
% \inferrule
% {\Gamma \turns e : A_1 \andalso \Gamma \turns e : A_2}
% {\Gamma \turns e : A_1 \inter A_2}
% into one rule:
% \inferrule [Merge]
% {\Gamma \turns e_1 : A_1 \andalso \Gamma \turns e_2 : A_2}
% {\Gamma \turns e_1 \mergeop e_2 : A_1 \inter A_2}
Reynolds invented Forsythe~\cite{reynolds1997design} in the 1980s. Our merge
operator is analogous to his $ p_1, p_2 $. As Dunfield
has noted, in Forsythe merges can be only used unambiguously.
For instance, it is not allowed in Forsythe to merge two functions.

%Castagna, and Dunfield describe
%elaborating multi-fields records into merge of single-field records.
% Reynolds and Castagna do not consider elaboration and Dunfield do not
% formalize elaborating records.

% Both Pierce and Dunfield's system include a subsumption rule, which states that
% if an term has been inferred of type $ A $, then it is also of any
% supertype of $ A $. Our system does not have this rule.

Refinement
intersection~\cite{dunfield2007refined,davies2005practical,freeman1991refinement}
is the more conservative approach of adopting intersection types. It increases
only the expressiveness of types but not terms. But without a term-level
construct like ``merge'', it is not possible to encode various language
features. As an alternative to syntatic subtyping described in this paper,
Frisch et al.~\cite{frisch2008semantic} studied semantic subtyping.

\paragraph{Languages for extensibility.}
To improve support for extensibility various researchers have proposed
new OOP languages or programming mechanisms. It is interesting to
note that design patterns such as object algebras or modular visitors
provide a considerably different approach to extensibility when
compared to some previous proposals for language designs for
extensibility. Therefore the requirements in terms of type system
features are quite different.  One popular approach is \emph{family
  polymorphism}~\cite{Ernst01family}, which allows whole class hierarchies to be
captured as a family of classes. Such a family can be later reused to
create a derived family with potentially new class members, and
additional methods in the existing classes.  \emph{Virtual
  classes}~\cite{ernst2006virtual} are a concrete realization of this idea, where a
container class can hold nested inner \emph{virtual} classes (forming
the family of classes). In a subclass of the container class, the
inner classes can themselves be \emph{overriden}, which is why they
are called virtual. There are many language mechanisms that provide
variants of virtual classes or similar mechanisms~\cite{McDirmid01Jiazzi,Aracic06CaesarJ,Smaragdakis98mixin,nystrom2006j}. The work by
Nystrom on \emph{nested intersection}~\cite{nystrom2006j} uses a
form of intersection types to support the composition of
families of classes. Ostermann's \emph{delegation layers}~\cite{Ostermann02dynamically}
use delegation for doing dynamic composition in a system
with virtual classes. This in contrast with most other approaches
that use class-based composition, but closer to the dynamic
composition that we use in \name.
\begin{comment}
In contrast to type systems for virtual classes
and similar mechanisms, the goal of our work is to study the type
systems and basic language mechanism to better support such design patterns.
 some researchers have designed new type
system features such as virtual classes~\cite{ernst2006virtual}, polymorphic
variants~\cite{garrigue1998programming}, while others have shown employing
programming pattern such as object algebras~\cite{oliveira2012extensibility} by
using features within existing programming languages. Both of the two approaches
have drawbacks of some kind. The first approach often involves heavyweight
designs, while the second approach still sacrifices the readability for
extensibility.
\bruno{fill me in with more details and more references!}
\end{comment}
% Intersection types have been shown to be useful in designing languages that
% support modularity.~\cite{nystrom2006j}

% \paragraph{Extensible records.}

%\george{Record field deletion is also possible.}

% http://elm-lang.org/learn/Records.elm

% Encoding records using intersection types appeared in
% Reynolds~\cite{reynolds1997design} and Castagna et
% al.~\cite{castagna1995calculus}. Although Dunfield also discussed this idea in
% his paper \cite{dunfield2014elaborating}, he only provided an implementation but
% not a formalization. Very similar to our treatment of elaborating records is
% Cardelli's work~\cite{cardelli1992extensible} on translating a calculus, named
% $ F_{\subtype \rho}$, with extensible records to a simpler calculus that without
% records primitives (in which case is $ F_{\subtype} $). But he did not consider
% encoding multi-field records as intersections; hence his translation is more
% heavyweight. Crary~\cite{crary1998simple} used intersection types and
% existential types to address the problem that arises when interpreting method
% dispatch as self-application. But in his paper, intersection types are not used
% to encode multi-field records.

% Wand~\cite{wand1987complete} started the work on extensible records and proposed
% row types~\cite{wand1989type} for records. Cardelli and
% Mitchell~\cite{cardelli1990operations} defined three primitive operations on
% records that are similar to ours: \emph{selection}, \emph{restriction}, and
% \emph{extension}. The merge operator in \name plays the same role as extension.
% Following Cardelli and Mitchell's approach,
% of restriction and extension. Both Leijen's systems~\cite{leijen2004first,leijen2005extensible}
% and ours allow records that contain
% duplicate labels. Leijen's system is more sophisticated. For example, it supports
% passing record labels as arguments to functions. He also showed an encoding of
% intersection types using first-class labels.

% Chlipala's
% \texttt{Ur}~\cite{chlipala2010ur} explains record as type level
% constructs.\bruno{What is the point of citing Chlipala's paper?}

% Our system can be adapted to simulate systems that support extensible
% records but not intersection of ordinary types like \texttt{Int} and
% \texttt{Float} by allowing only intersection of record types.
%
% $ \turnsrec A $ states that $ A $ is a record type, or the intersection of
% record types, and so forth.
%
% \inferrule [RecBase] {} {\turnsrec \recordType l A}
%
% \inferrule [RecStep]
% {\turnsrec A_1 \andalso \turnsrec A_2}
% {\turnsrec A_1 \inter A_2}
%
% \inferrule [Merge']
% {\Gamma \turns e_1 : A_1 \yields {E_1} \andalso \turnsrec A_1 \\
%  \Gamma \turns e_2 : A_2 \yields {E_2} \andalso \turnsrec A_2}
% {\Gamma \turns e_1 \mergeop e_2 : A_1 \inter A_2 \yields {\pair {E_1} {E_2}}}
%
% Of course our approach has its limitation as duplicated labels in a record are
% allowed. This has been discussed in a larger issue by
% Dunfield~\cite{dunfield2014elaborating}.
%
% R{\'e}my~\cite{remy1989type}

\section{Conclusion and Future Work}
\label{sec:conclusion}

This paper described \name: a System $F$-based language that combines
intersection types, parametric polymorphism and a merge operator.
The language is proved to be type-safe and coherent.
To ensure coherence the type system accepts only
disjoint intersections. To provide flexibility in the presence of parametric polymorphism,
universal quantification is extended with
disjointness constraints. We believe that disjoint intersection types
and disjoint quantification are intuitive, and at the same time
flexible enough to enable practical applications.

%We implemented the core functionalities of the \namedis as part of a JVM-based
%compiler. Based on the type system of \namedis, we have built an ML-like
%source language compiler that offers interoperability with Java (such as object
%creation and method calls). The source language is loosely based on the more
%general System $F_{\omega}$ and supports a
%number of other features, including records, mutually recursive
%\code{let} bindings, type aliases, algebraic data types, pattern matching, and
%first-class modules that are encoded using \code{letrec} and records.

For the future, we intend to create a prototype-based statically typed
source language based on \name.  We are also interested in extending
our work to systems with union types and a $\bot$ type. Union types
are also widely used in languages such as Ceylon or Flow, but
preserving coherence in the presence of union types is
challenging. The naive addition of $\bot$ seems to be problematic. 
The proofs for \name rely on the invariant that a type variable $\alpha$ can never be disjoint 
to another type that contains $\alpha$. The addition of $\bot$ breaks
this invariant, allowing us to derive, for example, $\jdis \Gamma
\alpha \alpha$.
Finally, we could study a similar system with implicit polymorphism.
Such system would require some changes in the subtyping and disjointness relations.
For instance, subtyping should allow  
${\for \alpha {\alpha \to \alpha}} \subtype \tyint \to \tyint$.
Consequently, the disjointness relation would have to be modified,
since valid statements in \name such as 
$\jdis \Gamma {\for \alpha {\alpha \to \alpha}} {\tyint \to \tyint}$ 
would no longer hold under the more powerful subtyping relation. 


\section*{Acknowledgments}
We would like to thank the ICFP reviewers for their helpful comments.
This work has been sponsored by the Hong Kong Research Grant Council Early Career Scheme project number 27200514.

%\newpage
\bibliographystyle{abbrvnat}
\bibliography{references}

%\clearpage
%\onecolumn
%
%\appendix
%\section{Target Type System}

\begin{figure}[h]
  \framebox{$ \jtype G E T $}
  \begin{mathpar}

    \ruletargetvar

    \ruletargetlam

    \ruletargetapp

    \ruletargetblam

    \ruletargettapp

    \ruletargetpair

    \ruletargetprojl

    \ruletargetprojr

  \end{mathpar}

  \caption{Target type system.}
\end{figure}

%\section{Proofs}

\begin{lemma}[\rulelabelget~rules produce type-correct coercion]
  If $ \judgeget \tau l {\tau_1} \yields C $, then $ \epsilon \turns C :
  \im \tau \to \im {\tau_1} $.
\end{lemma}

\begin{proof}
  By induction of the derivation.

  \begin{itemize}

  \item \textbf{Case}
    \begin{flalign*}
      & \rulegetelab &
    \end{flalign*}

    \begin{tabular}{ll}
      $ \epsilon \turns \Lam x {\im {\RecTy l \tau}} x : \im {\RecTy l \tau} \to \im {\RecTy l \tau} $ & By $\rulelabeltlam$ and $\rulelabeltvar$ \\
      $ \epsilon \turns \Lam x {\im {\RecTy l \tau}} x : \im {\RecTy l \tau} \to \im \tau $ & By the definition of $\im \cdot$
    \end{tabular} \\

  \item \textbf{Case}
    \begin{flalign*}
      & \rulegetleftelab &
    \end{flalign*}

    \begin{tabular}{ll}
      $\epsilon, x \hast \im {\tau_1 \Intersect \tau_2} \turns x : \im {\tau_1 \Intersect \tau_2}$ & By $\rulelabeltvar$ \\
      $\epsilon, x \hast \im {\tau_1 \Intersect \tau_2} \turns x : \Pair {\im {\tau_1}} {\im {\tau_2}}$ & By the definition of $\im \cdot$ \\
      $\epsilon, x \hast \im {\tau_1 \Intersect \tau_2} \turns \Proj 1 x : \im {\tau_1}$ & By $\rulelabeltprojleft$ \\
      $\epsilon \turns C : \im {\tau_1} \to \im \tau$ & By i.h. \\
      $\epsilon, x \hast \im {\tau_1 \Intersect \tau_2} \turns C : \im
      {\tau_1} \to \im \tau$ & \george{What should this be called?} \\
      $\epsilon, x \hast \im {\tau_1 \Intersect \tau_2} \turns \App C (\Proj 1 x) : \im {\tau}$ & By $\rulelabeltapp$ \\
      $\epsilon \turns \Lam x {\im {\tau_1 \Intersect \tau_2}} {C (\Proj 1 x)} : \im {\tau_1 \Intersect \tau_2} \to \im {\tau}$ & By $\rulelabeltlam$
    \end{tabular} \\

  \item \textbf{Case}
    \begin{flalign*}
      & \rulegetrightelab &
    \end{flalign*}

    By symmetry with the above case. \\

\end{itemize}
\end{proof}


\begin{lemma}[\rulelabelput~rules produce type-correct coercion]
  If $ \judgeput \tau l {\tau_1 \yields E} {\tau_2} {\tau_3} \yields C $ and $
  \Gamma \turns E : \im {\tau_1} $ for some $ \Gamma $, then
  $ \Gamma \turns C : \im \tau \to \im {\tau_2} $.
\end{lemma}

\begin{proof}
  By induction of the derivation.

  \begin{itemize}

  \item \textbf{Case}
    \begin{flalign*}
      & \ruleputelab &
    \end{flalign*}

    \begin{tabular}{ll}
      $ \Gamma \turns \Lam \_ {\im {\RecTy l \tau}} E : \im {\RecTy l \tau} \to
      \im {\tau_1} $ & By $\rulelabeltlam$, $\rulelabeltvar$, and the hypothesis \\
    \end{tabular} \\

  \item \textbf{Case}
    \begin{flalign*}
      & \ruleputleftelab &
    \end{flalign*}

    \begin{tabular}{ll}
      $\Gamma, x \hast \im {\tau_1 \Intersect \tau_2} \turns x : \im {\tau_1 \Intersect \tau_2} $ & By $\rulelabeltvar$ \\
      $\Gamma, x \hast \im {\tau_1 \Intersect \tau_2} \turns x : \Pair {\im {\tau_1}} {\im {\tau_2}} $ & By the definition of $\im \cdot$ \\
      $\Gamma, x \hast \im {\tau_1 \Intersect \tau_2} \turns \Proj 1 x : \im {\tau_1} $ & By $\rulelabeltprojleft$ \\
      $\Gamma \turns C : \im {\tau_1} \to \im {\tau_3}$ & By i.h. \\ 
      $\Gamma, x \hast \im {\tau_1 \Intersect \tau_2} \turns C : \im {\tau_1} \to \im {\tau_3}$ & \george{Really?} \\ 
      $\Gamma, x \hast \im {\tau_1 \Intersect \tau_2} \turns \App C {(\Proj 1 x)} : \im {\tau_3} $ & By $\rulelabeltapp$ \\
      $\Gamma \turns \Lam x {\im {\tau_1 \Intersect \tau_2}} {\App C {(\Proj 1 x)}} : \im {\tau_1 \Intersect \tau_2} \to \im {\tau_3} $ & By $\rulelabeltlam$ \\
    \end{tabular} \\

  \item \textbf{Case}
    \begin{flalign*}
      & \ruleputrightelab &
    \end{flalign*}

    By symmetry with the above case. \\

\end{itemize}
\end{proof}



\begin{proof}
By structural induction on the types and the corresponding inference rule. \\

\rulename{SubVar}

\rulename{SubFun}

\rulename{SubForall}

\rulename{SubAnd1}

\rulename{SubAnd2}

\rulename{SubAnd3}

\rulename{SubRcd}

\end{proof}

\begin{lemma}
  If $$ \Gamma \turnsput \tau ; l ; E = C ; \tau_1 $$
  then $$ \im \Gamma \turns C : \im \tau \to \im \tau $$
\end{lemma}

\begin{proof}
By structural induction on the type and the corresponding inference rule. \\

\rulename{Put-Base} \\
\rulename{Put-Left} \\
\rulename{Put-Right} \\
\end{proof}

\begin{lemma} \label{preserve-wf}
  If   $$ \Gamma \turns \tau $$
  then $$ \im \Gamma \turns \im \tau $$
\end{lemma}

\begin{proof}
Since $$ \Gamma \turns \tau $$
It follows from \rulename{FI-WF} that
  $$ \ftv \tau  \subseteq \ftv {\Gamma} $$
And hence
  $$ \ftv {\im \tau} \subseteq \ftv {\im \Gamma} $$
By \rulename{F-WF} we have
  $$ \Gamma \turns \tau $$
\end{proof}

\begin{theorem}[Type preserving translation]
  If   $$ \Gamma \turns e : \tau \yields E  $$
  then $$ \im \Gamma \turns E : \im \tau $$
\end{theorem}

\begin{proof}
By structural induction on the expression and the corresponding inference rule. \\

\rulename{Var} $ \Gamma \turns x : \tau \yields x $ \\

It follows from \rulename{Var} that
  $$ (x : t) \in \Gamma $$
Based on the definition of $ \im \cdot $,
  $$ (x : \im t) \in \im \Gamma $$
Thus we have by \rulename{F-Var} that
  $$ \im \Gamma \turns x : \im \tau $$

\rulename{Abs} $ \Gamma \turns \lambda (x : \tau_1). e : \tau_1 \to \tau_2 \yields {\Lam x {\im {\tau_1}} E} $ \\

It follows from \rulename{Abs} that
  $$ \Gamma, x : \tau_1 \turns e : \tau_2 \yields E $$
And by the induction hypothesis that
  $$ \im \Gamma, x : \im {\tau_1} \turns E : \im {\tau_2} $$
By \rulename{Abs} we also have
  $$ \Gamma \turns \tau_1 $$
It follows from Lemma \ref{preserve-wf} that
  $$ \im \Gamma \turns \im {\tau_1} $$
Hence by \rulename{F-Abs} and the definition of $ \im \cdot $ we have
  $$ \im \Gamma \turns \Lam x {\im {\tau_1}} E : \im {\tau_1 \to \tau_2} $$

\rulename{(TrApp)} $ \Gamma \turns \App {e_1} {e_2} : \tau_2 \yields {E_1 (\App C {E_2})} $ \\

From \rulename{(TrApp)} we have
  $$ \Gamma \turns \tau_3 <: \tau_1 \yields C $$
Applying Lemma \ref{type-coerce} to the above we have
  $$ \im \Gamma \turns C : \im {\tau_3} \to \im {\tau_1} $$
Also from \rulename{(TrApp)} and the induction hypothesis
  $$ \im \Gamma \turns E_1 : \im {\tau_1} \to \im {\tau_2} $$
Also from \rulename{(TrApp)} and the induction hypothesis
  $$ \im \Gamma \turns E_2 : \im {\tau_3} $$
Assembling those parts using \rulename{(F-App)} we come to
  $$ \im \Gamma \turns E_1 (\App C {E_2}) : \im {\tau_2} $$
\end{proof}

\rulename{TAbs} $ \Gamma \turns \Lambda \alpha. e : \forall \alpha. \tau \yields {\forall \alpha. E} $ \\

From \rulename{TAbs} we have
  $$ \Gamma \turns e : \tau \yields E $$
By the induction hypothesis we have
  $$ \im \Gamma \turns E : \im \tau $$
Thus by \rulename{F-TAbs} and the definition of $ \im \cdot $
  $$ \Gamma \turns \Lambda \alpha. E : \im {\forall \alpha. \tau} $$


\rulename{TAp} $ \Gamma \turns e \; \tau  : \subst \tau \alpha  \tau_1 \yields {E \; \im \tau} $ \\

From \rulename{TApp} we have
  $$ \Gamma \turns e : \forall \alpha. \tau_1 \yields E $$
And by the induction hypothesis that
  $$ \im \Gamma \turns E : \forall \alpha. \im {\tau_1} $$
Also from \rulename{TApp} and Lemma \ref{preserve-wf} we have
  $$ \im \Gamma \turns \im \tau $$
Then by \rulename{F-TApp} that
  $$ \im \Gamma \turns E \; \im \tau : \subst {\im \tau} \alpha \im {\tau_1} $$
Therefore
  $$ \im \Gamma \turns E \; \im \tau : \im {\subst \tau \alpha \im {\tau_1}} $$

% \rulename{(TrMerge)} $ \Gamma \turns e_1 \merge e_2 : \tau_1 \& \tau_2 % % \yields {\Pair {E1} {E2} $ \\

From \rulename{(TrMerge)} and the induction hypothesis we have
  $$ \im \Gamma \turns E_1 : \im {\tau_1} $$
and
  $$ \im \Gamma \turns E_2 : \im {\tau_2} $$
Hence by \rulename{F-Pair}
  $$ \im \Gamma \turns \Pair {E_1} {E_2} : \Pair {\im {\tau_1}} {\im {\tau_2}} $$
Hence by the definition of $ \im \cdot $
  $$ \im \Gamma \turns \Pair {E_1} {E_2} : \im {\tau_1 \& \tau_2} $$

\rulename{RecIntro} $ \Gamma \turns \RecCon l e : \RecTy l \tau \yields E $ \\

From \rulename{RcdIntro} we have
  $$ \Gamma \turns e : \tau \yields E $$
And by the induction hypothesis that
  $$ \im \Gamma \turns E : \im \tau $$
Thus by the definition of $ \im \cdot $
  $$ \im \Gamma \turns E : \im {\RecTy l \tau} $$

\rulename{RcdElim} $ \Gamma \turns e.l : \tau_1 \yields {\App C E} $ \\

From \rulename{RcdElim}
  $$ \Gamma \turns e : \tau \yields E $$
And by the induction hypothesis that
  $$ \im \Gamma \turns E : \im \tau $$
Also from \rulename{RcdEim}
  $$ \Gamma \turnsget e ; l = C ; \tau_1 $$
Applying Lemma \ref{type-get} to the above we have
  $$ \im \Gamma \turns C : \im \tau \to \im {\tau_1}  $$
Hence by \rulename{F-App} we have
  $$ \im \Gamma \turns \App C E : \im {\tau_1} $$

From \rulename{RcdUpd}
  $$ \Gamma \turns e : \tau \yields E $$
And by the induction hypothesis that
  $$ \im \Gamma \turns E : \im \tau $$
Also from \rulename{RcdUpd}
  $$ \Gamma \turnsput \tau ; l; E = C ; \tau_1 $$
Applying Lemma \ref{type-put} to the above we have
  $$ \im \Gamma \turns C : \im \tau \to \im \tau  $$
Hence by \rulename{F-App} we have
  $$ \im \Gamma \turns \App C E : \im \tau $$



\end{document}
