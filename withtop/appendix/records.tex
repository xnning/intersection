\subsection{Record System}

This section describes the record system used in the examples of this paper. We
first describe the basic operations along with the novelty in our system and
then discuss the extension of disjointness to record types.

\paragraph{Operations.}

We define three operations with records: \emph{construction},
\emph{selection}, and \emph{exclusion}. The construction operation builds a record literal such as

\begin{lstlisting}
{x = 3, y = 4, z = 12}
\end{lstlisting}

\noindent The selection operation allows retrieving a specific field (which must
be present) from a record, such as

\begin{lstlisting}
point.x
\end{lstlisting}

\noindent The exclusion operation, denoted by a backslash, removes a field from
a record. For example, the following code

\begin{lstlisting}
{x = 3, y = 4, z = 12} \ z
\end{lstlisting}

\noindent shows how to remove the $z$-coordinate from a three-dimensional point. More formally, the syntax is

\[
e \setminus l
\]

\noindent where $l$ denotes a field name. Since the type of $e$ is statically
known and $l$ is just a name, errors can be detected during typechecking. A type
error is reported unless $e$ is of a record type where field $l$ is present.

One interesting aspect of our system is that two records can be merged using
our merge operator to form a bigger record. For example, we identify

\begin{lstlisting}
{x = 3} ,, {y = 4}
\end{lstlisting}

\noindent with

\begin{lstlisting}
{x = 3, y = 4}
\end{lstlisting}

In fact, in the intermediate representation of our implementation, multi-field
records are regarded just as merges of single-field records. Similarly, we
regard multi-field record types as intersections of single-field record types.
For example,

\begin{lstlisting}
{x: Int, y: Int}
\end{lstlisting}

\noindent is just

\begin{lstlisting}
{x: Int} & {y: Int}
\end{lstlisting}

\paragraph{Disjointness.}

Of course, such merge and intersection is not arbitrary: the two parts must be
``disjoint'' in some sense. Otherwise, nondeterministic behavior can still
occur. For instance, we are reluctant to merge \lstinline${x = 3}$ and
\lstinline${x = 4}$ to form \lstinline${x = 3, x = 4}$, since it is not very
clear what \lstinline${x = 3, x = 4}.x$ should evaluate to.

We define two record types to be disjoint if the intersection of the field names
is empty. This is actually a stronger condition than what is derived from our
specification of disjointness, but it removes the ambiguity of field selection.
For example, \lstinline${x: Int}$ and \lstinline${x: String}$ are disjoint
according to the specification since we cannot find a type that is a supertype
of both. But still, we would like to reject a merge of terms with those two
types because the resulting term will contain duplicate fields.
