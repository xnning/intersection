\subsection{Desugaring of Traits}

This section provides a more general explanation of the idea of desugaring that
appeared in Section~\ref{sec:trait}.

\paragraph{Trait Declarations.} Method parameters, denoted by \lstinline$...$,
are ``pulled right'' to become a tupled parameter of lambdas. In the case that
the method parameter list is empty, it becomes an empty tuple type, which turns
out to be the unit type. To account for the call-by-value semantics of the
target language, the type of the self reference, \lstinline$A0$, is replaced by
$\code{()} \to \code{A0}$ during the desugaring and the occurrences of the self
reference \lstinline$s$ are replaced by \lstinline$s ()$, denoted by
\lstinline$[s / s ()]$.

\begin{lstlisting}
trait T(x1: A1, ..., xn: An) { s: A0 (*$ \to $*)
  m1(...) = e1
  ...
  mn(...) = en
} in ...

(*$\hookrightarrow$*)

let T = (*$ \lambda $*)(x1: A1, ..., xn: An) (s: () (*$ \to $*) A0) (*$ \to $*) {
  m1 = [s / s ()] (*$ \lambda $*)(...) (*$ \to $*) e1,
  ...
  mn = [s / s ()] (*$ \lambda $*)(...) (*$ \to $*) en
} in ...
\end{lstlisting}

\paragraph{Instantiation of Traits.} Our source language allows creating a
single object from one or more traits becomes an
inlined fixpoint, where \lstinline$(E1,...,En)$ are trait names with their
required constructor parameters fully supplied. The self reference is passed to
\lstinline$E1 & ... & En$, as each of them is an open term in the target
language. The closed items are then merged using the \lstinline$,,$ operator.
The unit type and unit literal in the translation result (both written as
\lstinline$()$) are added, purely due to the fact that the target language is
call-by-value and we need them to prevent premature evaluation.

\begin{lstlisting}
new[T] (E1 & ... & En)

(*$\hookrightarrow$*)

let rec self : () (*$ \to $*) T = (*$ \lambda $*)(_: ()) (*$ \to $*) ((E1 self) ,, ... ,, (En self))
in self ()
\end{lstlisting}

\paragraph{The Type for Traits.} \lstinline$Trait[T]$ denotes the type for those
traits which provide an interface described by the type \lstinline$T$. In fact,
it is just like a type constructor except for the fact that it is built-in in
the language. In this way, the encoding is not exposed to the user of the source
language.

\begin{lstlisting}
Trait[T]

(*$\hookrightarrow$*)

(() (*$ \to $*) T) (*$ \to $*) T
\end{lstlisting}
