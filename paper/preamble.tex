% Remote packages

% For pdflatex, replaced by fontspec:
% \usepackage{tgpagella}
\usepackage[T1]{fontenc}
\usepackage[utf8]{inputenc}

% For xelatex or lualatex
% \usepackage{fontspec}
% \setmainfont{Times New Roman}

\usepackage{amsmath}
\usepackage{amsthm}
\usepackage{amssymb}
\usepackage{mathtools} % For \Coloneqq
\usepackage{bm}        % Bold symbols in maths mode
\usepackage{fixltx2e}
\usepackage{stmaryrd}
\usepackage[dvipsnames]{xcolor}
\usepackage{listings} % For code listings
% \usepackage{minted}
% \usemintedstyle{murphy}
\usepackage{fancyvrb}
\usepackage{url}
\usepackage{xspace}
\usepackage{comment}
\usepackage{mdwlist}

% Typography
\usepackage[euler-digits,euler-hat-accent]{eulervm}

% Copied from the FCore paper:
\usepackage[colorlinks=true,allcolors=black,breaklinks,draft=false]{hyperref}   % hyperlinks, including DOIs and URLs in bibliography
% known bug: http://tex.stackexchange.com/questions/1522/pdfendlink-ended-up-in-different-nesting-level-than-pdfstartlink

% Figures with borders
% http://en.wikibooks.org/wiki/LaTeX/Floats,_Figures_and_Captions
% \usepackage{float}
% \floatstyle{boxed}
% \restylefloat{figure}

% Local packages

\usepackage{styles/bcprules}    % by Benjamin C. Pierce
\usepackage{styles/cmll}
\usepackage{styles/mathpartir}  % by Didier Rémy


% ! Always load mathastext last
% http://mirrors.ibiblio.org/CTAN/macros/latex/contrib/mathastext/mathastext.pdf
% \renewcommand\familydefault\ttdefault
% \usepackage{mathastext}
% \renewcommand\familydefault\rmdefault

% http://tex.stackexchange.com/questions/114151/how-do-i-reference-in-appendix-a-theorem-given-in-the-body
\usepackage{thmtools, thm-restate}

\newtheorem{theorem}{Theorem}
\newtheorem{lemma}{Lemma}

% Define macros immediately before the \begin{document} command
% General
\newcommand{\code}[1]{\texttt {#1}}
\newcommand{\highlight}[1]{\colorbox{GreenYellow}{#1}}

% Logic
\newcommand{\turns}{\vdash}

% Math
\newcommand{\im}[1]{\lvert #1 \rvert}

% PL
\newcommand{\subst}[2]{\lbrack #1 / #2 \rbrack}
\newcommand{\concatOp}{+\kern-1.3ex+\kern0.8ex}  % http://tex.stackexchange.com/a/4195/73122

% Constructors
\newcommand{\for}[2]{\forall #1. \, #2}
% \newcommand{\lam}[2]{\lambda #1. \, #2}
\newcommand{\app}[2]{#1 \; #2}
\newcommand{\blam}[2]{\Lambda #1. #2}
\newcommand{\tapp}[2]{#1 \; #2}

\newcommand{\pair}[2]{(#1, #2)}
\newcommand{\proj}[2]{{\code{proj}}_{#1} #2}

\newcommand{\recordType}[2]{\{ #1 : #2 \}}
\newcommand{\recordCon}[2]{\{ #1 = #2 \}}

\newcommand{\ifThenElse}[3]{\code{if} \; #1 \; \code{then} \; #2 \; \code{else} \; #3}
% math-mode versions of \rlap, etc
% from Alexander Perlis, "A complement to \smash, \llap, and lap"
%   http://math.arizona.edu/~aprl/publications/mathclap/
\def\clap#1{\hbox to 0pt{\hss#1\hss}}
\def\mathllap{\mathpalette\mathllapinternal}
\def\mathrlap{\mathpalette\mathrlapinternal}
\def\mathclap{\mathpalette\mathclapinternal}
\def\mathllapinternal#1#2{\llap{$\mathsurround=0pt#1{#2}$}}
\def\mathrlapinternal#1#2{\rlap{$\mathsurround=0pt#1{#2}$}}
\def\mathclapinternal#1#2{\clap{$\mathsurround=0pt#1{#2}$}}

% math-mode versions of \rlap, etc
% from Alexander Perlis, "A complement to \smash, \llap, and lap"
%   http://math.arizona.edu/~aprl/publications/mathclap/
\def\clap#1{\hbox to 0pt{\hss#1\hss}}
\def\mathllap{\mathpalette\mathllapinternal}
\def\mathrlap{\mathpalette\mathrlapinternal}
\def\mathclap{\mathpalette\mathclapinternal}
\def\mathllapinternal#1#2{\llap{$\mathsurround=0pt#1{#2}$}}
\def\mathrlapinternal#1#2{\rlap{$\mathsurround=0pt#1{#2}$}}
\def\mathclapinternal#1#2{\clap{$\mathsurround=0pt#1{#2}$}}

\newcommand{\horizontalrule}{
  \begin{center}
    \line(1,0){250}
  \end{center}
}

\newcommand{\code}[1]{\texttt{#1}}
\definecolor{light-gray}{gray}{0.9}
\newcommand{\highlight}[1]{\colorbox{light-gray}{#1}}

\newcommand{\im}[1]{\lvert #1 \rvert}
\newcommand{\powerset}[1]{\mathcal{P}(#1)}
\newcommand{\universe}{\mathbb{U}}

\newcommand{\turns}{\vdash}
\newcommand{\oftype}{\!:\!}
\newcommand{\subtype}{<:}
\newcommand{\commonsuper}{\Uparrow}
\newcommand{\commonsub}{\Downarrow}
\newcommand{\disjoint}{*}
\newcommand{\disjointimpl}{*_\textnormal{i}}
\newcommand{\disjointax}{*_\textnormal{ax}}
\newcommand{\subst}[2]{\lbrack #2 := #1 \rbrack~}

\newcommand{\yields}[1]{\highlight{$\; \hookrightarrow #1$}}
% \newcommand{\yields}[1]{}

\newcommand{\ftv}[1]{\textsf{ftv}(#1)}

\newcommand{\binderspace}{\,}
\newcommand{\appspace}{\;}

\newcommand{\inter}{\&}
\newcommand{\union}{|}
\newcommand{\for}[2]{\forall #1.\binderspace #2}
\newcommand{\fordis}[3]{\for {(#1 \disjoint #2)} {#3}}
\newcommand{\lam}[2]{\lambda #1.\binderspace #2}
\newcommand{\lamty}[3]{\lam {(#1 \oftype #2)} #3}
\newcommand{\blam}[2]{\Lambda #1.\binderspace #2}
\newcommand{\blamdis}[3]{\blam {(#1 \disjoint #2)} #3}
\newcommand{\mergeop}{,,}
\newcommand{\app}[2]{#1 \; #2}
\newcommand{\tapp}[2]{#1 \appspace #2}
\newcommand{\pair}[2]{(#1, #2)}
\newcommand{\proj}[2]{{\code{proj}}_{#1} #2}
\newcommand{\fst}[1]{\app {\code{fst}} {#1}}
\newcommand{\snd}[1]{\app {\code{snd}} {#1}}
\newcommand{\recordType}[2]{\{ #1 : #2 \}}
\newcommand{\recordCon}[2]{\{ #1 = #2 \}}

\newcommand{\true}{\code{True}}
\newcommand{\tyint}{\code{Int}}
\newcommand{\tybool}{\code{Bool}}
\newcommand{\tychar}{\code{Char}}
\newcommand{\tystring}{\code{String}}

% Judgements
\newcommand{\jwf}[2]{#1 \turns #2}
\newcommand{\jatomic}[1]{#1 \ \textnormal{atomic}}
\newcommand{\jtype}[3]{\turns #2 \ \oftype \ #3}
\newcommand{\jdis}[3]{\turns #2 \disjoint #3}
\newcommand{\jdisimpl}[3]{\turns #2 \disjointimpl #3}

\newcommand{\reflabel}[1]{(\textsc{#1})}

% \newcommand{\name}{$ \lambda_{\&} $\xspace}
\newcommand{\name}{\xspace[name]\xspace}

\newcommand{\authornote}[3]{\textcolor{#2}{\textsc{#1}: #3}}
\newcommand\bruno[1]{\authornote{bruno}{Red}{#1}}
\newcommand\george[1]{\authornote{george}{Blue}{#1}}


\colorlet{orthogcolor}{BrickRed}

\newcommand{\orthog}{\perp}

\newcommand{\rulelabelorthog}{\bm{\textcolor{orthogcolor}{orthog}}}
\newcommand{\rulelabelorthogvar}{\rulelabelorthog\text{var}}
\newcommand{\rulelabelorthogtop}{\rulelabelorthog\text{top}}
\newcommand{\rulelabelorthogfun}{\rulelabelorthog\text{fun}}
\newcommand{\rulelabelorthogforall}{\rulelabelorthog\text{forall}}
\newcommand{\rulelabelorthogandleft}{\rulelabelorthog{\text{and}_1}}
\newcommand{\rulelabelorthogandright}{\rulelabelorthog{\text{and}_2}}
\newcommand{\rulelabelorthogreclab}{\rulelabelorthog\text{reclab}}
\newcommand{\rulelabelorthogrec}{\rulelabelorthog\text{rec}}

\newcommand{\ruleorthogvar}{
\inferrule* [right=$\rulelabelorthogvar$]
  {\alpha_1 \neq \alpha_2}
  {\alpha_1 \orthog \alpha_2}
}

\newcommand{\ruleorthogfun}{
\inferrule* [right=$\rulelabelorthogfun$]
  {\tau_3 \orthog \tau_1 \andalso \tau_2 \orthog \tau_4}
  {\tau_1 \to \tau_2 \orthog \tau_3 \to \tau_4}
}

\newcommand{\ruleorthogforall}{
\inferrule* [right=$\rulelabelorthogforall$]
  {\tau_1 \orthog \subst {\alpha_1} {\alpha_2} \tau_2}
  {\for {\alpha_1} \tau_1 \orthog \for {\alpha_2} \tau_2}
}

\newcommand{\ruleorthogandleft}{
\inferrule* [right=$\rulelabelorthogandleft$]
  {\tau_1 \orthog \tau_3 \andalso \tau_2 \orthog \tau_3}
  {\tau_1 \andop \tau_2 \orthog \tau_3}
}

\newcommand{\ruleorthogandright}{
\inferrule* [right=$\rulelabelorthogandright$]
  {\tau_1 \orthog \tau_2 \andalso \tau_1 \orthog \tau_3}
  {\tau_1 \orthog \tau_2 \andop \tau_3}
}

\newcommand{\ruleorthogreclab}{
\inferrule* [right=$\rulelabelorthogreclab$]
  {l_1 \neq l_2}
  {\recty {l_1} {\tau_1} \orthog \recty {l_2} {\tau_2}}
}

\newcommand{\ruleorthogrec}{
\inferrule* [right=$\rulelabelorthogrec$]
  {\tau_1 \orthog \tau_2}
  {\recty l {\tau_1} \orthog \recty l {\tau_2}}
}
\newcommand{\formsub}{\framebox{$ A \subtype B \yields E $}}

\newcommand{\makelabelsub}[1]{Sub\_#1}

\newcommand{\labelsubint}{\makelabelsub Int}
\newcommand{\rulesubint}{
  \inferrule* [right=\labelsubint]
    { }
    {\tyint \subtype \tyint \yields {\lamty x {\im \alpha} x}}
}

\newcommand{\labelsubtop}{\makelabelsub Top}
\newcommand{\rulesubtop}{
  \inferrule* [right=\labelsubtop]
    { }
    {A \subtype \top \yields {\lamty x {\im A} \unit}}
}

\newcommand{\labelsubvar}{\makelabelsub Var}
\newcommand{\rulesubvar}{
  \inferrule* [right=\labelsubvar]
    { }
    {\alpha \subtype \alpha \yields {\lamty x {\im \alpha} x}}
}

\newcommand{\labelsubfun}{\makelabelsub Fun}
\newcommand{\rulesubfun}{
  \inferrule* [right=\labelsubfun]
    {{B_1} \subtype {A_1} \yields {E_1} \\
     {A_2} \subtype {B_2} \yields {E_2}}
    {{A_1 \to A_2} \subtype {B_1 \to B_2}
    \yields
        {\lamty f {\im {A_1 \to A_2}}
        {\lamty x {\im {B_1}}
            {\app {E_2} {(\app f {(\app {E_1} x)})}}}}}
}

\newcommand{\labelsubforall}{\makelabelsub Forall}
\newcommand{\rulesubforall}{
  \inferrule* [right=\labelsubforall]
    {{A_1} \subtype {A_2} \yields E}
    {\for {\alpha} {A_1} \subtype \for {\alpha} {A_2}
      \yields
        {\lamty f {\im {\for {\alpha} {A_1}}}
          {\blam \alpha {\app E {(\app f \alpha)}}}}}
}

\newcommand{\rulesubforalldis}{
  \inferrule* [right=\labelsubforall]
    {{B_1} \subtype {B_2} \yields E}
    {\fordis {\alpha} A {B_1} \subtype \fordis {\alpha} A {B_2}
      \yields
        {\lamty f {\im {\for {\alpha} {B_1}}}
          {\blam \alpha {\app E {(\app f \alpha)}}}}}
}

\newcommand{\rulesubforallext}{
  \inferrule* [right=\labelsubforall]
    {{B_1} \subtype {B_2} \yields E \\
     {A_2} \subtype {A_1} \yields {E_d}}
    {\fordis {\alpha} {A_1} {B_1} \subtype \fordis {\alpha} {A_2} {B_2}
      \yields
        {\lamty f {\im {\for {\alpha} {B_1}}}
          {\blam \alpha {\app E {(\app f \alpha)}}}}}
}

\newcommand{\labelsubinter}{\makelabelsub Inter}
\newcommand{\rulesubinter}{
  \inferrule* [right=\labelsubinter]
    {{A_1} \subtype {A_2} \yields {E_1} \\
     {A_1} \subtype {A_3} \yields {E_2}}
    {{A_1} \subtype {A_2 \inter A_3}
      \yields
        {\lamty x {\im {A_1}}
          {\pair {\app {E_1} x} {\app {E_2} x}}}}
}

\newcommand{\labelsubinterl}{\makelabelsub {Inter\_1}}
\newcommand{\rulesubinterl}{
  \inferrule* [right=\labelsubinterl]
    {{A_1} \subtype {A_3} \yields E}
    {{A_1 \inter A_2} \subtype {A_3}
      \yields
        {\lamty x {\im {A_1 \inter A_2}}
          {\app E {(\proj 1 x)}}}}
}

\newcommand{\rulesubinterldis}{
\inferrule* [right=\labelsubinterl]
    {{A_1} \subtype {A_3} \yields E \\ \jatomic {A_3}}
    {{A_1 \inter A_2} \subtype {A_3}
      \yields
        {\lamty x {\im {A_1 \inter A_2}}
          {\app E {(\proj 1 x)}}}}
}

\newcommand{\labelsubinterr}{\makelabelsub {Inter\_2}}
\newcommand{\rulesubinterr}{
  \inferrule* [right=\labelsubinterr]
    {{A_2} \subtype {A_3} \yields E}
    {{A_1 \inter A_2} \subtype {A_3}
      \yields
        {\lamty x {\im {A_1 \inter A_2}}
          {\app E {(\proj 2 x)}}}}
}

\newcommand{\rulesubinterrdis}{
  \inferrule* [right=\labelsubinterr]
    {{A_2} \subtype {A_3} \yields E \\ \jatomic {A_3}}
    {{A_1 \inter A_2} \subtype {A_3}
      \yields
        {\lamty x {\im {A_1 \inter A_2}}
          {\app E {(\proj 2 x)}}}}
}

\newcommand{\rulesubinterlcoerce}{
  \inferrule* [right=\labelsubinterl]
    {{A_1} \subtype {A_3} \yields E \\ \jatomic {A_3} }
    {{A_1 \inter A_2} \subtype {A_3}
      \yields
        { \lamty x {\im {A_1 \inter A_2}} {\andcoerce{A_3}_{{(
          {\app E {(\proj 1 x)}})}}}}}
}

\newcommand{\rulesubinterrcoerce}{
  \inferrule* [right=\labelsubinterr]
    {{A_2} \subtype {A_3} \yields E \\ \jatomic {A_3} }
    {{A_1 \inter A_2} \subtype {A_3}
      \yields 
        { \lamty x {\im {A_1 \inter A_2}} {\andcoerce{A_3}_{{(
          {\app E {(\proj 2 x)}})}}}}}
}

\newcommand{\rulelabel}{\text{Ty}}
\newcommand{\rulelabelSelect}{\text{Sel}}
\newcommand{\rulelabelRestrict}{\text{Res}}

% Var
\newcommand{\rulelabelVar}{\rulelabel\text{Var}}
\newcommand{\ruleVar} {
\inferrule* [right=$\rulelabelVar$]
  {x \hast A \in \Gamma}
  {\hastype \Gamma x A \yields x}
}

% Top
\newcommand{\rulelabelTop}{\rulelabel\text{Top}}
\newcommand{\ruleTop} {
\inferrule* [right=$\rulelabelTop$]
  { }
  {\hastype \Gamma \top \top \yields {()}}
}

% Lam
\newcommand{\rulelabelLam}{\rulelabel\text{Lam}}
\newcommand{\ruleLam} {
\inferrule* [right=$\rulelabelLam$]
  {\istype \Gamma A \\ \hastype {\Gamma, x \hast A} e B \yields E}
  {\hastype \Gamma {\lam x A e} {A \to B} \yields {\lam x {\im A} E}}
}

% App
\newcommand{\rulelabelApp}{\rulelabel\text{App}}
\newcommand{\ruleApp}{
\inferrule* [right=$\rulelabelApp$]
  {\hastype \Gamma {e_1} {A_1 \to A_2} \yields {E_1} \\
   \hastype \Gamma {e_2} {A_3} \yields {E_2} \\
   A_3 \subtype A_1 \yields C}
  {\hastype \Gamma {\app {e_1} {e_2}} {A_2} \yields {\app {E_1} {(\app C E_2)}}}
}

% BLam
\newcommand{\rulelabelBLam}{\rulelabel\text{BLam}}
\newcommand{\ruleBLam}{
\inferrule* [right=$\rulelabelBLam$]
  {\hastype {\Gamma, \alpha * B} e A \yields E}
  {\hastype \Gamma {\blam {\alpha * B} e} {\for {\alpha * B} A} \yields {\blam \alpha E}}
}

% TApp
\newcommand{\rulelabelTApp}{\rulelabel\text{TApp}}
\newcommand{\ruleTApp}{
\inferrule* [right=$\rulelabelTApp$]
  {\hastype \Gamma e {\for {\alpha * C} B} \yields E \\ \isdisjoint
  \Gamma A C \\\istype \Gamma A}
  {\hastype \Gamma {\tapp e A} {\subst A \alpha B} \yields {\tapp E {\im A}}}
}

% Merge
\newcommand{\rulelabelMerge}{\rulelabel\text{Merge}}
\newcommand{\ruleMerge}{
\inferrule* [right=$\rulelabelMerge$]
  {\hastype \Gamma {e_1} A \yields {E_1} \\
   \hastype \Gamma {e_2} B \yields {E_2} \\
   % A \bot B}
   \isdisjoint \Gamma A B}
  {\hastype \Gamma {e_1 \mergeOp e_2} {A \intersect B} \yields {\pair {E_1} {E_2}}}
}

% ConstraintIntro
\newcommand{\rulelabelConstraintIntro}{\rulelabel\text{ConstraintIntro}}
\newcommand{\ruleConstraintIntro}{
  \inferrule* [right=$\rulelabelConstraintIntro$]
    {\istype \Gamma {A_1} \\ \istype \Gamma {A_2} \\
     \hastype {\Gamma, A_1 \disjoint A_2} e B \yields E}
    {\hastype \Gamma {\assume {(A_1 \disjoint A_1)} e} {\constraints {A_1   \disjoint A_2} B} \yields E}
}

% ConstraintElim
\newcommand{\rulelabelConstraintElim}{\rulelabel\text{ConstraintElim}}
\newcommand{\ruleConstraintElim}{
\inferrule* [right=$\rulelabelConstraintElim$]
  {\hastype \Gamma e {\constraints {A_1 \disjoint A_2} B} \yields E \\
  \isdisjoint \Gamma {A_1} {A_2}}
  {\hastype \Gamma {\app e {\_}} B \yields E}
}

% rec-con
\newcommand{\rulelabelRecConstruct}{\rulelabel\text{rec-construct}}
\newcommand{\rulerecordConstruct}{
\inferrule* [right=$\rulelabelRecConstruct$]
  {\hastype \Gamma e A \yields E}
  {\hastype \Gamma {\recordCon l e} {\recordType l A} \yields E}
}

% rec-select
\newcommand{\rulelabelRecSelect}{\rulelabel\text{rec-select}}
\newcommand{\ruleRecSelect}{
\inferrule* [right=$\rulelabelRecSelect$]
  {\hastype \Gamma e A \yields E \\
   \judgeSelect A l {A_1} \yields C}
  {\hastype \Gamma {e.l} {A_1} \yields {\app C E}}
}

% rec-restrict
\newcommand{\rulelabelRecRestrict}{\rulelabel\text{rec-restrict}}
\newcommand{\ruleRecRestrict}{
\inferrule* [right=$\rulelabelRecRestrict$]
  {\hastype \Gamma e A \yields E \\
   \judgeRestrict A l {A_1} \yields C}
  {\hastype \Gamma {e \restrictOp l} {A_1} \yields {\app C E}}
}

\newcommand{\judgeSelect}[3]{#1 \bullet #2 = #3}

% select
\newcommand{\ruleGet}{
  \inferrule* [right=$\rulelabelSelect$]
  { }
  {\judgeSelect {\recordType l A} l A \yields {\lam x {\im {\recordType l A}} x}}
}

% select1
\newcommand{\rulelabelSelectLeft}{{\rulelabelSelect}_1}
\newcommand{\ruleGetLeft}{
  \inferrule* [right=$\rulelabelSelectLeft$]
  {\judgeSelect {A_1} l A \yields C}
  {\judgeSelect {A_1 \intersect A_2} l A \yields {\lam x {\im {A_1
          \intersect A_2}} {\app C {(\proj 1 x)}}}}
}

% select2
\newcommand{\rulelabelSelectRight}{{\rulelabelSelect}_2}
\newcommand{\ruleGetRight}{
  \inferrule* [right=$\rulelabelSelectRight$]
  {\judgeSelect {A_2} l A \yields C}
  {\judgeSelect {A_1 \intersect A_2} l A \yields {\lam x {\im {A_1
          \intersect A_2}} {\app C {(\proj 2 x)}}}}
}

\newcommand{\judgeRestrict}[3]{#1 \bm{\restrictOp} #2 = #3}

% restrict
\newcommand{\ruleRestrict}{
  \inferrule* [right=$\rulelabelRestrict$]
  { }
  {\judgeRestrict {\recordType l A} l \top \yields {\lam x {\im {\recordType l A}} {()}}}
}

% restrict1
\newcommand{\rulelabelRestrictleft}{{\rulelabelRestrict}_1}
\newcommand{\ruleRestrictLeft}{
  \inferrule* [right=$\rulelabelRestrictleft$]
  {\judgeRestrict {A_1} l A \yields C}
  {\judgeRestrict {A_1 \intersect A_2} l {A \intersect A_2} \yields {\lam x {\im {A_1
          \intersect A_2}} {\pair {\app C {(\proj 1 x)}} {\proj 2 x}}}}
}

% restrict2
\newcommand{\rulelabelRestrictRight}{{\rulelabelRestrict}_2}
\newcommand{\ruleRestrictRight}{
  \inferrule* [right=$\rulelabelRestrictRight$]
  {\judgeRestrict {A_2} l A \yields C}
  {\judgeRestrict {A_1 \intersect A_2} l {A_1 \intersect A} \yields {\lam x {\im {A_1
          \intersect A_2}} {\pair {\proj 1 x} {\app C {(\proj 2 x)}}}}}
}

\newcommand{\makelabeltgt}[1]{Tgt\_#1}

% Target WF
\newcommand{\formtgtwf}{\framebox{$ \jwf G T $}}

\newcommand{\makelabeltgtwf}[1]{\makelabeltgt {WF\_#1}}

\newcommand{\labeltgtwffv}{\makelabeltgtwf FV}
\newcommand{\ruletgtwffv} {
\inferrule* [right=\labeltgtwffv]
  {\ftv T \in G}
  {\jwf G T}
}

% Target typing
\newcommand{\formtgt}{\framebox{$ \jtype G E T $}}

\newcommand{\makelabeltgtty}[1]{\makelabeltgt {Ty\_#1}}

\newcommand{\labeltgtvar}{\makelabeltgtty Var}
\newcommand{\ruletgtvar} {
\inferrule* [right=\labeltgtvar]
  {(x,T) \in \Gamma}
  {\jtype \Gamma x T}
}

\newcommand{\labeltgtlam}{\makelabeltgtty Lam}
\newcommand{\ruletgtlam} {
\inferrule* [right=\labeltgtlam]
  {\jtype {\Gamma, x \oftype T} E {T_1} \andalso \jwf \Gamma T}
  {\jtype \Gamma {\lamty x T E} {T \to T_1}}
}

\newcommand{\labeltgtapp}{\makelabeltgtty App}
\newcommand{\ruletgtapp}{
\inferrule* [right=\labeltgtapp]
  {\jtype \Gamma {E_1} {T_1 \to T_2} \andalso \jtype \Gamma {E_2} {T_1}}
  {\jtype \Gamma {\app {E_1} {E_2}} {T_2}}
}

\newcommand{\labeltgtblam}{\makelabeltgtty BLam}
\newcommand{\ruletgtblam}{
\inferrule* [right=\labeltgtblam]
  {\jtype {\Gamma, \alpha} E T}
  {\jtype \Gamma {\blam \alpha E} {\for \alpha T}}
}

\newcommand{\labeltgttapp}{\makelabeltgtty TApp}
\newcommand{\ruletgttapp}{
\inferrule* [right=\labeltgttapp]
  {\jtype \Gamma E {\for \alpha {T_1}} \andalso \jwf \Gamma T}
  {\jtype \Gamma {\tapp E T} {\subst T \alpha T_1}}
}

\newcommand{\labeltgtpair}{\makelabeltgtty Pair}
\newcommand{\ruletgtpair}{
\inferrule* [right=\labeltgtpair]
  {\jtype \Gamma {E_1} {T_1} \andalso \jtype \Gamma {E_2} {T_2}}
  {\jtype \Gamma {\pair {E_1} {E_2}} {\pair {T_1} {T_2}}}
}

\newcommand{\labeltgtprojl}{\makelabeltgtty {Proj\_1}}
\newcommand{\ruletgtprojl}{
\inferrule* [right=\labeltgtprojl]
  {\jtype \Gamma E {\pair {T_1} {T_2}}}
  {\jtype \Gamma {\proj 1 E} {T_1}}
}

\newcommand{\labeltgtprojr}{\makelabeltgtty {Proj\_2}}
\newcommand{\ruletgtprojr}{
\inferrule* [right=\labeltgtprojr]
  {\jtype \Gamma E {\pair {T_1} {T_2}}}
  {\jtype \Gamma {\proj 2 E} {T_2}}
}

\begin{mathpar}
  \framebox{$ \judgeSourceWF \gamma \tau $}

  \ruleSourceWFVar

  \ruleSourceWFTop

  \ruleSourceWFFun

  \ruleSourceWFForall

  \ruleSourceWFAnd

  \ruleSourceWFRec
\end{mathpar}

\newcommand{\name}{{\bf $F_{\&}$}\xspace}
%%\newcommand{\Name}{{\bf fi}}

\newcommand{\target}{{\bf f}\xspace}
\newcommand{\Target}{{\bf f}\xspace}

\newcommand{\authornote}[3]{{\color{#2} {\sc #1}: #3}}
% \newcommand\bruno[1]{\authornote{bruno}{red}{#1}}
% \newcommand\george[1]{\authornote{george}{blue}{#1}}
\newcommand\bruno[1]{}
\newcommand\george[1]{}

\lstdefinelanguage{scala}{
  morekeywords={abstract,case,catch,class,def,%
    do,else,extends,false,final,finally,%
    for,if,implicit,import,match,mixin,%
    new,null,object,override,package,%
    private,protected,requires,return,sealed,%
    super,this,throw,trait,true,try,%
    type,val,var,while,with,yield},
  otherkeywords={=>,<-,<\%,<:,>:,\#,@},
  sensitive=true,
  morecomment=[l]{//},
  morecomment=[n]{/*}{*/},
  morestring=[b]",
  morestring=[b]',
  morestring=[b]"""
}

\lstdefinelanguage{F2J}{
  morekeywords={let,rec,type,in},
  otherkeywords={->},
  sensitive=true,
  morecomment=[l]{--},
  morestring=[b]", % 'b' means inside a string delimiters are escaped by a backslash.
  morestring=[b]'
}

\lstset{ %
  % language=F2J,                % choose the language of the code
  columns=flexible,
  lineskip=-1pt,
  basicstyle=\ttfamily\small,       % the size of the fonts that are used for the code
  numbers=none,                   % where to put the line-numbers
  stepnumber=1,                   % the step between two line-numbers. If it's 1 each line will be numbered
  numbersep=5pt,                  % how far the line-numbers are from the code
  backgroundcolor=\color{white},  % choose the background color. You must add \usepackage{color}
  showspaces=false,               % show spaces adding particular underscores
  showstringspaces=false,         % underline spaces within strings
  showtabs=false,                 % show tabs within strings adding particular underscores
  tabsize=2,                  % sets default tabsize to 2 spaces
  captionpos=none,                   % sets the caption-position to bottom
  breaklines=true,                % sets automatic line breaking
  breakatwhitespace=false,        % sets if automatic breaks should only happen at whitespace
  title=\lstname,                 % show the filename of files included with \lstinputlisting; also try caption instead of title
  escapeinside={(*}{*)},          % if you want to add a comment within your code
  keywordstyle=\ttfamily\bfseries,
% commentstyle=\color{Gray}
% stringstyle=\color{Green}
}
