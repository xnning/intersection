\documentclass[preprint]{sigplanconf}

% Use packages immediately following the \documentclass command
\usepackage{amsmath}
\usepackage{amsthm}
\usepackage{fixltx2e}
\usepackage{listings}
\usepackage{mdframed}
\usepackage{xcolor}

\newmdtheoremenv{theorem}{Theorem}
\newmdtheoremenv{lemma}{Lemma}

\definecolor{github}{HTML}{4183C4}

% Define macros immediately before the \begin{document} command
\newcommand{\yields}[1] {\textcolor{github}{\; \hookrightarrow #1}}
\newcommand{\meta}[1]   {{\rm #1 }}
\newcommand{\ftv}[1]    {\meta{ftv}(#1)}
\newcommand{\merge}{,\!,}
\newcommand{\rcdupd}[3]{#1 \; \textrm{with} \; \{#2 = #3\}}


\newcommand{\FI}{{\bf FI} }
\newcommand{\F}{{\bf F} }

\newcommand{\FIend}{{\bf FI}}
\newcommand{\Fend}{{\bf F}}

% "define" code highlights for Java and Scala
\lstdefinelanguage{JavaScala}{
  morekeywords={public,int,interface,implements,default,
    abstract,case,catch,class,def,%
    do,else,extends,false,final,finally,%
    for,if,implicit,import,match,mixin,%
    new,null,object,override,package,%
    private,protected,requires,return,sealed,%
    super,this,throw,trait,true,try,%
    type,val,var,while,with,yield},
  otherkeywords={=>,<-,<\%,<:,>:,\#,@},
  sensitive=true,
  morecomment=[l]{//},
  morecomment=[n]{/*}{*/},
  morestring=[b]",
  morestring=[b]',
  morestring=[b]"""
}

\lstset{ %
language=Haskell,                % choose the language of the code
columns=flexible,
lineskip=-1pt,
basicstyle=\ttfamily\small,       % the size of the fonts that are used for the code
numbers=none,                   % where to put the line-numbers
numberstyle=\ttfamily\tiny,      % the size of the fonts that are used for the line-numbers
stepnumber=1,                   % the step between two line-numbers. If it's 1 each line will be numbered
numbersep=5pt,                  % how far the line-numbers are from the code
backgroundcolor=\color{white},  % choose the background color. You must add \usepackage{color}
showspaces=false,               % show spaces adding particular underscores
showstringspaces=false,         % underline spaces within strings
showtabs=false,                 % show tabs within strings adding particular underscores
morekeywords={var},
%  frame=single,                   % adds a frame around the code
tabsize=2,                  % sets default tabsize to 2 spaces
captionpos=none,                   % sets the caption-position to bottom
breaklines=true,                % sets automatic line breaking
breakatwhitespace=false,        % sets if automatic breaks should only happen at whitespace
title=\lstname,                 % show the filename of files included with \lstinputlisting; also try caption instead of title
escapeinside={(*}{*)},          % if you want to add a comment within your code
keywordstyle=\ttfamily\bfseries,
% commentstyle=\color{Gray},
% stringstyle=\color{Green}
}

\begin{document}

\special{papersize=8.5in,11in}
\setlength{\pdfpageheight}{\paperheight}
\setlength{\pdfpagewidth}{\paperwidth}

\title{\FI}

\authorinfo{Name1}
           {Affiliation1}
           {Email1}
\authorinfo{Name2\and Name3}
           {Affiliation2/3}
           {Email2/3}

\maketitle

\begin{abstract}
\end{abstract}

\keywords
intersecion types, inheritance

\section{Introduction}

There has been a remarkable number of works aimed at improving support
for extensibility in programming languages. These works include:
visions of new programming models~\cite{}; new programming languages or
language extensions~\cite{}, and \emph{design patterns} that can be
used with existing mainstream languages~\cite{}.

%\cite{family polymorphism and virtual
%classes.}. Another line of work are proposals for precise formal models or new 
%programming languages. Yet another line are \emph{design patterns}
%that can be used  with existing mainstream languages. 
%%Part of the motivation behind 

Some of the more recent work on extensibility is focused on various
proposals for design patterns.  Examples include \emph{Object
  Algebras}~\cite{}, \emph{Modular Visitors}~\cite{} or
Torgersen's~\cite{} four design patterns using generics. In those
approaches the idea is to use some advanced (but already available)
features, such as \emph{generics}, in combination with conventional
OOP features to model more extensible designs.  Those designs work in
modern OOP languages such as Java, C\# or Scala.

Although such design patterns give practical benefits in terms of
extensibility, they also expose limitations in existing mainstream OOP
languages. In particular there are three pressing limitations: 
1) lack of good mechanisms for
  \emph{object-level} composition; 2) \emph{conflation of 
    (type) inheritance with subtyping}; 3) \emph{heavy reliance on generics}.

  The first limitation shows up, for example, in Oliveira et
  al.~\cite{} encodings of Feature-Oriented Programming using Object
  Algebras~\cite{}. These programs are best expressed using a form of
  \emph{type-safe}, \emph{dynamic}, \emph{delegation}-based
  composition. Although such form of composition can be encoded in
  languages like Scala, it requires the use of low-level reflection
  techniques, such as dynamic proxies, reflection or other forms of
  meta-programming~\cite{}. It is clear that better language support
  would be desirable.

  The second limitation shows up in designs for modelling
  modular or extensible visitors~\cite{}.  The vast majority of modern
  OOP languages combines type inheritance and subtyping. 
  That is a type extension induces a subtype. However
  as Cook et al.~\cite{} famously argued there are programs where
  ``\emph{subtyping is not inheritance}''. Interestingly previously
  not many practical programs have been reported in the literature
  where the distinction between subtyping and inheritance is
  relevant. However, as shown in this paper, it turns out that this
  difference does show up in practice when designing modular
  (extensible) visitors.  We believe that modular visitors provide a
  compeling practical example where inheritance and subtyping should
  not be conflated!

  Finally, the third limitation is prevalent in many extensible
  designs~\cite{}. Such designs rely on advanced features of generics,
  such as \emph{f-bounded polymorphism}~\cite{}, \emph{variance
    annotations}~\cite{}, \emph{wildcards}~\cite{} and/or \emph{higher-kinded
    types}~\cite{} to achieve type-safety. Sadly, the amount of
  type-annotations, combined with the lack of understanding of these
  features, usually deters programmers from using such designs.

\begin{comment}
Motivated by the insights gained in previous work, this paper presents 
a minimal core calculus that addresses current limitations and
provides a better foundational model for statically typed
delegation-based OOP? We show that Object Algebras fit nicely in this
model. 
\end{comment}

This paper presents System \name: an extension of System F~\cite{}
with intersection types and a merge operator~\cite{}.  
The goal of System \name is to study the \emph{minimal} foundational language
constructs that are needed to support various extensible designs,
while at the same time addressing the limitations of existing OOP
languages. To address the lack of good object-level composition
mechanisms, System \name uses the merge operator to allow dynamic
composition of values/objects. Moreover, in System \name (type-level)
extension is independent of subtyping, and it is possible for an
extension to be a supertype of a base object type.


Technically speaking System \name is mainly inspired by the work of
Dundfield~\cite{}.  Dundfield shows how to model a simply typed
calculus with intersection types and a merge operator. The presence of
a merge operator adds significant expressiveness to the language,
allowing encodings for many other language constructs as syntactic
sugar. System \name differs from Dundfield's work in a few
ways. Firstly it adds parametric polymorphism and formalizes a
extension for records to support a basic form of objects. Secondly,
the elaboration semantics into System F is done directly from the
source calculus with subtyping. In contrast Dunfield has an additional
step which eliminates subtyping.  Finally a non-technical difference
is that System \name is aimed at studying issues of OOP languages and
extensibility, whereas Dunfield's work was aimed at Functional
Programming and he did not consider applications to extensibility.
Like many other foundational formal models for OOP (for
example~\cite{}), System \name is purely functional and it uses
structural typing.

System \name is
formalized and implemented. Furthermore the paper illustrates how
various extensible designs can be encoded in System \name.

\begin{comment}
We present a polymorphic calculus containing intersection types and records, and show
how this language can be used to solve various common tasks in functional
programming in a nicer way.Intersection types provides a power mechanism for functional programming, in
particular for extensibility and allowing new forms of composition.

Prototype-based programming is one of the two major styles of object-oriented
programming, the other being class-based programming which is featured in
languages such as Java and C\#. It has gained increasing popularity recently
with the prominence of JavaScript in web applications. Prototype-based
programming supports highly dynamic behaviors at run time that are not possible
with traditional class-based programming. However, despite its flexibility,
prototype-based programming is often criticized over concerns of correctness and
safety. Furthermore, almost all prototype-based systems rely on the fact that
the language is dynamically typed and interpreted.
\end{comment}

In summary, the contributions of this paper are:

\begin{itemize}

\item {\bf A Minimal Core Language for Extensibility:} This paper
  identifies a minimal core language, System \name, capable of
  expressing various extensibility designs in the literature.
  System \name also addresses limitations of existing OOP
  languages that complicate extensible designs. 
  
\item {\bf Formalization of System \name:} An elaboration semantics of
  System \name into System F is given, and type-soundness is proved.

\item {\bf Encodings of Extensible Designs:} Various encodings of
  extensible designs into System \name, including \emph{Object
    Algebras} and \emph{Modular Visitors}. 

\item {\bf Implementation and Examples:} An implementation of an
  extension of System \name, as well as the examples presented in the
  paper, are publicly available. 

\begin{comment}

\item{elaboration typing rules which given a source expression with intersection
    types, typecheck and translate it into an ordinary F term. Prove a type
    preservation result: if a term $ e $ has type $ \tau $ in the source language,
    then the translated term $ \image e $ is well-typed and has type $ \image \tau $ in the
    target language.}

\item{present an algorithm for detecting incoherence which can be very important
    in practice.}

\item{explores the connection between intersection types and object algebra by
    showing various examples of encoding object algebra with intersection
    types.}

\end{comment}

\end{itemize}

\begin{comment}
\subsection{Other Notes}

finitary overloading: yes
but have other merits of intersection been explored?

-- Compare Scala:
-- merge[A,B] = new A with B

-- type IEval  = { eval :  Int }
-- type IPrint = { print : String }

-- F[\_]
\end{comment}
\section{A Taste of FI}

\begin{footnote}
  Change the examples later to something very simple.
\end{footnote}

This section provides the reader with the intuition of \systemFI, while we postpone
the presentation of the details in later sections.

In short, \systemfi generalizes \systemf by adding intersection polymorphism. \systemfi terms
are elaborated into \systemF, a variant of System F. System F, or polymorphic
lambda calculus lays the foundation of functional programming languages such as
Haskell.

The type system of \systemfi permits a subtyping relation naturally and enables
prototype-based inheritance. We will explore the usefulness of such a type
system in practice by showing various examples.

\subsection{Intersection Types}

The central addition to the type system of \systemf in \systemfi is intersection types. What
is an intersection type? One classic view is from set-theoretic interpretation
of types: \lstinline{A \& B} stands for the intersection of the set of values of
\lstinline{A} and \lstinline{B}. The other view, adopted in this paper, regards
types as a kind of interface: a value of type \lstinline{A \& B} satisfies both
of the interfaces of \lstinline{A} and \lstinline{B}. For example,
\lstinline{eval : Int} is the interface that supports evaluation to integers,
while \lstinline{ eval : Int \& print : String } supports both evaluation and
pretty printing. Those interfaces are akin to interfaces in Java or traits in
Scala. But one key difference is that they are unnamed in \systemFI.

Intersection types provide a simple mechanism for ad-hoc polymorphism, similar
to what type classes in Haskell achieve. The key constructs are the ``merge''
operator, denoted by ``\lstinline{,,}'', at the value level and the corresponding type
intersection operator, denoted by, ``\lstinline{\&}'' at the type level.

For example, we can define an (ad-hoc)-polymorphic \lstinline{show} function
that is able to convert integers and booleans to strings. In \systemfi such function
can be given the type
\begin{lstlisting}
  (Int -> String) & (Bool -> String)
\end{lstlisting}
and be defined using the merge operator $ ,, $ as
\begin{lstlisting}
  let show = showInt ,, showBool
\end{lstlisting}
where \lstinline{showString} and \lstinline{showBool} are ordinary monomorphic
functions. Later suppose the integer \lstinline{1} is applied to the \lstinline{show} function,
the first component \lstinline{showInt} will be picked because the type of \lstinline{showInt}
is compatible with \lstinline{1} while \lstinline{showBool} is not.

% The merge construct in the original function is elaborated into a pair in the
% target language:

% \begin{verbatim}
% show = (showInt, showBool)
% \end{verbatim}

% In the target language where there is no intersection types, the application
% of the integer \texttt{1} to this function does not typecheck. However, we may
% rescue this situtation by inserting a coercion that extracts the first item
% out of this pair.

% Thus \texttt{show 1} in FI corresponds to \texttt{(fst show) 1} in F.

% While elaborating intersection types, this paper is the first that presents a
% type system that incorporates both parametric polymorphism and intersection
% polymorphism.

% Describe intersection types, encoding records with Intersecion types

% \lstinputlisting[linerange=-]{} % APPLY:linerange=MIXIN_LIB

\subsection{Encoding Records}

In addition to introduction of record literals using the usual notation, \systemfi
support two more operations on records: record elimination and record update.

A record type of the form \lstinline{l : t} can be thought as a normal type \lstinline{t}
tagged by the label \lstinline{l}.

% A basic example

% \lstinputlisting[linerange=-]{} % APPLY:linerange=BASICS_ADD

\lstinline{e1} and \lstinline{e2} are two expressions that support both evaluation and pretty
printing and each has type \lstinline{eval : Int, print : String}. \lstinline{add} takes
two expressions and computes their sum. Note that in order to compute a sum,
\lstinline{add} only requires that the two expressions support evaluation and hence the
type of the parameter \lstinline{eval : Int}. As a result, the type of \lstinline{e1} and
\lstinline{e2} are not exactly the same with that of the parameters of \lstinline{add}. However,
under a structural type system, this program should typecheck anyway because the
arguments being passed has more information than required. In other words,
\lstinline{eval : Int, print : String} is a subtype of \lstinline{eval : Int}.

How is this subtyping relation derived? In \systemFI, multi-field record types are
excluded from the type system because \lstinline{eval : Int, print : String} can
be encoded as \lstinline{eval : Int \& print : String}. And by one of
subtyping rules derives that \lstinline{eval : Int \& print : String} is a
subtype of \lstinline{eval : Int}.

% This example is elaborated into the following in \systemF.

% \lstinputlisting[linerange=-]{} % APPLY:linerange=BASICSELAB_ADD

\subsection{Parametric Polymorphism}

The presence of both parametric polymorphism and intersection is critical, as we
shall see in the next section, in solving modularity problems. Here is a code
snippet from the next section (The reader is not required to understand the
purpose of this code at this stage; just recognizing the two types of
polymorphism is enough.)
\begin{lstlisting}
type SubExpAlg E = (ExpAlg E) \& { sub : E -> E -> E };
let e2 E (f : SubExpAlg E) = f.sub (exp1 E f) (f.lit 2);
\end{lstlisting}
\lstinline{SubExpAlg} is a type synonym (a la Haskell) defined as the intersection of
\lstinline{ExpAlg E} and \lstinline{sub : E -> E -> E}, parametrized by a type parameter
\lstinline{E}. \lstinline{e2} exhibits parametric polymorphism as it takes a type argument
\lstinline{E}.

\section{Application} \label{sec:application}

Structural subtyping facilitates reuse~\cite{malayeri2008integrating}.
\bruno{orphan sentence!}

This section shows that the System $ F $ plus intersection types are enough for
encoding extensible designs, and even beat the designs in languages with a much
more sophisticated type system. In particular, \name has two main advantages
over existing languages:

\begin{enumerate}
\item It supports dynamic composition of intersecting values.
\item It supports contravariant parameter types in the subtyping relation.
\end{enumerate}

Various solutions have been proposed to deal with the extensibility problems and
many rely on heavyweight language features such as abstract methods and classes
in Java.

\bruno{I would like to see a story about Church Encodings in
  \name. Can you look at Pierce's papers and try to write something
  along those lines? That will be a good intro for object algebras and
visitors!}

These two features can be used to improve existing designs of modular programs.

% Introduce the expression problem

The expression problem refers to the difficulty of adding a new operations and a
new data variant without changing or duplicating existing code.

There has been recently a lightweight solution to the expression problem that
takes advantage of covariant return types in Java. We show that FI is able to
solve the expression problem in the same spirit. The
A)


% - Object/Fold Algebras. How to support extensibility in an easier way.

% See Datatypes a la Carte

% - Mixins

% - Lenses? Can intersection types help with lenses? Perhaps making the
% types more natural and easy to understand/use?

% - Embedded DSLs? Extensibility in DSLs? Composing multiple DSL interpretations?

% http://www.cs.ox.ac.uk/jeremy.gibbons/publications/embedding.pdf

\begin{minted}{scala}
trait Expr {
  def eval: Int
}

class Lit(n: Int) extends Expr {
  def eval: Int = n
}

class Add(n: Int) extends Expr {
  def eval: Int = e1.eval + e2.eval
}
\end{minted}

\bruno{You already talk about overloading in
  the previous section. Need to decide where to put the text!}

Dunfield~\cite{dunfield2014elaborating} notes that using merges as a mechanism
of overloading is not as powerful as type classes.

\subsection{Encoding Bounded Polymorphism}\bruno{IMPORTANT: 
Always capitalize the relevant words in a title!. This title should be
"Encoding Bounded Polymorphism''}

As \name extends System $ F $ with intersection types, $ F_{\subtype} $ extends
System $ F $ with bounded polymorphism. $ F_{\subtype} $~\cite{pierce2002types}
allows giving an upper bound to the type variable in type abstractions. The idea
of bounded universal quantification was discussed in the seminal paper by
Cardelli and Wegner ~\cite{cardelli1985understanding}. They show that bounded
quantifiers are useful because it is able to solve the ``loss of information''
problem.

In fact, the extension of System $ F $ in the other direction, i.e., with
intersection types, is able to address the same problem effectively. Suppose we
have the following definitions:
\begin{minted}{scala}
def user = { name = "George", admin = true }
def id(user: {name: String}) = user
\end{minted}
Under a structural type system, programmers would expect that passing the user
to the function is allowed. They are correct. Note that in the source language
multi-field records are just syntatic sugars for merges of single-field records.
Therefore, user is of a subtype of the parameter due to subtyping introduced by
intersection types. So far so good. But there is a problem: what if programmers
wants to access the \texttt{admin} field later, like:
\begin{minted}{scala}
(id user).admin
\end{minted}
They cannot do so as the above will not typecheck. After going through the
function, the user now has only the type
\begin{minted}{scala}
{name: String}
\end{minted}
This is rather undesired because it indeed has an \texttt{admin} field!

Bounded polymorphism enable the function to return the exact type of the
argument so that we have no problem in accessing the \texttt{admin} field later.
Consider the example below:
\begin{minted}{scala}
def id [A <: {name: String}] (user: A) = user
(id [{name: String, admin: Bool}] user).admin
\end{minted}

We do not have bounded polymorphism in the source language. But we can encode
that via intersection types:
\begin{minted}{scala}
def id [A] (user: A & {name: String}) = user in
(id [{admin: Bool}] user).admin
\end{minted}

Polymorphism plus intersecion types is as powerful as bounded polymorphism.

\cite{pierce1997intersection}\bruno{I don't think this has been
  shown. What we can say is: we can encode a form of bounded
  polymorphism with intersection types.}

% Multiple inheritance?
% Algebra -> P1,2
% Visitor -> P2

% Yanlin
% Mixin

% \begin{lstlisting}
% let merge A B (f : ExpAlg A) (g : ExpAlg B) = {
%   lit = \(x : Int). f.lit x ,, g.lit x,
%   add = \(x : A & B). \(y : A & B). f.add x y ,, g.add x y
% };
% \end{lstlisting}

\subsection{Object Algebras}

Object algebras provide an alternative to \emph{algebraic data types}
(ADT).\bruno{We are targeting an OO crowd. Mentioning algebraic
  datatypes is not going to be very useful there.}
 For example, the
following Haskell definition of the type of simple expressions
\begin{minted}{haskell}
data Exp where
  Lit :: Int -> Exp
  Add :: Exp -> Exp -> Exp
\end{minted}
can be expressed by the \emph{interface} of an object algebra of
simple expressions:
\begin{minted}{scala}
trait ExpAlg[E] {
  def lit(x: Int): E
  def add(e1: E, e2: E): E
}
\end{minted}
Similar to ADT, data constructors in object algebras are represented by functions such as
\lstinline{lit} and \lstinline{add} inside an interface \lstinline{ExpAlg}.
Different with ADT, the type of the expression itself is abtracted by a type
parameter \lstinline{E}.

which can be expressed similarly in \name as:
\begin{lstlisting}
type ExpAlg E = {
  lit : Int -> E,
  add : E -> E -> E
}
\end{lstlisting}

% Introduce Scala's intersection types

Scala supports intersection types via the \lstinline{with} keyword. The type
\lstinline{A with B} expresses the combined interface of \lstinline{A} and
\lstinline{B}. The idea is similar to
\begin{minted}{java}
interface AwithB extends A, B {}
\end{minted}
in Java.
\footnote{However, Java would require the \lstinline{A} and \lstinline{B} to be
  concrete types, whereas in Scala, there is no such restriction.}

The value level counterpart are functions of the type \lstinline
{A => B => A with B}. \footnote{FIXME}

Our type system is a simple extension of System $ F $; yet surprisingly, it
is able to solve the limitations of using object algebras in languages such as
Java and Scala. We will illustrate this point with an step-by-step of solving
the expression problem using a source language built on top of \name.

Oliveira noted that composition of object algebras can be cumbersome and
intersection types provides a solution to that problem.

We first define an interface that supports the evaluation operation:

% #include ../src/Algebra.sf  @base
\begin{lstlisting}
type IEval  = { eval : Int };
type ExpAlg E = { lit : Int -> E, add : E -> E -> E };
let evalAlg = {
  lit = \(x : Int). { eval = x },
  add = \(x : IEval). \(y : IEval). { eval = x.eval + y.eval }
};
\end{lstlisting}
% #end

The interface is just a type synonym \lstinline{IEval}. In \name, record
types are structural and hence any value that satisfies this interface is of
type \lstinline{IEval} or of a subtype of \lstinline{IEval}. \footnote{Should be
mentioned in S2.}

In the following, \lstinline{ExpAlg} is an object algebra interface of
expressions with literal and addition case. And \lstinline{evalAlg} is an object
algebra for evluation of those expressions, which has type \lstinline{ExpAlg Int}

% #include ../src/Algebra.sf  @variant
\begin{lstlisting}
type SubExpAlg E = (ExpAlg E) & { sub : E -> E -> E };
let subEvalAlg = evalAlg ,, { sub = \ (x : IEval). \ (y : IEval). { eval = x.eval - y.eval } };
\end{lstlisting}
% #end

Next, we define an interface that supports pretty printing.

% #include ../src/Algebra.sf  @operation
\begin{lstlisting}
type IPrint = { print : String };
let printAlg = {
  lit = \(x : Int). { print = x.toString() },
  add = \(x : IPrint). \(y : IPrint). { print = x.print.concat(" + ").concat(y.print) },
  sub = \(x : IPrint). \(y : IPrint). { print = x.print.concat(" - ").concat(y.print) }
};
\end{lstlisting}
% #end

Provided with the definitions above, we can then create values using the
appropriate algebras. For example:
defines two expressions.

The expressions are unusual in the sense that they are functions that take an
extra argument \lstinline{f}, the object algebras, and use the data constructors
provided by the object algebra (factory) \lstinline{f} such as \lstinline{lit},
\lstinline{add} and \lstinline{sub} to create values. Moreover, The algebras
themselves are abstracted over the allowed operations such as evaluation and
pretty printing by requiring the expression functions to take an extra argument
\lstinline{E}.

% #include ../src/Algebra.sf  @merge
\begin{lstlisting}
let merge A B (f : ExpAlg A) (g : ExpAlg B) = {
  lit = \(x : Int). f.lit x ,, g.lit x,
  add = \(x : A & B). \(y : A & B).
          f.add x y ,, g.add x y
};
\end{lstlisting}
% #end

If we would like to have an expression that supports both evaluation and pretty
printing, we will need a mechanism to combine the evaluation and printing
algebras. Intersection types allows such composition: the \lstinline{merge}
function, which takes two expression algebras to create a combined algebra. It
does so by constructing a new expression algebra, a record whose each field is a
function that delegates the input to the two algebras taken.

% #include ../src/Algebra.sf  @usage
\begin{lstlisting}
let newAlg = merge IEval IPrint subEvalAlg printAlg in
let o1 = e1 (IEval & IPrint) newAlg in
o1.print
\end{lstlisting}
% #end

\bruno{Don't start a sentence with \lstinline{o1}.}
\lstinline{o1} is a single object created that supports both evaluation and
printing, thus achieving full feature-oriented programming.

\subsection{Visitors}

Constructing instances seems clumsy!

The visitor pattern allows adding new operations to existing structures without
modifying those structures. The type of expressions are defined as follows:

\begin{minted}{scala}
trait Exp[A] {
  def accept(f: ExpAlg[A]): A
}

trait SubExp[A] extends Exp[A] {
  override def accept(f: SubExpAlg[A]): A
}
\end{minted}

The body of \lstinline{Exp} and \lstinline{SubExp} are almost the same: they
both contain an \lstinline{accept} method that takes an algebra \lstinline{f}
and returns a value of the carrier type \lstinline{A}. The only difference is at
\lstinline{f} --- \lstinline{SubExpAlg[A]} is a subtype of
\lstinline{ExpAlg[A]}. Since \lstinline{f} appear in parameter position of
\lstinline{accept} and function parameters are contravariant, naturally we would
hope that \lstinline{SubExp[A]} is a supertype of \lstinline{Exp[A]}. However,
such subtyping relation does not fit well in Scala because inheritance implies
subtyping in such languages \footnote{It is still possible to encode
  contravariant parameter types in Scala but doing so would require some
  technique.\bruno{what technique?}}. As \lstinline{SubExp[A]} extends \lstinline{Exp[A]}, the former
becomes a subtype of the latter.

Such limitation does not exist in \name. For example, we can define the similar interfaces \lstinline{Exp} and \lstinline{SubExp}:
\begin{lstlisting}
type Exp    A = { accept: forall A. ExpAlg A -> A };
type SubExp A = { accept: forall A. SubExpAlg A -> A };
\end{lstlisting}
Then by the typing judgment it holds that \lstinline{SubExp} is a supertype of
\lstinline{Exp}. This relation gives desired results. To give a concrete example:

A is called is the \emph{interpretation}. It works for any interpretation you want.

First we define two data constructors for simple expressions:
\begin{lstlisting}
let lit (n : Int): Exp A = {
  accept = /\A. \(f : ExpAlg A). f.lit n
};

let add (e1 : Exp) (e2 : Exp): Exp A = {
  accept = /\A. \(f : ExpAlg A).
             f.add (e1.accept A f) (e2.accept A f)
};
\end{lstlisting}

Suppose later we decide to augment the expressions with subtraction:
\begin{lstlisting}
let sub (e1 : SubExp) (e2 : SubExp): SubExp A =
  { accept = /\A. \(f : SubExpAlg A).
               f.sub (e1.accept A f) (e2.accept A f) };
\end{lstlisting}

One big benefit of using the visitor pattern is that programmers is able to
write in the same way that would do in Haskell.
For example, \lstinline{e2 = sub (lit 2) (lit 3)} defines an expression.

Another important property that does not exist in Scala is that programmer is
able to pass \lstinline{lit 2}, which is of type \lstinline{Exp A}, to
\lstinline{sub}, which expects a \lstinline{SubExp A} because of the subtyping
relation we have. After all, it is known statically that \lstinline{lit 2} can
be passed into \lstinline{sub} and nothing will go
wrong.\bruno{Subtyping needs to be much more emphasized! See Modular
  Visitor Components! }

% \subsection{Yanlin stuff}
% \bruno{This can be dropped.}

% This subsection presents yet another lightweight solution to the Expression
% Problem, inspired by the recent work by Wang. It has been shown that
% contravariant return types allows refinement of the types of extended
% expressions.

% First, we define the type of expressions that support evaluation and implement
% two constructors:
% \begin{lstlisting}
% type Exp = { eval: Int }
% let lit (n: Int) = { eval = n }
% let add (e1: Exp) (e2: Exp)
%   = { eval = e1.eval + e2.eval }
% \end{lstlisting}

% If we would like to add a new operation, say pretty printing, it is nothing more
% than refining the original \lstinline{Exp} interface by \emph{intersecting} the
% original type with the new \lstinline{print} interface using the \lstinline{&}
% primitive and \emph{merging} the original data constructors using the \lstinline{,,}
% primitive.
% \begin{lstlisting}
% type ExpExt = Exp & { print: String }
% let litExt (n: Int) = lit n ,, { print = n.toString() }
% let addExt (e1: ExpExt) (e2: ExpExt)
%   = add e1 e2 ,,
%     { print = e1.print.concat(" + ").concat(e2.print) }
% \end{lstlisting}

% Now we can construct expressions using the constructors defined above:
% \begin{lstlisting}
% let e1: ExpExt = addExt (litExt 2) (litExt 3)
% let e2: Exp = add (lit 2) (lit 4)
% \end{lstlisting}
% \lstinline{e1} is an expression capable of both evaluation and printing, while
% \lstinline{e2} supports evaluation only.

% We can also add a new variant to our expression:
% \begin{lstlisting}
% let sub (e1: Exp) (e2: Exp) = { eval = e1.eval - e2.eval }
% let subExt (e1: ExpExt) (e2: ExpExt)
%   = sub e1 e2 ,, { print = e1.print.concat(" - ").concat(e2.print) }
% \end{lstlisting}

% Finally we are able to manipulate our expressions with the power of both
% subtraction and pretty printing.
% \begin{lstlisting}
% (subExt e1 e1).print
% \end{lstlisting}

\subsection{Mixins}

Mixins are useful programming technique wildly adopted in dynamic programming
languages such as JavaScript and Ruby. But obviously it is the programmers'
responsbility to make sure that the mixin does not try to access methods or
fields that are not present in the base class.

In Haskell, one is also able to write programs in mixin style using records.
However, this approach has a serious drawback: since there is no subtyping in
Haskell, it is not possible to refine the mixin by adding more fields to the
records. This means that the type of the family of the mixins has to be
determined upfront, which undermines extensibility.

\name is able to overcome both of the problems: it allows composing mixins
that (1) extends the base behavior, (2) while ensuring type safety.

The figure defines a mini mixin library. The apostrophe in front of types
denotes call-by-name arguments similar to the \lstinline{=>} notation in the
Scala language.

\begin{lstlisting}
type Mixin S = 'S -> 'S -> S;
let zero S (super : 'S) (this : 'S) : S = super;
let rec mixin S (f : Mixin S) : S
  = let m = mixin S in f (\ (_ : Unit). m f) (\ (_ : Unit). m f);
let extends S (f : Mixin S) (g : Mixin S) : Mixin S
  = \ (super : 'S). \ (this  : 'S). f (\ (d : Unit). g super this) this;
\end{lstlisting}

We define a factorial function in mixin style and make a \lstinline{noisy} mixin
that prints ``Hello'' and delegates to its superclass. Then the two functions
are composed using the \lstinline{mixin} and \lstinline{extends} combinators.
The result is the \lstinline{noisyFact} function that prints ``Hello'' every
time it is called and computes factorial.
\begin{lstlisting}
let fact (super : 'Int -> Int) (this : 'Int -> Int) : Int -> Int
  = \ (n : Int). if n == 0 then 1 else n * this (n - 1)
let noisy (super : 'Int -> Int) (this : 'Int -> Int) : Int -> Int
  = \ (n : Int). { println("Hello"); super n }
let noisyFact = mixin (Int -> Int) (extends (Int -> Int) foolish fact)
noisy 5
\end{lstlisting}

% \subsection{Composing Mixins and Object Algebras}

\newcommand{\makelabeltgt}[1]{Tgt\_#1}

% Target WF
\newcommand{\formtgtwf}{\framebox{$ \jwf G T $}}

\newcommand{\makelabeltgtwf}[1]{\makelabeltgt {WF\_#1}}

\newcommand{\labeltgtwffv}{\makelabeltgtwf FV}
\newcommand{\ruletgtwffv} {
\inferrule* [right=\labeltgtwffv]
  {\ftv T \in G}
  {\jwf G T}
}

% Target typing
\newcommand{\formtgt}{\framebox{$ \jtype G E T $}}

\newcommand{\makelabeltgtty}[1]{\makelabeltgt {Ty\_#1}}

\newcommand{\labeltgtvar}{\makelabeltgtty Var}
\newcommand{\ruletgtvar} {
\inferrule* [right=\labeltgtvar]
  {(x,T) \in \Gamma}
  {\jtype \Gamma x T}
}

\newcommand{\labeltgtlam}{\makelabeltgtty Lam}
\newcommand{\ruletgtlam} {
\inferrule* [right=\labeltgtlam]
  {\jtype {\Gamma, x \oftype T} E {T_1} \andalso \jwf \Gamma T}
  {\jtype \Gamma {\lamty x T E} {T \to T_1}}
}

\newcommand{\labeltgtapp}{\makelabeltgtty App}
\newcommand{\ruletgtapp}{
\inferrule* [right=\labeltgtapp]
  {\jtype \Gamma {E_1} {T_1 \to T_2} \andalso \jtype \Gamma {E_2} {T_1}}
  {\jtype \Gamma {\app {E_1} {E_2}} {T_2}}
}

\newcommand{\labeltgtblam}{\makelabeltgtty BLam}
\newcommand{\ruletgtblam}{
\inferrule* [right=\labeltgtblam]
  {\jtype {\Gamma, \alpha} E T}
  {\jtype \Gamma {\blam \alpha E} {\for \alpha T}}
}

\newcommand{\labeltgttapp}{\makelabeltgtty TApp}
\newcommand{\ruletgttapp}{
\inferrule* [right=\labeltgttapp]
  {\jtype \Gamma E {\for \alpha {T_1}} \andalso \jwf \Gamma T}
  {\jtype \Gamma {\tapp E T} {\subst T \alpha T_1}}
}

\newcommand{\labeltgtpair}{\makelabeltgtty Pair}
\newcommand{\ruletgtpair}{
\inferrule* [right=\labeltgtpair]
  {\jtype \Gamma {E_1} {T_1} \andalso \jtype \Gamma {E_2} {T_2}}
  {\jtype \Gamma {\pair {E_1} {E_2}} {\pair {T_1} {T_2}}}
}

\newcommand{\labeltgtprojl}{\makelabeltgtty {Proj\_1}}
\newcommand{\ruletgtprojl}{
\inferrule* [right=\labeltgtprojl]
  {\jtype \Gamma E {\pair {T_1} {T_2}}}
  {\jtype \Gamma {\proj 1 E} {T_1}}
}

\newcommand{\labeltgtprojr}{\makelabeltgtty {Proj\_2}}
\newcommand{\ruletgtprojr}{
\inferrule* [right=\labeltgtprojr]
  {\jtype \Gamma E {\pair {T_1} {T_2}}}
  {\jtype \Gamma {\proj 2 E} {T_2}}
}

\section{Source Language}

The source language, System FI, is identical to the source language described in
the previous section, except for the two additions: intersection types and
records. The formalization includes only single records and single record types as the multi-records can be desugared into the merge of multiple single records.

Dunfield has described a language that includes a ``top'' type but it does not appear in our language. Our work differs from Dunfield in that ...

Remark. The operational semantics of FI is not presented in this paper. However,

\subsection{Source Syntax}

\subsection{Source Subtyping}

\subsection{Source Typing}

\section{Elaboration Typing}

In order to give the reader an intuitive idea of how the elaboration works,
let's first imagine a manual translation.

First, multi-field record literals are desugared into merges of single-field
record literals. Therefore $ \{ eval = 4, print = ``4'' \} $ becomes
$ \{ eval = 4 \} ,, \{ print = ``4'' \} $. Merges of two values are elaborated
into just a pair of them and single-field record literals lose their field
labels during the elaboration. Hence $ \{ eval = 4 \} ,, \{ print = ``4'' \} $
becomes $ (4, ``4'') $.

Finally, $ e1 $ and $ e2 $ are both coerced by a projection function
$ \\(x:(Int,String)). x.\_1 $ before being applied to $ add $. We adopt a
Scala-like syntax where $ .\_1 $ denotes the projection of a tuple on the first
element, and so on.

\framebox{$|\tau| = T$}

\[
\begin{array}{rcl}
  |\alpha|               & = & \alpha \\
  |\tau_1 \to \tau_2|    & = & |\tau_1| \to |\tau_2| \\
  |\forall \alpha. \tau| & = & \forall \alpha. |\tau| \\
  |t_1 \& t_2|           & = & \langle |\tau_1|, |\tau_2| \rangle \\
  |\{ l : \tau \}|       & = & |\tau|
\end{array}
\]

\begin{lemma} \label{type-coerce}
  If $$ \Gamma \vdash \tau_1 <: \tau_2 \yields{C} $$
  then $$ |\Gamma| \vdash C : |\tau_1| \to |\tau_2| $$
\end{lemma}

In this section, we present a relatively lightweight type-directed elaboration
from FI to F. The elaboration consists of four sets of rules, which are
explained below:

\begin{itemize}

\item{\bf Coercion}

  The coercion judgment $ \Gamma \vdash \tau_1 <: \tau_2 \yields{C} $ extends
  the subtyping judgment with a coercion on the right hand side of
  $ \hookrightarrow $. A coercion, which is just an expression in the target
  language, is guaranteed to have type $ \tau_1 \to \tau_2 $, as proved by Lemma
  \ref{type-coerce}. It is read ``In the environment $ \Gamma $, $ \tau_1 $ is a
  subtype of $ \tau_2 $; and if any expression $ e $ has a type $ t_1 $ that is
  a subtype of the type of $ t_2 $, the elaborated $ e $, when applied to the
  corresponding coercion $ C $, has exactly type $ |t_2| $''. For example,
  $\Gamma \vdash Int \& Bool <: Bool \yields{fst} $, where $ fst $ is the
  projection of a tuple on the first element. The coercion judgment is only used
  in the \texttt{TrApp} case.

\item{\bf Elaboration}

  The elaboration judgment $ \Gamma \vdash e : \tau \yields{E} $ extends the
  typing judgment with an elaborated expression on the right hand side of
  $ \hookrightarrow $. It is also standard, except for the case of
  \texttt{TrApp}, in which a coercion from the inferred type of the argument to
  the expected type of the parameter is inserted before the argument; and the
  case of \texttt{TrRcdEim} and \texttt{TrRcdUpd}, where the ``get'' and ``put''
  rules will be used. The two set of rules are explained below.

\item{\bf ``get'' rules}

  The ``get'' judgment can be thought as producing a field accessor.

\item{\bf ``put'' rules}

  The ``put'' judgment can be thought as producing a field updater.

\end{itemize}

Type-Directed Translation to System F.
Main results: type-preservation + coherence.
\section{Implementation}

We implemented the core functionalities of the \name as part of a JVM-based
compiler. The implementation supports record update instead of restriction as a
primitive; however the former is formalized with the same underlying idea of
elaborating records. Based on the type system of \name, we built an ML-like
source language compiler that offers interoperability with Java (such as object
creation and method calls). The source language is loosely based on the more
general System $F_{\omega}$ (compared to our target, System $F$) and supports a
number of other features, including multi-field records, mutually recursive
\code{let} bindings, type aliases, algebraic data types, pattern matching, and
first-class modules that are encoded with \code{letrec} and records.

Relevant to this paper are the three following phases in the compiler that
collectively turn source programs into System $F$:

\begin{enumerate}
\item A \emph{typechecking} phase that checks the usage of \name features and
  other source language features against an abstract syntax tree that follows
  the source syntax.

\item A \emph{desugaring} phase that translates well-typed source terms into
  \name terms. Source-level features such as multi-field records, type aliases
  are removed at this phase. The resulting program is just an \name expression
  extended with some other constructs necessary for code generation.

\item A \emph{translation} phase that turns well-typed \name terms into System
  $F$ ones.
\end{enumerate}

Phase 3 is what we have formalized in this paper.

\paragraph{Removing identity functions.} Our translation inserts identity
functions whenever subtyping or record operation occurs, which could mean
notable run-time overhead. But in practice this is not an issue. In the current
implementation, we introduced a partial evaluator with three simple rewriting
rules to eliminate the redundant identity functions as another compiler phase
after the translation. In another version of our implementation, partial
evaluation is weaved into the process of translation so that the unwanted
identity functions are not introduced during the translation.

\section{Related Work} \label{sec:related-work}

% \url{http://homepages.inf.ed.ac.uk/gdp/publications/Sub_Par.pdf}

% \cite{plotkin1994subtyping}

% Also discussed intersection types!~\cite{malayeri2008integrating}.

% Pierce Ph.D thesis: F<: + /|
%        technical report: F + /|, closer to ours

% \cite{barbanera1995intersection}

\paragraph{Intersection types with polymorphism.}
Our type system combines intersection types and parametric polymorphism. Closest
to us is Pierce's work~\cite{pierce1991programming1} on a prototype
compiler for a language with both intersection types, union types, and
parametric polymorphism. Similarly to \name in his system universal
quantifiers do not support bounded quantification. However Pierce did not try to prove any
meta-theoretical results and his calculus does not have a merge
operator.  Pierce also studied a system where both intersection
types and bounded polymorphism are present in his Ph.D.
dissertation~\cite{pierce1991programming2} and a 1997
report~\cite{pierce1997intersection}. Going in the direction of higher
kinds, Compagnoni and Pierce~\cite{compagnoni1996higher} added
intersection types to System $ F_{\omega} $ and used the new calculus,
$ F^{\omega}_{\wedge} $, to model multiple inheritance. In their
system, types include the construct of intersection of types of the
same kind $ K $. Davies and Pfenning
\cite{davies2000intersection} studied the interactions between
intersection types and effects in call-by-value languages. And they
proposed a ``value restriction'' for intersection types, similar to
value restriction on parametric polymorphism. Although they proposed a system with
parametric polymorphism, their subtyping rules are significantly different from ours,
since they consider parametric polymorphism
as the ``infinit analog'' of intersection polymorphism.
There have been attempts to provide a foundational calculus
for Scala that incorporates intersection
types~\cite{amin2014foundations,amin2012dependent}.
Although the minimal Scala-like calculus does not natively support
parametric polymorphism, it is possible to encode parametric
polymorphism with abstract type members. Thus it can be argued that
this calculus also supports intersection types and parametric
polymorphism. However, the type-soundness of a minimal Scala-like
calculus with intersection types and parametric polymorphism is not
yet proven. Recently, some form of intersection
types has been adopted in object-oriented languages such as Scala,
Ceylon, and Grace. Generally speaking,
the most significant difference to \name is that in all previous systems
there is no explicit introduction construct like our merge operator. As shown in
Section~\ref{sec:overview}, this feature is pivotal in supporting modularity
and extensibility because it allows dynamic composition of values.

\begin{comment}
only allow intersections of concrete types (classes),
whereas our language allows intersections of type variables, such as
\texttt{A \& B}. Without that vehicle, we would not be able to define
the generic \texttt{merge} function (below) for all interpretations of
a given algebra, and would incur boilerplate code:

\begin{lstlisting}{language=haskell}
let merge [A, B] (f: ExpAlg A) (g: ExpAlg B) = {
  lit (x : Int) = f.lit x ,, g.lit x,
  add (x : A & B) (y : A & B) =
    f.add x y ,, g.add x y
}
\end{lstlisting}
\end{comment}


\paragraph{Other type systems with intersection types.}
Intersection types date back to as early as Coppo et
al.~\cite{coppo1981functional}. As emphasized throughout the paper our
work is inspired by Dunfield~\cite{dunfield2014elaborating}. He described a similar approach to ours:
compiling a system with intersection types into ordinary $ \lambda $-calculus
terms. The major difference is that his system does not include parametric
polymorphism, while ours does not include unions. Besides, our rules are
algorithmic and we formalize a record system.
% Although similar in spirit,
% our elaboration typing is simpler: we require subtyping in the case of
% applications, thus avoiding the subsumption rule. Besides, our treatment
% combines the merge rules ($ k $ ranges over $ \{1, 2\} $)
% \inferrule
% {\Gamma \turns e_k : A}
% {\Gamma \turns e_1 \mergeop e_2 : A}
% and the standard intersection-introduction rule
% \inferrule
% {\Gamma \turns e : A_1 \andalso \Gamma \turns e : A_2}
% {\Gamma \turns e : A_1 \inter A_2}
% into one rule:
% \inferrule [Merge]
% {\Gamma \turns e_1 : A_1 \andalso \Gamma \turns e_2 : A_2}
% {\Gamma \turns e_1 \mergeop e_2 : A_1 \inter A_2}
Reynolds invented Forsythe~\cite{reynolds1997design} in the 1980s. Our merge
operator is analogous to his $ p_1, p_2 $. As Dunfield
has noted, in Forsythe merges can be only used unambiguously.
For instance, it is not allowed in Forsythe to merge two functions.

%Castagna, and Dunfield describe
%elaborating multi-fields records into merge of single-field records.
% Reynolds and Castagna do not consider elaboration and Dunfield do not
% formalize elaborating records.

% Both Pierce and Dunfield's system include a subsumption rule, which states that
% if an term has been inferred of type $ A $, then it is also of any
% supertype of $ A $. Our system does not have this rule.

Refinement
intersection~\cite{dunfield2007refined,davies2005practical,freeman1991refinement}
is the more conservative approach of adopting intersection types. It increases
only the expressiveness of types but not terms. But without a term-level
construct like ``merge'', it is not possible to encode various language
features. As an alternative to syntatic subtyping described in this paper,
Frisch et al.~\cite{frisch2008semantic} studied semantic subtyping.

\paragraph{Languages for extensibility.}
To improve support for extensibility various researchers have proposed
new OOP languages or programming mechanisms. It is interesting to
note that design patterns such as object algebras or modular visitors
provide a considerably different approach to extensibility when
compared to some previous proposals for language designs for
extensibility. Therefore the requirements in terms of type system
features are quite different.  One popular approach is \emph{family
  polymorphism}~\cite{Ernst01family}, which allows whole class hierarchies to be
captured as a family of classes. Such a family can be later reused to
create a derived family with potentially new class members, and
additional methods in the existing classes.  \emph{Virtual
  classes}~\cite{ernst2006virtual} are a concrete realization of this idea, where a
container class can hold nested inner \emph{virtual} classes (forming
the family of classes). In a subclass of the container class, the
inner classes can themselves be \emph{overriden}, which is why they
are called virtual. There are many language mechanisms that provide
variants of virtual classes or similar mechanisms~\cite{McDirmid01Jiazzi,Aracic06CaesarJ,Smaragdakis98mixin,nystrom2006j}. The work by
Nystrom on \emph{nested intersection}~\cite{nystrom2006j} uses a
form of intersection types to support the composition of
families of classes. Ostermann's \emph{delegation layers}~\cite{Ostermann02dynamically}
use delegation for doing dynamic composition in a system
with virtual classes. This in contrast with most other approaches
that use class-based composition, but closer to the dynamic
composition that we use in \name.
\begin{comment}
In contrast to type systems for virtual classes
and similar mechanisms, the goal of our work is to study the type
systems and basic language mechanism to better support such design patterns.
 some researchers have designed new type
system features such as virtual classes~\cite{ernst2006virtual}, polymorphic
variants~\cite{garrigue1998programming}, while others have shown employing
programming pattern such as object algebras~\cite{oliveira2012extensibility} by
using features within existing programming languages. Both of the two approaches
have drawbacks of some kind. The first approach often involves heavyweight
designs, while the second approach still sacrifices the readability for
extensibility.
\bruno{fill me in with more details and more references!}
\end{comment}
% Intersection types have been shown to be useful in designing languages that
% support modularity.~\cite{nystrom2006j}

% \paragraph{Extensible records.}

%\george{Record field deletion is also possible.}

% http://elm-lang.org/learn/Records.elm

% Encoding records using intersection types appeared in
% Reynolds~\cite{reynolds1997design} and Castagna et
% al.~\cite{castagna1995calculus}. Although Dunfield also discussed this idea in
% his paper \cite{dunfield2014elaborating}, he only provided an implementation but
% not a formalization. Very similar to our treatment of elaborating records is
% Cardelli's work~\cite{cardelli1992extensible} on translating a calculus, named
% $ F_{\subtype \rho}$, with extensible records to a simpler calculus that without
% records primitives (in which case is $ F_{\subtype} $). But he did not consider
% encoding multi-field records as intersections; hence his translation is more
% heavyweight. Crary~\cite{crary1998simple} used intersection types and
% existential types to address the problem that arises when interpreting method
% dispatch as self-application. But in his paper, intersection types are not used
% to encode multi-field records.

% Wand~\cite{wand1987complete} started the work on extensible records and proposed
% row types~\cite{wand1989type} for records. Cardelli and
% Mitchell~\cite{cardelli1990operations} defined three primitive operations on
% records that are similar to ours: \emph{selection}, \emph{restriction}, and
% \emph{extension}. The merge operator in \name plays the same role as extension.
% Following Cardelli and Mitchell's approach,
% of restriction and extension. Both Leijen's systems~\cite{leijen2004first,leijen2005extensible}
% and ours allow records that contain
% duplicate labels. Leijen's system is more sophisticated. For example, it supports
% passing record labels as arguments to functions. He also showed an encoding of
% intersection types using first-class labels.

% Chlipala's
% \texttt{Ur}~\cite{chlipala2010ur} explains record as type level
% constructs.\bruno{What is the point of citing Chlipala's paper?}

% Our system can be adapted to simulate systems that support extensible
% records but not intersection of ordinary types like \texttt{Int} and
% \texttt{Float} by allowing only intersection of record types.
%
% $ \turnsrec A $ states that $ A $ is a record type, or the intersection of
% record types, and so forth.
%
% \inferrule [RecBase] {} {\turnsrec \recordType l A}
%
% \inferrule [RecStep]
% {\turnsrec A_1 \andalso \turnsrec A_2}
% {\turnsrec A_1 \inter A_2}
%
% \inferrule [Merge']
% {\Gamma \turns e_1 : A_1 \yields {E_1} \andalso \turnsrec A_1 \\
%  \Gamma \turns e_2 : A_2 \yields {E_2} \andalso \turnsrec A_2}
% {\Gamma \turns e_1 \mergeop e_2 : A_1 \inter A_2 \yields {\pair {E_1} {E_2}}}
%
% Of course our approach has its limitation as duplicated labels in a record are
% allowed. This has been discussed in a larger issue by
% Dunfield~\cite{dunfield2014elaborating}.
%
% R{\'e}my~\cite{remy1989type}

\section{Conclusion and Future Work}
\label{sec:conclusion}

This paper described \name: a System $F$-based language that combines
intersection types, parametric polymorphism and a merge operator.
The language is proved to be type-safe and coherent.
To ensure coherence the type system accepts only
disjoint intersections. To provide flexibility in the presence of parametric polymorphism,
universal quantification is extended with
disjointness constraints. We believe that disjoint intersection types
and disjoint quantification are intuitive, and at the same time
flexible enough to enable practical applications.

%We implemented the core functionalities of the \namedis as part of a JVM-based
%compiler. Based on the type system of \namedis, we have built an ML-like
%source language compiler that offers interoperability with Java (such as object
%creation and method calls). The source language is loosely based on the more
%general System $F_{\omega}$ and supports a
%number of other features, including records, mutually recursive
%\code{let} bindings, type aliases, algebraic data types, pattern matching, and
%first-class modules that are encoded using \code{letrec} and records.

For the future, we intend to create a prototype-based statically typed
source language based on \name.  We are also interested in extending
our work to systems with union types and a $\bot$ type. Union types
are also widely used in languages such as Ceylon or Flow, but
preserving coherence in the presence of union types is
challenging. The naive addition of $\bot$ seems to be problematic. 
The proofs for \name rely on the invariant that a type variable $\alpha$ can never be disjoint 
to another type that contains $\alpha$. The addition of $\bot$ breaks
this invariant, allowing us to derive, for example, $\jdis \Gamma
\alpha \alpha$.
Finally, we could study a similar system with implicit polymorphism.
Such system would require some changes in the subtyping and disjointness relations.
For instance, subtyping should allow  
${\for \alpha {\alpha \to \alpha}} \subtype \tyint \to \tyint$.
Consequently, the disjointness relation would have to be modified,
since valid statements in \name such as 
$\jdis \Gamma {\for \alpha {\alpha \to \alpha}} {\tyint \to \tyint}$ 
would no longer hold under the more powerful subtyping relation. 


\acks

Acknowledgments, if needed.

\appendix

\section{Proofs}

\begin{proof}
By structural induction on the types and the corresponding inference rule. \\

\texttt{(SubVar)}

\texttt{(SubFun)}

\texttt{(SubForall)}

\texttt{(SubAnd1)}

\texttt{(SubAnd2)}

\texttt{(SubAnd3)}

\texttt{(SubRcd)}

\end{proof}

\begin{lemma} \label{type-get}
  If $$ \Gamma \vdash_{get} \tau ; l = C ; \tau_1 $$
  then $$ |\Gamma| \vdash C : |\tau| \to |\tau_1| $$
\end{lemma}

\begin{proof}
By structural induction on the type and the corresponding inference rule. \\

\texttt{(Get-Base)} $ \Gamma \vdash_{get} \{ l : \tau \} ; l = \lambda (x : |\{ l : \tau \}|). x ; \tau $ \\

By the induction hypothesis
$$ |\Gamma| \vdash \lambda (x : |\{ l : \tau \}|). x : |\{ l : \tau \}| \to |\tau| $$

\texttt{(Get-Left)} \\
\texttt{(Get-Right)} \\

\end{proof}

\begin{lemma} \label{type-put}
  If $$ \Gamma \vdash_{put} \tau ; l ; E = C ; \tau_1 $$
  then $$ |\Gamma| \vdash C : |\tau| \to |\tau| $$
\end{lemma}

\begin{proof}
By structural induction on the type and the corresponding inference rule. \\

\texttt{(Put-Base)} \\
\texttt{(Put-Left)} \\
\texttt{(Put-Right)} \\
\end{proof}

\begin{lemma} \label{preserve-wf}
  If   $$ \Gamma \vdash \tau $$
  then $$ |\Gamma| \vdash |\tau| $$
\end{lemma}

\begin{proof}
Since $$ \Gamma \vdash \tau $$
It follows from \texttt{(FI-WF)} that
  $$ \ftv{\tau} \subseteq \ftv{\Gamma} $$
And hence
  $$ \ftv{|\tau|} \subseteq \ftv{|\Gamma|} $$
By \texttt{(F-WF)} we have
  $$ \Gamma \vdash \tau $$
\end{proof}

\begin{theorem}[Type preserving translation] \label{preserve-tr}
  If   $$ \Gamma \vdash e : \tau \yields{E} $$
  then $$ |\Gamma| \vdash E : \left| \tau \right| $$
\end{theorem}

\begin{proof}
By structural induction on the expression and the corresponding inference rule. \\

\texttt{(TrVar)} $ \Gamma \vdash x : \tau \yields{x} $ \\

It follows from \texttt{(TrVar)} that
  $$ (x : \tau) \in \Gamma $$
Based on the definition of $ |\cdot| $,
  $$ (x : |\tau|) \in |\Gamma| $$
Thus we have by \texttt{(F-Var)} that
  $$ |\Gamma| \vdash x : |\tau| $$

\texttt{(TrAbs)} $ \Gamma \vdash \lambda (x : \tau_1). e : \tau_1 \to \tau_2 \yields{\lambda x : |\tau_1|. E} $ \\

It follows from \texttt{(TrAbs)} that
  $$ \Gamma, x : \tau_1 \vdash e : \tau_2 \yields{E} $$
And by the induction hypothesis that
  $$ |\Gamma|, x : |\tau_1| \vdash E : |\tau_2| $$
By \texttt{(TrAbs)} we also have
  $$ \Gamma \vdash \tau_1 $$
It follows from Lemma \ref{preserve-wf} that
  $$ |\Gamma| \vdash |\tau_1| $$
Hence by \texttt{(F-Abs)} and the definition of $|\cdot|$ we have
  $$ |\Gamma| \vdash \lambda x : |\tau_1|. E : |\tau_1 \to \tau_2| $$

\texttt{(TrApp)} $ \Gamma \vdash e_1 e_2 : \tau_2 \yields{E_1 (C E_2)} $ \\

From \texttt{(TrApp)} we have
  $$ \Gamma \vdash \tau_3 <: \tau_1 \yields{C} $$
Applying Lemma \ref{type-coerce} to the above we have
  $$ |\Gamma| \vdash C : |\tau_3| \to |\tau_1| $$
Also from \texttt{(TrApp)} and the induction hypothesis
  $$ |\Gamma| \vdash E_1 : |\tau_1| \to |\tau_2| $$
Also from \texttt{(TrApp)} and the induction hypothesis
  $$ |\Gamma| \vdash E_2 : |\tau_3| $$
Assembling those parts using \texttt{(F-App)} we come to
  $$ |\Gamma| \vdash E_1 (C E_2) : |\tau_2| $$
\end{proof}

\texttt{(TrTAbs)} $ \Gamma \vdash \Lambda \alpha. e : \forall \alpha. \tau \yields{\forall \alpha. E} $ \\

From \texttt{(TrTAbs)} we have
  $$ \Gamma \vdash e : \tau \yields{E} $$
By the induction hypothesis we have
  $$ |\Gamma| \vdash E : |\tau| $$
Thus by \texttt{(F-TAbs)} and the definition of $|\cdot|$
  $$ \Gamma \vdash \Lambda \alpha. E : |\forall \alpha. \tau| $$


\texttt{(TrTApp)} $ \Gamma \vdash e \; \tau  : [\alpha := \tau]\tau_1 \yields{E \; |\tau|} $ \\

From \texttt{(TrTApp)} we have
  $$ \Gamma \vdash e : \forall \alpha. \tau_1 \yields{E} $$
And by the induction hypothesis that
  $$ |\Gamma| \vdash E : \forall \alpha. |\tau_1| $$
Also from \texttt{(TrTApp)} and Lemma \ref{preserve-wf} we have
  $$ |\Gamma| \vdash |\tau| $$
Then by \texttt{(F-TApp)} that
  $$ |\Gamma| \vdash E \; |\tau| : [\alpha := |\tau| ]|\tau_1| $$
Therefore
  $$ |\Gamma| \vdash E \; |\tau| : | [\alpha := \tau ] | \tau_1 | | $$

\texttt{(TrMerge)} $ \Gamma \vdash e_1 \merge e_2 : \tau_1 \& \tau_2 \yields{\langle E1, E2  \rangle}$ \\

From \texttt{(TrMerge)} and the induction hypothesis we have
  $$ |\Gamma| \vdash E_1 : |\tau_1| $$
and
  $$ |\Gamma| \vdash E_2 : |\tau_2| $$
Hence by \texttt{(F-Pair)}
  $$ |\Gamma| \vdash \langle E_1, E_2 \rangle : \langle |\tau_1|, |\tau_2| \rangle $$
Hence by the definition of $|\cdot|$
  $$ |\Gamma| \vdash \langle E_1, E_2 \rangle : |\tau_1 \& \tau_2| $$

\texttt{(TrRcdIntro)} $ \Gamma \vdash \{ l = e \} : \{ l : \tau \} \yields{E} $ \\

From \texttt{(TrRcdIntro)} we have
  $$ \Gamma \vdash e : \tau \yields{E} $$
And by the induction hypothesis that
  $$ |\Gamma| \vdash E : |\tau| $$
Thus by the definition of $|\cdot|$
  $$ |\Gamma| \vdash E : |\{ l : \tau \}| $$

\texttt{(TrRcdElim)} $ \Gamma \vdash e.l : \tau_1 \yields{C E} $ \\

From \texttt{(TrRcdElim)}
  $$ \Gamma \vdash e : \tau \yields{E} $$
And by the induction hypothesis that
  $$ |\Gamma| \vdash E : |\tau| $$
Also from \texttt{(TrRcdEim)}
  $$ \Gamma \vdash_{get} e ; l = C ; \tau_1 $$
Applying Lemma \ref{type-get} to the above we have
  $$ |\Gamma| \vdash C : |\tau| \to |\tau_1|  $$
Hence by \texttt{(F-App)} we have
  $$ |\Gamma| \vdash C E : |\tau_1| $$

\texttt{(TrRcdUpd)} $ \Gamma \vdash \rcdupd{e}{l}{e_1} : \tau \yields{C E} $ \\

From \texttt{(TrRcdUpd)}
  $$ \Gamma \vdash e : \tau \yields{E} $$
And by the induction hypothesis that
  $$ |\Gamma| \vdash E : |\tau| $$
Also from \texttt{(TrRcdUpd)}
  $$ \Gamma \vdash_{put} t ; l; E = C ; \tau_1 $$
Applying Lemma \ref{type-put} to the above we have
  $$ |\Gamma| \vdash C : |\tau| \to |\tau|  $$
Hence by \texttt{(F-App)} we have
  $$ |\Gamma| \vdash C E : |\tau| $$


% We recommend abbrvnat bibliography style.

\bibliographystyle{abbrvnat}

% The bibliography should be embedded for final submission.

\bibliography{references}

\begin{thebibliography}{}
\softraggedright

\bibitem[Smith et~al.(2009)Smith, Jones]{smith02}
P. Q. Smith, and X. Y. Jones. ...reference text...

\end{thebibliography}

\end{document}
