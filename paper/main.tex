\documentclass[times,10pt]{sigplanconf}

% Remote packages

% For pdflatex, replaced by fontspec:
\usepackage{tgpagella}
\usepackage[T1]{fontenc}
\usepackage[utf8]{inputenc}

% For xelatex or lualatex
% \usepackage{fontspec}
% \setmainfont{Times New Roman}

\usepackage{amsmath}
\usepackage{amsthm}
\usepackage{amssymb}
\usepackage{mathtools} % For \Coloneqq
\usepackage{bm}        % Bold symbols in maths mode
\usepackage{fixltx2e}
\usepackage{stmaryrd}
\usepackage[dvipsnames]{xcolor}
\usepackage{listings} % For code listings
% \usepackage{minted}
% \usemintedstyle{murphy}
\usepackage{fancyvrb}
\usepackage{url}
\usepackage{xspace}
\usepackage{comment}

% Typography
\usepackage[euler-digits,euler-hat-accent]{eulervm}

% Copied from the FCore paper:
\usepackage[colorlinks=true,allcolors=black,breaklinks,draft=false]{hyperref}   % hyperlinks, including DOIs and URLs in bibliography
% known bug: http://tex.stackexchange.com/questions/1522/pdfendlink-ended-up-in-different-nesting-level-than-pdfstartlink

% Figures with borders
% http://en.wikibooks.org/wiki/LaTeX/Floats,_Figures_and_Captions
% \usepackage{float}
% \floatstyle{boxed}
% \restylefloat{figure}

% Local packages

\usepackage{styles/bcprules}    % by Benjamin C. Pierce
\usepackage{styles/cmll}
\usepackage{styles/mathpartir}  % by Didier Rémy


% ! Always load mathastext last
% http://mirrors.ibiblio.org/CTAN/macros/latex/contrib/mathastext/mathastext.pdf
% \renewcommand\familydefault\ttdefault
% \usepackage{mathastext}
% \renewcommand\familydefault\rmdefault


\newtheorem{theorem}{Theorem}
\newtheorem{lemma}{Lemma}

% Define macros immediately before the \begin{document} command
% math-mode versions of \rlap, etc
% from Alexander Perlis, "A complement to \smash, \llap, and lap"
%   http://math.arizona.edu/~aprl/publications/mathclap/
\def\clap#1{\hbox to 0pt{\hss#1\hss}}
\def\mathllap{\mathpalette\mathllapinternal}
\def\mathrlap{\mathpalette\mathrlapinternal}
\def\mathclap{\mathpalette\mathclapinternal}
\def\mathllapinternal#1#2{\llap{$\mathsurround=0pt#1{#2}$}}
\def\mathrlapinternal#1#2{\rlap{$\mathsurround=0pt#1{#2}$}}
\def\mathclapinternal#1#2{\clap{$\mathsurround=0pt#1{#2}$}}

% math-mode versions of \rlap, etc
% from Alexander Perlis, "A complement to \smash, \llap, and lap"
%   http://math.arizona.edu/~aprl/publications/mathclap/
\def\clap#1{\hbox to 0pt{\hss#1\hss}}
\def\mathllap{\mathpalette\mathllapinternal}
\def\mathrlap{\mathpalette\mathrlapinternal}
\def\mathclap{\mathpalette\mathclapinternal}
\def\mathllapinternal#1#2{\llap{$\mathsurround=0pt#1{#2}$}}
\def\mathrlapinternal#1#2{\rlap{$\mathsurround=0pt#1{#2}$}}
\def\mathclapinternal#1#2{\clap{$\mathsurround=0pt#1{#2}$}}

\newcommand{\horizontalrule}{
  \begin{center}
    \line(1,0){250}
  \end{center}
}

\newcommand{\code}[1]{\texttt{#1}}
\definecolor{light-gray}{gray}{0.9}
\newcommand{\highlight}[1]{\colorbox{light-gray}{#1}}

\newcommand{\im}[1]{\lvert #1 \rvert}
\newcommand{\powerset}[1]{\mathcal{P}(#1)}
\newcommand{\universe}{\mathbb{U}}

\newcommand{\turns}{\vdash}
\newcommand{\oftype}{\!:\!}
\newcommand{\subtype}{<:}
\newcommand{\commonsuper}{\Uparrow}
\newcommand{\commonsub}{\Downarrow}
\newcommand{\disjoint}{*}
\newcommand{\disjointimpl}{*_\textnormal{i}}
\newcommand{\disjointax}{*_\textnormal{ax}}
\newcommand{\subst}[2]{\lbrack #2 := #1 \rbrack~}

\newcommand{\yields}[1]{\highlight{$\; \hookrightarrow #1$}}
% \newcommand{\yields}[1]{}

\newcommand{\ftv}[1]{\textsf{ftv}(#1)}

\newcommand{\binderspace}{\,}
\newcommand{\appspace}{\;}

\newcommand{\inter}{\&}
\newcommand{\union}{|}
\newcommand{\for}[2]{\forall #1.\binderspace #2}
\newcommand{\fordis}[3]{\for {(#1 \disjoint #2)} {#3}}
\newcommand{\lam}[2]{\lambda #1.\binderspace #2}
\newcommand{\lamty}[3]{\lam {(#1 \oftype #2)} #3}
\newcommand{\blam}[2]{\Lambda #1.\binderspace #2}
\newcommand{\blamdis}[3]{\blam {(#1 \disjoint #2)} #3}
\newcommand{\mergeop}{,,}
\newcommand{\app}[2]{#1 \; #2}
\newcommand{\tapp}[2]{#1 \appspace #2}
\newcommand{\pair}[2]{(#1, #2)}
\newcommand{\proj}[2]{{\code{proj}}_{#1} #2}
\newcommand{\fst}[1]{\app {\code{fst}} {#1}}
\newcommand{\snd}[1]{\app {\code{snd}} {#1}}
\newcommand{\recordType}[2]{\{ #1 : #2 \}}
\newcommand{\recordCon}[2]{\{ #1 = #2 \}}

\newcommand{\true}{\code{True}}
\newcommand{\tyint}{\code{Int}}
\newcommand{\tybool}{\code{Bool}}
\newcommand{\tychar}{\code{Char}}
\newcommand{\tystring}{\code{String}}

% Judgements
\newcommand{\jwf}[2]{#1 \turns #2}
\newcommand{\jatomic}[1]{#1 \ \textnormal{atomic}}
\newcommand{\jtype}[3]{\turns #2 \ \oftype \ #3}
\newcommand{\jdis}[3]{\turns #2 \disjoint #3}
\newcommand{\jdisimpl}[3]{\turns #2 \disjointimpl #3}

\newcommand{\reflabel}[1]{(\textsc{#1})}

% \newcommand{\name}{$ \lambda_{\&} $\xspace}
\newcommand{\name}{\xspace[name]\xspace}

\newcommand{\authornote}[3]{\textcolor{#2}{\textsc{#1}: #3}}
\newcommand\bruno[1]{\authornote{bruno}{Red}{#1}}
\newcommand\george[1]{\authornote{george}{Blue}{#1}}

\newcommand{\formsub}{\framebox{$ A \subtype B \yields E $}}

\newcommand{\makelabelsub}[1]{Sub\_#1}

\newcommand{\labelsubint}{\makelabelsub Int}
\newcommand{\rulesubint}{
  \inferrule* [right=\labelsubint]
    { }
    {\tyint \subtype \tyint \yields {\lamty x {\im \alpha} x}}
}

\newcommand{\labelsubtop}{\makelabelsub Top}
\newcommand{\rulesubtop}{
  \inferrule* [right=\labelsubtop]
    { }
    {A \subtype \top \yields {\lamty x {\im A} \unit}}
}

\newcommand{\labelsubvar}{\makelabelsub Var}
\newcommand{\rulesubvar}{
  \inferrule* [right=\labelsubvar]
    { }
    {\alpha \subtype \alpha \yields {\lamty x {\im \alpha} x}}
}

\newcommand{\labelsubfun}{\makelabelsub Fun}
\newcommand{\rulesubfun}{
  \inferrule* [right=\labelsubfun]
    {{B_1} \subtype {A_1} \yields {E_1} \\
     {A_2} \subtype {B_2} \yields {E_2}}
    {{A_1 \to A_2} \subtype {B_1 \to B_2}
    \yields
        {\lamty f {\im {A_1 \to A_2}}
        {\lamty x {\im {B_1}}
            {\app {E_2} {(\app f {(\app {E_1} x)})}}}}}
}

\newcommand{\labelsubforall}{\makelabelsub Forall}
\newcommand{\rulesubforall}{
  \inferrule* [right=\labelsubforall]
    {{A_1} \subtype {A_2} \yields E}
    {\for {\alpha} {A_1} \subtype \for {\alpha} {A_2}
      \yields
        {\lamty f {\im {\for {\alpha} {A_1}}}
          {\blam \alpha {\app E {(\app f \alpha)}}}}}
}

\newcommand{\rulesubforalldis}{
  \inferrule* [right=\labelsubforall]
    {{B_1} \subtype {B_2} \yields E}
    {\fordis {\alpha} A {B_1} \subtype \fordis {\alpha} A {B_2}
      \yields
        {\lamty f {\im {\for {\alpha} {B_1}}}
          {\blam \alpha {\app E {(\app f \alpha)}}}}}
}

\newcommand{\rulesubforallext}{
  \inferrule* [right=\labelsubforall]
    {{B_1} \subtype {B_2} \yields E \\
     {A_2} \subtype {A_1} \yields {E_d}}
    {\fordis {\alpha} {A_1} {B_1} \subtype \fordis {\alpha} {A_2} {B_2}
      \yields
        {\lamty f {\im {\for {\alpha} {B_1}}}
          {\blam \alpha {\app E {(\app f \alpha)}}}}}
}

\newcommand{\labelsubinter}{\makelabelsub Inter}
\newcommand{\rulesubinter}{
  \inferrule* [right=\labelsubinter]
    {{A_1} \subtype {A_2} \yields {E_1} \\
     {A_1} \subtype {A_3} \yields {E_2}}
    {{A_1} \subtype {A_2 \inter A_3}
      \yields
        {\lamty x {\im {A_1}}
          {\pair {\app {E_1} x} {\app {E_2} x}}}}
}

\newcommand{\labelsubinterl}{\makelabelsub {Inter\_1}}
\newcommand{\rulesubinterl}{
  \inferrule* [right=\labelsubinterl]
    {{A_1} \subtype {A_3} \yields E}
    {{A_1 \inter A_2} \subtype {A_3}
      \yields
        {\lamty x {\im {A_1 \inter A_2}}
          {\app E {(\proj 1 x)}}}}
}

\newcommand{\rulesubinterldis}{
\inferrule* [right=\labelsubinterl]
    {{A_1} \subtype {A_3} \yields E \\ \jatomic {A_3}}
    {{A_1 \inter A_2} \subtype {A_3}
      \yields
        {\lamty x {\im {A_1 \inter A_2}}
          {\app E {(\proj 1 x)}}}}
}

\newcommand{\labelsubinterr}{\makelabelsub {Inter\_2}}
\newcommand{\rulesubinterr}{
  \inferrule* [right=\labelsubinterr]
    {{A_2} \subtype {A_3} \yields E}
    {{A_1 \inter A_2} \subtype {A_3}
      \yields
        {\lamty x {\im {A_1 \inter A_2}}
          {\app E {(\proj 2 x)}}}}
}

\newcommand{\rulesubinterrdis}{
  \inferrule* [right=\labelsubinterr]
    {{A_2} \subtype {A_3} \yields E \\ \jatomic {A_3}}
    {{A_1 \inter A_2} \subtype {A_3}
      \yields
        {\lamty x {\im {A_1 \inter A_2}}
          {\app E {(\proj 2 x)}}}}
}

\newcommand{\rulesubinterlcoerce}{
  \inferrule* [right=\labelsubinterl]
    {{A_1} \subtype {A_3} \yields E \\ \jatomic {A_3} }
    {{A_1 \inter A_2} \subtype {A_3}
      \yields
        { \lamty x {\im {A_1 \inter A_2}} {\andcoerce{A_3}_{{(
          {\app E {(\proj 1 x)}})}}}}}
}

\newcommand{\rulesubinterrcoerce}{
  \inferrule* [right=\labelsubinterr]
    {{A_2} \subtype {A_3} \yields E \\ \jatomic {A_3} }
    {{A_1 \inter A_2} \subtype {A_3}
      \yields 
        { \lamty x {\im {A_1 \inter A_2}} {\andcoerce{A_3}_{{(
          {\app E {(\proj 2 x)}})}}}}}
}

\newcommand{\ruleevarelab} {
\inferrule* [right=$\rulelabelevar$]
  {(x,\tau) \in \gamma}
  {\judgee \gamma x \tau \yields x}
}

\newcommand{\ruleelamelab} {
\inferrule* [right=$\rulelabelelam$]
  {\judgee {\gamma, x \hast \tau} e {\tau_1} \yields E \andalso \judgeewf \gamma \tau}
  {\judgee \gamma {\lam x \tau e} {\tau \to \tau_1} \yields {\lam x {\im \tau} E}}
}

\newcommand{\ruleeappelab}{
\inferrule* [right=$\rulelabeleapp$]
  {\judgee \gamma {e_1} {\tau_1 \to \tau_2} \yields {E_1} \\
   \judgee \gamma {e_2} {\tau_3} \yields {E_2} \andalso
   \tau_3 \subtype \tau_1 \yields C}
  {\judgee \gamma {\app {e_1} {e_2}} {\tau_2} \yields {\app {E_1} {(\app C E_2)}}}
}

\newcommand{\ruleeblamelab}{
\inferrule* [right=$\rulelabeleblam$]
  {\judgee {\gamma, \alpha} e \tau \yields E}
  {\judgee \gamma {\blam \alpha e} {\for \alpha \tau} \yields {\blam \alpha E}}
}

\newcommand{\ruleetappelab}{
\inferrule* [right=$\rulelabeletapp$]
  {\judgee \gamma e {\for \alpha {\tau_1}} \yields E \andalso \judgeewf \gamma \tau}
  {\judgee \gamma {\tapp e \tau} {\subst \tau \alpha \tau_1} \yields {\tapp E {\im \tau}}}
}

\newcommand{\ruleemergeelab}{
\inferrule* [right=$\rulelabelemerge$]
  {\judgee \gamma {e_1} {\tau_1} \yields {E_1} \andalso
   \judgee \gamma {e_2} {\tau_2} \yields {E_2}}
  {\judgee \gamma {e_1 \mergeop e_2} {\tau_1 \andop \tau_2} \yields {\pair {E_1} {E_2}}}
}

\newcommand{\ruleerecconelab}{
\inferrule* [right=$\rulelabelereccon$]
  {\judgee \gamma e \tau \yields E}
  {\judgee \gamma {\reccon l e} {\recty l \tau} \yields E}
}

\newcommand{\ruleerecprojelab}{
\inferrule* [right=$\rulelabelerecproj$]
  {\judgee \gamma e \tau \yields E \andalso
   \judgeget \tau l {\tau_1} \yields C}
  {\judgee \gamma {e.l} {\tau_1} \yields {\app C E}}
}

\newcommand{\ruleerecupdelab}{
\inferrule* [right=$\rulelabelerecupd$]
  {\judgee \gamma e \tau \yields E \andalso
   \judgee \gamma {e_1} {\tau_1} \yields {E_1} \\
   \judgeput \tau l {\tau_1 \yields {E_1}} {\tau_2} {\tau_3} \yields C \andalso
   \tau_1 \subtype \tau_3}
  {\judgee \gamma {\recupd e l {e_1}} {\tau_2} \yields {\app C E}}
}
\newcommand{\ruleget}{
  \inferrule* [right=$\rulelabelget$]
  { }
  {\judgeget {\recty l \tau} l \tau}
}

\newcommand{\rulegetelab}{
  \inferrule* [right=$\rulelabelget$]
  { }
  {\judgeget {\recty l \tau} l \tau \yields {\lam x {\im {\recty l \tau}} x}}
}

\newcommand{\rulegetleft}{
  \inferrule* [right=$\rulelabelgetleft$]
  {\judgeget {\tau_1} l \tau}
  {\judgeget {\tau_1 \andop \tau_2} l \tau}
}

\newcommand{\rulegetleftelab}{
  \inferrule* [right=$\rulelabelgetleft$]
  {\judgeget {\tau_1} l \tau \yields C}
  {\judgeget {\tau_1 \andop \tau_2} l \tau \yields {\lam x {\im {\tau_1
          \andop \tau_2}} {\app C {(\proj 1 x)}}}}
}

\newcommand{\rulegetright}{
  \inferrule* [right=$\rulelabelgetright$]
  {\judgeget {\tau_2} l \tau}
  {\judgeget {\tau_1 \andop \tau_2} l \tau}
}

\newcommand{\rulegetrightelab}{
  \inferrule* [right=$\rulelabelgetright$]
  {\judgeget {\tau_2} l \tau \yields C}
  {\judgeget {\tau_1 \andop \tau_2} l \tau \yields {\lam x {\im {\tau_1
          \andop \tau_2}} {\app C {(\proj 2 x)}}}}
}

\newcommand{\ruleput}{
\inferrule* [right=$\rulelabelput$]
  { }
  {\judgeput {\recty l \tau} l {\tau_1} {\recty l {\tau_1}} \tau}
}

\newcommand{\ruleputelab}{
\inferrule* [right=$\rulelabelput$]
  { }
  {\judgeput {\recty l \tau} l {\tau_1 \yields E} {\recty l {\tau_1}} \tau
  \yields {\lam \_ {\im {\recty l \tau}} E}}
}

\newcommand{\ruleputleft}{
\inferrule* [right=$\rulelabelputleft$]
  {\judgeput {\tau_1} l \tau {\tau_3} {\tau_4}}
  {\judgeput {\tau_1 \andop \tau_2} l \tau {\tau_3 \andop \tau_2} {\tau_4}}
}

\newcommand{\ruleputleftelab}{
\inferrule* [right=$\rulelabelputleft$]
  {\judgeput {\tau_1} l {\tau \yields E} {\tau_3} {\tau_4} \yields C}
  {\judgeput {\tau_1 \andop \tau_2} l {\tau \yields E} {\tau_3 \andop \tau_2} {\tau_4}
  \yields {\lam x {\im {\tau_1 \andop \tau_2}} {\app C {(\proj 1 x)}}}}
}

\newcommand{\ruleputright}{
\inferrule* [right=$\rulelabelputright$]
  {\judgeput {\tau_2} l \tau {\tau_3} {\tau_4}}
  {\judgeput {\tau_1 \andop \tau_2} l \tau {\tau_1 \andop \tau_3} {\tau_4}}
}

\newcommand{\ruleputrightelab}{
\inferrule* [right=$\rulelabelputright$]
  {\judgeput {\tau_2} l {\tau \yields E} {\tau_3} {\tau_4} \yields C}
  {\judgeput {\tau_1 \andop \tau_2} l {\tau \yields E} {\tau_1 \andop \tau_3} {\tau_4}
  \yields {\lam x {\im {\tau_1 \andop \tau_2}} {\app C {(\proj 2 x)}}}}
}
\newcommand{\makelabeltgt}[1]{Tgt\_#1}

% Target WF
\newcommand{\formtgtwf}{\framebox{$ \jwf G T $}}

\newcommand{\makelabeltgtwf}[1]{\makelabeltgt {WF\_#1}}

\newcommand{\labeltgtwffv}{\makelabeltgtwf FV}
\newcommand{\ruletgtwffv} {
\inferrule* [right=\labeltgtwffv]
  {\ftv T \in G}
  {\jwf G T}
}

% Target typing
\newcommand{\formtgt}{\framebox{$ \jtype G E T $}}

\newcommand{\makelabeltgtty}[1]{\makelabeltgt {Ty\_#1}}

\newcommand{\labeltgtvar}{\makelabeltgtty Var}
\newcommand{\ruletgtvar} {
\inferrule* [right=\labeltgtvar]
  {(x,T) \in \Gamma}
  {\jtype \Gamma x T}
}

\newcommand{\labeltgtlam}{\makelabeltgtty Lam}
\newcommand{\ruletgtlam} {
\inferrule* [right=\labeltgtlam]
  {\jtype {\Gamma, x \oftype T} E {T_1} \andalso \jwf \Gamma T}
  {\jtype \Gamma {\lamty x T E} {T \to T_1}}
}

\newcommand{\labeltgtapp}{\makelabeltgtty App}
\newcommand{\ruletgtapp}{
\inferrule* [right=\labeltgtapp]
  {\jtype \Gamma {E_1} {T_1 \to T_2} \andalso \jtype \Gamma {E_2} {T_1}}
  {\jtype \Gamma {\app {E_1} {E_2}} {T_2}}
}

\newcommand{\labeltgtblam}{\makelabeltgtty BLam}
\newcommand{\ruletgtblam}{
\inferrule* [right=\labeltgtblam]
  {\jtype {\Gamma, \alpha} E T}
  {\jtype \Gamma {\blam \alpha E} {\for \alpha T}}
}

\newcommand{\labeltgttapp}{\makelabeltgtty TApp}
\newcommand{\ruletgttapp}{
\inferrule* [right=\labeltgttapp]
  {\jtype \Gamma E {\for \alpha {T_1}} \andalso \jwf \Gamma T}
  {\jtype \Gamma {\tapp E T} {\subst T \alpha T_1}}
}

\newcommand{\labeltgtpair}{\makelabeltgtty Pair}
\newcommand{\ruletgtpair}{
\inferrule* [right=\labeltgtpair]
  {\jtype \Gamma {E_1} {T_1} \andalso \jtype \Gamma {E_2} {T_2}}
  {\jtype \Gamma {\pair {E_1} {E_2}} {\pair {T_1} {T_2}}}
}

\newcommand{\labeltgtprojl}{\makelabeltgtty {Proj\_1}}
\newcommand{\ruletgtprojl}{
\inferrule* [right=\labeltgtprojl]
  {\jtype \Gamma E {\pair {T_1} {T_2}}}
  {\jtype \Gamma {\proj 1 E} {T_1}}
}

\newcommand{\labeltgtprojr}{\makelabeltgtty {Proj\_2}}
\newcommand{\ruletgtprojr}{
\inferrule* [right=\labeltgtprojr]
  {\jtype \Gamma E {\pair {T_1} {T_2}}}
  {\jtype \Gamma {\proj 2 E} {T_2}}
}

\begin{mathpar}
  \framebox{$ \judgeSourceWF \gamma \tau $}

  \ruleSourceWFVar

  \ruleSourceWFTop

  \ruleSourceWFFun

  \ruleSourceWFForall

  \ruleSourceWFAnd

  \ruleSourceWFRec
\end{mathpar}

\newcommand{\name}{{\bf $F_{\&}$}\xspace}
%%\newcommand{\Name}{{\bf fi}}

\newcommand{\target}{{\bf f}\xspace}
\newcommand{\Target}{{\bf f}\xspace}

\newcommand{\authornote}[3]{{\color{#2} {\sc #1}: #3}}
\newcommand\bruno[1]{\authornote{bruno}{red}{#1}}
\newcommand\george[1]{\authornote{george}{blue}{#1}}

\lstdefinelanguage{scala}{
  morekeywords={abstract,case,catch,class,def,%
    do,else,extends,false,final,finally,%
    for,if,implicit,import,match,mixin,%
    new,null,object,override,package,%
    private,protected,requires,return,sealed,%
    super,this,throw,trait,true,try,%
    type,val,var,while,with,yield},
  otherkeywords={=>,<-,<\%,<:,>:,\#,@},
  sensitive=true,
  morecomment=[l]{//},
  morecomment=[n]{/*}{*/},
  morestring=[b]",
  morestring=[b]',
  morestring=[b]"""
}

\lstdefinelanguage{F2J}{
  morekeywords={let,rec,type,in},
  otherkeywords={->},
  sensitive=true,
  morecomment=[l]{--},
  morestring=[b]", % 'b' means inside a string delimiters are escaped by a backslash.
  morestring=[b]'
}

\lstset{ %
  % language=F2J,                % choose the language of the code
  columns=flexible,
  lineskip=-1pt,
  basicstyle=\ttfamily\small,       % the size of the fonts that are used for the code
  numbers=none,                   % where to put the line-numbers
  stepnumber=1,                   % the step between two line-numbers. If it's 1 each line will be numbered
  numbersep=5pt,                  % how far the line-numbers are from the code
  backgroundcolor=\color{white},  % choose the background color. You must add \usepackage{color}
  showspaces=false,               % show spaces adding particular underscores
  showstringspaces=false,         % underline spaces within strings
  showtabs=false,                 % show tabs within strings adding particular underscores
  tabsize=2,                  % sets default tabsize to 2 spaces
  captionpos=none,                   % sets the caption-position to bottom
  breaklines=true,                % sets automatic line breaking
  breakatwhitespace=false,        % sets if automatic breaks should only happen at whitespace
  title=\lstname,                 % show the filename of files included with \lstinputlisting; also try caption instead of title
  escapeinside={(*}{*)},          % if you want to add a comment within your code
  keywordstyle=\ttfamily\bfseries,
% commentstyle=\color{Gray}
% stringstyle=\color{Green}
}


\begin{document}

\special{papersize=8.5in,11in}
\setlength{\pdfpageheight}{\paperheight}
\setlength{\pdfpagewidth}{\paperwidth}

\conferenceinfo{CONF 'yy}{Month d--d, 20yy, City, ST, Country}
\copyrightyear{20yy}
\copyrightdata{978-1-nnnn-nnnn-n/yy/mm}
\doi{nnnnnnn.nnnnnnn}

\titlebanner{banner above paper title}        % These are ignored unless
\preprintfooter{\name}                        % 'preprint' option specified.

\title{System \name: A Simple Core Language for Extensibility}
%%\subtitle{Subtitle Text, if any}

\authorinfo{Name1}
           {Affiliation1}
           {Email1}
\authorinfo{Name2\and Name3}
           {Affiliation2/3}
           {Email2/3}

\maketitle

\begin{abstract}
  
  Over the years there have been various proposals for \emph{design
    patterns} for improved \emph{extensibility} of programs.
  Examples include \emph{Object Algebras}, \emph{Modular Visitors} or
  Torgersen's design patterns using generics.
  Although those design patterns give practical
  benefits in terms of extensibility, they also expose limitations in
  existing mainstream OOP languages. Some pressing
  limitations are: 1) lack of good mechanisms for
  \emph{object-level} composition; 2) \emph{conflation of 
    (type) inheritance with subtyping}; 3) \emph{heavy reliance on generics}.

  This paper presents System \name: an extension of System F with
  \emph{intersection types} and a \emph{merge operator}.  The goal of System \name
  is to study the minimal language constructs needed to support
  various extensible designs, while at the same time addressing the
  limitations of existing OOP languages. To address the lack of good
  object-level composition mechanisms, System \name uses the merge
  operator to do dynamic composition of values/objects. Moreover, in
  System \name type inheritance is independent of subtyping, and an
  extension can be a supertype of a base object type.  Finally, System
  \name replaces many uses of generics by intersection types or
  conventional subtyping. System \name is formalized and
  implemented. Moreover the paper shows how various extensible designs
  can be encoded in System \name.

\end{abstract}

%%\category{CR-number}{subcategory}{third-level}

% general terms are not compulsory anymore,
% you may leave them out
%%\terms
%%Design, Languages, Theory

%%\keywords
%%Intersecion Types, Polymorphism, Type System

\bruno{Make all figures fit and be well-formated!}
\section{Introduction}

There has been a remarkable number of works aimed at improving support
for extensibility in programming languages. These works include:
visions of new programming models~\cite{}; new programming languages or
language extensions~\cite{}, and \emph{design patterns} that can be
used with existing mainstream languages~\cite{}.

%\cite{family polymorphism and virtual
%classes.}. Another line of work are proposals for precise formal models or new 
%programming languages. Yet another line are \emph{design patterns}
%that can be used  with existing mainstream languages. 
%%Part of the motivation behind 

Some of the more recent work on extensibility is focused on various
proposals for design patterns.  Examples include \emph{Object
  Algebras}~\cite{}, \emph{Modular Visitors}~\cite{} or
Torgersen's~\cite{} four design patterns using generics. In those
approaches the idea is to use some advanced (but already available)
features, such as \emph{generics}, in combination with conventional
OOP features to model more extensible designs.  Those designs work in
modern OOP languages such as Java, C\# or Scala.

Although such design patterns give practical benefits in terms of
extensibility, they also expose limitations in existing mainstream OOP
languages. In particular there are three pressing limitations: 
1) lack of good mechanisms for
  \emph{object-level} composition; 2) \emph{conflation of 
    (type) inheritance with subtyping}; 3) \emph{heavy reliance on generics}.

  The first limitation shows up, for example, in Oliveira et
  al.~\cite{} encodings of Feature-Oriented Programming using Object
  Algebras~\cite{}. These programs are best expressed using a form of
  \emph{type-safe}, \emph{dynamic}, \emph{delegation}-based
  composition. Although such form of composition can be encoded in
  languages like Scala, it requires the use of low-level reflection
  techniques, such as dynamic proxies, reflection or other forms of
  meta-programming~\cite{}. It is clear that better language support
  would be desirable.

  The second limitation shows up in designs for modelling
  modular or extensible visitors~\cite{}.  The vast majority of modern
  OOP languages combines type inheritance and subtyping. 
  That is a type extension induces a subtype. However
  as Cook et al.~\cite{} famously argued there are programs where
  ``\emph{subtyping is not inheritance}''. Interestingly previously
  not many practical programs have been reported in the literature
  where the distinction between subtyping and inheritance is
  relevant. However, as shown in this paper, it turns out that this
  difference does show up in practice when designing modular
  (extensible) visitors.  We believe that modular visitors provide a
  compeling practical example where inheritance and subtyping should
  not be conflated!

  Finally, the third limitation is prevalent in many extensible
  designs~\cite{}. Such designs rely on advanced features of generics,
  such as \emph{f-bounded polymorphism}~\cite{}, \emph{variance
    annotations}~\cite{}, \emph{wildcards}~\cite{} and/or \emph{higher-kinded
    types}~\cite{} to achieve type-safety. Sadly, the amount of
  type-annotations, combined with the lack of understanding of these
  features, usually deters programmers from using such designs.

\begin{comment}
Motivated by the insights gained in previous work, this paper presents 
a minimal core calculus that addresses current limitations and
provides a better foundational model for statically typed
delegation-based OOP? We show that Object Algebras fit nicely in this
model. 
\end{comment}

This paper presents System \name: an extension of System F~\cite{}
with intersection types and a merge operator~\cite{}.  
The goal of System \name is to study the \emph{minimal} foundational language
constructs that are needed to support various extensible designs,
while at the same time addressing the limitations of existing OOP
languages. To address the lack of good object-level composition
mechanisms, System \name uses the merge operator to allow dynamic
composition of values/objects. Moreover, in System \name (type-level)
extension is independent of subtyping, and it is possible for an
extension to be a supertype of a base object type.


Technically speaking System \name is mainly inspired by the work of
Dundfield~\cite{}.  Dundfield shows how to model a simply typed
calculus with intersection types and a merge operator. The presence of
a merge operator adds significant expressiveness to the language,
allowing encodings for many other language constructs as syntactic
sugar. System \name differs from Dundfield's work in a few
ways. Firstly it adds parametric polymorphism and formalizes a
extension for records to support a basic form of objects. Secondly,
the elaboration semantics into System F is done directly from the
source calculus with subtyping. In contrast Dunfield has an additional
step which eliminates subtyping.  Finally a non-technical difference
is that System \name is aimed at studying issues of OOP languages and
extensibility, whereas Dunfield's work was aimed at Functional
Programming and he did not consider applications to extensibility.
Like many other foundational formal models for OOP (for
example~\cite{}), System \name is purely functional and it uses
structural typing.

System \name is
formalized and implemented. Furthermore the paper illustrates how
various extensible designs can be encoded in System \name.

\begin{comment}
We present a polymorphic calculus containing intersection types and records, and show
how this language can be used to solve various common tasks in functional
programming in a nicer way.Intersection types provides a power mechanism for functional programming, in
particular for extensibility and allowing new forms of composition.

Prototype-based programming is one of the two major styles of object-oriented
programming, the other being class-based programming which is featured in
languages such as Java and C\#. It has gained increasing popularity recently
with the prominence of JavaScript in web applications. Prototype-based
programming supports highly dynamic behaviors at run time that are not possible
with traditional class-based programming. However, despite its flexibility,
prototype-based programming is often criticized over concerns of correctness and
safety. Furthermore, almost all prototype-based systems rely on the fact that
the language is dynamically typed and interpreted.
\end{comment}

In summary, the contributions of this paper are:

\begin{itemize}

\item {\bf A Minimal Core Language for Extensibility:} This paper
  identifies a minimal core language, System \name, capable of
  expressing various extensibility designs in the literature.
  System \name also addresses limitations of existing OOP
  languages that complicate extensible designs. 
  
\item {\bf Formalization of System \name:} An elaboration semantics of
  System \name into System F is given, and type-soundness is proved.

\item {\bf Encodings of Extensible Designs:} Various encodings of
  extensible designs into System \name, including \emph{Object
    Algebras} and \emph{Modular Visitors}. 

\item {\bf Implementation and Examples:} An implementation of an
  extension of System \name, as well as the examples presented in the
  paper, are publicly available. 

\begin{comment}

\item{elaboration typing rules which given a source expression with intersection
    types, typecheck and translate it into an ordinary F term. Prove a type
    preservation result: if a term $ e $ has type $ \tau $ in the source language,
    then the translated term $ \image e $ is well-typed and has type $ \image \tau $ in the
    target language.}

\item{present an algorithm for detecting incoherence which can be very important
    in practice.}

\item{explores the connection between intersection types and object algebra by
    showing various examples of encoding object algebra with intersection
    types.}

\end{comment}

\end{itemize}

\begin{comment}
\subsection{Other Notes}

finitary overloading: yes
but have other merits of intersection been explored?

-- Compare Scala:
-- merge[A,B] = new A with B

-- type IEval  = { eval :  Int }
-- type IPrint = { print : String }

-- F[\_]
\end{comment}
\section{Overview} \label{sec:overview}

\joao{review this paragraph once the section is finished}
This section introduces \name and its support for intersection types,
parametric polymorphism and the merge operator. 
It then discusses the issue of coherence and shows how the notion of disjoint
intersection types and disjoint quantification achieve a coherent semantics.

Note that this section uses some syntactic sugar, as well as standard
programming language features, to illustrate the various concepts in \name. 
Although the minimal core language that we formalize in
Section~\ref{sec:fi} does not present all such features, our implementation
supports them.

\subsection{Intersection Types and the Merge Operator}
%%\subsection{Intersection Types, Merge and Polymorphism in \namedis}
%\bruno{We should't simply copy and paste the text from the previous
%  paper. We should try to at least rephrase some things.}

Intersection types date back as early as Coppo et al.'s work~\cite{coppo1981functional}. 
Since then various researchers have studied intersection types, and some languages have 
adopted them in one form or another.
%However, as we shall see in
%Section~\ref{subsec:incoherence}, it also introduces difficulties. In what follows
%intersection types and the merge operator are informally introduced.

\paragraph{Intersection types.}
The intersection of type $A$ and $B$ (denoted as \lstinline{A & B} in
\name) contains exactly those values
which can be used as either values of type $A$ or of type $B$. 
For instance, consider the following program in \name:

\begin{lstlisting}
let x : Int & Char = (*$ \ldots $*) in -- definition omitted
let idInt (y : Int) : Int = y in
let idChar (y : Char) : Char = y in
(idInt x, idChar x)
\end{lstlisting}
\bruno{I had comment about not having the same text. I should have
  perhaps have been a bit more clear and say that having the same code
is ok, it is just the text that needs a little rewriting. In fact now
the code examples are worst: identity functions are pretty boring and
they don't really illustrate the point.}

\noindent If a value \lstinline{x} has type \lstinline{Int & Char} then
\lstinline{x} can be used anywhere where either values of type \lstinline{Int} or a 
\lstinline{Char} are expected. 
This means that, in the program above
the functions \lstinline{idInt} and \lstinline{idChar} -- the
identity functions on integers and characters, respectively -- 
both accept \lstinline{x} as an argument. 

\paragraph{Merge operator.}
The previous program deliberately ommitted the introduction of values of an
intersection type. 
There are many variants of intersection types in the literature. 
Our work follows a particular formulation, where
intersection types are introduced by a \emph{merge operator}. \joao{add citations?}
As Dunfield~\cite{dunfield2014elaborating} has argued a merge operator adds considerable
expressiveness to a calculus. 
The merge operator allows two values to be merged in a single intersection type. 
For example, an implementation of \lstinline{x} is constructed in \name as follows:

\begin{lstlisting}
let x : Int & Char = 1,,'c' in (*$ \ldots $*)
\end{lstlisting}
\noindent In \name (following Dunfield's notation), the
merge of two values $v_1$ and $v_2$ is denoted as $v_1 ,, v_2$.

\paragraph{Merge vs Pairs}
The significant difference between intersection types with a
merge operator and regular pairs is in the elimination construct. 
With pairs there are explicit eliminators (\lstinline{fst} and
\lstinline{snd}), and these eliminators must be used to extract the
components of the right type.
With intersection types and a merge operator, elminators are implicit in the language,
meaning no uses of projection functions are necessary.

%\paragraph{Merge operator and pairs.}
%The merge operator is similar to the introduction construct on pairs.
%An analogous implementation of \lstinline{x} with pairs would be:

%\begin{lstlisting}
%let xPair : (Int, Char) = (1, 'c') in (*$ \ldots $*)
%\end{lstlisting}

% For example, in order to use
%\lstinline{idInt} and \lstinline{idChar} with pairs, we would need to
%write a program such as:

%\begin{lstlisting}
%(idInt (fst xPair), idChar (snd xPair))
%\end{lstlisting}

%\noindent In contrast the elimination of intersection types is done
%implicitly, by following a type-directed process. For example,
%when a value of type \lstinline{Int} is needed, but an intersection
%of type \lstinline{Int & Char} is found, the compiler uses the
%type system to extract the corresponding value.

\subsection{Coherence and Disjointness}
\label{subsec:coherence}
Coherence is a desirable property for a semantics. 
A semantics is said to be coherent if any \emph{valid program} has exactly one
meaning~\cite{reynolds1991coherence} (that is, the semantics is not ambiguous).
Unfortunately the implicit nature of elimination for intersection
types built with a merge operator can lead to incoherence,
This is due to intersections with overlapping types, as in
$\tyint \inter \tyint$.
For example, the result of the program (\lstinline$(1,,2) : Int$)
can be either \lstinline$1$ or \lstinline$2$, depending on the implementation 
of the language.

One option to restore coherence is to reject programs which may have
multiple meanings.
The \oldname \joao{add reference} system -- a calculus with
a simply-typed calculus with intersection types and 
a merge operator, in the same flavour of Dunfield's -- solves this problem
by using the concept of disjoint intersections. 

\begin{comment}
If both results are accepted, we say that the semantics is
\emph{incoherent}: there are multiple possible meanings for the same
valid program. 
Dunfield's calculus~\cite{dunfield2014elaborating} is incoherent and accepts the
program above.\bruno{Well this text needs a significant revision
  because now our ICFP paper largely solves this problem. So you want,
at some point, to summarise the key result of the ICFP paper: we know
how todo coherent intersection types for a simply typed calculus. 
Then you want to setup the problem for this paper: but how about
polymorphism? what are the challenges, why is it hard?}
\end{comment}

\paragraph{Disjoint intersections}
%Of course, when rejecting programs it is important
%not to be too conservative, and reject too many programs which are
%actually coherent.
The incoherence problem with the expression $1 \mergeop 2$
happens because there are two overlapping integers in the merge. 
Generally speaking, if both terms can be assigned some type $C$
then both of them can be chosen as the meaning of the merge,
which in its turn leads to multiple meanings of a term.
Thus a natural option is to forbid such overlapping
values of the same type in a merge.

This is precisely the approach taken in \oldname: a merge can only be composed of
two values as long as their types are \emph{disjoint}.  
Disjointness is a binary relation between two types, defined for any types which
do not contain any overlapping types.
It can be specified as: given any two types, they are disjoint if there does not 
exist any common super-type.

\begin{comment}
More formally, the notion of disjointness can be specified as follows:
\bruno{I think we should avoid presenting the specification, since we
  do not have one for this paper. We can refer to this in the related
  work.}

\begin{definition}[Disjointness]
  Given two types $A$ and $B$, two types are disjoint
  (written $A \disjoint B$) if there is no type $C$ such that both $A$ and $B$ are
  subtypes of $C$:
  \[A \disjoint B \equiv \not\exists C.~A \subtype C \wedge B \subtype C\]
\end{definition}
\end{comment}

With this concept of disjointness in mind, it is easy to verify that the previous example 
will no longer be accepted since \lstinline$((1,,2) : Int)$ is no longer well-typed!
The merge operator requires two types to be disjoint in order to type-check it into
an intersection.
In the example \lstinline$(1,,2)$ is rejected by the compiler, since \lstinline$Int$ is not
disjoint with \lstinline$Int$.
In other words, there exists a super-type of both \lstinline$Int$ and \lstinline$Int$, 
namely \lstinline$Int$ itself.
This result can be generalized and \oldname has shown that it can lead to a coherent calculus. 
Although this is a promising result, the question remains: is it possible to incorporate 
parametric polymorphism in such calculus, while retaining coherence?

%\oldname not only provided a specification for disjointness but also an algorithmic version
%of it, and proved that both versions are equivalent.
%This turned disjointness as a concept which is both easy to understand and easy to implement. 

\begin{comment}
\subsection{Top-like types}\bruno{You are spending too much time to
  get to the point. When writting a paper you want to get to the point
  (what's the problem) 
as fast as possible. So you should have enough background to understand
the paper, but keep that background minimal. Look at the ICFP paper: 
we got to Section 2.2 (the problem) in less than a column.
I think we don't need to
cover top-like types here. They are not essential. }
The \oldname calculus also showed how to extend the type system with a type $\top$, the supertype 
of all types.
Since introducing $\top$ leads to a useless disjointness specification (i.e. no type is disjoint to
any other type) and introduces some ambiguity because $\top \subtype \top \inter \top$ and
$\top \inter \top \subtype \top$.
Therefore, the specification was changed to the following:

\begin{definition}[$\top$-disjointness]
  Two types $A$ and $B$ are disjoint
  (written $A \disjoint B$) if the following two conditions are satisfied:
\begin{enumerate}
  \item $(\text{not}~\toplike{A})~\text{and}~(\text{not}~\toplike{B}) $
  \item $\forall C.~\text{if}~ A \subtype C~\text{and}~B \subtype C~\text{then}~\toplike{C}$
\end{enumerate}
\end{definition}
The unary relation $\toplike{\cdot}$ represents the so-called top-like types, which are types that resemble 
$\top$.
This set of types includes $\top$ itself, intersections composed of other 
top-like types (i.e. $\top \inter \top$), and pre-top-types, which are functions that have
$\top$ as their co-domain (i.e. $\tyint \to \tychar \to \top$).
\end{comment}

\subsection{Parametric Polymorphism}
Unfortunately, adding parametric polymorphism is non-trivial.
A naive attempt to add polymorphism would consist in introducing a forall type, type variables, and a 
big lambda at term level.
Type variables can be assumed as disjoint to any other type, as a starting point.
Now consider the attempt to write the following polymorphic function in such system (we will use
uppercase Latin letters to denote type variables): 
\begin{lstlisting}
let fst A B (x: A & B) : A = (x : A) in (*$ \ldots $*)
\end{lstlisting}
The \code{fst} function is supposed to extract a value of type
\lstinline{A} from the merge value $x$ (of type \lstinline{A&B}). 
This function is problematic: when
\lstinline{A} and \lstinline{B} are instantiated to non-disjoint
types, then uses of \lstinline{fst} may lead to incoherence.
For example, consider the following use of \lstinline{fst}:
\begin{lstlisting}
fst Int Int (1,,2)
\end{lstlisting}
\noindent This program is clearly incoherent as both
$1$ and $2$ can be extracted from the merge and still match the type
of the first argument of \lstinline{fst}.

\paragraph{Biased choice breaks equational reasoning.} 
At first sight, one option
to workaround the incoherence issue would be to bias the type-based merge lookup
to the left or to the right. %(as discussed in Section~\ref{subsec:incoherence}). 
However, biased choice is
very problematic when parametric polymorphism is present in the language.
To see the issue, suppose we chose to always pick the
rightmost value in a merge when multiple values of same type exist.
Intuitively, it would appear that the result of the use of
\lstinline{fst} above is $2$. 
Indeed simple equational reasoning seems to validate such result:
\begin{lstlisting}
   fst Int Int (1,,2)
(*$ \rightsquigarrow $*) ((fun z (*$ \to $*) z) : Int (*$ \to $*) Int) (1,,2) -- (* \textnormal{By the definition of \code{fst}} *)
(*$ \rightsquigarrow $*) ((fun z (*$ \to $*) z) : Int (*$ \to $*) Int) 2      -- (* \textnormal{Right-biased coercion} *)
(*$ \rightsquigarrow $*) 2                                  -- (* \textnormal{By $\beta$-reduction} *)
\end{lstlisting}

However (assuming a straightforward implementation of right-biased
choice) the result of the program would be 1! The reason for this has
todo with \emph{when} the type-based lookup on the merge happens. In
the case of \lstinline{fst}, lookup is triggered by a coercion
function inserted in the definition of \lstinline{fst} at
compile-time.
In the definition of \lstinline$fst$ all it is known is that a
value of type $A$ should be returned from a merge with an intersection
type $A\&B$.  
Clearly the only type-safe choice to coerce the value of type $A\&B$ into $A$ is to
take the left component of the merge. 
This works perfectly for merges
such as \lstinline$(1,,'c')$, where the types of the first and second components
of the merge are disjoint. 
For the merge \lstinline$(1,,'c')$, if a integer lookup
is needed, then \lstinline$1$ is the rightmost integer, which is consistent with the
biased choice. 
Unfortunately, when given the merge \lstinline$(1,,2)$ the
left-component \lstinline$1$ is also picked up, even though in this case \lstinline$2$
is the rightmost integer in the merge. 

The subtle interaction of polymorphism and type-based lookup
means that equational reasoning is broken.
In the equational reasoning steps above, doing apparently correct
substitutions lead us to a wrong result. 
This is a major problem for biased choice and a reason to dismiss it as a possible implementation
choice for \name.

\paragraph{A more conservative attempt.}
Another attempt at restoring coherence can be to forbid type variables inside
intersections (i.e. type variables are not disjoint to any type).
This conservative approach would solve the problem of coherence, but it would also greatly 
restrict the expressiveness of the resulting language.
For example, the function $fst$ defined above, would no longer be accepted by the system.
In fact, parametric polymorphism and intersection types could only be mixed in a very limited manner -
as long as variables do not reside under intersections - and this is arguably a useful improvement
in respect to other standard type systems, such as System $F$.

\begin{comment}
\subsection{Disjoint Polymorphism}
A more liberal solution,which enables the combination of type
variables and intersection types, is disjoint polymorphism.
Disjoint polymorphism assign constraints to each type variable, which
allows delaying the check for disjointness until type application/instantiation. 
\end{comment}

\subsection{Disjoint Quantification}
To avoid being overly conservative, while still retaining coherence in the
presence of parametric polymorphism and intersection types, \name uses
an extension to universal quantification called \emph{disjoint quantification}.
Inspired by bounded quantification~\cite{Cardelli:1994},
where a type variable is constrained by a type bound, disjoint quantification 
allows a type variable to be constrained so that it is disjoint with a given type. 
With disjoint quantification a variant of the program $fst$, which
is accepted by \name, would be written as:
\begin{lstlisting}
let fst A ((*$ \highlight {B *~A} $*)) (x: A & B) = (x : A)
in (*$ \ldots $*)
\end{lstlisting}
The small change is in the declaration of the type parameter $B$. The notation
$B*A$ means that in this program the type variable $B$ is constrained so that
it can only be instantiated with any type disjoint to $A$.
This ensures that the
merge denoted by $x$ is disjoint for all valid instantiations of $A$ and $B$.
In other words, only coherent uses of \lstinline$fst$ will be accepted.
For example, the following use of \lstinline$fst$:
\begin{lstlisting}
fst Int Char (1,,'c')
\end{lstlisting}
is accepted since \lstinline$Int$ and \lstinline$Char$ are disjoint, thus satisfying the constraint
on the second type parameter of \lstinline$fst$.
Furthermore, problematic uses of \lstinline$fst$ are rejected. 
However, the following use of \lstinline$fst$:
\begin{lstlisting}
fst Int Int (1,,2)
\end{lstlisting}
is rejected because \lstinline$Int$ is not disjoint with \lstinline$Int$, thus failing to satisfy the
disjointness constraint on the second type parameter of \lstinline$fst$.

\paragraph{Empty constraint}
Even though disjoint quantification solves the problem of coherence, there is still one detail 
that needs further justification.
The reader might have noticed how we ommitted the disjointness constraint of 
the type variable $A$ in the \lstinline$fst$ function.
This actually means that $A$ should be associated with the empty contraint,
which raises the question: which type should be used to represent such empty constraint?
Or, in other words, which type is disjoint to every other type? 
It is obvious that this type should be one of the bounds of the subtyping lattice: either $\bot$ or
$\top$.

The essential intuition here is that the more specific a type in the subtyping relation is, the less types
exist that are disjoint to it.
For example, $\tyint$ is disjoint with all types except the intersection that contain $\tyint$, $\tyint$
itself, and $\bot$; while $\tyint \inter \tychar$ is disjoint to all types that $\tyint$ is, plus the
types disjoint to $\tychar$.
Thus, the more specific a type variable constraint is, the less options we have to instantiate it with.
This reasoning implies that $\top$ should be treated as the empty constraint.
Indeed, in \name, a single type variable $A$ is only syntactic sugar for $A * \top$.
\joao{should we say anything here about this going against our
  previous top-disjointness formulation?}
\bruno{yes, but not here. We can discuss this in later sections when
  discussing the technical details.}
For instance, the type of the identity function in System $F$ that reads $\forall A. A \to A$ is 
equivalent to the \name's type $\forall (A * \top). A \to A$.

%Let us assume two distinct interpretations for the subtyping relation.
%Given two types $A$ and $B$, we can say relation is either:
%\begin{enumerate}
%\item Subset relation ($A \subseteq B$): every element of type $A$ is also of type $B$; or
%\item Coercion ($\exists t. t : A \to B$): every element of type $A$ can be coerced into type $B$.
%\end{enumerate}

\begin{comment}
First, if we consider our subtyping lattice as unbounded (i.e. no $\top$ and no $\bot$), then we have that
disjointness is covariant with respect to the subtyping relation.
More formally:
\[ \inferrule {\jwf \Gamma {A \disjoint B} \\ B \subtype C }
              {\jwf \Gamma {A \disjoint C}} \]

To illustrate this, take $A$ as $\tyint$ and $B$ as $\tybool \inter \tychar$.
This lemma states that every supertype of $\tybool \inter \tychar$, namely $\tybool$, $\tychar$ and 
$\tybool \inter \tychar$ itself are also disjoint with $\tyint$.
Coming back to a bounded subtyping lattice, let us now consider both bounds. 
If some type $A$ were to be disjoint with $\bot$, then by the lemma above $A$ will be disjoint to
virtually any type.
This means that, if $A$ is a type variable, then the possible types that it can be instantiated with
are the ones which are disjoint with every other type (otherwise the lemma above will no longer hold).
Clearly $\bot$ is not a suitable option,  
In other words, we can think of $\bot$ as the type as specific as the infinite intersection.
Conversely, $\top$ can be thought as specific - or rather, as general - as the 0-ary intersection.
\end{comment}

%\joao{do we want to say this here?:}
%However, the previous specification of $\top$-disjointness does not reflect this, since it states that $\top$ 
%is not disjoint to any other type, not even to itself.
%Thus, in this paper, we slightly changed the notion of $\top$-disjointness to read instead:

%\begin{definition}[$\top$-disjointness]
%  Two types $A$ and $B$ are disjoint (written $A \disjoint B$) if:
%  \[\forall C.~\text{if}~ A \subtype C~\text{and}~B \subtype C~\text{then}~\toplike{C}\]
%\end{definition}
%\joao{say that we manage to retain coherence wrt to the simply typed version?}

\subsection{Stability of Substitutions}
From the technical point of view, the main challenge in the design of
System \name is that, in general, types are not stable under
substitution. This contrasts, for example, with System $F$ where types
are stable under substitution. That is in System $F$ the following
property holds: 
\bruno{state usual property here}

In \name if a type variable $A$ is substituted in a type $T_1$, for a type $T_2$ 
(written $\subst {T_2} A {T_1}$), where $T_1$ and $T_2$ are well-formed, the resulting type might be ill-formed. 
To understand why, recall the previous example: 
\begin{lstlisting}
fst Int Int (1,,2)
\end{lstlisting}
The type signature of \lstinline$fst$ may be read as $\forall A. (B * A) . (A \inter B) \to A$.
An application to the type $\tyint$ will lead to instantiation of the variable $A$, leading to the type
$\forall (B * \tyint). (\tyint \inter B) \to \tyint$.
Now, the second $\tyint$ application is problematic, since instantiating $B$ with $\tyint$ will lead to
the ill-formed type $(\tyint \inter \tyint) \to \tyint$.
However, from this example it is easy to see that all types which are not problematic are exactly the
the ones disjoint with $A$.
This paper shows how a weaker version of the usual type substituition stability still holds, 
namely by requiring that the type varible's disjointness constraint is compatible
with the type as target of the instantiation. 

\begin{comment}
This section introduces \namedis and its support for intersection types,
parametric polymorphism and the merge operator. It then discusses
the issue of coherence and shows how the notion of disjoint
intersection types and disjoint quantification achieve a coherent semantics.
%Finally we illustrate the expressive power of \namedis by encoding
%extensible type-theoretic encodings of datatypes.

Note that this section uses some syntactic sugar, as well as standard
programming language features, to illustrate the various concepts in
\namedis. Although the minimal core language that we formalize in
Section~\ref{sec:fi} does not present all such features, our implementation
supports them.

%\bruno{Need to type-check the programs!}

%\begin{comment}
%It then shows that,
%with unrestricted intersection types, the system
%lacks \emph{coherence}. This motivates the introduction of
%disjoint intersection types and extending universal quantification to
%disjoint quantification, which is enough to ensure coherence.
%\end{comment}

\subsection{Intersection Types and the Merge Operator}
%%\subsection{Intersection Types, Merge and Polymorphism in \namedis}

Intersection types date back as early as Coppo et
al.'s work~\cite{coppo1981functional}. Since then various researchers have
studied intersection types, and some languages have adopted them in one
form or another.
%However, as we shall see in
%Section~\ref{subsec:incoherence}, it also introduces difficulties. In what follows
%intersection types and the merge operator are informally introduced.

\paragraph{Intersection types.}
The intersection of type $A$ and $B$ (denoted as \lstinline{A & B} in
\namedis) contains exactly those values
which can be used as either values of type $A$ or of type $B$. For instance,
consider the following program in \namedis:

\begin{lstlisting}
let x : Int & Char = (*$ \ldots $*) in -- definition omitted
let idInt (y : Int) : Int = y in
let idChar (y : Char) : Char = y in
(idInt x, idChar x)
\end{lstlisting}

\noindent If a value \lstinline{x} has type \lstinline{Int & Char} then
\lstinline{x} can be used as an integer or as a character. Therefore,
\lstinline{x} can be used as an argument to any function that takes
an integer as an argument, or any
function that take a character as an argument. In the program above
the functions \lstinline{idInt} and \lstinline{idChar} are the
identity functions on integers and characters, respectively.
Passing \lstinline{x} as an argument to either one (or both) of the
functions is valid.

\paragraph{Merge operator.}
In the previous program we deliberately did not show how to introduce values of an
intersection type. There are many variants of intersection types
in the literature. Our work follows a particular formulation, where
intersection types are introduced by a \emph{merge operator}.
As Dunfield~\cite{dunfield2014elaborating} has argued a merge operator adds considerable
expressiveness to a calculus. The merge operator allows
two values to be merged in a single intersection type. For example, an
implementation of \lstinline{x} is constructed in \namedis as follows:

\begin{lstlisting}
let x : Int & Char = 1,,'c' in (*$ \ldots $*)
\end{lstlisting}

\noindent In \namedis (following Dunfield's notation), the
merge of two values $v_1$ and $v_2$ is denoted as $v_1 ,, v_2$.

\paragraph{Merge operator and pairs.}
The merge operator is similar to the introduction construct on pairs.
An analogous implementation of \lstinline{x} with pairs would be:

\begin{lstlisting}
let xPair : (Int, Char) = (1, 'c') in (*$ \ldots $*)
\end{lstlisting}

\noindent The significant difference between intersection types with a
merge operator and pairs is in the elimination construct. With pairs
there are explicit eliminators (\lstinline{fst} and
\lstinline{snd}). These eliminators must be used to extract the
components of the right type. For example, in order to use
\lstinline{idInt} and \lstinline{idChar} with pairs, we would need to
write a program such as:

\begin{lstlisting}
(idInt (fst xPair), idChar (snd xPair))
\end{lstlisting}

\noindent In contrast the elimination of intersection types is done
implicitly, by following a type-directed process. For example,
when a value of type \lstinline{Int} is needed, but an intersection
of type \lstinline{Int & Char} is found, the compiler uses the
type system to extract the corresponding value.

\subsection{Incoherence}\label{subsec:incoherence}
Unfortunatelly the implicit nature of elimination for intersection
types built with a merge operator can lead to incoherence.
The merge operator combines two terms, of type $A$ and $B$
respectively, to form a term of type $A \inter B$. For example,
$1 \mergeop `c'$ is of type $\tyint \inter \tychar$. In this case, no
matter if $1 \mergeop `c'$ is used as $\tyint$ or $\tychar$, the result
of evaluation is always clear. However, with overlapping types, it is
not straightforward anymore to see the result. For example, what
should be the result of this program, which asks for an integer out of
a merge of two integers:
\begin{lstlisting}
(fun (x: Int) (*$ \to $*) x) (1,,2)
\end{lstlisting}
Should the result be \lstinline$1$ or \lstinline$2$?

If both results are accepted, we say that the semantics is
\emph{incoherent}: there are multiple possible meanings for the same
valid program. Dunfield's calculus~\cite{dunfield2014elaborating} is incoherent and accepts the
program above.

\paragraph{Getting around incoherence: biased choice.}
In a real implementation of Dunfield calculus a choice has to be made
on which value to compute. For example, one potential option is to
always take the left-most value matching the type in the
merge. Similarly, one could always take the right-most
value matching the type in the merge. Either way, the meaning
of a program will depend on a biased implementation choice,
which is clearly unsatisfying from the theoretical point of view
(although perhaps acceptable in practice).
Moreover, even if we accept a particular biased choice as
being good enough, the approach cannot be easily
extended to systems with parametric polymorphism, as we illustrate
in Section~\ref{subsec:polymorphism}.

\subsection{Restoring Coherence: Disjoint Intersection Types}\label{sec:restoring}
Coherence is a desirable property for a semantics. A semantics is said
to be coherent if any \emph{valid program} has exactly one
meaning~\cite{reynolds1991coherence} (that is, the semantics is not ambiguous).
One option to restore coherence is to reject programs which may have
multiple meanings.
%Of course, when rejecting programs it is important
%not to be too conservative, and reject too many programs which are
%actually coherent.
Analyzing the expression $1 \mergeop 2$, we can see that the reason
for incoherence is that there are multiple, overlapping, integers in the
merge. Generally speaking, if both terms can be assigned some type $C$,
both of them can be chosen as the meaning of the merge,
which leads to multiple meanings of a term.
Thus a natural option is to try to forbid such overlapping
values of the same type in a merge.

This is precisely the approach taken in \namedis. \namedis requires that the
two types of in intersection must be \emph{disjoint}.  However,
although disjointness seems a natural restriction to impose on
intersection types, it is not obvious to formalize it. Indeed Dunfield
has mentioned disjointness as an option to restore coherence, but he
left it for future work due to the non-triviality of the approach.

\paragraph{Searching for a definition of disjointness.}
The first step towards disjoint intersection types is to come up
with a definition of disjointness. A first attempt at such definition would
be to require that, given two types $A$ and $B$, both types are not
subtypes of each other. Thus, denoting disjointness as $A * B$, we would have:
\[A * B \equiv A \not<: B \wedge B \not<: A\]
At first sight this seems a reasonable definition and it does prevent
merges such as \lstinline{1,,2}. However some moments of thought are enough to realize that
such definition does not ensure disjointness. For example, consider
the following merge:

\begin{lstlisting}
((1,,'c') ,, (2,,True))
\end{lstlisting}

\noindent This merge has two components which are also intersection
types. The first component (\lstinline{(1,,'c')}) has type $\tyint \inter
\tychar$, whereas the second component (\lstinline{(2 ,, True)}) has type
$\tyint \inter \tybool$. Clearly,
\[ \tyint \inter \tychar \not \subtype \tyint \inter \tybool \wedge \tyint \inter \tybool \not \subtype \tyint \inter \tychar \]
Nevertheless the following program still leads to
incoherence:
\begin{lstlisting}
(fun (x: Int) (*$ \to $*) x) ((1,,'c'),,(2,,True))
\end{lstlisting}
as both \lstinline{1} or \lstinline{2} are possible outcomes
of the program. Although this attempt to define disjointness failed,
it did bring us some additional insight: although the types of the two
components of the merge are not subtypes of each other, they share
some types in common.

\paragraph{A proper definition of disjointness.} In order for two types
to be trully disjoint, they must not have any subcomponents sharing
the same type. In a system with intersection types this can be ensured
by requiring the two types do not share a common supertype. The
following definition captures this idea more formally.

\begin{definition}[Disjointness]
  Given two types $A$ and $B$, two types are disjoint
  (written $A \disjoint B$) if there is no type $C$ such that both $A$ and $B$ are
  subtypes of $C$:
  \[A \disjoint B \equiv \not\exists C.~A \subtype C \wedge B \subtype C\]
\end{definition}

\noindent This definition of disjointness prevents the problematic
merge. Since $Int$ is a common supertype of both $Int \& Char$ and
$Int \& Bool$, those two types are not disjoint.

\namedis's type system only accepts programs that use disjoint
intersection types. As shown in Section~\ref{sec:disjoint} disjoint intersection
types will play a crutial rule in guaranteeing that the semantics is coherent.

\subsection{Parametric Polymorphism and Intersection Types}\label{subsec:polymorphism}
Before we show how \namedis extends the idea of disjointness to parametric
polymorphism, we discuss some non-trivial issues that arise from
the interaction between parametric polymorphism and intersection types.
%The interaction between parametric polymorphism and
%intersection types when coherence is a goal is non-trivial.
%In particular biased choice .
%The key challenge is to have a type
%system that still ensures coherence, but at the same time is not too
%restrictive in the programs that can be accepted.
%\begin{comment}
%Dunfield~\cite{} provides a
%good illustrative example of the issues that arise when combining
%disjoint intersection types and parametric polymorphism:
%\[\lambda x. {\bf let}~y = 0 \mergeop x~{\bf in}~x\]
%\end{comment}
Consider the attempt to write
the following polymorphic function in \namedis (we use
uppercase Latin letters to denote type variables):
\begin{lstlisting}
let fst A B (x: A & B) = (fun (z:A) (*$ \to $*) z) x in (*$ \ldots $*)
\end{lstlisting}
The
\code{fst} function is supposed to extract a value of type
(\lstinline{A}) from the merge value $x$ (of type \lstinline{A&B}). However
this function is problematic.  The reason is that when
\lstinline{A} and \lstinline{B} are instantiated to non-disjoint
types, then uses of \lstinline{fst} may lead to incoherence.
For example, consider the following use of \lstinline{fst}:
\begin{lstlisting}
fst Int Int (1,,2)
\end{lstlisting}
\noindent This program is clearly incoherent as both
$1$ and $2$ can be extracted from the merge and still match the type
of the first argument of \lstinline{fst}.

\paragraph{Biased choice breaks equational reasoning.} At first sight, one option
to workaround the issue incoherence would be to bias the type-based merge lookup
to the left or to the right (as discussed in
Section~\ref{subsec:incoherence}). Unfortunately, biased choice is
very problematic when parametric polymorphism is present in the language.
To see the issue, suppose we chose to always pick the
rightmost value in a merge when multiple values of same type exist.
Intuitively, it would appear that the result of the use of
\lstinline{fst} above is $2$. Indeed simple equational reasoning
seems to validate such result:
\begin{lstlisting}
   fst Int Int (1,,2)
(*$ \rightsquigarrow $*) (fun (z: Int) (*$ \to $*) z) (1,,2) -- (* \textnormal{By the definition of \code{fst}} *)
(*$ \rightsquigarrow $*) (fun (z: Int) (*$ \to $*) z) 2      -- (* \textnormal{Right-biased coercion} *)
(*$ \rightsquigarrow $*) 2                          -- (* \textnormal{By $\beta$-reduction} *)
\end{lstlisting}

However (assumming a straightforward implementation of right-biased
choice) the result of the program would be 1! The reason for this has
todo with \emph{when} the type-based lookup on the merge happens. In
the case of \lstinline{fst}, lookup is triggered by a coercion
function inserted in the definition of \lstinline{fst} at
compile-time.
In the definition of \lstinline$fst$ all it is known is that a
value of type $A$ should be returned from a merge with an intersection
type $A\&B$.  Clearly the only type-safe choice to coerce the value of
type $A\&B$ into $A$ is to
take the left component of the merge. This works perfectly for merges
such as \lstinline$(1,,'c')$, where the types of the first and second components
of the merge are disjoint. For the merge \lstinline$(1,,'c')$, if a integer lookup
is needed, then \lstinline$1$ is the rightmost integer, which is consistent with the
biased choice. Unfortunately, when given the merge \lstinline$(1,,2)$ the
left-component (\lstinline$1$) is also picked up, even though in this case \lstinline$2$
is the rightmost integer in the merge. Clearly this is inconsistent
with the biased choice!

Unfortunately this subtle interaction of polymorphism and type-based lookup
 means that equational reasoning is broken!
In the equational reasoning steps above, doing apparently correct
substitutions lead us to a wrong result. This is a major problem for
biased choice and a reason to dismiss it as a possible implementation
choice for \namedis.

\paragraph{Conservatively rejecting intersections.}
To avoid incoherence, and the issues of biased choice, another option
is simply to reject programs where the
instantiations of type variables may lead to incoherent programs.
In this case the definition of \lstinline$fst$ would be rejected, since there
are indeed some cases that may lead to incoherent programs.
Unfortunately this is too restrictive and prevents many useful
programs using both parametric polymorphism and intersection types.
In particular, in the case of \lstinline{fst}, if the two type
parameters are used with two disjoint intersection
types, then the merge will not lead to ambiguity.

In summary, it seems hard to have parametric polymorphism, intersection
types and coherence without being overly conservative.


%\begin{comment}
%\subsection{Intersection Types in Existing Languages}
%
%What is an intersection type? The intersection of types $A$ and $B$
%contains exactly those values which can be used as either of type $A$
%or of type $B$.  Just as not all intersection of sets are nonempty,
%not all intersections of types are inhabited.  For example, the
%intersection of a base type $\tyint$ and a function type
%$\tyint \to \tyint$ is not inhabited.\bruno{put this text somewhere?}
%
%Since then various researchers have
%studied intersection types, and some languages have adopted in one
%form or another. However, while intersection types are already used
%in various languages, the lack of a merge operator removes
%considerable expressiveness.
%
%
%A number of OO languages, such as
%Java, C\#, Scala, and Ceylon\footnote{\url{http://ceylon-lang.org/}},
%already support intersection types to different degrees. Intersection
%types are particularly relevant for OOP as they can be used to model
%multiple interface inheritance. In Java, for example,
%
%\begin{lstlisting}
%interface AwithB extends A, B {}
%\end{lstlisting}
%
%\noindent introduces a new interface \lstinline{AwithB} that satisfies the interfaces of
%both \lstinline{A} and \lstinline{B}. Arguably such type can be considered as a nominal
%intersection type. Scala takes one step further by eliminating the
%need of a nominal type. For example, given two concrete traits, it is possible to
%use \emph{mixin composition} to create an object that implements both
%traits. Such an object has a (structural) intersection type:
%
%\begin{lstlisting}
%trait A
%trait B
%
%val newAB : A with B = new A with B
%\end{lstlisting}
%
%\noindent Scala also allows intersection of type parameters. For example:
%\begin{lstlisting}
%def merge[A,B] (x: A) (y: B) : A with B = ...
%\end{lstlisting}
%uses the anonymous intersection of two type parameters \lstinline{A} and
%\lstinline{B}. However, in Scala it is not possible to dynamically
%compose two objects. For example, the following code:
%
%\begin{lstlisting}
%// Invalid Scala code:
%def merge[A,B] (x: A) (y: B) : A with B = x with y
%\end{lstlisting}
%
%\noindent is rejected by the Scala compiler. The problem is that the
%\lstinline{with} construct for Scala expressions can only be used to
%mixin traits or classes, and not arbitrary objects. Note that in the
%definition \lstinline{newAB} both \lstinline{A} and \lstinline{B} are
%\emph{traits}, whereas in the definition of \lstinline{merge} the variables
%\lstinline{x} and \lstinline{y} denote \emph{objects}.
%
%This limitation essentially put intersection types in Scala in a second-class
%status. Although \lstinline{merge} returns an intersection type, it is
%hard to actually build values with such types. In essence an
%object-level introduction construct for intersection types is missing.
%As it turns out using low-level type-unsafe programming features such
%as dynamic proxies, reflection or other meta-programming techniques,
%it is possible to implement such an introduction
%construct in Scala~\cite{oliveira2013feature,rendel14attributes}. However, this
%is clearly a hack and it would be better to provide proper language
%support for such a feature.
%
%
%
%
%\paragraph{Parametric polymorphism and intersection types.}
%Both universal quantification and intersection types provide a kind of
%polymorphism. While the former provides parametric polymorphism, the latter
%provides ad-hoc polymorphism. In some systems, parametric polymorphism is
%considered the infinite analog of intersection polymorphism. But in our system
%we do not consider this relationship. \george{Need to argue that why their
%coexistence might be a good thing.} \george{May use the merge example}
%\bruno{Some more examples in following subsections?}
%
%
%To address the limitations of intersection types in languages like
%Scala, \namedis allows intersecting any two terms at run time using a
%\emph{merge} operator (denoted by $ \mergeop $)~\cite{dunfield2014elaborating}.  With the merge
%operator it is trivial to implement the \lstinline{merge} function in \namedis:
%
%\begin{lstlisting}
%let merge[A, B * A] (x : A) (y : B) : A & B = x ,, y in (*$ \ldots $*)
%\end{lstlisting}
%
%\noindent In contrast to Scala's term-level \lstinline{with}
%construct, the operator \lstinline{,,} allows two arbitrary values \lstinline{x}
%and \lstinline{y} to be merged. The resulting type is a \emph{disjoint}
%intersection of the types of  \lstinline{x}
%and \lstinline{y} (\lstinline{A & B} in this case).
%
%\paragraph{Incoherence and parametric Polymorphism}
%We can define a \code{fst} function that extracts the first item of a merged value:
%\[
%\code{fst} \ \alpha \ \beta \ (x : \alpha \inter \beta) = \app {(\lam y \alpha y)} x
%\]
%What should be the result of this program?
%\begin{lstlisting}
%fst Int Int (1,,2)
%\end{lstlisting}
%
%Then we have the following equational reasoning:
%\begin{lstlisting}
%fst Int Int (1,,2) => (\(y : Int). y) (1,,2)
%\end{lstlisting}
%If we favor the second item, the program seems to evaluate to $2$. But in
%reality, the result is $2$. No matter we favor the first or the second item,
%we can always construct a program such that for that program, equational
%reasoning is broken.
%
%Therefore, we require that the two types of an intersection must be not
%overlapping, or \emph{disjoint}, and add this requirement to the well-formedness of types.
%
%A well-formed type is such that given any query type,
%it is always clear which subpart the query is referring to.
%In terms of rules, this notion of well-formedness is almost the same as the one in System $F$
%except for intersection types we require the two components to be disjoint.
%
%With parametric polymorphism, disjointness is harder to determine due to type variables.
%Consider this program:
%\[
%\blam \alpha {\lam x {\alpha \inter \tyint} x}
%\]
%$x$ in the body is of type $\alpha \inter \tyint$ and if $\alpha$ and $\tyint$ are
%disjoint depends on the instantiation of $\alpha$.
%\end{comment}

\subsection{Disjoint Quantification}
To avoid being overly conservative, while still retaining coherence in the
presence of parametric polymorphism and intersection types, \namedis uses
an extension to universal quantification called \emph{disjoint quantification}.
Inspired by
bounded quantification~\cite{Cardelli:1994},
where a type variable is constrained by a type
bound, disjoint quantification allows a type variable to be
constrained so that it is disjoint with a
given type. With disjoint quantification a variant of the program $fst$, which
is accepted by \namedis, would be written as:
\begin{lstlisting}
let fst A ((*$ \highlight {B *~A} $*)) (x: A & B) = (fun (z: A) (*$ \to $*) z) x
in (*$ \ldots $*)
\end{lstlisting}
The small change is in the declaration of the type parameter $B$. The notation
$B*A$ means that in this program the type variable $B$ is constrained so that
it can only be instantiated with any type disjoint to $A$.
This ensures that the
merge denoted by $x$ is disjoint for all valid instantiations of $A$ and $B$.

The nice thing about this solution is that many uses of \lstinline$fst$ are accepted.
For example, the following use of \lstinline$fst$:
\begin{lstlisting}
fst Int Char (1,,'c')
\end{lstlisting}
is accepted since \lstinline$Int$ and \lstinline$Char$ are disjoint, thus satisfying the constraint
on the second type parameter of \lstinline$fst$.
However, problematic uses of \lstinline$fst$ are rejected. For example:
\begin{lstlisting}
fst Int Int (1,,2)
\end{lstlisting}
is rejected because \lstinline$Int$ is not disjoint with \lstinline$Int$, thus failing to satisfy the
disjointness constraint on the second type parameter of \lstinline$fst$.

%\begin{comment}
%Note that there is a nice symmetry between bounded quantification and disjoint quantification.
%In systems with bounded quantification,
%the usual unconstrained quantifier $\for {\alpha} \ldots$
%is a syntactic sugar for $\for {\alpha \subtype \top} \ldots$, and
%$\blam \alpha \ldots$ for $\blam {\alpha \subtype \top} \ldots$.
%In parellel, in our system with disjoint quantification,
%the usual unconstrained quantifier $\for {\alpha} \ldots$
%is a syntactic sugar for $\for {\alpha \disjoint \bot} \ldots$, and
%$\blam \alpha \ldots$ for $\blam {\alpha \disjoint \top} \ldots$.
%The intuition is that since the bottom type is akin to the empty set,
%no other type overlaps with it.\george{Format this paragraph better.}
%\end{comment}
%
%
%\begin{comment}
%With this tool in hand, we can rewrite the program above to:
%\[
%\blam {\alpha \disjoint \tyint} {\lam x {\alpha \inter \tyint} x}
%\]
%
%This program typechecks because while $x$ is of type $\alpha \inter \tyint$,
%and $\alpha$ is disjoint with $\tyint$. Similarly, in the new system,
%the original program no longer typechecks, thus preventing overlapping types.
%\end{comment}
\end{comment}

\section{Application} \label{sec:application}

Structural subtyping facilitates reuse~\cite{malayeri2008integrating}.
\bruno{orphan sentence!}

This section shows that the System $ F $ plus intersection types are enough for
encoding extensible designs, and even beat the designs in languages with a much
more sophisticated type system. In particular, \name has two main advantages
over existing languages:

\begin{enumerate}
\item It supports dynamic composition of intersecting values.
\item It supports contravariant parameter types in the subtyping relation.
\end{enumerate}

Various solutions have been proposed to deal with the extensibility problems and
many rely on heavyweight language features such as abstract methods and classes
in Java.

\bruno{I would like to see a story about Church Encodings in
  \name. Can you look at Pierce's papers and try to write something
  along those lines? That will be a good intro for object algebras and
visitors!}

These two features can be used to improve existing designs of modular programs.

% Introduce the expression problem

The expression problem refers to the difficulty of adding a new operations and a
new data variant without changing or duplicating existing code.

There has been recently a lightweight solution to the expression problem that
takes advantage of covariant return types in Java. We show that FI is able to
solve the expression problem in the same spirit. The
A)


% - Object/Fold Algebras. How to support extensibility in an easier way.

% See Datatypes a la Carte

% - Mixins

% - Lenses? Can intersection types help with lenses? Perhaps making the
% types more natural and easy to understand/use?

% - Embedded DSLs? Extensibility in DSLs? Composing multiple DSL interpretations?

% http://www.cs.ox.ac.uk/jeremy.gibbons/publications/embedding.pdf

\begin{minted}{scala}
trait Expr {
  def eval: Int
}

class Lit(n: Int) extends Expr {
  def eval: Int = n
}

class Add(n: Int) extends Expr {
  def eval: Int = e1.eval + e2.eval
}
\end{minted}

\bruno{You already talk about overloading in
  the previous section. Need to decide where to put the text!}

Dunfield~\cite{dunfield2014elaborating} notes that using merges as a mechanism
of overloading is not as powerful as type classes.

\subsection{Encoding Bounded Polymorphism}\bruno{IMPORTANT: 
Always capitalize the relevant words in a title!. This title should be
"Encoding Bounded Polymorphism''}

As \name extends System $ F $ with intersection types, $ F_{\subtype} $ extends
System $ F $ with bounded polymorphism. $ F_{\subtype} $~\cite{pierce2002types}
allows giving an upper bound to the type variable in type abstractions. The idea
of bounded universal quantification was discussed in the seminal paper by
Cardelli and Wegner ~\cite{cardelli1985understanding}. They show that bounded
quantifiers are useful because it is able to solve the ``loss of information''
problem.

In fact, the extension of System $ F $ in the other direction, i.e., with
intersection types, is able to address the same problem effectively. Suppose we
have the following definitions:
\begin{minted}{scala}
def user = { name = "George", admin = true }
def id(user: {name: String}) = user
\end{minted}
Under a structural type system, programmers would expect that passing the user
to the function is allowed. They are correct. Note that in the source language
multi-field records are just syntatic sugars for merges of single-field records.
Therefore, user is of a subtype of the parameter due to subtyping introduced by
intersection types. So far so good. But there is a problem: what if programmers
wants to access the \texttt{admin} field later, like:
\begin{minted}{scala}
(id user).admin
\end{minted}
They cannot do so as the above will not typecheck. After going through the
function, the user now has only the type
\begin{minted}{scala}
{name: String}
\end{minted}
This is rather undesired because it indeed has an \texttt{admin} field!

Bounded polymorphism enable the function to return the exact type of the
argument so that we have no problem in accessing the \texttt{admin} field later.
Consider the example below:
\begin{minted}{scala}
def id [A <: {name: String}] (user: A) = user
(id [{name: String, admin: Bool}] user).admin
\end{minted}

We do not have bounded polymorphism in the source language. But we can encode
that via intersection types:
\begin{minted}{scala}
def id [A] (user: A & {name: String}) = user in
(id [{admin: Bool}] user).admin
\end{minted}

Polymorphism plus intersecion types is as powerful as bounded polymorphism.

\cite{pierce1997intersection}\bruno{I don't think this has been
  shown. What we can say is: we can encode a form of bounded
  polymorphism with intersection types.}

% Multiple inheritance?
% Algebra -> P1,2
% Visitor -> P2

% Yanlin
% Mixin

% \begin{lstlisting}
% let merge A B (f : ExpAlg A) (g : ExpAlg B) = {
%   lit = \(x : Int). f.lit x ,, g.lit x,
%   add = \(x : A & B). \(y : A & B). f.add x y ,, g.add x y
% };
% \end{lstlisting}

\subsection{Object Algebras}

Object algebras provide an alternative to \emph{algebraic data types}
(ADT).\bruno{We are targeting an OO crowd. Mentioning algebraic
  datatypes is not going to be very useful there.}
 For example, the
following Haskell definition of the type of simple expressions
\begin{minted}{haskell}
data Exp where
  Lit :: Int -> Exp
  Add :: Exp -> Exp -> Exp
\end{minted}
can be expressed by the \emph{interface} of an object algebra of
simple expressions:
\begin{minted}{scala}
trait ExpAlg[E] {
  def lit(x: Int): E
  def add(e1: E, e2: E): E
}
\end{minted}
Similar to ADT, data constructors in object algebras are represented by functions such as
\lstinline{lit} and \lstinline{add} inside an interface \lstinline{ExpAlg}.
Different with ADT, the type of the expression itself is abtracted by a type
parameter \lstinline{E}.

which can be expressed similarly in \name as:
\begin{lstlisting}
type ExpAlg E = {
  lit : Int -> E,
  add : E -> E -> E
}
\end{lstlisting}

% Introduce Scala's intersection types

Scala supports intersection types via the \lstinline{with} keyword. The type
\lstinline{A with B} expresses the combined interface of \lstinline{A} and
\lstinline{B}. The idea is similar to
\begin{minted}{java}
interface AwithB extends A, B {}
\end{minted}
in Java.
\footnote{However, Java would require the \lstinline{A} and \lstinline{B} to be
  concrete types, whereas in Scala, there is no such restriction.}

The value level counterpart are functions of the type \lstinline
{A => B => A with B}. \footnote{FIXME}

Our type system is a simple extension of System $ F $; yet surprisingly, it
is able to solve the limitations of using object algebras in languages such as
Java and Scala. We will illustrate this point with an step-by-step of solving
the expression problem using a source language built on top of \name.

Oliveira noted that composition of object algebras can be cumbersome and
intersection types provides a solution to that problem.

We first define an interface that supports the evaluation operation:

% #include ../src/Algebra.sf  @base
\begin{lstlisting}
type IEval  = { eval : Int };
type ExpAlg E = { lit : Int -> E, add : E -> E -> E };
let evalAlg = {
  lit = \(x : Int). { eval = x },
  add = \(x : IEval). \(y : IEval). { eval = x.eval + y.eval }
};
\end{lstlisting}
% #end

The interface is just a type synonym \lstinline{IEval}. In \name, record
types are structural and hence any value that satisfies this interface is of
type \lstinline{IEval} or of a subtype of \lstinline{IEval}. \footnote{Should be
mentioned in S2.}

In the following, \lstinline{ExpAlg} is an object algebra interface of
expressions with literal and addition case. And \lstinline{evalAlg} is an object
algebra for evluation of those expressions, which has type \lstinline{ExpAlg Int}

% #include ../src/Algebra.sf  @variant
\begin{lstlisting}
type SubExpAlg E = (ExpAlg E) & { sub : E -> E -> E };
let subEvalAlg = evalAlg ,, { sub = \ (x : IEval). \ (y : IEval). { eval = x.eval - y.eval } };
\end{lstlisting}
% #end

Next, we define an interface that supports pretty printing.

% #include ../src/Algebra.sf  @operation
\begin{lstlisting}
type IPrint = { print : String };
let printAlg = {
  lit = \(x : Int). { print = x.toString() },
  add = \(x : IPrint). \(y : IPrint). { print = x.print.concat(" + ").concat(y.print) },
  sub = \(x : IPrint). \(y : IPrint). { print = x.print.concat(" - ").concat(y.print) }
};
\end{lstlisting}
% #end

Provided with the definitions above, we can then create values using the
appropriate algebras. For example:
defines two expressions.

The expressions are unusual in the sense that they are functions that take an
extra argument \lstinline{f}, the object algebras, and use the data constructors
provided by the object algebra (factory) \lstinline{f} such as \lstinline{lit},
\lstinline{add} and \lstinline{sub} to create values. Moreover, The algebras
themselves are abstracted over the allowed operations such as evaluation and
pretty printing by requiring the expression functions to take an extra argument
\lstinline{E}.

% #include ../src/Algebra.sf  @merge
\begin{lstlisting}
let merge A B (f : ExpAlg A) (g : ExpAlg B) = {
  lit = \(x : Int). f.lit x ,, g.lit x,
  add = \(x : A & B). \(y : A & B).
          f.add x y ,, g.add x y
};
\end{lstlisting}
% #end

If we would like to have an expression that supports both evaluation and pretty
printing, we will need a mechanism to combine the evaluation and printing
algebras. Intersection types allows such composition: the \lstinline{merge}
function, which takes two expression algebras to create a combined algebra. It
does so by constructing a new expression algebra, a record whose each field is a
function that delegates the input to the two algebras taken.

% #include ../src/Algebra.sf  @usage
\begin{lstlisting}
let newAlg = merge IEval IPrint subEvalAlg printAlg in
let o1 = e1 (IEval & IPrint) newAlg in
o1.print
\end{lstlisting}
% #end

\bruno{Don't start a sentence with \lstinline{o1}.}
\lstinline{o1} is a single object created that supports both evaluation and
printing, thus achieving full feature-oriented programming.

\subsection{Visitors}

Constructing instances seems clumsy!

The visitor pattern allows adding new operations to existing structures without
modifying those structures. The type of expressions are defined as follows:

\begin{minted}{scala}
trait Exp[A] {
  def accept(f: ExpAlg[A]): A
}

trait SubExp[A] extends Exp[A] {
  override def accept(f: SubExpAlg[A]): A
}
\end{minted}

The body of \lstinline{Exp} and \lstinline{SubExp} are almost the same: they
both contain an \lstinline{accept} method that takes an algebra \lstinline{f}
and returns a value of the carrier type \lstinline{A}. The only difference is at
\lstinline{f} --- \lstinline{SubExpAlg[A]} is a subtype of
\lstinline{ExpAlg[A]}. Since \lstinline{f} appear in parameter position of
\lstinline{accept} and function parameters are contravariant, naturally we would
hope that \lstinline{SubExp[A]} is a supertype of \lstinline{Exp[A]}. However,
such subtyping relation does not fit well in Scala because inheritance implies
subtyping in such languages \footnote{It is still possible to encode
  contravariant parameter types in Scala but doing so would require some
  technique.\bruno{what technique?}}. As \lstinline{SubExp[A]} extends \lstinline{Exp[A]}, the former
becomes a subtype of the latter.

Such limitation does not exist in \name. For example, we can define the similar interfaces \lstinline{Exp} and \lstinline{SubExp}:
\begin{lstlisting}
type Exp    A = { accept: forall A. ExpAlg A -> A };
type SubExp A = { accept: forall A. SubExpAlg A -> A };
\end{lstlisting}
Then by the typing judgment it holds that \lstinline{SubExp} is a supertype of
\lstinline{Exp}. This relation gives desired results. To give a concrete example:

A is called is the \emph{interpretation}. It works for any interpretation you want.

First we define two data constructors for simple expressions:
\begin{lstlisting}
let lit (n : Int): Exp A = {
  accept = /\A. \(f : ExpAlg A). f.lit n
};

let add (e1 : Exp) (e2 : Exp): Exp A = {
  accept = /\A. \(f : ExpAlg A).
             f.add (e1.accept A f) (e2.accept A f)
};
\end{lstlisting}

Suppose later we decide to augment the expressions with subtraction:
\begin{lstlisting}
let sub (e1 : SubExp) (e2 : SubExp): SubExp A =
  { accept = /\A. \(f : SubExpAlg A).
               f.sub (e1.accept A f) (e2.accept A f) };
\end{lstlisting}

One big benefit of using the visitor pattern is that programmers is able to
write in the same way that would do in Haskell.
For example, \lstinline{e2 = sub (lit 2) (lit 3)} defines an expression.

Another important property that does not exist in Scala is that programmer is
able to pass \lstinline{lit 2}, which is of type \lstinline{Exp A}, to
\lstinline{sub}, which expects a \lstinline{SubExp A} because of the subtyping
relation we have. After all, it is known statically that \lstinline{lit 2} can
be passed into \lstinline{sub} and nothing will go
wrong.\bruno{Subtyping needs to be much more emphasized! See Modular
  Visitor Components! }

% \subsection{Yanlin stuff}
% \bruno{This can be dropped.}

% This subsection presents yet another lightweight solution to the Expression
% Problem, inspired by the recent work by Wang. It has been shown that
% contravariant return types allows refinement of the types of extended
% expressions.

% First, we define the type of expressions that support evaluation and implement
% two constructors:
% \begin{lstlisting}
% type Exp = { eval: Int }
% let lit (n: Int) = { eval = n }
% let add (e1: Exp) (e2: Exp)
%   = { eval = e1.eval + e2.eval }
% \end{lstlisting}

% If we would like to add a new operation, say pretty printing, it is nothing more
% than refining the original \lstinline{Exp} interface by \emph{intersecting} the
% original type with the new \lstinline{print} interface using the \lstinline{&}
% primitive and \emph{merging} the original data constructors using the \lstinline{,,}
% primitive.
% \begin{lstlisting}
% type ExpExt = Exp & { print: String }
% let litExt (n: Int) = lit n ,, { print = n.toString() }
% let addExt (e1: ExpExt) (e2: ExpExt)
%   = add e1 e2 ,,
%     { print = e1.print.concat(" + ").concat(e2.print) }
% \end{lstlisting}

% Now we can construct expressions using the constructors defined above:
% \begin{lstlisting}
% let e1: ExpExt = addExt (litExt 2) (litExt 3)
% let e2: Exp = add (lit 2) (lit 4)
% \end{lstlisting}
% \lstinline{e1} is an expression capable of both evaluation and printing, while
% \lstinline{e2} supports evaluation only.

% We can also add a new variant to our expression:
% \begin{lstlisting}
% let sub (e1: Exp) (e2: Exp) = { eval = e1.eval - e2.eval }
% let subExt (e1: ExpExt) (e2: ExpExt)
%   = sub e1 e2 ,, { print = e1.print.concat(" - ").concat(e2.print) }
% \end{lstlisting}

% Finally we are able to manipulate our expressions with the power of both
% subtraction and pretty printing.
% \begin{lstlisting}
% (subExt e1 e1).print
% \end{lstlisting}

\subsection{Mixins}

Mixins are useful programming technique wildly adopted in dynamic programming
languages such as JavaScript and Ruby. But obviously it is the programmers'
responsbility to make sure that the mixin does not try to access methods or
fields that are not present in the base class.

In Haskell, one is also able to write programs in mixin style using records.
However, this approach has a serious drawback: since there is no subtyping in
Haskell, it is not possible to refine the mixin by adding more fields to the
records. This means that the type of the family of the mixins has to be
determined upfront, which undermines extensibility.

\name is able to overcome both of the problems: it allows composing mixins
that (1) extends the base behavior, (2) while ensuring type safety.

The figure defines a mini mixin library. The apostrophe in front of types
denotes call-by-name arguments similar to the \lstinline{=>} notation in the
Scala language.

\begin{lstlisting}
type Mixin S = 'S -> 'S -> S;
let zero S (super : 'S) (this : 'S) : S = super;
let rec mixin S (f : Mixin S) : S
  = let m = mixin S in f (\ (_ : Unit). m f) (\ (_ : Unit). m f);
let extends S (f : Mixin S) (g : Mixin S) : Mixin S
  = \ (super : 'S). \ (this  : 'S). f (\ (d : Unit). g super this) this;
\end{lstlisting}

We define a factorial function in mixin style and make a \lstinline{noisy} mixin
that prints ``Hello'' and delegates to its superclass. Then the two functions
are composed using the \lstinline{mixin} and \lstinline{extends} combinators.
The result is the \lstinline{noisyFact} function that prints ``Hello'' every
time it is called and computes factorial.
\begin{lstlisting}
let fact (super : 'Int -> Int) (this : 'Int -> Int) : Int -> Int
  = \ (n : Int). if n == 0 then 1 else n * this (n - 1)
let noisy (super : 'Int -> Int) (this : 'Int -> Int) : Int -> Int
  = \ (n : Int). { println("Hello"); super n }
let noisyFact = mixin (Int -> Int) (extends (Int -> Int) foolish fact)
noisy 5
\end{lstlisting}

% \subsection{Composing Mixins and Object Algebras}

\section{The \name calculus}

\footnote{Joshua Dunfield}

This section formalizes the syntax, subtyping, and typing of \name. In the next
section, we will go through the type-directed translation from \name to System
F.

% Note the semantics of this language is not defined formally, instead, by a
% translation into the target language, System F.

\subsection{Syntax}

The syntax of the \name calculus extends System F by adding the two features:
intersection types and records. The formalization includes only single records
because single record types as the multi-records can be desugared into the merge
of multiple single records.

\[
\begin{array}{l}
  \begin{array}{llrl}
    \text{Types} 
    & \tau & \Coloneqq & \alpha \mid \highlight{\top} \mid \tau_1 \to \tau_2 \mid \for \alpha \tau \\
    &      & \mid      & \highlight{\tau_1 \andop \tau_2} \mid \highlight{\recty l \tau} \\
    \text{Expressions} 
    & e & \Coloneqq & x \mid \highlight{\top} \mid \lam x \tau e \mid \app e e \mid \blam \alpha e \mid \tapp e \tau \\
    &   & \mid      & \highlight {e \mergeop e} \mid \highlight {\reccon l e} \mid
                      \highlight {e.l} \mid \highlight{e \restrictop l} \\
    \text{Contexts} 
    & \gamma & \Coloneqq & \epsilon \mid \gamma, \alpha \mid \gamma, x \hast \tau \\
    \text{Labels} & l
  \end{array} 
\end{array}
\]


% Types
Types $ t $ have five constructs. The first three are standard (present in
System F): type variable $ \alpha $, function types $ t \to t $, and type
abstraction $ \forall \alpha. t $; while the last two, intersection types
$ t \with t $ and record types $ \recordtype{l}{t} $, are novel in \Name. In
record types, $ l $ is the label and $ t $ the type.

% Expressions - standard ones
First five constructs of expressions are also standard: variables $ x $ and two
abstraction-elimination pairs. $ \abs{\rel{x}{t}}{e} $ abstracts expression
$ e $ over values of type $ t $ and is eliminated by application $ \app{e}{e} $;
$ \Abs{\alpha}{e} $ abstracts expression $ e $ over types and is eliminated by
type application $ \app{e}{t} $.

% Expressions - new ones
The last four constructs are novel. $ e \dcomma e $ is the \emph{merge} of two
terms. $ \recordintro{l}{e} $ introduces a record literal having $ l $ as the
label for field containing expression $ e $ . $ e.l $ access the field with
label $ l $ in $ e $. Finally, $ \recordupdate{e}{l}{e} $ updates the field
labelled $ l $ in expression $ e $. For simplicity, we omit other constructs in
order to focus on the essence of the calculus. For example, fixpoints can be
added in standard ways.

% Fields
The field $ F $ is non-standard and introduced to deal with records. It is an
associative list. Each item is a pair whose first item is either empty or a
label and the second the types.

% The most central construct of our language is ...

% Dunfield has described a language that includes a ``top'' type but it does not appear in our language. Our work differs from Dunfield in that ...

% Remark. The operational semantics of FI is not presented in this paper. However,

\subsection{Subtyping}

\begin{figure*}
\framebox{\( \ty \subtype \ty \yieldsnothing C \)}

\infax[SVar]
{\alpha \subtype \alpha \yieldsnothing {\abs {\rel x {\image \alpha}} x}}

\infrule[SFun]
{\ty_3 \subtype \ty_1 \yieldsnothing {C_1} \andalso \ty_2 \subtype \ty_4 \yieldsnothing {C_2}}
{\ty_1 \to \ty_2 \subtype \ty_3 \to \ty_4
  \yieldsnothing
    {\abs {\rel f {\image {\ty_1 \to \ty_2}}}
      {\abs {\rel x {\image {\ty_3}}}
        {\app {C_2} {(\app {f} {(\app {C_1} {x})})}}}}}

\infrule[SForall]
{\ty_1 \subtype \subst {\alpha_2} {\alpha_1} \ty_2 \yieldsnothing C}
{\forall \alpha_1. \ty_1 \subtype \forall \alpha_2. \ty_2
  \yieldsnothing
    {\abs {\rel{f} {\image {\forall \alpha. \ty_1}}}
      {\Abs \alpha {\app C {(\app f \alpha)}}}}}

\infrule[SAnd1]
{\ty_1 \subtype \ty_3 \yieldsnothing C}
{\ty_1 \with \ty_2 \subtype \ty_3
  \yieldsnothing
    {\abs {\rel x {\image {\ty_1 \with \ty_2}}}
      {\app C {(\app \fst x)}}}}

\infrule[SAnd2]
{\ty_2 \subtype \ty_3 \yieldsnothing C}
{\ty_1 \with \ty_2 \subtype \ty_3
  \yieldsnothing
    {\abs {\rel x {\image {\ty_1 \with \ty_2}}}
      {\app C {(\app \snd x)}}}}

\infrule[SAnd3]
{\ty_1 \subtype \ty_2 \yieldsnothing {C_1} \andalso \ty_1 \subtype \ty_3 \yieldsnothing {C_2}}
{\ty_1 \subtype \ty_2 \with \ty_3
  \yieldsnothing
    {\abs {\rel x {\image {\ty_1}}}
      {\tupled {\app {C_1} x, \app {C_2} x}}}}

\infrule[SRcd]
{\ty_1 \subtype \ty_2 \yieldsnothing C}
{\recordtype l {\ty_1} \subtype \recordtype l {\ty_2}
  \yieldsnothing
    {\abs {\rel x {\image {\recordtype l {\ty_1}}}} {\app C x}}}

\caption{Subtyping}
\end{figure*}

Thanks to intersection types, we have natural subtyping relations among types.
For example, $ Int \with Bool $ should be a subtype of $ Int $, since the former
can be viewed as either $ Int $ or $ Bool $. The subtyping rules are standard
except for three points listed below:
\begin{enumerate}
\item $ t_1 \with t_2 $ is a subtype of $ t_3 $, if \emph{either} $ t_1 $ or
  $ t_2 $ are subtypes of $ t_3 $,

\item $ t_1 $ is a subtype of $ t_2 \with t_3 $, if $ t_1 $ is a subtype of
  both $ t_2 $ and $ t_3 $.

\item $ \recordtype{l_1}{t_1} $ is a subtype of $ \recordtype{l_2}{t_2} $, if
  $ l_1 $ and $ l_2 $ are identical and $ t_1 $ is a subtype of $ t_2 $.
\end{enumerate}
The first point is captured by two rules \texttt{S-And-1} and \texttt{S-And-2},
whereas the second point by \texttt{S-And-3}. Note that the last point means
that record types are covariant in the type of the fields.

\subsection{Typing}

\begin{figure*}
\framebox{\(\Gamma \turns {t}\)}

\infrule[WF-Var]
{\alpha \in \Gamma}
{\Gamma \turns \alpha}

\infrule[WF-Fun]
{\Gamma \turns \ty_1 \andalso \Gamma \turns \ty_2}
{\Gamma \turns \ty_1 \to \ty_2}

\infrule[WF-Forall]
{\Gamma, \alpha \turns \ty}
{\Gamma \turns \forall \alpha. \ty}

\infrule[WF-And]
{\Gamma \turns \ty_1 \andalso \Gamma \turns \ty_2}
{\Gamma \turns \ty_1 \intersects \ty_2}

\infrule[WF-Rcd]
{\Gamma \turns \ty}
{\Gamma \turns \recordtype l \ty}

\caption{Well-formedness}
\end{figure*}

\begin{figure*}
\framebox{\( \Gamma \turns e : \ty \yieldsnothing E \)}

\infrule[Var]
{(x,\ty) \in \Gamma}
{\Gamma \turns x : \ty \yieldsnothing x}

\infrule[Abs]
{\Gamma, \rel x \ty \turns e : \ty_1 \yieldsnothing E \andalso
 \Gamma \turns \ty}
{\Gamma \turns \abs {\rel x \ty} e : \ty \to \ty_1 \yieldsnothing {\abs {\rel x {\image \ty}} E}}

\infrule[TAbs]
{\Gamma, \alpha \turns e : \ty \yieldsnothing E}
{\Gamma \turns \Abs \alpha e : \forall \alpha. \ty \yieldsnothing {\Abs \alpha E}}

\infrule[App]
{\Gamma \turns e_1 : \ty_1 \to \ty_2 \yieldsnothing {E_1} \\
 \Gamma \turns e_2 : \ty_3 \yieldsnothing {E_2} \andalso
 \ty_3 \subtype \ty_1 \yieldsnothing C}
{\Gamma \turns \app {e_1} {e_2} : \ty_2 \yieldsnothing {\app {E_1} {(\app C E_2)}}}

\infrule[TApp]
{\Gamma \turns e : \forall \alpha. \ty_1 \yieldsnothing E \andalso
 \Gamma \turns \ty}
{\Gamma \turns \app e \ty : \subst \alpha \ty \ty_1 \yieldsnothing {\app E {\image \ty}}}

\infrule[Merge]
{\Gamma \turns e_1 : \ty_1 \yieldsnothing {E_1} \andalso
 \Gamma \turns e_2 : \ty_2 \yieldsnothing {E_2}}
{\Gamma \turns e_1 \dcomma e_2 : \ty_1 \intersects \ty_2 \yieldsnothing {\tupled {E_1, E_2}}}

\infrule[RecCon]
{\Gamma \turns e : \ty \yieldsnothing E}
{\Gamma \turns \reccon l e : \recty l \ty \yieldsnothing E}

\infrule[RecProj]
{\Gamma \turns e : \ty \yieldsnothing E \andalso
 \Gamma \turnsget (\ty; l) : \ty_1 \yieldsnothing C}
{\Gamma \turns e.l : \ty_1 \yieldsnothing {\app C E}}

\infrule[RecUpd]
{\Gamma \turns e : \ty \yieldsnothing E \andalso
 \Gamma \turns e_1 : \ty_1 \yieldsnothing {E_1} \\
 \turnsput (t; l; e_1 : \ty_1 \yieldsnothing {E_1}) : (\ty_2, \ty_3) \yieldsnothing C \andalso
 \ty_1 \subtype \ty_2}
{\Gamma \turns \recupd e l {e_1} : \ty_3 \yieldsnothing {\app C E}}

\framebox{\( \vdash_{get} (t; l) : t \yieldsnothing C \)}

\infax[GetBase]
{\vdash_{get} (\recordtype l t; l) : t
  \yieldsnothing {\abs {\rel x {\imageof {\recordtype l t}}} x}}

\infrule[GetLeft]
{\vdash_{get} (t_1; l) : t \yieldsnothing C}
{\vdash_{get} (t_1 \with t_2; l) : t
  \yieldsnothing {\abs {\rel x {\imageof {t_1 \with t_2}}} {C (\app \fst x)}}}

\infrule[GetRight]
{\vdash_{get} (t_2; l) : t \yieldsnothing C}
{\vdash_{get} (t_1 \with t_2; l) : t
  \yieldsnothing {\abs {\rel x {\imageof {t_1 \with t_2}}} {C (\app \snd x)}}}

\begin{mathpar}
\framebox{\( \turnsput (\tau; l; e : \tau \yieldsnothing E) : (\tau, \tau) \yieldsnothing C \)}

\inferrule* [right=Put]
{ }
{\turnsput (\RecTy l \tau; l; e : \tau_1) : (\tau, \RecTy l {\tau_1})}

\inferrule* [right=Put1]
{\turnsput (\tau_1; l; e : \tau) : (\tau_3, \tau_4)}
{\turnsput (\tau_1 \Intersect \tau_2; l; e : \tau) : (\tau_3, \tau_4 \Intersect \tau_2)}

\inferrule* [right=Put2]
{\turnsput (\tau_2; l; e : \tau) : (\tau_3, \tau_4)}
{\turnsput (\tau_1 \Intersect \tau_2; l; e : \tau) : (\tau_3, \tau_1 \Intersect \tau_4)}
\end{mathpar}
\caption{Typing}
\end{figure*}

The typing judgment for \name is of the form: $ \Gamma \vdash e : t $. This
judgment uses the context $ \Gamma $. The typing rules for our core languages
are mostly standard ones for System F. In particular we introduce a
\texttt{T-Merge} rule that applies to \emph{merge} constructs.

The last two rules make use of the $ \texttt{fields} $ function just to make
sure that the field being accessed (\texttt{T-RcdElim}) or updated
(\texttt{T-RcdUpd}) actually exists. The function is defined recursively, in
Haskell pseudocode, as:
\[ \begin{array}{rll}
  \fields{\alpha} & = & \rel{\cdot}{\alpha} \\
  \fields{t_1 \to t_2} & = & \rel{\cdot}{t_1 \to t_2} \\
  \fields{\forall \alpha. t} & = & \rel{\cdot}{\forall \alpha. t} \\
  \fields{t_1 \with t_2} & = & \fields{t_1} \dplus \fields{t_2} \\
  \fields{\recordtype{l}{t}} & = & \rel{l}{t}
\end{array} \]
where $ \cdot $ means empty list, $ \dplus $ list concatenation, and $ : $ is an
infix operator that prepend the first argument to the second. The function
returns an associative list whose domain is field labels and range types.

\textit{dom} reads: ``the domain of''. $ F(l) $ means the result of lookup for
$ l $ inside the associative list $ F $. The order of lookup can be either from
left to right or from right to left but has to be consistent inside one
implementation. We prefer the order from the right to the left because it make
possible record overriding. For example,
$ (\recordintro{count}{1} ,, \recordintro{count}{2}).count $ will evaluate to
$ 2 $ in this case.
\section{Type-directed Translation to System F}

In this section we define the semantics of \name by means of a type-directed
translation to System F. This translation removes the labels of records and
turns intersections into products, much like Dunfield's elaboration. But our
translation also deals with parametric polymorphism and records.

\subsection{Informal Discussion}

To help the reader have a high-level understanding of how the translation
works, in this subsection we present the translation informally. Take the \name
expression for example:

\begin{lstlisting}
{ eval = 4, print = ``4'' }.eval
\end{lstlisting}

First, multi-field record literals are desugared into merges of single-field
record literals. Therefore $ \{ eval = 4, print = ``4'' \} $ becomes
$ \{ eval = 4 \} ,, \{ print = ``4'' \} $. Merges of two values are translated
into just a pair of them by \texttt{TrMerge} and single-field record literals lose their field
labels by \texttt{TrRcdIntro}. Hence $ \{ eval = 4 \} ,, \{ print = ``4'' \} $
becomes $ (4, ``4'') $.

Finally, $ e1 $ and $ e2 $ are both coerced by a projection function
$ \\(x:(Int,String)). \texttt{fst} x $.

\subsection{Target Language}

The target language is System F extended with pairs. The syntax and typing is
completely standard. The syntax of Systm F is as follows:

\[
\begin{array}{llrl}
  \text{Types}       & T    & \Coloneqq & \alpha \mid () \mid {T_1} \to {T_2} \mid \for \alpha T \mid \pair {T_1} {T_2} \\
  \text{Terms} & E, C & \Coloneqq & x \mid () \mid \lam x T E \mid \app {E_1} {E_2} \mid \blam \alpha E \\ 
                     &      & \mid      & \tapp E T \mid \pair {E_1} {E_2} \mid \proj k E \\
  \text{Contexts} & \Gamma & \Coloneqq & \epsilon \mid \Gamma, \alpha \mid \Gamma, x \hast T \\
\end{array}
\]


The dynamic semantics of System F can be found in ...

\begin{lemma} \label{type-coerce}
  If $$ \Gamma \vdash \tau_1 <: \tau_2 \yields{C} $$
  then $$ |\Gamma| \vdash C : |\tau_1| \to |\tau_2| $$
\end{lemma}

The main translation judgment is $ \Gamma \vdash e : \tau \hookrightarrow E $ which
states that with respect to the environment $ \Gamma $, the \name expression
$ e $ is of a \name type $ \tau $ and its translation is a System F expression $ E $.

We also define the type translation function $ | \cdot | $ from \name types
$ \tau $ to System F types $ T $.
\framebox{$ \image \ty = T $}

\[
\begin{array}{rcl}
  \image \alpha                     & = & \alpha \\
  \image {\ty_1} \to \image {\ty_2} & = & \image {\ty_1} \to \image {\ty_2} \\
  \image {\Forall \alpha \ty}      & = & \Forall \alpha \image \ty \\
  \image {\ty_1 \Intersect \ty_2}   & = & \Pair {\image {\ty_1}} {\image {\ty_2}} \\
  \image {\RecTy l \ty}            & = & \image t
\end{array}
\]
The first three rules of the translation is standard. For the last two, the
intersection of two types are translated into a product of them, and the label
of record types are erased.

The translation consists of four sets of rules, which are explained below:

\subsection{Subtyping (Coercion)}

\framebox{\( t \subtype t \yields C \)}

\infax[SVar]
{\alpha \subtype \alpha \yields {\abs {\rel x {\imageof \alpha}} x}}

\infrule[SFun]
{t_3 \subtype t_1 \yields {C_1} \andalso t_2 \subtype t_4 \yields {C_2}}
{t_1 \to t_2 \subtype t_3 \to t_4
  \yields
    {\abs {\rel f {\imageof {t_1 \to t_2}}}
      {\abs {\rel x {\imageof {t_3}}}
        {\app {C_2} {(\app{f} {(\app{C_1} {x})})}}}}}

\infrule[SForall]
{t_1 \subtype \subst {\alpha_2} {\alpha_1} t_2 \yields{C}}
{\forall \alpha_1. t_1 \subtype \forall \alpha_2. t_2
  \yields
    {\abs{\rel{f} {\imageof {\forall \alpha. t_1}}}
      {\Abs \alpha {\app C {(\app f \alpha)}}}}}

\infrule[SAnd1]
{t_1 \subtype t_3 \yields C}
{t_1 \with t_2 \subtype t_3
  \yields
    {\abs{\rel x {\imageof {t_1 \with t_2}}}
      {\app C {(\proj 1 x)}}}}

\infrule[SAnd2]
{t_2 \subtype t_3 \yields C}
{t_1 \with t_2 \subtype t_3
  \yields
    {\abs {\rel x {\imageof {t_1 \with t_2}}}
      {\app C {(\proj 2 x)}}}}

\infrule[SAnd3]
{t_1 \subtype t_2 \yields {C_1} \andalso t_1 \subtype t_3 \yields {C_2}}
{t_1 \subtype t_2 \with t_3
  \yields
    {\abs {\rel x {\imageof {t_1}}}
      {\tupled {\app {C_1} x, \app {C_2} x}}}}

\infrule[SRcd]
{t_1 \subtype t_2 \yields C}
{\recordtype l {t_1} \subtype \recordtype l {t_2}
  \yields
    {\abs {\rel x {\imageof {\recordtype l {t_1}}}} {\tupled {\app C {(\proj 1 x)}}}}}


The coercion judgment $ \Gamma \vdash \tau_1 \subtype \tau_2 \yields{C} $
extends the subtyping judgment with a coercion on the right hand side of
$ \hookrightarrow $. A coercion $ C $ is an expression in the target language
and has type $ \tau_1 \to \tau_2 $, as proved by Lemma \ref{type-coerce}. It is
read ``In the environment $ \Gamma $, $ \tau_1 $ is a subtype of $ \tau_2 $; and
if any expression $ e $ has a type $ t_1 $ that is a subtype of the type of
$ t_2 $, the elaborated $ e $, when applied to the corresponding coercion $ C $,
has exactly type $ |t_2| $''. For example,
$\Gamma \vdash Int \& Bool <: Bool \yields{fst} $, where $ fst $ is the
projection of a tuple on the first element. The coercion judgment is only used
in the \texttt{TrApp} case. As \texttt{SFun} supports contravariant parameter
type and covariant return type, the coercion of the parameter types and that of
the return types are used to create a coercion for the function type.
\texttt{SAnd1}, \texttt{SAnd2}, and \texttt{SAnd3} deal with intersection types.
The first two are complementary to each other. Take \texttt{SAnd1} for example,
if we know $ t_1 $ is a subtype of $ t_3 $ and $ C $ is a coercion from $ t_1 $
to $ t_3 $, then we can conclude that $ t_1 \with t_2 $ is also a subtype of
$ t_3 $ and the new coercion is a function that takes a value $ x $ of type
$ t_1 \with t_2 $, project $ x $ on the first item, and apply $ C $ to it.
\texttt{SAnd3} uses both of two coercions and constructs a pair.

\subsection{Typing (Translation)}

In this subsection we now present formally the translation rules that convert
\name expressions into System F ones. This set of rules essentially extends
those in the previous section with the light-blue part for the translation.

\framebox{\( \Gamma \vdash e : t \yields E \)}

\infrule[TrVar]
{(x,t) \in \Gamma}
{\Gamma \vdash x : t \yields x}

\infrule[TrAbs]
{\Gamma, \rel x t \vdash e : t_1 \yields E \andalso
 \Gamma \vdash_{F} t}
{\Gamma \vdash \abs {\rel x t} e : t \to t_1 \yields {\abs {\rel x {\imageof t}} E}}

\infrule[TrTAbs]
{\Gamma, \alpha \vdash e : t \yields E}
{\Gamma \vdash \Abs \alpha e : \forall \alpha. t \yields {\Abs \alpha E}}

\infrule[TrApp]
{\Gamma \vdash e_1 : t_1 \to t_2 \yields {E_1} \\
 \Gamma \vdash e_2 : t_3 \yields {E_2} \\
 t_3 \subtype t_1 \yields C}
{\Gamma \vdash \app {e_1} {e_2} : t_2 \yields {\app E_1 {(\app C E_2)}}}

\infrule[TrTApp]
{\Gamma \vdash e : \forall \alpha. t_1 \yields E \andalso
 \Gamma \vdash_{F} t}
{\Gamma \vdash \app e t : \subst \alpha t t_1 \yields {\app E {\imageof t}}}

\infrule[TrMerge]
{\Gamma \vdash e_1 : t_1 \yields {E_1} \\
 \Gamma \vdash e_2 : t_2 \yields {E_2}}
{\Gamma \vdash e_1 \dcomma e_2 : t_1 \with t_2 \yields {\tupled {E_1, E_2}}}

\infrule[TrRcdIntro]
{\Gamma \vdash e : t \yields E}
{\Gamma \vdash \recordintro l e : \recordtype l t \yields E}

\infrule[TrRcdElim]
{\Gamma \vdash e : t \yields E \andalso
 \Gamma \vdash_{get} (t; l) : t_1 \yields C}
{\Gamma \vdash e.l : t_1 \yields {\app C E}}

\infrule[TrRcdUpd]
{\Gamma \vdash e : t \yields E \andalso
 \Gamma \vdash e_1 : t_1 \yields {E_1} \\
 \vdash_{put} (t; l; e_1 : t_1 \yields {E_1}) : (t_1', t') \yields C \andalso
 t_1 \subtype t_1'}
{\Gamma \vdash \recordupdate e l {e_1} : t' \yields {\app C E}}

\framebox{\( \turnsget (\ty_1; l) : \ty_2 \yields C \)}

The field with label $ l $ inside $ \ty_1 $ is of type $ \ty_2 $.

\infax[GetBase]
{\turnsget (\recty l \ty; l) : \ty
  \yields {\idmono {\image {\recty l \ty}}}}

\infrule[GetLeft]
{\turnsget (\ty_1; l) : \ty \yields C}
{\turnsget (\ty_1 \intersects \ty_2; l) : \ty
  \yields {\absty x {\image {\ty_1 \intersects \ty_2}} {C (\app \fst x)}}}

\infrule[GetRight]
{\turnsget (\ty_2; l) : \ty \yields C}
{\turnsget (\ty_1 \intersects \ty_2; l) : \ty
  \yields {\absty x {\image {\ty_1 \intersects \ty_2}} {C (\app \snd x)}}}

\framebox{\( \vdash_{put} (t; l; e : t \yields E) : (t, t) \yields C \)}

\infax[PutBase]
{\vdash_{put} (\recordtype l t; l; e : t' \yields E) : (t, \recordtype l {t'})
  \yields {\abs {\rel x {\image {\recordtype l t}}} E}}

\infrule[PutLeft]
{\vdash_{put} (t_1; l; e : t \yields E) : (t', t_1') \yields C}
{\vdash_{put} (t_1 \with t_2; l; e : t \yields E) : (t', t_1' \with t_2)
  \yields {\abs {\rel x {\image {t_1 \with t_2}}} {C (\fst ~ x)}}}

\infrule[PutRight]
{\vdash_{put} (t_2; l; e : t \yields E) : (t', t_2') \yields C}
{\vdash_{put} (t_1 \with t_2; l; e : t \yields E) : (t', t_1' \with t_2)
  \yields {\abs {\rel x {\image {t_1 \with t_2}}} {C (\fst ~ x)}}}


\begin{itemize}

\item{\bf Coercion}

  Explained in the previous subsection.

\item{\bf Elaboration}

  The elaboration judgment $ \Gamma \vdash e : \tau \yields{E} $ extends the
  typing judgment with an elaborated expression on the right hand side of
  $ \hookrightarrow $. It is also standard, except for the case of
  \texttt{TrApp}, in which a coercion from the inferred type of the argument,
  $ e_2 $ , to the expected type of the parameter, $ t_1 $, is inserted before
  the argument; \texttt{TrMerge} translates merges into pairs.
  \texttt{TrRcdIntro} uses the same System F expression $ E $ for $ e $ as for
  $ \{ l = e \} $. And in \texttt{TrRcdEim} and \texttt{TrRcdUpd} the coercions
  generated by the ``get'' and ``put'' rules will be used to coece the main
  \name expression. The two set of rules are explained below.

\item{\bf ``get'' rules}

  The ``get'' judgment deals spefically with record elimination and yields a
  coercion can be thought as a field accessor. For example:

  $ \Gamma \vdash_{get} (\{ eval : Int \}, eval) : \{ eval : Int \} \yields {\abs {\rel x Int} x} $

  The lambda is the field accessor and when applied to a translated expression
  of type $ \{ eval : Int \}$, it is able to give the desired field.
  \texttt{GetBase} is the base case: the type of the field labelled $ l $ in a $
  \recordtype l t $ is just $ t $ and the coercion is an identity function
  specialized to type $ \imageof {\recordtype l t} $
  \texttt{GetLeft} and \texttt{GetRight} are complementary to each other.

\item{\bf ``put'' rules}

  The ``put'' judgment deals spefically with record update can be thought as
  producing a field updater. Compared to the ``get'' rules, the ``put'' rules
  take an extra input $ e $, which is the desired expression to replace the
  field lablled $ l $ in values of type $ t $. \texttt{PutBase} is the base
  case. This rule allows refinement of record fields in the sense that the type
  of $ e $ can be a subtype of the type of the field labelled by $ l $. The
  resulting type is $ \recordtype l {t'} $ and the generated coercion is a
  constant function that always returns $ E $. \texttt{PutLeft} and
  \texttt{PutRight} are complementary to each other: the idea is exactly the
  same as \texttt{GetLeft} and \texttt{GetRight} except that the refined type
  $ t_1' $ and $ t_2' $ is used.

\end{itemize}

\subsection{Meta-theory}

\begin{lemma}{Subtyping is reflexive} \label{sub-refl}
Given a type $ \tau $, $ \tau \subtype \tau $.
\end{lemma}

\begin{lemma}{Subtyping is transitive} \label{sub-trans}
If $ \tau_1 \subtype \tau_2 $ and $ \tau_2 \subtype \tau_3$,
then $ \tau_1 \subtype \tau_3$.
\end{lemma}


\begin{lemma}[Get rules produce the correct coercion] \label{type-get}
  If $$ \Gamma \vdash_{get} \tau ; l = C ; \tau_1 $$
  then $$ |\Gamma| \vdash C : |\tau| \to |\tau_1| $$
\end{lemma}

\begin{proof}
By induction on the given derivation.
\end{proof}

\begin{lemma}[Put rules produce the correct coercion] \label{type-put}
  If $$ \Gamma \vdash_{put} \tau ; l ; E = C ; \tau_1 $$
  then $$ |\Gamma| \vdash C : |\tau| \to |\tau| $$
\end{lemma}

\begin{proof}
By induction on the given derivation.
\end{proof}

\begin{lemma}[Translation preserves well-formedness] \label{preserve-wf}
  If   $$ \Gamma \vdash \tau $$
  then $$ |\Gamma| \vdash |\tau| $$
\end{lemma}

\begin{proof}
By induction on the given derivation.
\end{proof}

\begin{theorem}[Type preserving translation] \label{preserve-tr}
  If   $$ \Gamma \vdash e : \tau \yields{E} $$
  then $$ |\Gamma| \vdash E : \left| \tau \right| $$
\end{theorem}

\begin{proof}
(Sketch) By structural induction on the expression and the corresponding
inference rule. The full proof can be found in the appendix.
\end{proof}

Type-Directed Translation to System F.
Main results: type-preservation + coherence.

\section{Implementation}

We implemented the core functionalities of the \name as part of a JVM-based
compiler. The implementation supports record update instead of restriction as a
primitive; however the former is formalized with the same underlying idea of
elaborating records. Based on the type system of \name, we built an ML-like
source language compiler that offers interoperability with Java (such as object
creation and method calls). The source language is loosely based on the more
general System $F_{\omega}$ (compared to our target, System $F$) and supports a
number of other features, including multi-field records, mutually recursive
\code{let} bindings, type aliases, algebraic data types, pattern matching, and
first-class modules that are encoded with \code{letrec} and records.

Relevant to this paper are the three following phases in the compiler that
collectively turn source programs into System $F$:

\begin{enumerate}
\item A \emph{typechecking} phase that checks the usage of \name features and
  other source language features against an abstract syntax tree that follows
  the source syntax.

\item A \emph{desugaring} phase that translates well-typed source terms into
  \name terms. Source-level features such as multi-field records, type aliases
  are removed at this phase. The resulting program is just an \name expression
  extended with some other constructs necessary for code generation.

\item A \emph{translation} phase that turns well-typed \name terms into System
  $F$ ones.
\end{enumerate}

Phase 3 is what we have formalized in this paper.

\paragraph{Removing identity functions.} Our translation inserts identity
functions whenever subtyping or record operation occurs, which could mean
notable run-time overhead. But in practice this is not an issue. In the current
implementation, we introduced a partial evaluator with three simple rewriting
rules to eliminate the redundant identity functions as another compiler phase
after the translation. In another version of our implementation, partial
evaluation is weaved into the process of translation so that the unwanted
identity functions are not introduced during the translation.

\section{Related Work} \label{sec:related-work}

% \url{http://homepages.inf.ed.ac.uk/gdp/publications/Sub_Par.pdf}

% \cite{plotkin1994subtyping}

% Also discussed intersection types!~\cite{malayeri2008integrating}.

% Pierce Ph.D thesis: F<: + /|
%        technical report: F + /|, closer to ours

% \cite{barbanera1995intersection}

\paragraph{Intersection types with polymorphism.}
Our type system combines intersection types and parametric polymorphism. Closest
to us is Pierce's work~\cite{pierce1991programming1} on a prototype
compiler for a language with both intersection types, union types, and
parametric polymorphism. Similarly to \name in his system universal
quantifiers do not support bounded quantification. However Pierce did not try to prove any
meta-theoretical results and his calculus does not have a merge
operator.  Pierce also studied a system where both intersection
types and bounded polymorphism are present in his Ph.D.
dissertation~\cite{pierce1991programming2} and a 1997
report~\cite{pierce1997intersection}. Going in the direction of higher
kinds, Compagnoni and Pierce~\cite{compagnoni1996higher} added
intersection types to System $ F_{\omega} $ and used the new calculus,
$ F^{\omega}_{\wedge} $, to model multiple inheritance. In their
system, types include the construct of intersection of types of the
same kind $ K $. Davies and Pfenning
\cite{davies2000intersection} studied the interactions between
intersection types and effects in call-by-value languages. And they
proposed a ``value restriction'' for intersection types, similar to
value restriction on parametric polymorphism. Although they proposed a system with
parametric polymorphism, their subtyping rules are significantly different from ours,
since they consider parametric polymorphism
as the ``infinit analog'' of intersection polymorphism.
There have been attempts to provide a foundational calculus
for Scala that incorporates intersection
types~\cite{amin2014foundations,amin2012dependent}.
Although the minimal Scala-like calculus does not natively support
parametric polymorphism, it is possible to encode parametric
polymorphism with abstract type members. Thus it can be argued that
this calculus also supports intersection types and parametric
polymorphism. However, the type-soundness of a minimal Scala-like
calculus with intersection types and parametric polymorphism is not
yet proven. Recently, some form of intersection
types has been adopted in object-oriented languages such as Scala,
Ceylon, and Grace. Generally speaking,
the most significant difference to \name is that in all previous systems
there is no explicit introduction construct like our merge operator. As shown in
Section~\ref{sec:overview}, this feature is pivotal in supporting modularity
and extensibility because it allows dynamic composition of values.

\begin{comment}
only allow intersections of concrete types (classes),
whereas our language allows intersections of type variables, such as
\texttt{A \& B}. Without that vehicle, we would not be able to define
the generic \texttt{merge} function (below) for all interpretations of
a given algebra, and would incur boilerplate code:

\begin{lstlisting}{language=haskell}
let merge [A, B] (f: ExpAlg A) (g: ExpAlg B) = {
  lit (x : Int) = f.lit x ,, g.lit x,
  add (x : A & B) (y : A & B) =
    f.add x y ,, g.add x y
}
\end{lstlisting}
\end{comment}


\paragraph{Other type systems with intersection types.}
Intersection types date back to as early as Coppo et
al.~\cite{coppo1981functional}. As emphasized throughout the paper our
work is inspired by Dunfield~\cite{dunfield2014elaborating}. He described a similar approach to ours:
compiling a system with intersection types into ordinary $ \lambda $-calculus
terms. The major difference is that his system does not include parametric
polymorphism, while ours does not include unions. Besides, our rules are
algorithmic and we formalize a record system.
% Although similar in spirit,
% our elaboration typing is simpler: we require subtyping in the case of
% applications, thus avoiding the subsumption rule. Besides, our treatment
% combines the merge rules ($ k $ ranges over $ \{1, 2\} $)
% \inferrule
% {\Gamma \turns e_k : A}
% {\Gamma \turns e_1 \mergeop e_2 : A}
% and the standard intersection-introduction rule
% \inferrule
% {\Gamma \turns e : A_1 \andalso \Gamma \turns e : A_2}
% {\Gamma \turns e : A_1 \inter A_2}
% into one rule:
% \inferrule [Merge]
% {\Gamma \turns e_1 : A_1 \andalso \Gamma \turns e_2 : A_2}
% {\Gamma \turns e_1 \mergeop e_2 : A_1 \inter A_2}
Reynolds invented Forsythe~\cite{reynolds1997design} in the 1980s. Our merge
operator is analogous to his $ p_1, p_2 $. As Dunfield
has noted, in Forsythe merges can be only used unambiguously.
For instance, it is not allowed in Forsythe to merge two functions.

%Castagna, and Dunfield describe
%elaborating multi-fields records into merge of single-field records.
% Reynolds and Castagna do not consider elaboration and Dunfield do not
% formalize elaborating records.

% Both Pierce and Dunfield's system include a subsumption rule, which states that
% if an term has been inferred of type $ A $, then it is also of any
% supertype of $ A $. Our system does not have this rule.

Refinement
intersection~\cite{dunfield2007refined,davies2005practical,freeman1991refinement}
is the more conservative approach of adopting intersection types. It increases
only the expressiveness of types but not terms. But without a term-level
construct like ``merge'', it is not possible to encode various language
features. As an alternative to syntatic subtyping described in this paper,
Frisch et al.~\cite{frisch2008semantic} studied semantic subtyping.

\paragraph{Languages for extensibility.}
To improve support for extensibility various researchers have proposed
new OOP languages or programming mechanisms. It is interesting to
note that design patterns such as object algebras or modular visitors
provide a considerably different approach to extensibility when
compared to some previous proposals for language designs for
extensibility. Therefore the requirements in terms of type system
features are quite different.  One popular approach is \emph{family
  polymorphism}~\cite{Ernst01family}, which allows whole class hierarchies to be
captured as a family of classes. Such a family can be later reused to
create a derived family with potentially new class members, and
additional methods in the existing classes.  \emph{Virtual
  classes}~\cite{ernst2006virtual} are a concrete realization of this idea, where a
container class can hold nested inner \emph{virtual} classes (forming
the family of classes). In a subclass of the container class, the
inner classes can themselves be \emph{overriden}, which is why they
are called virtual. There are many language mechanisms that provide
variants of virtual classes or similar mechanisms~\cite{McDirmid01Jiazzi,Aracic06CaesarJ,Smaragdakis98mixin,nystrom2006j}. The work by
Nystrom on \emph{nested intersection}~\cite{nystrom2006j} uses a
form of intersection types to support the composition of
families of classes. Ostermann's \emph{delegation layers}~\cite{Ostermann02dynamically}
use delegation for doing dynamic composition in a system
with virtual classes. This in contrast with most other approaches
that use class-based composition, but closer to the dynamic
composition that we use in \name.
\begin{comment}
In contrast to type systems for virtual classes
and similar mechanisms, the goal of our work is to study the type
systems and basic language mechanism to better support such design patterns.
 some researchers have designed new type
system features such as virtual classes~\cite{ernst2006virtual}, polymorphic
variants~\cite{garrigue1998programming}, while others have shown employing
programming pattern such as object algebras~\cite{oliveira2012extensibility} by
using features within existing programming languages. Both of the two approaches
have drawbacks of some kind. The first approach often involves heavyweight
designs, while the second approach still sacrifices the readability for
extensibility.
\bruno{fill me in with more details and more references!}
\end{comment}
% Intersection types have been shown to be useful in designing languages that
% support modularity.~\cite{nystrom2006j}

% \paragraph{Extensible records.}

%\george{Record field deletion is also possible.}

% http://elm-lang.org/learn/Records.elm

% Encoding records using intersection types appeared in
% Reynolds~\cite{reynolds1997design} and Castagna et
% al.~\cite{castagna1995calculus}. Although Dunfield also discussed this idea in
% his paper \cite{dunfield2014elaborating}, he only provided an implementation but
% not a formalization. Very similar to our treatment of elaborating records is
% Cardelli's work~\cite{cardelli1992extensible} on translating a calculus, named
% $ F_{\subtype \rho}$, with extensible records to a simpler calculus that without
% records primitives (in which case is $ F_{\subtype} $). But he did not consider
% encoding multi-field records as intersections; hence his translation is more
% heavyweight. Crary~\cite{crary1998simple} used intersection types and
% existential types to address the problem that arises when interpreting method
% dispatch as self-application. But in his paper, intersection types are not used
% to encode multi-field records.

% Wand~\cite{wand1987complete} started the work on extensible records and proposed
% row types~\cite{wand1989type} for records. Cardelli and
% Mitchell~\cite{cardelli1990operations} defined three primitive operations on
% records that are similar to ours: \emph{selection}, \emph{restriction}, and
% \emph{extension}. The merge operator in \name plays the same role as extension.
% Following Cardelli and Mitchell's approach,
% of restriction and extension. Both Leijen's systems~\cite{leijen2004first,leijen2005extensible}
% and ours allow records that contain
% duplicate labels. Leijen's system is more sophisticated. For example, it supports
% passing record labels as arguments to functions. He also showed an encoding of
% intersection types using first-class labels.

% Chlipala's
% \texttt{Ur}~\cite{chlipala2010ur} explains record as type level
% constructs.\bruno{What is the point of citing Chlipala's paper?}

% Our system can be adapted to simulate systems that support extensible
% records but not intersection of ordinary types like \texttt{Int} and
% \texttt{Float} by allowing only intersection of record types.
%
% $ \turnsrec A $ states that $ A $ is a record type, or the intersection of
% record types, and so forth.
%
% \inferrule [RecBase] {} {\turnsrec \recordType l A}
%
% \inferrule [RecStep]
% {\turnsrec A_1 \andalso \turnsrec A_2}
% {\turnsrec A_1 \inter A_2}
%
% \inferrule [Merge']
% {\Gamma \turns e_1 : A_1 \yields {E_1} \andalso \turnsrec A_1 \\
%  \Gamma \turns e_2 : A_2 \yields {E_2} \andalso \turnsrec A_2}
% {\Gamma \turns e_1 \mergeop e_2 : A_1 \inter A_2 \yields {\pair {E_1} {E_2}}}
%
% Of course our approach has its limitation as duplicated labels in a record are
% allowed. This has been discussed in a larger issue by
% Dunfield~\cite{dunfield2014elaborating}.
%
% R{\'e}my~\cite{remy1989type}

\section{Conclusion and Future Work}
\label{sec:conclusion}

This paper described \name: a System $F$-based language that combines
intersection types, parametric polymorphism and a merge operator.
The language is proved to be type-safe and coherent.
To ensure coherence the type system accepts only
disjoint intersections. To provide flexibility in the presence of parametric polymorphism,
universal quantification is extended with
disjointness constraints. We believe that disjoint intersection types
and disjoint quantification are intuitive, and at the same time
flexible enough to enable practical applications.

%We implemented the core functionalities of the \namedis as part of a JVM-based
%compiler. Based on the type system of \namedis, we have built an ML-like
%source language compiler that offers interoperability with Java (such as object
%creation and method calls). The source language is loosely based on the more
%general System $F_{\omega}$ and supports a
%number of other features, including records, mutually recursive
%\code{let} bindings, type aliases, algebraic data types, pattern matching, and
%first-class modules that are encoded using \code{letrec} and records.

For the future, we intend to create a prototype-based statically typed
source language based on \name.  We are also interested in extending
our work to systems with union types and a $\bot$ type. Union types
are also widely used in languages such as Ceylon or Flow, but
preserving coherence in the presence of union types is
challenging. The naive addition of $\bot$ seems to be problematic. 
The proofs for \name rely on the invariant that a type variable $\alpha$ can never be disjoint 
to another type that contains $\alpha$. The addition of $\bot$ breaks
this invariant, allowing us to derive, for example, $\jdis \Gamma
\alpha \alpha$.
Finally, we could study a similar system with implicit polymorphism.
Such system would require some changes in the subtyping and disjointness relations.
For instance, subtyping should allow  
${\for \alpha {\alpha \to \alpha}} \subtype \tyint \to \tyint$.
Consequently, the disjointness relation would have to be modified,
since valid statements in \name such as 
$\jdis \Gamma {\for \alpha {\alpha \to \alpha}} {\tyint \to \tyint}$ 
would no longer hold under the more powerful subtyping relation. 


% http://en.wikibooks.org/wiki/LaTeX/Bibliography_Management

% We recommend abbrvnat bibliography style.
\bibliographystyle{abbrvnat}

% The bibliography should be embedded for final submission.
\bibliography{bib/papers,bib/records}

% \begin{thebibliography}{}
% \softraggedright

% \bibitem[Smith et~al.(2009)Smith, Jones]{smith02}
% P. Q. Smith, and X. Y. Jones. ...reference text...

% Coppo, M., Dezani-Ciancaglini, M.: A new type-assignment for λ-terms. Archiv.
% Math. Logik 19, 139–156 (1978)

% \end{thebibliography}


\acks

Acknowledgments, if needed.

\clearpage
\onecolumn
\appendix
\section{Proofs}

\begin{lemma}[\rulelabelget~rules produce type-correct coercion]
  If $ \judgeget \tau l {\tau_1} \yields C $, then $ \epsilon \turns C :
  \im \tau \to \im {\tau_1} $.
\end{lemma}

\begin{proof}
  By induction of the derivation.

  \begin{itemize}

  \item \textbf{Case}
    \begin{flalign*}
      & \rulegetelab &
    \end{flalign*}

    \begin{tabular}{ll}
      $ \epsilon \turns \Lam x {\im {\RecTy l \tau}} x : \im {\RecTy l \tau} \to \im {\RecTy l \tau} $ & By $\rulelabeltlam$ and $\rulelabeltvar$ \\
      $ \epsilon \turns \Lam x {\im {\RecTy l \tau}} x : \im {\RecTy l \tau} \to \im \tau $ & By the definition of $\im \cdot$
    \end{tabular} \\

  \item \textbf{Case}
    \begin{flalign*}
      & \rulegetleftelab &
    \end{flalign*}

    \begin{tabular}{ll}
      $\epsilon, x \hast \im {\tau_1 \Intersect \tau_2} \turns x : \im {\tau_1 \Intersect \tau_2}$ & By $\rulelabeltvar$ \\
      $\epsilon, x \hast \im {\tau_1 \Intersect \tau_2} \turns x : \Pair {\im {\tau_1}} {\im {\tau_2}}$ & By the definition of $\im \cdot$ \\
      $\epsilon, x \hast \im {\tau_1 \Intersect \tau_2} \turns \Proj 1 x : \im {\tau_1}$ & By $\rulelabeltprojleft$ \\
      $\epsilon \turns C : \im {\tau_1} \to \im \tau$ & By i.h. \\
      $\epsilon, x \hast \im {\tau_1 \Intersect \tau_2} \turns C : \im
      {\tau_1} \to \im \tau$ & \george{What should this be called?} \\
      $\epsilon, x \hast \im {\tau_1 \Intersect \tau_2} \turns \App C (\Proj 1 x) : \im {\tau}$ & By $\rulelabeltapp$ \\
      $\epsilon \turns \Lam x {\im {\tau_1 \Intersect \tau_2}} {C (\Proj 1 x)} : \im {\tau_1 \Intersect \tau_2} \to \im {\tau}$ & By $\rulelabeltlam$
    \end{tabular} \\

  \item \textbf{Case}
    \begin{flalign*}
      & \rulegetrightelab &
    \end{flalign*}

    By symmetry with the above case. \\

\end{itemize}
\end{proof}


\begin{lemma}[\rulelabelput~rules produce type-correct coercion]
  If $ \judgeput \tau l {\tau_1 \yields E} {\tau_2} {\tau_3} \yields C $ and $
  \Gamma \turns E : \im {\tau_1} $ for some $ \Gamma $, then
  $ \Gamma \turns C : \im \tau \to \im {\tau_2} $.
\end{lemma}

\begin{proof}
  By induction of the derivation.

  \begin{itemize}

  \item \textbf{Case}
    \begin{flalign*}
      & \ruleputelab &
    \end{flalign*}

    \begin{tabular}{ll}
      $ \Gamma \turns \Lam \_ {\im {\RecTy l \tau}} E : \im {\RecTy l \tau} \to
      \im {\tau_1} $ & By $\rulelabeltlam$, $\rulelabeltvar$, and the hypothesis \\
    \end{tabular} \\

  \item \textbf{Case}
    \begin{flalign*}
      & \ruleputleftelab &
    \end{flalign*}

    \begin{tabular}{ll}
      $\Gamma, x \hast \im {\tau_1 \Intersect \tau_2} \turns x : \im {\tau_1 \Intersect \tau_2} $ & By $\rulelabeltvar$ \\
      $\Gamma, x \hast \im {\tau_1 \Intersect \tau_2} \turns x : \Pair {\im {\tau_1}} {\im {\tau_2}} $ & By the definition of $\im \cdot$ \\
      $\Gamma, x \hast \im {\tau_1 \Intersect \tau_2} \turns \Proj 1 x : \im {\tau_1} $ & By $\rulelabeltprojleft$ \\
      $\Gamma \turns C : \im {\tau_1} \to \im {\tau_3}$ & By i.h. \\ 
      $\Gamma, x \hast \im {\tau_1 \Intersect \tau_2} \turns C : \im {\tau_1} \to \im {\tau_3}$ & \george{Really?} \\ 
      $\Gamma, x \hast \im {\tau_1 \Intersect \tau_2} \turns \App C {(\Proj 1 x)} : \im {\tau_3} $ & By $\rulelabeltapp$ \\
      $\Gamma \turns \Lam x {\im {\tau_1 \Intersect \tau_2}} {\App C {(\Proj 1 x)}} : \im {\tau_1 \Intersect \tau_2} \to \im {\tau_3} $ & By $\rulelabeltlam$ \\
    \end{tabular} \\

  \item \textbf{Case}
    \begin{flalign*}
      & \ruleputrightelab &
    \end{flalign*}

    By symmetry with the above case. \\

\end{itemize}
\end{proof}



\begin{proof}
By structural induction on the types and the corresponding inference rule. \\

\rulename{SubVar}

\rulename{SubFun}

\rulename{SubForall}

\rulename{SubAnd1}

\rulename{SubAnd2}

\rulename{SubAnd3}

\rulename{SubRcd}

\end{proof}

\begin{lemma}
  If $$ \Gamma \turnsput \tau ; l ; E = C ; \tau_1 $$
  then $$ \im \Gamma \turns C : \im \tau \to \im \tau $$
\end{lemma}

\begin{proof}
By structural induction on the type and the corresponding inference rule. \\

\rulename{Put-Base} \\
\rulename{Put-Left} \\
\rulename{Put-Right} \\
\end{proof}

\begin{lemma} \label{preserve-wf}
  If   $$ \Gamma \turns \tau $$
  then $$ \im \Gamma \turns \im \tau $$
\end{lemma}

\begin{proof}
Since $$ \Gamma \turns \tau $$
It follows from \rulename{FI-WF} that
  $$ \ftv \tau  \subseteq \ftv {\Gamma} $$
And hence
  $$ \ftv {\im \tau} \subseteq \ftv {\im \Gamma} $$
By \rulename{F-WF} we have
  $$ \Gamma \turns \tau $$
\end{proof}

\begin{theorem}[Type preserving translation]
  If   $$ \Gamma \turns e : \tau \yields E  $$
  then $$ \im \Gamma \turns E : \im \tau $$
\end{theorem}

\begin{proof}
By structural induction on the expression and the corresponding inference rule. \\

\rulename{Var} $ \Gamma \turns x : \tau \yields x $ \\

It follows from \rulename{Var} that
  $$ (x : t) \in \Gamma $$
Based on the definition of $ \im \cdot $,
  $$ (x : \im t) \in \im \Gamma $$
Thus we have by \rulename{F-Var} that
  $$ \im \Gamma \turns x : \im \tau $$

\rulename{Abs} $ \Gamma \turns \lambda (x : \tau_1). e : \tau_1 \to \tau_2 \yields {\Lam x {\im {\tau_1}} E} $ \\

It follows from \rulename{Abs} that
  $$ \Gamma, x : \tau_1 \turns e : \tau_2 \yields E $$
And by the induction hypothesis that
  $$ \im \Gamma, x : \im {\tau_1} \turns E : \im {\tau_2} $$
By \rulename{Abs} we also have
  $$ \Gamma \turns \tau_1 $$
It follows from Lemma \ref{preserve-wf} that
  $$ \im \Gamma \turns \im {\tau_1} $$
Hence by \rulename{F-Abs} and the definition of $ \im \cdot $ we have
  $$ \im \Gamma \turns \Lam x {\im {\tau_1}} E : \im {\tau_1 \to \tau_2} $$

\rulename{(TrApp)} $ \Gamma \turns \App {e_1} {e_2} : \tau_2 \yields {E_1 (\App C {E_2})} $ \\

From \rulename{(TrApp)} we have
  $$ \Gamma \turns \tau_3 <: \tau_1 \yields C $$
Applying Lemma \ref{type-coerce} to the above we have
  $$ \im \Gamma \turns C : \im {\tau_3} \to \im {\tau_1} $$
Also from \rulename{(TrApp)} and the induction hypothesis
  $$ \im \Gamma \turns E_1 : \im {\tau_1} \to \im {\tau_2} $$
Also from \rulename{(TrApp)} and the induction hypothesis
  $$ \im \Gamma \turns E_2 : \im {\tau_3} $$
Assembling those parts using \rulename{(F-App)} we come to
  $$ \im \Gamma \turns E_1 (\App C {E_2}) : \im {\tau_2} $$
\end{proof}

\rulename{TAbs} $ \Gamma \turns \Lambda \alpha. e : \forall \alpha. \tau \yields {\forall \alpha. E} $ \\

From \rulename{TAbs} we have
  $$ \Gamma \turns e : \tau \yields E $$
By the induction hypothesis we have
  $$ \im \Gamma \turns E : \im \tau $$
Thus by \rulename{F-TAbs} and the definition of $ \im \cdot $
  $$ \Gamma \turns \Lambda \alpha. E : \im {\forall \alpha. \tau} $$


\rulename{TAp} $ \Gamma \turns e \; \tau  : \subst \tau \alpha  \tau_1 \yields {E \; \im \tau} $ \\

From \rulename{TApp} we have
  $$ \Gamma \turns e : \forall \alpha. \tau_1 \yields E $$
And by the induction hypothesis that
  $$ \im \Gamma \turns E : \forall \alpha. \im {\tau_1} $$
Also from \rulename{TApp} and Lemma \ref{preserve-wf} we have
  $$ \im \Gamma \turns \im \tau $$
Then by \rulename{F-TApp} that
  $$ \im \Gamma \turns E \; \im \tau : \subst {\im \tau} \alpha \im {\tau_1} $$
Therefore
  $$ \im \Gamma \turns E \; \im \tau : \im {\subst \tau \alpha \im {\tau_1}} $$

% \rulename{(TrMerge)} $ \Gamma \turns e_1 \merge e_2 : \tau_1 \& \tau_2 % % \yields {\Pair {E1} {E2} $ \\

From \rulename{(TrMerge)} and the induction hypothesis we have
  $$ \im \Gamma \turns E_1 : \im {\tau_1} $$
and
  $$ \im \Gamma \turns E_2 : \im {\tau_2} $$
Hence by \rulename{F-Pair}
  $$ \im \Gamma \turns \Pair {E_1} {E_2} : \Pair {\im {\tau_1}} {\im {\tau_2}} $$
Hence by the definition of $ \im \cdot $
  $$ \im \Gamma \turns \Pair {E_1} {E_2} : \im {\tau_1 \& \tau_2} $$

\rulename{RecIntro} $ \Gamma \turns \RecCon l e : \RecTy l \tau \yields E $ \\

From \rulename{RcdIntro} we have
  $$ \Gamma \turns e : \tau \yields E $$
And by the induction hypothesis that
  $$ \im \Gamma \turns E : \im \tau $$
Thus by the definition of $ \im \cdot $
  $$ \im \Gamma \turns E : \im {\RecTy l \tau} $$

\rulename{RcdElim} $ \Gamma \turns e.l : \tau_1 \yields {\App C E} $ \\

From \rulename{RcdElim}
  $$ \Gamma \turns e : \tau \yields E $$
And by the induction hypothesis that
  $$ \im \Gamma \turns E : \im \tau $$
Also from \rulename{RcdEim}
  $$ \Gamma \turnsget e ; l = C ; \tau_1 $$
Applying Lemma \ref{type-get} to the above we have
  $$ \im \Gamma \turns C : \im \tau \to \im {\tau_1}  $$
Hence by \rulename{F-App} we have
  $$ \im \Gamma \turns \App C E : \im {\tau_1} $$

From \rulename{RcdUpd}
  $$ \Gamma \turns e : \tau \yields E $$
And by the induction hypothesis that
  $$ \im \Gamma \turns E : \im \tau $$
Also from \rulename{RcdUpd}
  $$ \Gamma \turnsput \tau ; l; E = C ; \tau_1 $$
Applying Lemma \ref{type-put} to the above we have
  $$ \im \Gamma \turns C : \im \tau \to \im \tau  $$
Hence by \rulename{F-App} we have
  $$ \im \Gamma \turns \App C E : \im \tau $$


\end{document}
