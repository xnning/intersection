\documentclass[preprint]{sigplanconf}

% Use packages immediately following the \documentclass command
% Remote packages
\usepackage{amsmath}
\usepackage{amsthm}
\usepackage{fixltx2e}
\usepackage{mdframed}
\usepackage{stmaryrd}
\usepackage{xcolor}

% For documentation for the `Listings` package,
% see http://texdoc.net/texmf-dist/doc/latex/listings/listings.pdf
\usepackage{listings}

% Use the minted package to format code snippets
% Documentation: https://code.google.com/p/minted/downloads/detail?name=minted.pdf
\usepackage{minted}


% Local packages
\usepackage{bcprules}
\usepackage{cmll}
\usepackage{mathpartir}


% Always load mathastext last
% http://mirrors.ibiblio.org/CTAN/macros/latex/contrib/mathastext/mathastext.pdf
\renewcommand\familydefault\ttdefault
\usepackage{mathastext}
\renewcommand\familydefault\rmdefault


\newmdtheoremenv{theorem}{Theorem}
\newmdtheoremenv{lemma}{Lemma}

\definecolor{github}{HTML}{4183C4}

% Define macros immediately before the \begin{document} command
% math-mode versions of \rlap, etc
% from Alexander Perlis, "A complement to \smash, \llap, and lap"
%   http://math.arizona.edu/~aprl/publications/mathclap/
\def\clap#1{\hbox to 0pt{\hss#1\hss}}
\def\mathllap{\mathpalette\mathllapinternal}
\def\mathrlap{\mathpalette\mathrlapinternal}
\def\mathclap{\mathpalette\mathclapinternal}
\def\mathllapinternal#1#2{\llap{$\mathsurround=0pt#1{#2}$}}
\def\mathrlapinternal#1#2{\rlap{$\mathsurround=0pt#1{#2}$}}
\def\mathclapinternal#1#2{\clap{$\mathsurround=0pt#1{#2}$}}

% math-mode versions of \rlap, etc
% from Alexander Perlis, "A complement to \smash, \llap, and lap"
%   http://math.arizona.edu/~aprl/publications/mathclap/
\def\clap#1{\hbox to 0pt{\hss#1\hss}}
\def\mathllap{\mathpalette\mathllapinternal}
\def\mathrlap{\mathpalette\mathrlapinternal}
\def\mathclap{\mathpalette\mathclapinternal}
\def\mathllapinternal#1#2{\llap{$\mathsurround=0pt#1{#2}$}}
\def\mathrlapinternal#1#2{\rlap{$\mathsurround=0pt#1{#2}$}}
\def\mathclapinternal#1#2{\clap{$\mathsurround=0pt#1{#2}$}}

\newcommand{\horizontalrule}{
  \begin{center}
    \line(1,0){250}
  \end{center}
}

\newcommand{\code}[1]{\texttt{#1}}
\definecolor{light-gray}{gray}{0.9}
\newcommand{\highlight}[1]{\colorbox{light-gray}{#1}}

\newcommand{\im}[1]{\lvert #1 \rvert}
\newcommand{\powerset}[1]{\mathcal{P}(#1)}
\newcommand{\universe}{\mathbb{U}}

\newcommand{\turns}{\vdash}
\newcommand{\oftype}{\!:\!}
\newcommand{\subtype}{<:}
\newcommand{\commonsuper}{\Uparrow}
\newcommand{\commonsub}{\Downarrow}
\newcommand{\disjoint}{*}
\newcommand{\disjointimpl}{*_\textnormal{i}}
\newcommand{\disjointax}{*_\textnormal{ax}}
\newcommand{\subst}[2]{\lbrack #2 := #1 \rbrack~}

\newcommand{\yields}[1]{\highlight{$\; \hookrightarrow #1$}}
% \newcommand{\yields}[1]{}

\newcommand{\ftv}[1]{\textsf{ftv}(#1)}

\newcommand{\binderspace}{\,}
\newcommand{\appspace}{\;}

\newcommand{\inter}{\&}
\newcommand{\union}{|}
\newcommand{\for}[2]{\forall #1.\binderspace #2}
\newcommand{\fordis}[3]{\for {(#1 \disjoint #2)} {#3}}
\newcommand{\lam}[2]{\lambda #1.\binderspace #2}
\newcommand{\lamty}[3]{\lam {(#1 \oftype #2)} #3}
\newcommand{\blam}[2]{\Lambda #1.\binderspace #2}
\newcommand{\blamdis}[3]{\blam {(#1 \disjoint #2)} #3}
\newcommand{\mergeop}{,,}
\newcommand{\app}[2]{#1 \; #2}
\newcommand{\tapp}[2]{#1 \appspace #2}
\newcommand{\pair}[2]{(#1, #2)}
\newcommand{\proj}[2]{{\code{proj}}_{#1} #2}
\newcommand{\fst}[1]{\app {\code{fst}} {#1}}
\newcommand{\snd}[1]{\app {\code{snd}} {#1}}
\newcommand{\recordType}[2]{\{ #1 : #2 \}}
\newcommand{\recordCon}[2]{\{ #1 = #2 \}}

\newcommand{\true}{\code{True}}
\newcommand{\tyint}{\code{Int}}
\newcommand{\tybool}{\code{Bool}}
\newcommand{\tychar}{\code{Char}}
\newcommand{\tystring}{\code{String}}

% Judgements
\newcommand{\jwf}[2]{#1 \turns #2}
\newcommand{\jatomic}[1]{#1 \ \textnormal{atomic}}
\newcommand{\jtype}[3]{\turns #2 \ \oftype \ #3}
\newcommand{\jdis}[3]{\turns #2 \disjoint #3}
\newcommand{\jdisimpl}[3]{\turns #2 \disjointimpl #3}

\newcommand{\reflabel}[1]{(\textsc{#1})}

% \newcommand{\name}{$ \lambda_{\&} $\xspace}
\newcommand{\name}{\xspace[name]\xspace}

\newcommand{\authornote}[3]{\textcolor{#2}{\textsc{#1}: #3}}
\newcommand\bruno[1]{\authornote{bruno}{Red}{#1}}
\newcommand\george[1]{\authornote{george}{Blue}{#1}}


\lstdefinelanguage{F2J}{
  morekeywords={let,type,module,end,in},
  otherkeywords={->},
  sensitive=false, % whether keywords are case sensitive
  morecomment=[l]{--},
  morestring=[b]", % `b' means inside a string delimiters are escaped by a backslash.
  morestring=[b]'
}

\lstset{ %
  language=F2J,                % choose the language of the code
  columns=flexible,
  lineskip=-1pt,
  basicstyle=\ttfamily\small,       % the size of the fonts that are used for the code
  numbers=none,                   % where to put the line-numbers
  numberstyle=\ttfamily\tiny,      % the size of the fonts that are used for the line-numbers
  stepnumber=1,                   % the step between two line-numbers. If it's 1 each line will be numbered
  numbersep=5pt,                  % how far the line-numbers are from the code
  backgroundcolor=\color{white},  % choose the background color. You must add \usepackage{color}
  showspaces=false,               % show spaces adding particular underscores
  showstringspaces=false,         % underline spaces within strings
  showtabs=false,                 % show tabs within strings adding particular underscores
  morekeywords={var},
%  frame=single,                   % adds a frame around the code
  tabsize=2,                  % sets default tabsize to 2 spaces
  captionpos=none,                   % sets the caption-position to bottom
  breaklines=true,                % sets automatic line breaking
  breakatwhitespace=false,        % sets if automatic breaks should only happen at whitespace
  title=\lstname,                 % show the filename of files included with \lstinputlisting; also try caption instead of title
  escapeinside={(*}{*)},          % if you want to add a comment within your code
  keywordstyle=\ttfamily\bfseries,
% commentstyle=\color{Gray}
% stringstyle=\color{Green}
}

\begin{document}

\special{papersize=8.5in,11in}
\setlength{\pdfpageheight}{\paperheight}
\setlength{\pdfpagewidth}{\paperwidth}

\title{\name}

\authorinfo{Name1}
           {Affiliation1}
           {Email1}
\authorinfo{Name2\and Name3}
           {Affiliation2/3}
           {Email2/3}

\maketitle

\begin{abstract}
\end{abstract}

\keywords
intersecion types, inheritance
\section{Introduction}

-- Compare Scala:
-- merge[A,B] = new A with B

-- type IEval  = { eval :  Int }
-- type IPrint = { print : String }

-- F[\_]

We present a polymorphic calculus containing intersection types and records, and show
how this language can be used to solve various common tasks in functional
programming in a nicer way.

Intersection types provides a power mechanism for functional programming, in
particular for extensibility and allowing new forms of composition.

Prototype-based programming is one of the two major styles of object-oriented
programming, the other being class-based programming which is featured in
languages such as Java and C\#. It has gained increasing popularity recently
with the prominence of JavaScript in web applications. Prototype-based
programming supports highly dynamic behaviors at run time that are not possible
with traditional class-based programming. However, despite its flexibility,
prototype-based programming is often criticized over concerns of correctness and
safety. Furthermore, almost all prototype-based systems rely on the fact that
the language is dynamically typed and interpreted.

In summary, the contributions of this paper are:

\begin{itemize}

\item{elaboration typing rules which given a source expression with intersection
    types, typecheck and translate it into an ordinary F term. Prove a type
    preservation result: if a term $e$ has type $\tau$ in the source language,
    then the translated term $|e|$ is well-typed and has type $|\tau|$ in the
    target language.}

\item{present an algorithm for detecting incoherence which can be very important
    in practice.}

\item{explores the connection between intersection types and object algebra by
    showing various examples of encoding object algebra with intersection
    types.}

\end{itemize}
\section{A Taste of fi}

\begin{footnote}
  Change the examples later to something very simple.
\end{footnote}

This section provides the reader with the intuition of \name, while we postpone
the presentation of the details in later sections.

In short, \name generalizes \systemf by adding intersection polymorphism. \name terms
are elaborated into \systemF, a variant of System F. System F, or polymorphic
lambda calculus lays the foundation of functional programming languages such as
Haskell.

The type system of \name permits a subtyping relation naturally and enables
prototype-based inheritance. We will explore the usefulness of such a type
system in practice by showing various examples.

\subsection{Intersection Types}

The central addition to the type system of \systemf in \name is intersection types. What
is an intersection type? One classic view is from set-theoretic interpretation
of types: \lstinline{A \& B} stands for the intersection of the set of values of
\lstinline{A} and \lstinline{B}. The other view, adopted in this paper, regards
types as a kind of interface: a value of type \lstinline{A \& B} satisfies both
of the interfaces of \lstinline{A} and \lstinline{B}. For example,
\lstinline{eval : Int} is the interface that supports evaluation to integers,
while \lstinline{ eval : Int \& print : String } supports both evaluation and
pretty printing. Those interfaces are akin to interfaces in Java or traits in
Scala. But one key difference is that they are unnamed in \name.

Intersection types provide a simple mechanism for ad-hoc polymorphism, similar
to what type classes in Haskell achieve. The key constructs are the ``merge''
operator, denoted by ``\lstinline{,,}'', at the value level and the corresponding type
intersection operator, denoted by, ``\lstinline{\&}'' at the type level.

For example, we can define an (ad-hoc)-polymorphic \lstinline{show} function
that is able to convert integers and booleans to strings. In \name such function
can be given the type
\begin{lstlisting}
  (Int -> String) & (Bool -> String)
\end{lstlisting}
and be defined using the merge operator $ ,, $ as
\begin{lstlisting}
  let show = showInt ,, showBool
\end{lstlisting}
where \lstinline{showString} and \lstinline{showBool} are ordinary monomorphic
functions. Later suppose the integer \lstinline{1} is applied to the \lstinline{show} function,
the first component \lstinline{showInt} will be picked because the type of \lstinline{showInt}
is compatible with \lstinline{1} while \lstinline{showBool} is not.

% The merge construct in the original function is elaborated into a pair in the
% target language:

% \begin{verbatim}
% show = (showInt, showBool)
% \end{verbatim}

% In the target language where there is no intersection types, the application
% of the integer \texttt{1} to this function does not typecheck. However, we may
% rescue this situtation by inserting a coercion that extracts the first item
% out of this pair.

% Thus \texttt{show 1} in FI corresponds to \texttt{(fst show) 1} in F.

% While elaborating intersection types, this paper is the first that presents a
% type system that incorporates both parametric polymorphism and intersection
% polymorphism.

% Describe intersection types, encoding records with Intersecion types

% \lstinputlisting[linerange=-]{} % APPLY:linerange=MIXIN_LIB

\subsection{Encoding Records}

In addition to introduction of record literals using the usual notation, \name
support two more operations on records: record elimination and record update.

A record type of the form \lstinline{l : t} can be thought as a normal type \lstinline{t}
tagged by the label \lstinline{l}.

% A basic example

% \lstinputlisting[linerange=-]{} % APPLY:linerange=BASICS_ADD

\lstinline{e1} and \lstinline{e2} are two expressions that support both evaluation and pretty
printing and each has type \lstinline{eval : Int, print : String}. \lstinline{add} takes
two expressions and computes their sum. Note that in order to compute a sum,
\lstinline{add} only requires that the two expressions support evaluation and hence the
type of the parameter \lstinline{eval : Int}. As a result, the type of \lstinline{e1} and
\lstinline{e2} are not exactly the same with that of the parameters of \lstinline{add}. However,
under a structural type system, this program should typecheck anyway because the
arguments being passed has more information than required. In other words,
\lstinline{eval : Int, print : String} is a subtype of \lstinline{eval : Int}.

How is this subtyping relation derived? In \name, multi-field record types are
excluded from the type system because \lstinline{eval : Int, print : String} can
be encoded as \lstinline{eval : Int \& print : String}. And by one of
subtyping rules derives that \lstinline{eval : Int \& print : String} is a
subtype of \lstinline{eval : Int}.

% This example is elaborated into the following in \systemF.

% \lstinputlisting[linerange=-]{} % APPLY:linerange=BASICSELAB_ADD

\subsection{Parametric Polymorphism}

The presence of both parametric polymorphism and intersection is critical, as we
shall see in the next section, in solving modularity problems. Here is a code
snippet from the next section (The reader is not required to understand the
purpose of this code at this stage; just recognizing the two types of
polymorphism is enough.)
\begin{lstlisting}
type SubExpAlg E = (ExpAlg E) \& { sub : E -> E -> E };
let e2 E (f : SubExpAlg E) = f.sub (exp1 E f) (f.lit 2);
\end{lstlisting}
\lstinline{SubExpAlg} is a type synonym (a la Haskell) defined as the intersection of
\lstinline{ExpAlg E} and \lstinline{sub : E -> E -> E}, parametrized by a type parameter
\lstinline{E}. \lstinline{e2} exhibits parametric polymorphism as it takes a type argument
\lstinline{E}.

\section{Application} \label{sec:application}

Structural subtyping facilitates reuse~\cite{malayeri2008integrating}.
\bruno{orphan sentence!}

This section shows that the System $ F $ plus intersection types are enough for
encoding extensible designs, and even beat the designs in languages with a much
more sophisticated type system. In particular, \name has two main advantages
over existing languages:

\begin{enumerate}
\item It supports dynamic composition of intersecting values.
\item It supports contravariant parameter types in the subtyping relation.
\end{enumerate}

Various solutions have been proposed to deal with the extensibility problems and
many rely on heavyweight language features such as abstract methods and classes
in Java.

\bruno{I would like to see a story about Church Encodings in
  \name. Can you look at Pierce's papers and try to write something
  along those lines? That will be a good intro for object algebras and
visitors!}

These two features can be used to improve existing designs of modular programs.

% Introduce the expression problem

The expression problem refers to the difficulty of adding a new operations and a
new data variant without changing or duplicating existing code.

There has been recently a lightweight solution to the expression problem that
takes advantage of covariant return types in Java. We show that FI is able to
solve the expression problem in the same spirit. The
A)


% - Object/Fold Algebras. How to support extensibility in an easier way.

% See Datatypes a la Carte

% - Mixins

% - Lenses? Can intersection types help with lenses? Perhaps making the
% types more natural and easy to understand/use?

% - Embedded DSLs? Extensibility in DSLs? Composing multiple DSL interpretations?

% http://www.cs.ox.ac.uk/jeremy.gibbons/publications/embedding.pdf

\begin{minted}{scala}
trait Expr {
  def eval: Int
}

class Lit(n: Int) extends Expr {
  def eval: Int = n
}

class Add(n: Int) extends Expr {
  def eval: Int = e1.eval + e2.eval
}
\end{minted}

\bruno{You already talk about overloading in
  the previous section. Need to decide where to put the text!}

Dunfield~\cite{dunfield2014elaborating} notes that using merges as a mechanism
of overloading is not as powerful as type classes.

\subsection{Encoding Bounded Polymorphism}\bruno{IMPORTANT: 
Always capitalize the relevant words in a title!. This title should be
"Encoding Bounded Polymorphism''}

As \name extends System $ F $ with intersection types, $ F_{\subtype} $ extends
System $ F $ with bounded polymorphism. $ F_{\subtype} $~\cite{pierce2002types}
allows giving an upper bound to the type variable in type abstractions. The idea
of bounded universal quantification was discussed in the seminal paper by
Cardelli and Wegner ~\cite{cardelli1985understanding}. They show that bounded
quantifiers are useful because it is able to solve the ``loss of information''
problem.

In fact, the extension of System $ F $ in the other direction, i.e., with
intersection types, is able to address the same problem effectively. Suppose we
have the following definitions:
\begin{minted}{scala}
def user = { name = "George", admin = true }
def id(user: {name: String}) = user
\end{minted}
Under a structural type system, programmers would expect that passing the user
to the function is allowed. They are correct. Note that in the source language
multi-field records are just syntatic sugars for merges of single-field records.
Therefore, user is of a subtype of the parameter due to subtyping introduced by
intersection types. So far so good. But there is a problem: what if programmers
wants to access the \texttt{admin} field later, like:
\begin{minted}{scala}
(id user).admin
\end{minted}
They cannot do so as the above will not typecheck. After going through the
function, the user now has only the type
\begin{minted}{scala}
{name: String}
\end{minted}
This is rather undesired because it indeed has an \texttt{admin} field!

Bounded polymorphism enable the function to return the exact type of the
argument so that we have no problem in accessing the \texttt{admin} field later.
Consider the example below:
\begin{minted}{scala}
def id [A <: {name: String}] (user: A) = user
(id [{name: String, admin: Bool}] user).admin
\end{minted}

We do not have bounded polymorphism in the source language. But we can encode
that via intersection types:
\begin{minted}{scala}
def id [A] (user: A & {name: String}) = user in
(id [{admin: Bool}] user).admin
\end{minted}

Polymorphism plus intersecion types is as powerful as bounded polymorphism.

\cite{pierce1997intersection}\bruno{I don't think this has been
  shown. What we can say is: we can encode a form of bounded
  polymorphism with intersection types.}

% Multiple inheritance?
% Algebra -> P1,2
% Visitor -> P2

% Yanlin
% Mixin

% \begin{lstlisting}
% let merge A B (f : ExpAlg A) (g : ExpAlg B) = {
%   lit = \(x : Int). f.lit x ,, g.lit x,
%   add = \(x : A & B). \(y : A & B). f.add x y ,, g.add x y
% };
% \end{lstlisting}

\subsection{Object Algebras}

Object algebras provide an alternative to \emph{algebraic data types}
(ADT).\bruno{We are targeting an OO crowd. Mentioning algebraic
  datatypes is not going to be very useful there.}
 For example, the
following Haskell definition of the type of simple expressions
\begin{minted}{haskell}
data Exp where
  Lit :: Int -> Exp
  Add :: Exp -> Exp -> Exp
\end{minted}
can be expressed by the \emph{interface} of an object algebra of
simple expressions:
\begin{minted}{scala}
trait ExpAlg[E] {
  def lit(x: Int): E
  def add(e1: E, e2: E): E
}
\end{minted}
Similar to ADT, data constructors in object algebras are represented by functions such as
\lstinline{lit} and \lstinline{add} inside an interface \lstinline{ExpAlg}.
Different with ADT, the type of the expression itself is abtracted by a type
parameter \lstinline{E}.

which can be expressed similarly in \name as:
\begin{lstlisting}
type ExpAlg E = {
  lit : Int -> E,
  add : E -> E -> E
}
\end{lstlisting}

% Introduce Scala's intersection types

Scala supports intersection types via the \lstinline{with} keyword. The type
\lstinline{A with B} expresses the combined interface of \lstinline{A} and
\lstinline{B}. The idea is similar to
\begin{minted}{java}
interface AwithB extends A, B {}
\end{minted}
in Java.
\footnote{However, Java would require the \lstinline{A} and \lstinline{B} to be
  concrete types, whereas in Scala, there is no such restriction.}

The value level counterpart are functions of the type \lstinline
{A => B => A with B}. \footnote{FIXME}

Our type system is a simple extension of System $ F $; yet surprisingly, it
is able to solve the limitations of using object algebras in languages such as
Java and Scala. We will illustrate this point with an step-by-step of solving
the expression problem using a source language built on top of \name.

Oliveira noted that composition of object algebras can be cumbersome and
intersection types provides a solution to that problem.

We first define an interface that supports the evaluation operation:

% #include ../src/Algebra.sf  @base
\begin{lstlisting}
type IEval  = { eval : Int };
type ExpAlg E = { lit : Int -> E, add : E -> E -> E };
let evalAlg = {
  lit = \(x : Int). { eval = x },
  add = \(x : IEval). \(y : IEval). { eval = x.eval + y.eval }
};
\end{lstlisting}
% #end

The interface is just a type synonym \lstinline{IEval}. In \name, record
types are structural and hence any value that satisfies this interface is of
type \lstinline{IEval} or of a subtype of \lstinline{IEval}. \footnote{Should be
mentioned in S2.}

In the following, \lstinline{ExpAlg} is an object algebra interface of
expressions with literal and addition case. And \lstinline{evalAlg} is an object
algebra for evluation of those expressions, which has type \lstinline{ExpAlg Int}

% #include ../src/Algebra.sf  @variant
\begin{lstlisting}
type SubExpAlg E = (ExpAlg E) & { sub : E -> E -> E };
let subEvalAlg = evalAlg ,, { sub = \ (x : IEval). \ (y : IEval). { eval = x.eval - y.eval } };
\end{lstlisting}
% #end

Next, we define an interface that supports pretty printing.

% #include ../src/Algebra.sf  @operation
\begin{lstlisting}
type IPrint = { print : String };
let printAlg = {
  lit = \(x : Int). { print = x.toString() },
  add = \(x : IPrint). \(y : IPrint). { print = x.print.concat(" + ").concat(y.print) },
  sub = \(x : IPrint). \(y : IPrint). { print = x.print.concat(" - ").concat(y.print) }
};
\end{lstlisting}
% #end

Provided with the definitions above, we can then create values using the
appropriate algebras. For example:
defines two expressions.

The expressions are unusual in the sense that they are functions that take an
extra argument \lstinline{f}, the object algebras, and use the data constructors
provided by the object algebra (factory) \lstinline{f} such as \lstinline{lit},
\lstinline{add} and \lstinline{sub} to create values. Moreover, The algebras
themselves are abstracted over the allowed operations such as evaluation and
pretty printing by requiring the expression functions to take an extra argument
\lstinline{E}.

% #include ../src/Algebra.sf  @merge
\begin{lstlisting}
let merge A B (f : ExpAlg A) (g : ExpAlg B) = {
  lit = \(x : Int). f.lit x ,, g.lit x,
  add = \(x : A & B). \(y : A & B).
          f.add x y ,, g.add x y
};
\end{lstlisting}
% #end

If we would like to have an expression that supports both evaluation and pretty
printing, we will need a mechanism to combine the evaluation and printing
algebras. Intersection types allows such composition: the \lstinline{merge}
function, which takes two expression algebras to create a combined algebra. It
does so by constructing a new expression algebra, a record whose each field is a
function that delegates the input to the two algebras taken.

% #include ../src/Algebra.sf  @usage
\begin{lstlisting}
let newAlg = merge IEval IPrint subEvalAlg printAlg in
let o1 = e1 (IEval & IPrint) newAlg in
o1.print
\end{lstlisting}
% #end

\bruno{Don't start a sentence with \lstinline{o1}.}
\lstinline{o1} is a single object created that supports both evaluation and
printing, thus achieving full feature-oriented programming.

\subsection{Visitors}

Constructing instances seems clumsy!

The visitor pattern allows adding new operations to existing structures without
modifying those structures. The type of expressions are defined as follows:

\begin{minted}{scala}
trait Exp[A] {
  def accept(f: ExpAlg[A]): A
}

trait SubExp[A] extends Exp[A] {
  override def accept(f: SubExpAlg[A]): A
}
\end{minted}

The body of \lstinline{Exp} and \lstinline{SubExp} are almost the same: they
both contain an \lstinline{accept} method that takes an algebra \lstinline{f}
and returns a value of the carrier type \lstinline{A}. The only difference is at
\lstinline{f} --- \lstinline{SubExpAlg[A]} is a subtype of
\lstinline{ExpAlg[A]}. Since \lstinline{f} appear in parameter position of
\lstinline{accept} and function parameters are contravariant, naturally we would
hope that \lstinline{SubExp[A]} is a supertype of \lstinline{Exp[A]}. However,
such subtyping relation does not fit well in Scala because inheritance implies
subtyping in such languages \footnote{It is still possible to encode
  contravariant parameter types in Scala but doing so would require some
  technique.\bruno{what technique?}}. As \lstinline{SubExp[A]} extends \lstinline{Exp[A]}, the former
becomes a subtype of the latter.

Such limitation does not exist in \name. For example, we can define the similar interfaces \lstinline{Exp} and \lstinline{SubExp}:
\begin{lstlisting}
type Exp    A = { accept: forall A. ExpAlg A -> A };
type SubExp A = { accept: forall A. SubExpAlg A -> A };
\end{lstlisting}
Then by the typing judgment it holds that \lstinline{SubExp} is a supertype of
\lstinline{Exp}. This relation gives desired results. To give a concrete example:

A is called is the \emph{interpretation}. It works for any interpretation you want.

First we define two data constructors for simple expressions:
\begin{lstlisting}
let lit (n : Int): Exp A = {
  accept = /\A. \(f : ExpAlg A). f.lit n
};

let add (e1 : Exp) (e2 : Exp): Exp A = {
  accept = /\A. \(f : ExpAlg A).
             f.add (e1.accept A f) (e2.accept A f)
};
\end{lstlisting}

Suppose later we decide to augment the expressions with subtraction:
\begin{lstlisting}
let sub (e1 : SubExp) (e2 : SubExp): SubExp A =
  { accept = /\A. \(f : SubExpAlg A).
               f.sub (e1.accept A f) (e2.accept A f) };
\end{lstlisting}

One big benefit of using the visitor pattern is that programmers is able to
write in the same way that would do in Haskell.
For example, \lstinline{e2 = sub (lit 2) (lit 3)} defines an expression.

Another important property that does not exist in Scala is that programmer is
able to pass \lstinline{lit 2}, which is of type \lstinline{Exp A}, to
\lstinline{sub}, which expects a \lstinline{SubExp A} because of the subtyping
relation we have. After all, it is known statically that \lstinline{lit 2} can
be passed into \lstinline{sub} and nothing will go
wrong.\bruno{Subtyping needs to be much more emphasized! See Modular
  Visitor Components! }

% \subsection{Yanlin stuff}
% \bruno{This can be dropped.}

% This subsection presents yet another lightweight solution to the Expression
% Problem, inspired by the recent work by Wang. It has been shown that
% contravariant return types allows refinement of the types of extended
% expressions.

% First, we define the type of expressions that support evaluation and implement
% two constructors:
% \begin{lstlisting}
% type Exp = { eval: Int }
% let lit (n: Int) = { eval = n }
% let add (e1: Exp) (e2: Exp)
%   = { eval = e1.eval + e2.eval }
% \end{lstlisting}

% If we would like to add a new operation, say pretty printing, it is nothing more
% than refining the original \lstinline{Exp} interface by \emph{intersecting} the
% original type with the new \lstinline{print} interface using the \lstinline{&}
% primitive and \emph{merging} the original data constructors using the \lstinline{,,}
% primitive.
% \begin{lstlisting}
% type ExpExt = Exp & { print: String }
% let litExt (n: Int) = lit n ,, { print = n.toString() }
% let addExt (e1: ExpExt) (e2: ExpExt)
%   = add e1 e2 ,,
%     { print = e1.print.concat(" + ").concat(e2.print) }
% \end{lstlisting}

% Now we can construct expressions using the constructors defined above:
% \begin{lstlisting}
% let e1: ExpExt = addExt (litExt 2) (litExt 3)
% let e2: Exp = add (lit 2) (lit 4)
% \end{lstlisting}
% \lstinline{e1} is an expression capable of both evaluation and printing, while
% \lstinline{e2} supports evaluation only.

% We can also add a new variant to our expression:
% \begin{lstlisting}
% let sub (e1: Exp) (e2: Exp) = { eval = e1.eval - e2.eval }
% let subExt (e1: ExpExt) (e2: ExpExt)
%   = sub e1 e2 ,, { print = e1.print.concat(" - ").concat(e2.print) }
% \end{lstlisting}

% Finally we are able to manipulate our expressions with the power of both
% subtraction and pretty printing.
% \begin{lstlisting}
% (subExt e1 e1).print
% \end{lstlisting}

\subsection{Mixins}

Mixins are useful programming technique wildly adopted in dynamic programming
languages such as JavaScript and Ruby. But obviously it is the programmers'
responsbility to make sure that the mixin does not try to access methods or
fields that are not present in the base class.

In Haskell, one is also able to write programs in mixin style using records.
However, this approach has a serious drawback: since there is no subtyping in
Haskell, it is not possible to refine the mixin by adding more fields to the
records. This means that the type of the family of the mixins has to be
determined upfront, which undermines extensibility.

\name is able to overcome both of the problems: it allows composing mixins
that (1) extends the base behavior, (2) while ensuring type safety.

The figure defines a mini mixin library. The apostrophe in front of types
denotes call-by-name arguments similar to the \lstinline{=>} notation in the
Scala language.

\begin{lstlisting}
type Mixin S = 'S -> 'S -> S;
let zero S (super : 'S) (this : 'S) : S = super;
let rec mixin S (f : Mixin S) : S
  = let m = mixin S in f (\ (_ : Unit). m f) (\ (_ : Unit). m f);
let extends S (f : Mixin S) (g : Mixin S) : Mixin S
  = \ (super : 'S). \ (this  : 'S). f (\ (d : Unit). g super this) this;
\end{lstlisting}

We define a factorial function in mixin style and make a \lstinline{noisy} mixin
that prints ``Hello'' and delegates to its superclass. Then the two functions
are composed using the \lstinline{mixin} and \lstinline{extends} combinators.
The result is the \lstinline{noisyFact} function that prints ``Hello'' every
time it is called and computes factorial.
\begin{lstlisting}
let fact (super : 'Int -> Int) (this : 'Int -> Int) : Int -> Int
  = \ (n : Int). if n == 0 then 1 else n * this (n - 1)
let noisy (super : 'Int -> Int) (this : 'Int -> Int) : Int -> Int
  = \ (n : Int). { println("Hello"); super n }
let noisyFact = mixin (Int -> Int) (extends (Int -> Int) foolish fact)
noisy 5
\end{lstlisting}

% \subsection{Composing Mixins and Object Algebras}

\section{The \name calculus}

\footnote{Joshua Dunfield}

This section formalizes the syntax, subtyping, and typing of \name. In the next
section, we will go through the type-directed translation from \name to System
F.

% Note the semantics of this language is not defined formally, instead, by a
% translation into the target language, System F.

\subsection{Syntax}

The syntax of the \name calculus extends System F by adding the two features:
intersection types and records. The formalization includes only single records
because single record types as the multi-records can be desugared into the merge
of multiple single records.

\[
\begin{array}{l}
  \begin{array}{llrl}
    \text{Types} 
    & \tau & \Coloneqq & \alpha \mid \highlight{\top} \mid \tau_1 \to \tau_2 \mid \for \alpha \tau \\
    &      & \mid      & \highlight{\tau_1 \andop \tau_2} \mid \highlight{\recty l \tau} \\
    \text{Expressions} 
    & e & \Coloneqq & x \mid \highlight{\top} \mid \lam x \tau e \mid \app e e \mid \blam \alpha e \mid \tapp e \tau \\
    &   & \mid      & \highlight {e \mergeop e} \mid \highlight {\reccon l e} \mid
                      \highlight {e.l} \mid \highlight{e \restrictop l} \\
    \text{Contexts} 
    & \gamma & \Coloneqq & \epsilon \mid \gamma, \alpha \mid \gamma, x \hast \tau \\
    \text{Labels} & l
  \end{array} 
\end{array}
\]


% Types
Types $ t $ have five constructs. The first three are standard (present in
System F): type variable $ \alpha $, function types $ t \to t $, and type
abstraction $ \forall \alpha. t $; while the last two, intersection types
$ t \with t $ and record types $ \recordtype{l}{t} $, are novel in \Name. In
record types, $ l $ is the label and $ t $ the type.

% Expressions - standard ones
First five constructs of expressions are also standard: variables $ x $ and two
abstraction-elimination pairs. $ \abs{\rel{x}{t}}{e} $ abstracts expression
$ e $ over values of type $ t $ and is eliminated by application $ \app{e}{e} $;
$ \Abs{\alpha}{e} $ abstracts expression $ e $ over types and is eliminated by
type application $ \app{e}{t} $.

% Expressions - new ones
The last four constructs are novel. $ e \dcomma e $ is the \emph{merge} of two
terms. $ \recordintro{l}{e} $ introduces a record literal having $ l $ as the
label for field containing expression $ e $ . $ e.l $ access the field with
label $ l $ in $ e $. Finally, $ \recordupdate{e}{l}{e} $ updates the field
labelled $ l $ in expression $ e $. For simplicity, we omit other constructs in
order to focus on the essence of the calculus. For example, fixpoints can be
added in standard ways.

% Fields
The field $ F $ is non-standard and introduced to deal with records. It is an
associative list. Each item is a pair whose first item is either empty or a
label and the second the types.

% The most central construct of our language is ...

% Dunfield has described a language that includes a ``top'' type but it does not appear in our language. Our work differs from Dunfield in that ...

% Remark. The operational semantics of FI is not presented in this paper. However,

\subsection{Subtyping}

\begin{figure*}
\framebox{\( \ty \subtype \ty \yieldsnothing C \)}

\infax[SVar]
{\alpha \subtype \alpha \yieldsnothing {\abs {\rel x {\image \alpha}} x}}

\infrule[SFun]
{\ty_3 \subtype \ty_1 \yieldsnothing {C_1} \andalso \ty_2 \subtype \ty_4 \yieldsnothing {C_2}}
{\ty_1 \to \ty_2 \subtype \ty_3 \to \ty_4
  \yieldsnothing
    {\abs {\rel f {\image {\ty_1 \to \ty_2}}}
      {\abs {\rel x {\image {\ty_3}}}
        {\app {C_2} {(\app {f} {(\app {C_1} {x})})}}}}}

\infrule[SForall]
{\ty_1 \subtype \subst {\alpha_2} {\alpha_1} \ty_2 \yieldsnothing C}
{\forall \alpha_1. \ty_1 \subtype \forall \alpha_2. \ty_2
  \yieldsnothing
    {\abs {\rel{f} {\image {\forall \alpha. \ty_1}}}
      {\Abs \alpha {\app C {(\app f \alpha)}}}}}

\infrule[SAnd1]
{\ty_1 \subtype \ty_3 \yieldsnothing C}
{\ty_1 \with \ty_2 \subtype \ty_3
  \yieldsnothing
    {\abs {\rel x {\image {\ty_1 \with \ty_2}}}
      {\app C {(\app \fst x)}}}}

\infrule[SAnd2]
{\ty_2 \subtype \ty_3 \yieldsnothing C}
{\ty_1 \with \ty_2 \subtype \ty_3
  \yieldsnothing
    {\abs {\rel x {\image {\ty_1 \with \ty_2}}}
      {\app C {(\app \snd x)}}}}

\infrule[SAnd3]
{\ty_1 \subtype \ty_2 \yieldsnothing {C_1} \andalso \ty_1 \subtype \ty_3 \yieldsnothing {C_2}}
{\ty_1 \subtype \ty_2 \with \ty_3
  \yieldsnothing
    {\abs {\rel x {\image {\ty_1}}}
      {\tupled {\app {C_1} x, \app {C_2} x}}}}

\infrule[SRcd]
{\ty_1 \subtype \ty_2 \yieldsnothing C}
{\recordtype l {\ty_1} \subtype \recordtype l {\ty_2}
  \yieldsnothing
    {\abs {\rel x {\image {\recordtype l {\ty_1}}}} {\app C x}}}

\caption{Subtyping}
\end{figure*}

Thanks to intersection types, we have natural subtyping relations among types.
For example, $ Int \with Bool $ should be a subtype of $ Int $, since the former
can be viewed as either $ Int $ or $ Bool $. The subtyping rules are standard
except for three points listed below:
\begin{enumerate}
\item $ t_1 \with t_2 $ is a subtype of $ t_3 $, if \emph{either} $ t_1 $ or
  $ t_2 $ are subtypes of $ t_3 $,

\item $ t_1 $ is a subtype of $ t_2 \with t_3 $, if $ t_1 $ is a subtype of
  both $ t_2 $ and $ t_3 $.

\item $ \recordtype{l_1}{t_1} $ is a subtype of $ \recordtype{l_2}{t_2} $, if
  $ l_1 $ and $ l_2 $ are identical and $ t_1 $ is a subtype of $ t_2 $.
\end{enumerate}
The first point is captured by two rules \texttt{S-And-1} and \texttt{S-And-2},
whereas the second point by \texttt{S-And-3}. Note that the last point means
that record types are covariant in the type of the fields.

\subsection{Typing}

\begin{figure*}
\framebox{\(\Gamma \turns {t}\)}

\infrule[WF-Var]
{\alpha \in \Gamma}
{\Gamma \turns \alpha}

\infrule[WF-Fun]
{\Gamma \turns \ty_1 \andalso \Gamma \turns \ty_2}
{\Gamma \turns \ty_1 \to \ty_2}

\infrule[WF-Forall]
{\Gamma, \alpha \turns \ty}
{\Gamma \turns \forall \alpha. \ty}

\infrule[WF-And]
{\Gamma \turns \ty_1 \andalso \Gamma \turns \ty_2}
{\Gamma \turns \ty_1 \intersects \ty_2}

\infrule[WF-Rcd]
{\Gamma \turns \ty}
{\Gamma \turns \recordtype l \ty}

\caption{Well-formedness}
\end{figure*}

\begin{figure*}
\framebox{\( \Gamma \turns e : \ty \yieldsnothing E \)}

\infrule[Var]
{(x,\ty) \in \Gamma}
{\Gamma \turns x : \ty \yieldsnothing x}

\infrule[Abs]
{\Gamma, \rel x \ty \turns e : \ty_1 \yieldsnothing E \andalso
 \Gamma \turns \ty}
{\Gamma \turns \abs {\rel x \ty} e : \ty \to \ty_1 \yieldsnothing {\abs {\rel x {\image \ty}} E}}

\infrule[TAbs]
{\Gamma, \alpha \turns e : \ty \yieldsnothing E}
{\Gamma \turns \Abs \alpha e : \forall \alpha. \ty \yieldsnothing {\Abs \alpha E}}

\infrule[App]
{\Gamma \turns e_1 : \ty_1 \to \ty_2 \yieldsnothing {E_1} \\
 \Gamma \turns e_2 : \ty_3 \yieldsnothing {E_2} \andalso
 \ty_3 \subtype \ty_1 \yieldsnothing C}
{\Gamma \turns \app {e_1} {e_2} : \ty_2 \yieldsnothing {\app {E_1} {(\app C E_2)}}}

\infrule[TApp]
{\Gamma \turns e : \forall \alpha. \ty_1 \yieldsnothing E \andalso
 \Gamma \turns \ty}
{\Gamma \turns \app e \ty : \subst \alpha \ty \ty_1 \yieldsnothing {\app E {\image \ty}}}

\infrule[Merge]
{\Gamma \turns e_1 : \ty_1 \yieldsnothing {E_1} \andalso
 \Gamma \turns e_2 : \ty_2 \yieldsnothing {E_2}}
{\Gamma \turns e_1 \dcomma e_2 : \ty_1 \intersects \ty_2 \yieldsnothing {\tupled {E_1, E_2}}}

\infrule[RecCon]
{\Gamma \turns e : \ty \yieldsnothing E}
{\Gamma \turns \reccon l e : \recty l \ty \yieldsnothing E}

\infrule[RecProj]
{\Gamma \turns e : \ty \yieldsnothing E \andalso
 \Gamma \turnsget (\ty; l) : \ty_1 \yieldsnothing C}
{\Gamma \turns e.l : \ty_1 \yieldsnothing {\app C E}}

\infrule[RecUpd]
{\Gamma \turns e : \ty \yieldsnothing E \andalso
 \Gamma \turns e_1 : \ty_1 \yieldsnothing {E_1} \\
 \turnsput (t; l; e_1 : \ty_1 \yieldsnothing {E_1}) : (\ty_2, \ty_3) \yieldsnothing C \andalso
 \ty_1 \subtype \ty_2}
{\Gamma \turns \recupd e l {e_1} : \ty_3 \yieldsnothing {\app C E}}

\framebox{\( \vdash_{get} (t; l) : t \yieldsnothing C \)}

\infax[GetBase]
{\vdash_{get} (\recordtype l t; l) : t
  \yieldsnothing {\abs {\rel x {\imageof {\recordtype l t}}} x}}

\infrule[GetLeft]
{\vdash_{get} (t_1; l) : t \yieldsnothing C}
{\vdash_{get} (t_1 \with t_2; l) : t
  \yieldsnothing {\abs {\rel x {\imageof {t_1 \with t_2}}} {C (\app \fst x)}}}

\infrule[GetRight]
{\vdash_{get} (t_2; l) : t \yieldsnothing C}
{\vdash_{get} (t_1 \with t_2; l) : t
  \yieldsnothing {\abs {\rel x {\imageof {t_1 \with t_2}}} {C (\app \snd x)}}}

\begin{mathpar}
\framebox{\( \turnsput (\tau; l; e : \tau \yieldsnothing E) : (\tau, \tau) \yieldsnothing C \)}

\inferrule* [right=Put]
{ }
{\turnsput (\RecTy l \tau; l; e : \tau_1) : (\tau, \RecTy l {\tau_1})}

\inferrule* [right=Put1]
{\turnsput (\tau_1; l; e : \tau) : (\tau_3, \tau_4)}
{\turnsput (\tau_1 \Intersect \tau_2; l; e : \tau) : (\tau_3, \tau_4 \Intersect \tau_2)}

\inferrule* [right=Put2]
{\turnsput (\tau_2; l; e : \tau) : (\tau_3, \tau_4)}
{\turnsput (\tau_1 \Intersect \tau_2; l; e : \tau) : (\tau_3, \tau_1 \Intersect \tau_4)}
\end{mathpar}
\caption{Typing}
\end{figure*}

The typing judgment for \name is of the form: $ \Gamma \vdash e : t $. This
judgment uses the context $ \Gamma $. The typing rules for our core languages
are mostly standard ones for System F. In particular we introduce a
\texttt{T-Merge} rule that applies to \emph{merge} constructs.

The last two rules make use of the $ \texttt{fields} $ function just to make
sure that the field being accessed (\texttt{T-RcdElim}) or updated
(\texttt{T-RcdUpd}) actually exists. The function is defined recursively, in
Haskell pseudocode, as:
\[ \begin{array}{rll}
  \fields{\alpha} & = & \rel{\cdot}{\alpha} \\
  \fields{t_1 \to t_2} & = & \rel{\cdot}{t_1 \to t_2} \\
  \fields{\forall \alpha. t} & = & \rel{\cdot}{\forall \alpha. t} \\
  \fields{t_1 \with t_2} & = & \fields{t_1} \dplus \fields{t_2} \\
  \fields{\recordtype{l}{t}} & = & \rel{l}{t}
\end{array} \]
where $ \cdot $ means empty list, $ \dplus $ list concatenation, and $ : $ is an
infix operator that prepend the first argument to the second. The function
returns an associative list whose domain is field labels and range types.

\textit{dom} reads: ``the domain of''. $ F(l) $ means the result of lookup for
$ l $ inside the associative list $ F $. The order of lookup can be either from
left to right or from right to left but has to be consistent inside one
implementation. We prefer the order from the right to the left because it make
possible record overriding. For example,
$ (\recordintro{count}{1} ,, \recordintro{count}{2}).count $ will evaluate to
$ 2 $ in this case.
\section{Type-directed Translation to System F}

In this section we define the semantics of \name by means of a type-directed
translation to System F. This translation removes the labels of records and
turns intersections into products, much like Dunfield's elaboration. But our
translation also deals with parametric polymorphism and records.

\subsection{Informal Discussion}

To help the reader have a high-level understanding of how the translation
works, in this subsection we present the translation informally. Take the \name
expression for example:

\begin{lstlisting}
{ eval = 4, print = ``4'' }.eval
\end{lstlisting}

First, multi-field record literals are desugared into merges of single-field
record literals. Therefore $ \{ eval = 4, print = ``4'' \} $ becomes
$ \{ eval = 4 \} ,, \{ print = ``4'' \} $. Merges of two values are translated
into just a pair of them by \texttt{TrMerge} and single-field record literals lose their field
labels by \texttt{TrRcdIntro}. Hence $ \{ eval = 4 \} ,, \{ print = ``4'' \} $
becomes $ (4, ``4'') $.

Finally, $ e1 $ and $ e2 $ are both coerced by a projection function
$ \\(x:(Int,String)). \texttt{fst} x $.

\subsection{Target Language}

The target language is System F extended with pairs. The syntax and typing is
completely standard. The syntax of Systm F is as follows:

\[
\begin{array}{llrl}
  \text{Types}       & T    & \Coloneqq & \alpha \mid () \mid {T_1} \to {T_2} \mid \for \alpha T \mid \pair {T_1} {T_2} \\
  \text{Terms} & E, C & \Coloneqq & x \mid () \mid \lam x T E \mid \app {E_1} {E_2} \mid \blam \alpha E \\ 
                     &      & \mid      & \tapp E T \mid \pair {E_1} {E_2} \mid \proj k E \\
  \text{Contexts} & \Gamma & \Coloneqq & \epsilon \mid \Gamma, \alpha \mid \Gamma, x \hast T \\
\end{array}
\]


The dynamic semantics of System F can be found in ...

\begin{lemma} \label{type-coerce}
  If $$ \Gamma \vdash \tau_1 <: \tau_2 \yields{C} $$
  then $$ |\Gamma| \vdash C : |\tau_1| \to |\tau_2| $$
\end{lemma}

The main translation judgment is $ \Gamma \vdash e : \tau \hookrightarrow E $ which
states that with respect to the environment $ \Gamma $, the \name expression
$ e $ is of a \name type $ \tau $ and its translation is a System F expression $ E $.

We also define the type translation function $ | \cdot | $ from \name types
$ \tau $ to System F types $ T $.
\framebox{$ \image \ty = T $}

\[
\begin{array}{rcl}
  \image \alpha                     & = & \alpha \\
  \image {\ty_1} \to \image {\ty_2} & = & \image {\ty_1} \to \image {\ty_2} \\
  \image {\Forall \alpha \ty}      & = & \Forall \alpha \image \ty \\
  \image {\ty_1 \Intersect \ty_2}   & = & \Pair {\image {\ty_1}} {\image {\ty_2}} \\
  \image {\RecTy l \ty}            & = & \image t
\end{array}
\]
The first three rules of the translation is standard. For the last two, the
intersection of two types are translated into a product of them, and the label
of record types are erased.

The translation consists of four sets of rules, which are explained below:

\subsection{Subtyping (Coercion)}

\framebox{\( t \subtype t \yields C \)}

\infax[SVar]
{\alpha \subtype \alpha \yields {\abs {\rel x {\imageof \alpha}} x}}

\infrule[SFun]
{t_3 \subtype t_1 \yields {C_1} \andalso t_2 \subtype t_4 \yields {C_2}}
{t_1 \to t_2 \subtype t_3 \to t_4
  \yields
    {\abs {\rel f {\imageof {t_1 \to t_2}}}
      {\abs {\rel x {\imageof {t_3}}}
        {\app {C_2} {(\app{f} {(\app{C_1} {x})})}}}}}

\infrule[SForall]
{t_1 \subtype \subst {\alpha_2} {\alpha_1} t_2 \yields{C}}
{\forall \alpha_1. t_1 \subtype \forall \alpha_2. t_2
  \yields
    {\abs{\rel{f} {\imageof {\forall \alpha. t_1}}}
      {\Abs \alpha {\app C {(\app f \alpha)}}}}}

\infrule[SAnd1]
{t_1 \subtype t_3 \yields C}
{t_1 \with t_2 \subtype t_3
  \yields
    {\abs{\rel x {\imageof {t_1 \with t_2}}}
      {\app C {(\proj 1 x)}}}}

\infrule[SAnd2]
{t_2 \subtype t_3 \yields C}
{t_1 \with t_2 \subtype t_3
  \yields
    {\abs {\rel x {\imageof {t_1 \with t_2}}}
      {\app C {(\proj 2 x)}}}}

\infrule[SAnd3]
{t_1 \subtype t_2 \yields {C_1} \andalso t_1 \subtype t_3 \yields {C_2}}
{t_1 \subtype t_2 \with t_3
  \yields
    {\abs {\rel x {\imageof {t_1}}}
      {\tupled {\app {C_1} x, \app {C_2} x}}}}

\infrule[SRcd]
{t_1 \subtype t_2 \yields C}
{\recordtype l {t_1} \subtype \recordtype l {t_2}
  \yields
    {\abs {\rel x {\imageof {\recordtype l {t_1}}}} {\tupled {\app C {(\proj 1 x)}}}}}


The coercion judgment $ \Gamma \vdash \tau_1 \subtype \tau_2 \yields{C} $
extends the subtyping judgment with a coercion on the right hand side of
$ \hookrightarrow $. A coercion $ C $ is an expression in the target language
and has type $ \tau_1 \to \tau_2 $, as proved by Lemma \ref{type-coerce}. It is
read ``In the environment $ \Gamma $, $ \tau_1 $ is a subtype of $ \tau_2 $; and
if any expression $ e $ has a type $ t_1 $ that is a subtype of the type of
$ t_2 $, the elaborated $ e $, when applied to the corresponding coercion $ C $,
has exactly type $ |t_2| $''. For example,
$\Gamma \vdash Int \& Bool <: Bool \yields{fst} $, where $ fst $ is the
projection of a tuple on the first element. The coercion judgment is only used
in the \texttt{TrApp} case. As \texttt{SFun} supports contravariant parameter
type and covariant return type, the coercion of the parameter types and that of
the return types are used to create a coercion for the function type.
\texttt{SAnd1}, \texttt{SAnd2}, and \texttt{SAnd3} deal with intersection types.
The first two are complementary to each other. Take \texttt{SAnd1} for example,
if we know $ t_1 $ is a subtype of $ t_3 $ and $ C $ is a coercion from $ t_1 $
to $ t_3 $, then we can conclude that $ t_1 \with t_2 $ is also a subtype of
$ t_3 $ and the new coercion is a function that takes a value $ x $ of type
$ t_1 \with t_2 $, project $ x $ on the first item, and apply $ C $ to it.
\texttt{SAnd3} uses both of two coercions and constructs a pair.

\subsection{Typing (Translation)}

In this subsection we now present formally the translation rules that convert
\name expressions into System F ones. This set of rules essentially extends
those in the previous section with the light-blue part for the translation.

\framebox{\( \Gamma \vdash e : t \yields E \)}

\infrule[TrVar]
{(x,t) \in \Gamma}
{\Gamma \vdash x : t \yields x}

\infrule[TrAbs]
{\Gamma, \rel x t \vdash e : t_1 \yields E \andalso
 \Gamma \vdash_{F} t}
{\Gamma \vdash \abs {\rel x t} e : t \to t_1 \yields {\abs {\rel x {\imageof t}} E}}

\infrule[TrTAbs]
{\Gamma, \alpha \vdash e : t \yields E}
{\Gamma \vdash \Abs \alpha e : \forall \alpha. t \yields {\Abs \alpha E}}

\infrule[TrApp]
{\Gamma \vdash e_1 : t_1 \to t_2 \yields {E_1} \\
 \Gamma \vdash e_2 : t_3 \yields {E_2} \\
 t_3 \subtype t_1 \yields C}
{\Gamma \vdash \app {e_1} {e_2} : t_2 \yields {\app E_1 {(\app C E_2)}}}

\infrule[TrTApp]
{\Gamma \vdash e : \forall \alpha. t_1 \yields E \andalso
 \Gamma \vdash_{F} t}
{\Gamma \vdash \app e t : \subst \alpha t t_1 \yields {\app E {\imageof t}}}

\infrule[TrMerge]
{\Gamma \vdash e_1 : t_1 \yields {E_1} \\
 \Gamma \vdash e_2 : t_2 \yields {E_2}}
{\Gamma \vdash e_1 \dcomma e_2 : t_1 \with t_2 \yields {\tupled {E_1, E_2}}}

\infrule[TrRcdIntro]
{\Gamma \vdash e : t \yields E}
{\Gamma \vdash \recordintro l e : \recordtype l t \yields E}

\infrule[TrRcdElim]
{\Gamma \vdash e : t \yields E \andalso
 \Gamma \vdash_{get} (t; l) : t_1 \yields C}
{\Gamma \vdash e.l : t_1 \yields {\app C E}}

\infrule[TrRcdUpd]
{\Gamma \vdash e : t \yields E \andalso
 \Gamma \vdash e_1 : t_1 \yields {E_1} \\
 \vdash_{put} (t; l; e_1 : t_1 \yields {E_1}) : (t_1', t') \yields C \andalso
 t_1 \subtype t_1'}
{\Gamma \vdash \recordupdate e l {e_1} : t' \yields {\app C E}}

\framebox{\( \turnsget (\ty_1; l) : \ty_2 \yields C \)}

The field with label $ l $ inside $ \ty_1 $ is of type $ \ty_2 $.

\infax[GetBase]
{\turnsget (\recty l \ty; l) : \ty
  \yields {\idmono {\image {\recty l \ty}}}}

\infrule[GetLeft]
{\turnsget (\ty_1; l) : \ty \yields C}
{\turnsget (\ty_1 \intersects \ty_2; l) : \ty
  \yields {\absty x {\image {\ty_1 \intersects \ty_2}} {C (\app \fst x)}}}

\infrule[GetRight]
{\turnsget (\ty_2; l) : \ty \yields C}
{\turnsget (\ty_1 \intersects \ty_2; l) : \ty
  \yields {\absty x {\image {\ty_1 \intersects \ty_2}} {C (\app \snd x)}}}

\framebox{\( \vdash_{put} (t; l; e : t \yields E) : (t, t) \yields C \)}

\infax[PutBase]
{\vdash_{put} (\recordtype l t; l; e : t' \yields E) : (t, \recordtype l {t'})
  \yields {\abs {\rel x {\image {\recordtype l t}}} E}}

\infrule[PutLeft]
{\vdash_{put} (t_1; l; e : t \yields E) : (t', t_1') \yields C}
{\vdash_{put} (t_1 \with t_2; l; e : t \yields E) : (t', t_1' \with t_2)
  \yields {\abs {\rel x {\image {t_1 \with t_2}}} {C (\fst ~ x)}}}

\infrule[PutRight]
{\vdash_{put} (t_2; l; e : t \yields E) : (t', t_2') \yields C}
{\vdash_{put} (t_1 \with t_2; l; e : t \yields E) : (t', t_1' \with t_2)
  \yields {\abs {\rel x {\image {t_1 \with t_2}}} {C (\fst ~ x)}}}


\begin{itemize}

\item{\bf Coercion}

  Explained in the previous subsection.

\item{\bf Elaboration}

  The elaboration judgment $ \Gamma \vdash e : \tau \yields{E} $ extends the
  typing judgment with an elaborated expression on the right hand side of
  $ \hookrightarrow $. It is also standard, except for the case of
  \texttt{TrApp}, in which a coercion from the inferred type of the argument,
  $ e_2 $ , to the expected type of the parameter, $ t_1 $, is inserted before
  the argument; \texttt{TrMerge} translates merges into pairs.
  \texttt{TrRcdIntro} uses the same System F expression $ E $ for $ e $ as for
  $ \{ l = e \} $. And in \texttt{TrRcdEim} and \texttt{TrRcdUpd} the coercions
  generated by the ``get'' and ``put'' rules will be used to coece the main
  \name expression. The two set of rules are explained below.

\item{\bf ``get'' rules}

  The ``get'' judgment deals spefically with record elimination and yields a
  coercion can be thought as a field accessor. For example:

  $ \Gamma \vdash_{get} (\{ eval : Int \}, eval) : \{ eval : Int \} \yields {\abs {\rel x Int} x} $

  The lambda is the field accessor and when applied to a translated expression
  of type $ \{ eval : Int \}$, it is able to give the desired field.
  \texttt{GetBase} is the base case: the type of the field labelled $ l $ in a $
  \recordtype l t $ is just $ t $ and the coercion is an identity function
  specialized to type $ \imageof {\recordtype l t} $
  \texttt{GetLeft} and \texttt{GetRight} are complementary to each other.

\item{\bf ``put'' rules}

  The ``put'' judgment deals spefically with record update can be thought as
  producing a field updater. Compared to the ``get'' rules, the ``put'' rules
  take an extra input $ e $, which is the desired expression to replace the
  field lablled $ l $ in values of type $ t $. \texttt{PutBase} is the base
  case. This rule allows refinement of record fields in the sense that the type
  of $ e $ can be a subtype of the type of the field labelled by $ l $. The
  resulting type is $ \recordtype l {t'} $ and the generated coercion is a
  constant function that always returns $ E $. \texttt{PutLeft} and
  \texttt{PutRight} are complementary to each other: the idea is exactly the
  same as \texttt{GetLeft} and \texttt{GetRight} except that the refined type
  $ t_1' $ and $ t_2' $ is used.

\end{itemize}

\subsection{Meta-theory}

\begin{lemma}{Subtyping is reflexive} \label{sub-refl}
Given a type $ \tau $, $ \tau \subtype \tau $.
\end{lemma}

\begin{lemma}{Subtyping is transitive} \label{sub-trans}
If $ \tau_1 \subtype \tau_2 $ and $ \tau_2 \subtype \tau_3$,
then $ \tau_1 \subtype \tau_3$.
\end{lemma}


\begin{lemma}[Get rules produce the correct coercion] \label{type-get}
  If $$ \Gamma \vdash_{get} \tau ; l = C ; \tau_1 $$
  then $$ |\Gamma| \vdash C : |\tau| \to |\tau_1| $$
\end{lemma}

\begin{proof}
By induction on the given derivation.
\end{proof}

\begin{lemma}[Put rules produce the correct coercion] \label{type-put}
  If $$ \Gamma \vdash_{put} \tau ; l ; E = C ; \tau_1 $$
  then $$ |\Gamma| \vdash C : |\tau| \to |\tau| $$
\end{lemma}

\begin{proof}
By induction on the given derivation.
\end{proof}

\begin{lemma}[Translation preserves well-formedness] \label{preserve-wf}
  If   $$ \Gamma \vdash \tau $$
  then $$ |\Gamma| \vdash |\tau| $$
\end{lemma}

\begin{proof}
By induction on the given derivation.
\end{proof}

\begin{theorem}[Type preserving translation] \label{preserve-tr}
  If   $$ \Gamma \vdash e : \tau \yields{E} $$
  then $$ |\Gamma| \vdash E : \left| \tau \right| $$
\end{theorem}

\begin{proof}
(Sketch) By structural induction on the expression and the corresponding
inference rule. The full proof can be found in the appendix.
\end{proof}

Type-Directed Translation to System F.
Main results: type-preservation + coherence.


\section{Implementation}

\subsection{Type Synonyms}

We extend the implementation of the type system extended with type synonyms and
lazy arguments.

\begin{lstlisting}
type T A1 A2 = ... in
\end{lstlisting}

\subsection{Optimization}
\section{Related Work}

\begin{itemize}

\item{\bf Elaborating simply-typed lambda calculus}

  Dunfield has introduced a type system with intersection polymorphism but no
  parametric polymorphism.

\end{itemize}

Nystrom et. al. OOPSLA 06

Applications:

- Object/Fold Algebras. How to support extensibility in an easier way.

See Datatypes a la Carte

- Mixins

- Lenses? Can intersection types help with lenses? Perhaps making the
types more natural and easy to understand/use?

- Embedded DSLs? Extensibility in DSLs? Composing multiple DSL interpretations?

http://www.cs.ox.ac.uk/jeremy.gibbons/publications/embedding.pdf

\section{Conclusion}

% \acks

Acknowledgments, if needed.

\appendix

\section{Proofs}

\begin{proof}
By structural induction on the types and the corresponding inference rule. \\

\texttt{(SubVar)}

\texttt{(SubFun)}

\texttt{(SubForall)}

\texttt{(SubAnd1)}

\texttt{(SubAnd2)}

\texttt{(SubAnd3)}

\texttt{(SubRcd)}

\end{proof}

\begin{lemma} \label{type-get}
  If $$ \Gamma \vdash_{get} \tau ; l = C ; \tau_1 $$
  then $$ |\Gamma| \vdash C : |\tau| \to |\tau_1| $$
\end{lemma}

\begin{proof}
By structural induction on the type and the corresponding inference rule. \\

\texttt{(Get-Base)} $ \Gamma \vdash_{get} \{ l : \tau \} ; l = \lambda (x : |\{ l : \tau \}|). x ; \tau $ \\

By the induction hypothesis
$$ |\Gamma| \vdash \lambda (x : |\{ l : \tau \}|). x : |\{ l : \tau \}| \to |\tau| $$

\texttt{(Get-Left)} \\
\texttt{(Get-Right)} \\

\end{proof}

\begin{lemma} \label{type-put}
  If $$ \Gamma \vdash_{put} \tau ; l ; E = C ; \tau_1 $$
  then $$ |\Gamma| \vdash C : |\tau| \to |\tau| $$
\end{lemma}

\begin{proof}
By structural induction on the type and the corresponding inference rule. \\

\texttt{(Put-Base)} \\
\texttt{(Put-Left)} \\
\texttt{(Put-Right)} \\
\end{proof}

\begin{lemma} \label{preserve-wf}
  If   $$ \Gamma \vdash \tau $$
  then $$ |\Gamma| \vdash |\tau| $$
\end{lemma}

\begin{proof}
Since $$ \Gamma \vdash \tau $$
It follows from \texttt{(FI-WF)} that
  $$ \ftv{\tau} \subseteq \ftv{\Gamma} $$
And hence
  $$ \ftv{|\tau|} \subseteq \ftv{|\Gamma|} $$
By \texttt{(F-WF)} we have
  $$ \Gamma \vdash \tau $$
\end{proof}

\begin{theorem}[Type preserving translation] \label{preserve-tr}
  If   $$ \Gamma \vdash e : \tau \yields{E} $$
  then $$ |\Gamma| \vdash E : \left| \tau \right| $$
\end{theorem}

\begin{proof}
By structural induction on the expression and the corresponding inference rule. \\

\texttt{(TrVar)} $ \Gamma \vdash x : \tau \yields{x} $ \\

It follows from \texttt{(TrVar)} that
  $$ (x : \tau) \in \Gamma $$
Based on the definition of $ |\cdot| $,
  $$ (x : |\tau|) \in |\Gamma| $$
Thus we have by \texttt{(F-Var)} that
  $$ |\Gamma| \vdash x : |\tau| $$

\texttt{(TrAbs)} $ \Gamma \vdash \lambda (x : \tau_1). e : \tau_1 \to \tau_2 \yields{\lambda x : |\tau_1|. E} $ \\

It follows from \texttt{(TrAbs)} that
  $$ \Gamma, x : \tau_1 \vdash e : \tau_2 \yields{E} $$
And by the induction hypothesis that
  $$ |\Gamma|, x : |\tau_1| \vdash E : |\tau_2| $$
By \texttt{(TrAbs)} we also have
  $$ \Gamma \vdash \tau_1 $$
It follows from Lemma \ref{preserve-wf} that
  $$ |\Gamma| \vdash |\tau_1| $$
Hence by \texttt{(F-Abs)} and the definition of $|\cdot|$ we have
  $$ |\Gamma| \vdash \lambda x : |\tau_1|. E : |\tau_1 \to \tau_2| $$

\texttt{(TrApp)} $ \Gamma \vdash e_1 e_2 : \tau_2 \yields{E_1 (C E_2)} $ \\

From \texttt{(TrApp)} we have
  $$ \Gamma \vdash \tau_3 <: \tau_1 \yields{C} $$
Applying Lemma \ref{type-coerce} to the above we have
  $$ |\Gamma| \vdash C : |\tau_3| \to |\tau_1| $$
Also from \texttt{(TrApp)} and the induction hypothesis
  $$ |\Gamma| \vdash E_1 : |\tau_1| \to |\tau_2| $$
Also from \texttt{(TrApp)} and the induction hypothesis
  $$ |\Gamma| \vdash E_2 : |\tau_3| $$
Assembling those parts using \texttt{(F-App)} we come to
  $$ |\Gamma| \vdash E_1 (C E_2) : |\tau_2| $$
\end{proof}

\texttt{(TrTAbs)} $ \Gamma \vdash \Lambda \alpha. e : \forall \alpha. \tau \yields{\forall \alpha. E} $ \\

From \texttt{(TrTAbs)} we have
  $$ \Gamma \vdash e : \tau \yields{E} $$
By the induction hypothesis we have
  $$ |\Gamma| \vdash E : |\tau| $$
Thus by \texttt{(F-TAbs)} and the definition of $|\cdot|$
  $$ \Gamma \vdash \Lambda \alpha. E : |\forall \alpha. \tau| $$


\texttt{(TrTApp)} $ \Gamma \vdash e \; \tau  : [\alpha := \tau]\tau_1 \yields{E \; |\tau|} $ \\

From \texttt{(TrTApp)} we have
  $$ \Gamma \vdash e : \forall \alpha. \tau_1 \yields{E} $$
And by the induction hypothesis that
  $$ |\Gamma| \vdash E : \forall \alpha. |\tau_1| $$
Also from \texttt{(TrTApp)} and Lemma \ref{preserve-wf} we have
  $$ |\Gamma| \vdash |\tau| $$
Then by \texttt{(F-TApp)} that
  $$ |\Gamma| \vdash E \; |\tau| : [\alpha := |\tau| ]|\tau_1| $$
Therefore
  $$ |\Gamma| \vdash E \; |\tau| : | [\alpha := \tau ] | \tau_1 | | $$

% \texttt{(TrMerge)} $ \Gamma \vdash e_1 \merge e_2 : \tau_1 \& \tau_2 \yields{\langle E1, E2  \rangle}$ \\

From \texttt{(TrMerge)} and the induction hypothesis we have
  $$ |\Gamma| \vdash E_1 : |\tau_1| $$
and
  $$ |\Gamma| \vdash E_2 : |\tau_2| $$
Hence by \texttt{(F-Pair)}
  $$ |\Gamma| \vdash \langle E_1, E_2 \rangle : \langle |\tau_1|, |\tau_2| \rangle $$
Hence by the definition of $|\cdot|$
  $$ |\Gamma| \vdash \langle E_1, E_2 \rangle : |\tau_1 \& \tau_2| $$

\texttt{(TrRcdIntro)} $ \Gamma \vdash \{ l = e \} : \{ l : \tau \} \yields{E} $ \\

From \texttt{(TrRcdIntro)} we have
  $$ \Gamma \vdash e : \tau \yields{E} $$
And by the induction hypothesis that
  $$ |\Gamma| \vdash E : |\tau| $$
Thus by the definition of $|\cdot|$
  $$ |\Gamma| \vdash E : |\{ l : \tau \}| $$

\texttt{(TrRcdElim)} $ \Gamma \vdash e.l : \tau_1 \yields{C E} $ \\

From \texttt{(TrRcdElim)}
  $$ \Gamma \vdash e : \tau \yields{E} $$
And by the induction hypothesis that
  $$ |\Gamma| \vdash E : |\tau| $$
Also from \texttt{(TrRcdEim)}
  $$ \Gamma \vdash_{get} e ; l = C ; \tau_1 $$
Applying Lemma \ref{type-get} to the above we have
  $$ |\Gamma| \vdash C : |\tau| \to |\tau_1|  $$
Hence by \texttt{(F-App)} we have
  $$ |\Gamma| \vdash C E : |\tau_1| $$

% \texttt{(TrRcdUpd)} $ \Gamma \vdash \rcdupd{e}{l}{e_1} : \tau \yields{C E} $ \\

From \texttt{(TrRcdUpd)}
  $$ \Gamma \vdash e : \tau \yields{E} $$
And by the induction hypothesis that
  $$ |\Gamma| \vdash E : |\tau| $$
Also from \texttt{(TrRcdUpd)}
  $$ \Gamma \vdash_{put} t ; l; E = C ; \tau_1 $$
Applying Lemma \ref{type-put} to the above we have
  $$ |\Gamma| \vdash C : |\tau| \to |\tau|  $$
Hence by \texttt{(F-App)} we have
  $$ |\Gamma| \vdash C E : |\tau| $$


% We recommend abbrvnat bibliography style.

\bibliographystyle{abbrvnat}

% The bibliography should be embedded for final submission.

\bibliography{references}

\begin{thebibliography}{}
\softraggedright

\bibitem[Smith et~al.(2009)Smith, Jones]{smith02}
P. Q. Smith, and X. Y. Jones. ...reference text...

Coppo, M., Dezani-Ciancaglini, M.: A new type-assignment for λ-terms. Archiv.
Math. Logik 19, 139–156 (1978)

\end{thebibliography}

%% #include ../src/Algebra.sf

%% #end

\end{document}
