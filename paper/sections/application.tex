\section{Application}

% - Object/Fold Algebras. How to support extensibility in an easier way.

% See Datatypes a la Carte

% - Mixins

% - Lenses? Can intersection types help with lenses? Perhaps making the
% types more natural and easy to understand/use?

% - Embedded DSLs? Extensibility in DSLs? Composing multiple DSL interpretations?

% http://www.cs.ox.ac.uk/jeremy.gibbons/publications/embedding.pdf

\begin{minted}{scala}
trait Expr {
  def eval: Int
}

class Lit(n: Int) extends Expr {
  def eval: Int = n
}

class Add(n: Int) extends Expr {
  def eval: Int = e1.eval + e2.eval
}
\end{minted}

\subsection{Overloading}

Dunfield~\cite{dunfield2014elaborating} notes that using merges as a mechanism
of overloading is not as powerful as type classes.

\subsection{Bounded Polymorphism}

Ceylon

% motivate

Bounded polymorphism ...

\cite{pierce1997intersection}

\begin{figure}
  \begin{minted}{scala}
  def getName[A <: { def name: String }](v: A) = v.name
  case class Person(name: String)
  getName[Person](Person("Ek"))
  \end{minted}

  \begin{minted}{ocaml}
  let getName A (v : A & {name:String}) = v.name in
  getName [{name:String}] {name = "Ek", age = 10}
  \end{minted}

  \caption{Example of encoding records}
  \label{fig:encoding-records}
\end{figure}


% Multiple inheritance?
% Algebra -> P1,2
% Visitor -> P2

% Yanlin
% Mixin

% \begin{lstlisting}
% let merge A B (f : ExpAlg A) (g : ExpAlg B) = {
%   lit = \(x : Int). f.lit x ,, g.lit x,
%   add = \(x : A & B). \(y : A & B). f.add x y ,, g.add x y
% };
% \end{lstlisting}

This section shows that the System $ F $ plus intersection types are enough for
encoding extensible designs, and even beat the designs in languages with a much
more sophisticated type system. In particular, \name has two main advantages
over existing languages:

\begin{enumerate}
\item It supports dynamic composition of intersecting values.
\item It supports contravariant parameter types in the subtyping relation.
\end{enumerate}

Various solutions have been proposed to deal with the extensibility problems and
many rely on heavyweight language features such as abstract methods and classes
in Java.

These two features can be used to improve existing designs of modular programs.

% Introduce the expression problem

The expression problem refers to the difficulty of adding a new operations and a
new data variant without changing or duplicating existing code.

There has been recently a lightweight solution to the expression problem that
takes advantage of covariant return types in Java. We show that FI is able to
solve the expression problem in the same spirit. The
A)

\subsection{Object Algebras}

Object algebras provide an alternative to \emph{algebraic data types} (ADT). For example, the
following Haskell definition of the type of simple expressions
\begin{minted}[fontsize=\footnotesize]{haskell}
data Exp where
  Lit :: Int -> Exp
  Add :: Exp -> Exp -> Exp
\end{minted}
can be expressed by the \emph{interface} of an object algebra of simple expressions:
\begin{minted}[fontsize=\footnotesize]{scala}
trait ExpAlg[E] {
  def lit(x: Int): E
  def add(e1: E, e2: E): E
}
\end{minted}
Similar to ADT, data constructors in object algebras are represented by functions such as
\lstinline{lit} and \lstinline{add} inside an interface \lstinline{ExpAlg}.
Different with ADT, the type of the expression itself is abtracted by a type
parameter \lstinline{E}.

which can be expressed similarly in \name as:
\begin{lstlisting}
type ExpAlg E = {
  lit : Int -> E,
  add : E -> E -> E
}
\end{lstlisting}

% Introduce Scala's intersection types

Scala supports intersection types via the \lstinline{with} keyword. The type
\lstinline{A with B} expresses the combined interface of \lstinline{A} and
\lstinline{B}. The idea is similar to
\begin{minted}[fontsize=\footnotesize]{java}
interface AwithB extends A, B {}
\end{minted}
in Java.
\footnote{However, Java would require the \lstinline{A} and \lstinline{B} to be
  concrete types, whereas in Scala, there is no such restriction.}

The value level counterpart are functions of the type \lstinline
{A => B => A with B}. \footnote{FIXME}

Our type system is a simple extension of System $ F $; yet surprisingly, it
is able to solve the limitations of using object algebras in languages such as
Java and Scala. We will illustrate this point with an step-by-step of solving
the expression problem using a source language built on top of \name.

Oliveira noted that composition of object algebras can be cumbersome and
intersection types provides a solution to that problem.

We first define an interface that supports the evaluation operation:

% #include ../src/Algebra.sf  @base
\begin{lstlisting}
type IEval  = { eval : Int };
type ExpAlg E = { lit : Int -> E, add : E -> E -> E };
let evalAlg = {
  lit = \(x : Int). { eval = x },
  add = \(x : IEval). \(y : IEval). { eval = x.eval + y.eval }
};
\end{lstlisting}
% #end

The interface is just a type synonym \lstinline{IEval}. In \name, record
types are structural and hence any value that satisfies this interface is of
type \lstinline{IEval} or of a subtype of \lstinline{IEval}. \footnote{Should be
mentioned in S2.}

In the following, \lstinline{ExpAlg} is an object algebra interface of
expressions with literal and addition case. And \lstinline{evalAlg} is an object
algebra for evluation of those expressions, which has type \lstinline{ExpAlg Int}

% #include ../src/Algebra.sf  @variant
\begin{lstlisting}
type SubExpAlg E = (ExpAlg E) & { sub : E -> E -> E };
let subEvalAlg = evalAlg ,, { sub = \ (x : IEval). \ (y : IEval). { eval = x.eval - y.eval } };
\end{lstlisting}
% #end

Next, we define an interface that supports pretty printing.

% #include ../src/Algebra.sf  @operation
\begin{lstlisting}
type IPrint = { print : String };
let printAlg = {
  lit = \(x : Int). { print = x.toString() },
  add = \(x : IPrint). \(y : IPrint). { print = x.print.concat(" + ").concat(y.print) },
  sub = \(x : IPrint). \(y : IPrint). { print = x.print.concat(" - ").concat(y.print) }
};
\end{lstlisting}
% #end

Provided with the definitions above, we can then create values using the
appropriate algebras. For example:
defines two expressions.

The expressions are unusual in the sense that they are functions that take an
extra argument \lstinline{f}, the object algebras, and use the data constructors
provided by the object algebra (factory) \lstinline{f} such as \lstinline{lit},
\lstinline{add} and \lstinline{sub} to create values. Moreover, The algebras
themselves are abstracted over the allowed operations such as evaluation and
pretty printing by requiring the expression functions to take an extra argument
\lstinline{E}.

% #include ../src/Algebra.sf  @merge
\begin{lstlisting}
let merge A B (f : ExpAlg A) (g : ExpAlg B) = {
  lit = \(x : Int). f.lit x ,, g.lit x,
  add = \(x : A & B). \(y : A & B).
          f.add x y ,, g.add x y
};
\end{lstlisting}
% #end

If we would like to have an expression that supports both evaluation and pretty
printing, we will need a mechanism to combine the evaluation and printing
algebras. Intersection types allows such composition: the \lstinline{merge}
function, which takes two expression algebras to create a combined algebra. It
does so by constructing a new expression algebra, a record whose each field is a
function that delegates the input to the two algebras taken.

% #include ../src/Algebra.sf  @usage
\begin{lstlisting}
let newAlg = merge IEval IPrint subEvalAlg printAlg in
let o1 = e1 (IEval & IPrint) newAlg in
o1.print
\end{lstlisting}
% #end

\lstinline{o1} is a single object created that supports both evaluation and
printing, thus achieving full feature-oriented programming.

\subsection{Visitors}

Constructing instances seems clumsy!

The visitor pattern allows adding new operations to existing structures without
modifying those structures. The type of expressions are defined as follows:

\begin{minted}[fontsize=\footnotesize]{scala}
trait Exp[A] {
  def accept(f: ExpAlg[A]): A
}

trait SubExp[A] extends Exp[A] {
  override def accept(f: SubExpAlg[A]): A
}
\end{minted}

The body of \lstinline{Exp} and \lstinline{SubExp} are almost the same: they
both contain an \lstinline{accept} method that takes an algebra \lstinline{f}
and returns a value of the carrier type \lstinline{A}. The only difference is at
\lstinline{f} --- \lstinline{SubExpAlg[A]} is a subtype of
\lstinline{ExpAlg[A]}. Since \lstinline{f} appear in parameter position of
\lstinline{accept} and function parameters are contravariant, naturally we would
hope that \lstinline{SubExp[A]} is a supertype of \lstinline{Exp[A]}. However,
such subtyping relation does not fit well in Scala because inheritance implies
subtyping in such languages \footnote{It is still possible to encode
  contravariant parameter types in Scala but doing so would require some
  technique.}. As \lstinline{SubExp[A]} extends \lstinline{Exp[A]}, the former
becomes a subtype of the latter.

Such limitation does not exist in \name. For example, we can define the similar interfaces \lstinline{Exp} and \lstinline{SubExp}:
\begin{lstlisting}
type Exp    A = { accept: forall A. ExpAlg A -> A };
type SubExp A = { accept: forall A. SubExpAlg A -> A };
\end{lstlisting}
Then by the typing judgment it holds that \lstinline{SubExp} is a supertype of
\lstinline{Exp}. This relation gives desired results. To give a concrete example:

A is called is the \emph{interpretation}. It works for any interpretation you want.

First we define two data constructors for simple expressions:
\begin{lstlisting}
let lit (n : Int): Exp A = {
  accept = /\A. \(f : ExpAlg A). f.lit n
};

let add (e1 : Exp) (e2 : Exp): Exp A = {
  accept = /\A. \(f : ExpAlg A).
             f.add (e1.accept A f) (e2.accept A f)
};
\end{lstlisting}

Suppose later we decide to augment the expressions with subtraction:
\begin{lstlisting}
let sub (e1 : SubExp) (e2 : SubExp): SubExp A =
  { accept = /\A. \(f : SubExpAlg A).
               f.sub (e1.accept A f) (e2.accept A f) };
\end{lstlisting}

One big benefit of using the visitor pattern is that programmers is able to
write in the same way that would do in Haskell.
For example, \lstinline{e2 = sub (lit 2) (lit 3)} defines an expression.

Another important property that does not exist in Scala is that programmer is
able to pass \lstinline{lit 2}, which is of type \lstinline{Exp A}, to
\lstinline{sub}, which expects a \lstinline{SubExp A} because of the subtyping
relation we have. After all, it is known statically that \lstinline{lit 2} can
be passed into \lstinline{sub} and nothing will go wrong.

\subsection{Yanlin Stuff}

This subsection presents yet another lightweight solution to the Expression
Problem, inspired by the recent work by Wang. It has been shown that
contravariant return types allows refinement of the types of extended
expressions.

First, we define the type of expressions that support evaluation and implement
two constructors:
\begin{lstlisting}
type Exp = { eval: Int }
let lit (n: Int) = { eval = n }
let add (e1: Exp) (e2: Exp)
  = { eval = e1.eval + e2.eval }
\end{lstlisting}

If we would like to add a new operation, say pretty printing, it is nothing more
than refining the original \lstinline{Exp} interface by \emph{intersecting} the
original type with the new \lstinline{print} interface using the \lstinline{&}
primitive and \emph{merging} the original data constructors using the \lstinline{,,}
primitive.
\begin{lstlisting}
type ExpExt = Exp & { print: String }
let litExt (n: Int) = lit n ,, { print = n.toString() }
let addExt (e1: ExpExt) (e2: ExpExt)
  = add e1 e2 ,,
    { print = e1.print.concat(" + ").concat(e2.print) }
\end{lstlisting}

Now we can construct expressions using the constructors defined above:
\begin{lstlisting}
let e1: ExpExt = addExt (litExt 2) (litExt 3)
let e2: Exp = add (lit 2) (lit 4)
\end{lstlisting}
\lstinline{e1} is an expression capable of both evaluation and printing, while
\lstinline{e2} supports evaluation only.

We can also add a new variant to our expression:
\begin{lstlisting}
let sub (e1: Exp) (e2: Exp) = { eval = e1.eval - e2.eval }
let subExt (e1: ExpExt) (e2: ExpExt)
  = sub e1 e2 ,, { print = e1.print.concat(" - ").concat(e2.print) }
\end{lstlisting}

Finally we are able to manipulate our expressions with the power of both
subtraction and pretty printing.
\begin{lstlisting}
(subExt e1 e1).print
\end{lstlisting}

\subsection{Mixins}

Mixins are useful programming technique wildly adopted in dynamic programming
languages such as JavaScript and Ruby. But obviously it is the programmers'
responsbility to make sure that the mixin does not try to access methods or
fields that are not present in the base class.

In Haskell, one is also able to write programs in mixin style using records.
However, this approach has a serious drawback: since there is no subtyping in
Haskell, it is not possible to refine the mixin by adding more fields to the
records. This means that the type of the family of the mixins has to be
determined upfront, which undermines extensibility.

\name is able to overcome both of the problems: it allows composing mixins
that (1) extends the base behavior, (2) while ensuring type safety.

The figure defines a mini mixin library. The apostrophe in front of types
denotes call-by-name arguments similar to the \lstinline{=>} notation in the
Scala language.

\begin{lstlisting}
type Mixin S = 'S -> 'S -> S;
let zero S (super : 'S) (this : 'S) : S = super;
let rec mixin S (f : Mixin S) : S
  = let m = mixin S in f (\ (_ : Unit). m f) (\ (_ : Unit). m f);
let extends S (f : Mixin S) (g : Mixin S) : Mixin S
  = \ (super : 'S). \ (this  : 'S). f (\ (d : Unit). g super this) this;
\end{lstlisting}

We define a factorial function in mixin style and make a \lstinline{noisy} mixin
that prints ``Hello'' and delegates to its superclass. Then the two functions
are composed using the \lstinline{mixin} and \lstinline{extends} combinators.
The result is the \lstinline{noisyFact} function that prints ``Hello'' every
time it is called and computes factorial.
\begin{lstlisting}
let fact (super : 'Int -> Int) (this : 'Int -> Int) : Int -> Int
  = \ (n : Int). if n == 0 then 1 else n * this (n - 1)
let noisy (super : 'Int -> Int) (this : 'Int -> Int) : Int -> Int
  = \ (n : Int). { println("Hello"); super n }
let noisyFact = mixin (Int -> Int) (extends (Int -> Int) foolish fact)
noisy 5
\end{lstlisting}

% \subsection{Composing Mixins and Object Algebras}
