\section{An Overview of \name}

\george{Change the examples later to something very simple.}

This section provides the reader with necessary intuition of \name. In \name,
the central addition to the type system of System $ F $ is intersection types.
\bruno{mention (and cite) that there are a number of OO languages supporting
  intersection types } A number of OO languages, such as Java, Scala, and
Ceylon\george{cite?}, already support intersection types to different degrees. A
common limitation in those languages, though, is that there is no introduction
construct at the term level for intersection types. In contrast, there are
introduction construct for function types (lambdas) and universal quantification
(big lambdas) in most core calculi. To fill this gap, we allow intersecting any
two terms at run time using a \emph{merge} operator, similar to
Dunfield's~\cite{dunfield2014elaborating} approach. The key constructs are the
``merge'' operator, denoted by $ \dcomma $ at the term level and the
corresponding type intersection operator, denoted by $ \intersects $ at the type
level.\bruno{We take a particular approach to intersection types, namely that of
  Dundfield. Some other approaches are different. The \emph{key feature} in
  Dunfield's approach is the $\dcomma$ operator, which allows for run-time
  value/object composition.}

The addition of intersection types to System $ F $ has a number of consequences,
which we will explore one by one in the following subsections.

\bruno{Don't mention prototype-based inheritance in the summary, unless you are
  going to mention it later in the section. \emph{More generally don't mention
    things that you don't talk about later!}}. 

\subsection{Intersection Types}

% What is an intersection type? One classic view is from set-theoretic
% interpretation of types: $ A \intersects B $ stands for the intersection of
% the set of values of $ A $ and $ B $. A more practical view, adopted in this
% paper, regards types as a kind of interface: a value of type
% $ A \intersects B $ satisfies both of the interfaces of $ A $ and $ B $. For
% example, \lstinline{eval : Int} is the interface that supports evaluation to
% integers, while \lstinline{eval : Int & print : String } supports both
% evaluation and pretty printing. Those interfaces are akin to interfaces in
% Java or traits in Scala. But one key difference is that they are unnamed in
% \name.

We motivate the use of intersection types with overloaded function, as
intersection types provide a simple mechanism for ad-hoc polymorphism, similar
to what type classes in Haskell achieve. The benefit is that programmers can use
the same operation on different types and delegate the task of choosing a
concrete implementation to the type system. For example, we can define a
\lstinline{show} function that takes either an integer or a boolean and return
its string representation. In other words, it is also \emph{both} a function
from integers to strings as well as a function from boolean to strings.
Therefore, in \name it should be of following type:
\begin{lstlisting}
(Int -> String) & (Bool -> String)
\end{lstlisting}
Assuming we have the following two functions available,
\begin{lstlisting}
let showInt : Int  -> String = ...
let showBool : Bool -> String = ...
\end{lstlisting}
We may define \lstinline{show} by merging the them using the merge operator
({\lstinline{,,}):
\begin{lstlisting}
let show = showInt ,, showBool
\end{lstlisting}
\bruno{This is not a proper sentence. Polish text.}

To illustrate the usage, consider the function application \lstinline{show 100}.
The type system will pick the first component of \lstinline{show}, namely
\lstinline{showInt}, as the implementation being applied to \lstinline{100}
because the type of \lstinline{showInt} is compatible with \lstinline{100}, but
\lstinline{showBool} is not. This example shows that one may regard
intersections in our system as ``implicit pairs'' whose introduction is explicit
by the merge operator and elimination is implicit (with no source-level
construct for elimination).

% The merge construct in the original function is elaborated into a pair in the
% target language:

% \begin{verbatim}
% show = (showInt, showBool)
% \end{verbatim}

% In the target language where there is no intersection types, the application
% of the integer \texttt{1} to this function does not typecheck. However, we may
% rescue this situtation by inserting a coercion that extracts the first item
% out of this pair.

% Thus \texttt{show 1} in FI corresponds to \texttt{(fst show) 1} in F.

% While elaborating intersection types, this paper is the first that presents a
% type system that incorporates both parametric polymorphism and intersection
% polymorphism.

% Describe intersection types, encoding records with Intersecion types

% \lstinputlisting[linerange=-]{} % APPLY:linerange=MIXIN_LIB

\subsection{Subtyping}

As a result of intersection types, the type system of \name permits a subtyping
relation naturally. We will explore the usefulness of such a type system in
practice by showing various examples.

\subsection{Intersection Types and Records}
\bruno{The title of this section needs adjusting. Moreover 
we do not want to talk about record elimination and update. \name
supports a \emph{generalization} of record elimination and update 
that \emph{works for any type}. That's what we want to mention instead.}

In addition to introduction of record literals using the usual
notation, \name support two more operations on records: record
elimination and record update.

A record type of the form $ \recty l \ty $ can be thought as a normal type \lstinline{t}
tagged by the label \lstinline{l}.\bruno{I would call this a labelled
  type. Record types can be viewed as intersections of labelled types.}

% A basic example

% \lstinputlisting[linerange=-]{} % APPLY:linerange=BASICS_ADD

\bruno{Not a proper sentence! }
\lstinline{e1} and \lstinline{e2} are two expressions that support both evaluation and pretty
printing and each has type \lstinline{eval : Int, print : String}. \lstinline{add} takes
two expressions and computes their sum\bruno{Don't inline examples in the text. Give
  proper separate code examples.}. Note that in order to compute a sum,
\lstinline{add} only requires that the two expressions support evaluation and hence the
type of the parameter \lstinline{eval : Int}. As a result, the type of \lstinline{e1} and
\lstinline{e2} are not exactly the same with that of the parameters of \lstinline{add}. However,
under a structural type system, this program should typecheck anyway because the
arguments being passed has more information than required. In other words,
\lstinline{eval : Int, print : String} is a subtype of \lstinline{eval : Int}.

How is this subtyping relation derived? In \name, multi-field record types are
excluded from the type system because \lstinline{eval : Int, print : String} can
be encoded as \lstinline{eval : Int & print : String}. And by one of
subtyping rules derives that \lstinline{eval : Int & print : String} is a
subtype of \lstinline{eval : Int}.

\bruno{Examples illustrating update and elimination of intersections
  are missing!}

% This example is elaborated into the following in \Target.

% \lstinputlisting[linerange=-]{} % APPLY:linerange=BASICSELAB_ADD

\subsection{Intersection Types and Parametric Polymorphism}

The presence of both parametric polymorphism and intersection is critical, as we
shall see in the next section, in solving modularity problems. Here is a code
snippet from the next section (The reader is not required to understand the
purpose of this code at this stage; just recognizing the two types of
polymorphism is enough.)\bruno{use the join example here.}

\begin{lstlisting}
type SubExpAlg E = (ExpAlg E) \& { sub : E -> E -> E };
let e2 E (f : SubExpAlg E) = f.sub (exp1 E f) (f.lit 2);
\end{lstlisting}

\lstinline{SubExpAlg} is a type synonym (a la Haskell) defined as the intersection of
\lstinline{ExpAlg E} and \lstinline{sub : E -> E -> E}, parametrized by a type parameter
\lstinline{E}. \lstinline{e2} exhibits parametric polymorphism as it takes a type argument
\lstinline{E}.
