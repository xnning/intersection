\section{Type-directed Translation to System $ F $}

In this section we define the semantics of \name by means of a type-directed
translation to a variant of System $ F $ extended with tuples. This translation
removes the labels of records and turns intersections into products, much like
Dunfield's elaboration. But our translation also deals with parametric
polymorphism and records.

\subsection{Informal Discussion}

To help the reader have a high-level understanding of how the translation works,
in this subsection we present the translation informally. Take the \name
expression for example:
\begin{lstlisting}
{ eval = 4, print = "4" }.eval
\end{lstlisting}

First, multi-field record literals are desugared into merges of single-field
record literals. Therefore
\begin{lstlisting}
{ eval = 4, print = "4" }
\end{lstlisting}
becomes
\begin{lstlisting}
{ eval = 4 } ,, { print = "4" }
\end{lstlisting}
Merges of two values are translated into just a pair of them by $ \rulelabelemerge $
and single-field record literals lose their field labels by $ \rulelabelereccon $.
Hence \lstinline $ { eval = 4 } ,, { print = "4" } $ becomes \lstinline
$ (4, "4") $.

\bruno{Don't abuse inlining of examples in the text!}

Finally, $ e1 $ and $ e2 $ are both coerced by a projection function
$ \\(x:(Int,String)). \proj 1 x $.
\bruno{Show the source program, and the program that it gets
  translated to. Then explain how that translation works.}

\subsection{Target Language}

The target language is System $ F $ extended with pairs. The syntax and typing
is completely standard. The syntax of System $ F $ is as follows:
% \bruno{fill!}
\begin{figure}[h]
  \[
\begin{array}{llrl}
  \text{Types}       & T    & \Coloneqq & \alpha \mid () \mid {T_1} \to {T_2} \mid \for \alpha T \mid \pair {T_1} {T_2} \\
  \text{Terms} & E, C & \Coloneqq & x \mid () \mid \lam x T E \mid \app {E_1} {E_2} \mid \blam \alpha E \\ 
                     &      & \mid      & \tapp E T \mid \pair {E_1} {E_2} \mid \proj k E \\
  \text{Contexts} & \Gamma & \Coloneqq & \epsilon \mid \Gamma, \alpha \mid \Gamma, x \hast T \\
\end{array}
\]

  \caption{Target syntax.}
  \label{fig:f-syntax}
\end{figure}
while its semantics can be found in standard texts~\cite{pierce2002types}.

% \bruno{Why is this lemma placed here?}
% \bruno{Generaly Speaking this text seems out of place.Move to 5.4, maybe?}

The main translation judgment is $ \Gamma \turns e : \tau \hookrightarrow E $ which
states that with respect to the environment $ \Gamma $, the \name expression
$ e $ is of a \name type $ \tau $ and its translation is a System $ F $ expression $ E $.

We also define the type translation function $ \im \cdot $ from \name types
$ \tau $ to System $ F $ types $ T $.

\begin{figure}[h]
\framebox{$ \image \ty = T $}

\[
\begin{array}{rcl}
  \image \alpha                     & = & \alpha \\
  \image {\ty_1} \to \image {\ty_2} & = & \image {\ty_1} \to \image {\ty_2} \\
  \image {\Forall \alpha \ty}      & = & \Forall \alpha \image \ty \\
  \image {\ty_1 \Intersect \ty_2}   & = & \Pair {\image {\ty_1}} {\image {\ty_2}} \\
  \image {\RecTy l \ty}            & = & \image t
\end{array}
\]
\caption{Type translation.}
\end{figure}

The first three rules of the translation is standard. For the last two, the
intersection of two types are translated into a product of them, and the label
of record types are erased.

The translation consists of four sets of rules, which are explained below:

\subsection{Subtyping (Coercion)}

\george{Talk about $ \eta $-expansion.}

\begin{figure*}
\framebox{\( t \subtype t \yields C \)}

\infax[SVar]
{\alpha \subtype \alpha \yields {\abs {\rel x {\imageof \alpha}} x}}

\infrule[SFun]
{t_3 \subtype t_1 \yields {C_1} \andalso t_2 \subtype t_4 \yields {C_2}}
{t_1 \to t_2 \subtype t_3 \to t_4
  \yields
    {\abs {\rel f {\imageof {t_1 \to t_2}}}
      {\abs {\rel x {\imageof {t_3}}}
        {\app {C_2} {(\app{f} {(\app{C_1} {x})})}}}}}

\infrule[SForall]
{t_1 \subtype \subst {\alpha_2} {\alpha_1} t_2 \yields{C}}
{\forall \alpha_1. t_1 \subtype \forall \alpha_2. t_2
  \yields
    {\abs{\rel{f} {\imageof {\forall \alpha. t_1}}}
      {\Abs \alpha {\app C {(\app f \alpha)}}}}}

\infrule[SAnd1]
{t_1 \subtype t_3 \yields C}
{t_1 \with t_2 \subtype t_3
  \yields
    {\abs{\rel x {\imageof {t_1 \with t_2}}}
      {\app C {(\proj 1 x)}}}}

\infrule[SAnd2]
{t_2 \subtype t_3 \yields C}
{t_1 \with t_2 \subtype t_3
  \yields
    {\abs {\rel x {\imageof {t_1 \with t_2}}}
      {\app C {(\proj 2 x)}}}}

\infrule[SAnd3]
{t_1 \subtype t_2 \yields {C_1} \andalso t_1 \subtype t_3 \yields {C_2}}
{t_1 \subtype t_2 \with t_3
  \yields
    {\abs {\rel x {\imageof {t_1}}}
      {\tupled {\app {C_1} x, \app {C_2} x}}}}

\infrule[SRcd]
{t_1 \subtype t_2 \yields C}
{\recordtype l {t_1} \subtype \recordtype l {t_2}
  \yields
    {\abs {\rel x {\imageof {\recordtype l {t_1}}}} {\tupled {\app C {(\proj 1 x)}}}}}

\caption{Coersive subtyping.}
\end{figure*}

The coercion judgment $ \Gamma \turns \tau_1 \subtype \tau_2 \yields C $
extends the subtyping judgment with a coercion on the right hand side of
$ \hookrightarrow $. A coercion $ C $ is an expression in the target language
and has type $ \tau_1 \to \tau_2 $, as proved by Lemma \ref{type-coerce}. It is
read ``In the environment $ \Gamma $, $ \tau_1 $ is a subtype of $ \tau_2 $; and
if any expression $ e $ has a type $ \tau_1 $ that is a subtype of the type of
$ \tau_2 $, the elaborated $ e $, when applied to the corresponding coercion $ C $,
has exactly type $ \im \tau_2 $''. For example,
$\Gamma \turns Int \& Bool <: Bool \yields {fst} $, where $ fst $ is the
projection of a tuple on the first element. The coercion judgment is only used
in the $ \rulelabeleapp $ case. As $ \rulelabelsubfun $ supports contravariant parameter
type and covariant return type, the coercion of the parameter types and that of
the return types are used to create a coercion for the function type.
$ \rulelabelsubandleft $, $ \rulelabelsubandright $, and $ \rulelabelsuband $ deal with intersection types.
The first two are complementary to each other. Take $ \rulelabelsubandleft $ for example,
if we know $ \tau_1 $ is a subtype of $ \tau_3 $ and $ C $ is a coercion from $ \tau_1 $
to $ \tau_3 $, then we can conclude that $ \tau_1 \andop \tau_2 $ is also a subtype of
$ \tau_3 $ and the new coercion is a function that takes a value $ x $ of type
$ \tau_1 \andop \tau_2 $, project $ x $ on the first item, and apply $ C $ to it.
$ \rulelabelsuband $ uses both of two coercions and constructs a pair.
\bruno{Give a couple of concrete examples when explaining the rules.}

\subsection{Typing (Translation)}

In this subsection we now present formally the translation rules that convert
\name expressions into System $ F $ ones. This set of rules essentially extends
those in the previous section with the light-blue part for the translation.

\begin{figure*}
\newcommand{\ruleevarelab} {
\inferrule* [right=$\rulelabelevar$]
  {(x,\tau) \in \gamma}
  {\judgee \gamma x \tau \yields x}
}

\newcommand{\ruleelamelab} {
\inferrule* [right=$\rulelabelelam$]
  {\judgee {\gamma, x \hast \tau} e {\tau_1} \yields E \andalso \judgeewf \gamma \tau}
  {\judgee \gamma {\lam x \tau e} {\tau \to \tau_1} \yields {\lam x {\im \tau} E}}
}

\newcommand{\ruleeappelab}{
\inferrule* [right=$\rulelabeleapp$]
  {\judgee \gamma {e_1} {\tau_1 \to \tau_2} \yields {E_1} \\
   \judgee \gamma {e_2} {\tau_3} \yields {E_2} \andalso
   \tau_3 \subtype \tau_1 \yields C}
  {\judgee \gamma {\app {e_1} {e_2}} {\tau_2} \yields {\app {E_1} {(\app C E_2)}}}
}

\newcommand{\ruleeblamelab}{
\inferrule* [right=$\rulelabeleblam$]
  {\judgee {\gamma, \alpha} e \tau \yields E}
  {\judgee \gamma {\blam \alpha e} {\for \alpha \tau} \yields {\blam \alpha E}}
}

\newcommand{\ruleetappelab}{
\inferrule* [right=$\rulelabeletapp$]
  {\judgee \gamma e {\for \alpha {\tau_1}} \yields E \andalso \judgeewf \gamma \tau}
  {\judgee \gamma {\tapp e \tau} {\subst \tau \alpha \tau_1} \yields {\tapp E {\im \tau}}}
}

\newcommand{\ruleemergeelab}{
\inferrule* [right=$\rulelabelemerge$]
  {\judgee \gamma {e_1} {\tau_1} \yields {E_1} \andalso
   \judgee \gamma {e_2} {\tau_2} \yields {E_2}}
  {\judgee \gamma {e_1 \mergeop e_2} {\tau_1 \andop \tau_2} \yields {\pair {E_1} {E_2}}}
}

\newcommand{\ruleerecconelab}{
\inferrule* [right=$\rulelabelereccon$]
  {\judgee \gamma e \tau \yields E}
  {\judgee \gamma {\reccon l e} {\recty l \tau} \yields E}
}

\newcommand{\ruleerecprojelab}{
\inferrule* [right=$\rulelabelerecproj$]
  {\judgee \gamma e \tau \yields E \andalso
   \judgeget \tau l {\tau_1} \yields C}
  {\judgee \gamma {e.l} {\tau_1} \yields {\app C E}}
}

\newcommand{\ruleerecupdelab}{
\inferrule* [right=$\rulelabelerecupd$]
  {\judgee \gamma e \tau \yields E \andalso
   \judgee \gamma {e_1} {\tau_1} \yields {E_1} \\
   \judgeput \tau l {\tau_1 \yields {E_1}} {\tau_2} {\tau_3} \yields C \andalso
   \tau_1 \subtype \tau_3}
  {\judgee \gamma {\recupd e l {e_1}} {\tau_2} \yields {\app C E}}
}
\newcommand{\ruleget}{
  \inferrule* [right=$\rulelabelget$]
  { }
  {\judgeget {\recty l \tau} l \tau}
}

\newcommand{\rulegetelab}{
  \inferrule* [right=$\rulelabelget$]
  { }
  {\judgeget {\recty l \tau} l \tau \yields {\lam x {\im {\recty l \tau}} x}}
}

\newcommand{\rulegetleft}{
  \inferrule* [right=$\rulelabelgetleft$]
  {\judgeget {\tau_1} l \tau}
  {\judgeget {\tau_1 \andop \tau_2} l \tau}
}

\newcommand{\rulegetleftelab}{
  \inferrule* [right=$\rulelabelgetleft$]
  {\judgeget {\tau_1} l \tau \yields C}
  {\judgeget {\tau_1 \andop \tau_2} l \tau \yields {\lam x {\im {\tau_1
          \andop \tau_2}} {\app C {(\proj 1 x)}}}}
}

\newcommand{\rulegetright}{
  \inferrule* [right=$\rulelabelgetright$]
  {\judgeget {\tau_2} l \tau}
  {\judgeget {\tau_1 \andop \tau_2} l \tau}
}

\newcommand{\rulegetrightelab}{
  \inferrule* [right=$\rulelabelgetright$]
  {\judgeget {\tau_2} l \tau \yields C}
  {\judgeget {\tau_1 \andop \tau_2} l \tau \yields {\lam x {\im {\tau_1
          \andop \tau_2}} {\app C {(\proj 2 x)}}}}
}

\newcommand{\ruleput}{
\inferrule* [right=$\rulelabelput$]
  { }
  {\judgeput {\recty l \tau} l {\tau_1} {\recty l {\tau_1}} \tau}
}

\newcommand{\ruleputelab}{
\inferrule* [right=$\rulelabelput$]
  { }
  {\judgeput {\recty l \tau} l {\tau_1 \yields E} {\recty l {\tau_1}} \tau
  \yields {\lam \_ {\im {\recty l \tau}} E}}
}

\newcommand{\ruleputleft}{
\inferrule* [right=$\rulelabelputleft$]
  {\judgeput {\tau_1} l \tau {\tau_3} {\tau_4}}
  {\judgeput {\tau_1 \andop \tau_2} l \tau {\tau_3 \andop \tau_2} {\tau_4}}
}

\newcommand{\ruleputleftelab}{
\inferrule* [right=$\rulelabelputleft$]
  {\judgeput {\tau_1} l {\tau \yields E} {\tau_3} {\tau_4} \yields C}
  {\judgeput {\tau_1 \andop \tau_2} l {\tau \yields E} {\tau_3 \andop \tau_2} {\tau_4}
  \yields {\lam x {\im {\tau_1 \andop \tau_2}} {\app C {(\proj 1 x)}}}}
}

\newcommand{\ruleputright}{
\inferrule* [right=$\rulelabelputright$]
  {\judgeput {\tau_2} l \tau {\tau_3} {\tau_4}}
  {\judgeput {\tau_1 \andop \tau_2} l \tau {\tau_1 \andop \tau_3} {\tau_4}}
}

\newcommand{\ruleputrightelab}{
\inferrule* [right=$\rulelabelputright$]
  {\judgeput {\tau_2} l {\tau \yields E} {\tau_3} {\tau_4} \yields C}
  {\judgeput {\tau_1 \andop \tau_2} l {\tau \yields E} {\tau_1 \andop \tau_3} {\tau_4}
  \yields {\lam x {\im {\tau_1 \andop \tau_2}} {\app C {(\proj 2 x)}}}}
}
\caption{Elaboration typing from \name to System $ F $.}
\end{figure*}

% \bruno{Badly structured. Don't mention Coercion here, as it was already
% explained in the previous section.}
% \bruno{Don't use itemize and items. Use paragraphs instead!}

\paragraph{Translation}

  The elaboration judgment $ \Gamma \turns e : \tau \yields E $ extends the
  typing judgment with an elaborated expression on the right hand side of
  $ \hookrightarrow $. The translation ensures that $ E $ has type
  $ \im \tau $. It is also standard, except for the case of $ \rulelabeleapp $, in
  which a coercion from the inferred type of the argument, $ e_2 $ , to the
  expected type of the parameter, $ \tau_1 $, is inserted before the argument;
  $ \rulelabelemerge $ translates merges into pairs. $ \rulelabelereccon $ uses the
  same System $ F $ expression $ E $ for $ e $ as for $ \reccon l e $. And in
  $ \rulelabelerecsel $ and $ \rulelabelerecupd $ the coercions generated by the ``get''
  and ``put'' rules will be used to coece the main \name expression.

  $ \rulelabelerecsel $ typechecks $ e $ and use the ``get'' rule to return the
  type of the field $ \tau_1 $ and the coecion $ C $. The type of the whole
  expression is $ \tau_1 $ and its translation of $ \app C E $.

  $ \rulelabelerecupd $ is similar to $ \rulelabelerecsel $ in that it uses the
  auxiliary ``put'' rule. This rule typechecks $ e $ and $ e_1 $, and uses the
  ``put'' rule. Note that it allows refining of types by an $ e_1 $ that is of a
  subtype of $ \tau_1' $, which is the type of the field $ l $ in $ e $. The type
  of the updated expression then takes the type $ \tau' $ returned by the ``put''
  rule, while its translation is $ E $, applied to the coercion generated by the
  ``put'' rule, $ C $.

  The two set of rules are explained below.

\paragraph{``get'' Rules}

  The ``get'' judgment deals spefically with record elimination and yields a
  coercion can be thought as a field accessor. For
  example:\bruno{Still not showing the derivations!}

  $ \Gamma \turnsget (\{ \code{eval} : \code{Int} \}, \code{eval}) : \{ \code{eval} : \code{Int} \} \yields {\lam x \code{Int} x} $

  The lambda is the field accessor and when applied to a translated expression
  of type $ \{ \code{eval} : \code{Int} \}$, it is able to give the desired
  field. $ \rulelabelget $ is the base case: the type of the field labelled
  $ l $ in a $ \recty l \tau $ is just $ \tau $ and the coercion is an identity
  function specialized to type $ \im {\recty l \tau} $ $ \rulelabelgetleft $ and
  $ \rulelabelgetright $ are complementary to each other.

  Consider the source program:
  \begin{lstlisting}
    ({ name = "Isaac", age = 10 }).name
  \end{lstlisting}

  Multi-field records are desugared into merge of single-field records:
  \begin{lstlisting}
    ({ name = "Isaac"} ,, { age = 10 }).name
  \end{lstlisting}

  By $ \rulelabelget $,
  \[ \turnsget (\recty {name} {String}; {name}) : String \]

  we have the coercion
  \[ \lam x {\im {\recty {name} {String}}} x \]

  which is just $ \lam x {String} x $ according to type translation.

  By $ \rulelabelgetleft $,
  \[ \turnsget (\recty {name} {String} \andop \recty {age} {Int}; {name}) : String \]

  % we have the coercion
  % \[ \abs {\rel x {\im {\recty {name} {String} \andop \recty
  %         {age} {Int}}}} \app {(\abs {\rel x {\im {\recty {name} {String}}}} x)} {(\fst ~ x)} \]
  % which is just $ \abs {\rel x {(String, Int)}} {\app {(\abs {\rel x {String}} x)} {(\fst ~ x)}} $ by type translation.

  By typing rules, the translation of the program is
  \[ ("Isaac", 10) \]. If we apply the coercion to it, we get
  \[ "Isaac" \]


\paragraph{``put'' Rules}\bruno{Missing example (and derivation)}

  The ``put'' judgment deals spefically with record update can be thought as
  producing a field updater. Compared to the ``get'' rules, the ``put'' rules
  take an extra input $ e $, which is the desired expression to replace the
  field lablled $ l $ in values of type $ \tau $. $ \rulelabelput $ is the base
  case. This rule allows refinement of record fields in the sense that the type
  of $ e $ can be a subtype of the type of the field labelled by $ l $. The
  resulting type is $ \recty l {\tau'} $ and the generated coercion is a
  constant function that always returns $ E $. $ \rulelabelputleft $ and
  $ \rulelabelputright $ are complementary to each other: the idea is exactly the
  same as $ \rulelabelgetleft $ and $ \rulelabelgetright $ except that the refined type
  $ \tau_1' $ and $ \tau_2' $ is used.

\subsection{Meta-theory}

\begin{lemma}[Subtyping is reflexive.] \label{sub-refl}
Given a type $ \tau $, $ \tau \subtype \tau $.
\end{lemma}

\begin{lemma}[Subtyping is transitive.] \label{sub-trans}
If $ \tau_1 \subtype \tau_2 $ and $ \tau_2 \subtype \tau_3 $,
then $ \tau_1 \subtype \tau_3 $.
\end{lemma}

\begin{lemma} \label{type-coerce}
  If $$ \Gamma \turns \tau_1 <: \tau_2 \yields C $$
  then $$ \im \Gamma \turns C : \im {\tau_1} \to \im {\tau_2} $$
\end{lemma}

\begin{lemma}[Get rules produce the type-correct coercion.] \label{type-get}
  If $$ \Gamma \turnsget \tau ; l = C ; \tau_1 $$
  then $$ \im \Gamma \turns C : \im \tau \to \im {\tau_1} $$
\end{lemma}

\begin{proof}
By induction on the given derivation.
\end{proof}

\begin{lemma}[Put rules produce the type-correct coercion.] \label{type-put}
  If $$ \Gamma \turnsput \tau ; l ; E = C ; \tau_1 $$
  then $$ \im \Gamma \turns C : \im \tau \to \im \tau $$
\end{lemma}

\begin{proof}
By induction on the given derivation.
\end{proof}

\begin{lemma}[Translation preserves well-formedness.] \label{preserve-wf}
  If   $$ \Gamma \turns \tau $$
  then $$ \im \Gamma \turns \im \tau $$
\end{lemma}

\begin{proof}
By induction on the given derivation.
\end{proof}

\begin{theorem}[Type preserving translation.] \label{preserve-tr}
  If   $$ \Gamma \turns e : \tau \yields E $$
  then $$ \im \Gamma \turns E : \im \tau $$
\end{theorem}

\begin{proof}
(Sketch) By structural induction on the expression and the corresponding
inference rule. The full proof can be found in the appendix.
\end{proof}

Type-Directed Translation to System $ F $.
Main results: type-preservation + coherence.
