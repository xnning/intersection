\section{Related work} \label{sec:related-work}

% \url{http://homepages.inf.ed.ac.uk/gdp/publications/Sub_Par.pdf}

% \cite{plotkin1994subtyping}

% Also discussed intersection types!~\cite{malayeri2008integrating}.

% Pierce Ph.D thesis: F<: + /|
%        technical report: F + /|, closer to ours

% \cite{barbanera1995intersection}

\paragraph{Intersection types with polymorphism.}
Our type system combines intersection types and parametric polymorphism.  Closest
to us is Pierce's work~\cite{pierce1991programming1} on a prototype
compiler for a language with both intersection types, union types, and
parametric polymorphism. Similarly to \name in his system universal
quantifiers do not support bounded quantification. However Pierce did not try to prove any
meta-theoretical results and his calculus does not have a merge
operator.  Pierce has also studied a system where both intersection
types and bounded polymorphism are present in his Ph.D
dissertation~\cite{pierce1991programming2} and a 1997
report~\cite{pierce1997intersection}. Going in the direction of higher
kinds, Compagnoni and Pierce~\cite{compagnoni1996higher} add
intersection types to System $ F_{\omega} $ and use the new calculus,
$ F^{\omega}_{\wedge} $, to model multiple inheritance. In their
system, types include the construct of intersection of types of the
same kind $ K $. Davies and Pfenning
\cite{davies2000intersection} study the interactions between
intersection types and effects in call-by-value languages. And they
proposed a ``value restriction'' for intersection types, similar to
value restriction on parametric polymorphism. Although they proposed a system with 
parametric polymorphism, their subtyping rule is significantly different from ours, 
since they consider parametric polymorphism 
as the ``infinit analog'' of intersection polymorphism.
There have been attempts to provide a foundational calculus
for Scala that incorporates intersection
types~\cite{amin2014foundations,amin2012dependent}. 
Although the minimal Scala-like calculus does not natively support 
parametric polymorphism, it is possible to encode parametric
polymorphism with abstract type members. Thus it can be argued that 
this calculus also supports intersection types and parametric
polymorphism. However, the type-soundness of a minimal Scala-like 
calculus with intersection types and parametric polymorphism is not
yet proven. Recently, some form of intersection
types have been adopted in object-oriented languages such as Scala,
Ceylon, and Grace. Generally speaking,
the most significant difference to \name is that in all previous systems
there is no explicit introduction construct like our merge operator. As shown in
Section~\ref{sec:application}, this feature is pivotal in supporting modularity
and extensibility because it allows dynamic composition of values.

\begin{comment}
only allow intersections of concrete types (classes),
whereas our language allows intersections of type variables, such as
\texttt{A \& B}. Without that vehicle, we would not be able to define
the generic \texttt{merge} function (below) for all interpretations of
a given algebra, and would incur boilerplate code:

\begin{lstlisting}{language=haskell}
let merge [A, B] (f: ExpAlg A) (g: ExpAlg B) = {
  lit (x : Int) = f.lit x ,, g.lit x,
  add (x : A & B) (y : A & B) =
    f.add x y ,, g.add x y
}
\end{lstlisting}
\end{comment}


\paragraph{Other type systems with intersection types.}
Intersection types date back to as early as Coppo et
al.~\cite{coppo1981functional}. As emphasized throughout the paper our 
work is inspired by Dunfield~\cite{dunfield2014elaborating}. He describes a similar approach to ours:
compiling a system with intersection types into ordinary $ \lambda $-calculus
terms. The major difference is that his system does not include parametric
polymorphism, while ours does not include unions. Besides, our rules are
algorithmic and we formalize a record system.
% Although similar in spirit,
% our elaboration typing is simpler: we require subtyping in the case of
% applications, thus avoiding the subsumption rule. Besides, our treatment
% combines the merge rules ($ k $ ranges over $ \{1, 2\} $)
% \inferrule 
% {\Gamma \turns e_k : \tau}
% {\Gamma \turns e_1 \mergeop e_2 : \tau}
% and the standard intersection-introduction rule
% \inferrule 
% {\Gamma \turns e : \tau_1 \andalso \Gamma \turns e : \tau_2}
% {\Gamma \turns e : \tau_1 \andop \tau_2}
% into one rule:
% \inferrule [Merge]
% {\Gamma \turns e_1 : \tau_1 \andalso \Gamma \turns e_2 : \tau_2}
% {\Gamma \turns e_1 \mergeop e_2 : \tau_1 \andop \tau_2}
Reynolds invented Forsythe~\cite{reynolds1997design} in the 1980s. Our merge
operator is analogous to his $ p_1, p_2 $. As Dunfield
has noted, in Forsythe merges can be only used unambiguously. 
For instance, it is not allowed in Forsythe to merge two functions.

%Castagna, and Dunfield describe
%elaborating multi-fields records into merge of single-field records.
% Reynolds and Castagna do not consider elaboration and Dunfield do not
% formalize elaborating records.

% Both Pierce and Dunfield's system include a subsumption rule, which states that
% if an expression has been inferred of type $ \tau $, then it is also of any
% supertype of $ \tau $. Our system does not have this rule.

Refinement
intersection~\cite{dunfield2007refined,davies2005practical,freeman1991refinement}
is the more conservative approach of adopting intersection types. It increases
only the expressiveness of types but not terms. But without a term-level
construct like ``merge'', it is not possible to encode various language
features. As an alternative to syntatic subtyping described in this paper,
Frisch et al.~\cite{frisch2008semantic} study semantic subtyping.

\paragraph{Languages for extensibility.}
To improve support for extensibility various researchers have proposed
new OOP languages or programming mechanisms. It is interesting to
note that design patterns such as object algebras or modular visitors
provide a considerably different approach to extensibility when
compared to some previous proposals for language designs for
extensibility. Therefore the requirements in terms of type system
features are quite different.  One popular approach is \emph{family
  polymorphism}~\cite{Ernst01family}, which allows whole class hierarchies to be
captured as a family of classes. Such a family can be later reused to
create a derived family with potentially new class members, and
additional methods in the existing classes.  \emph{Virtual
  classes}~\cite{ernst2006virtual} are a concrete realization of this idea, where a
container class can hold nested inner \emph{virtual} classes (forming
the family of classes). In a subclass of the container class, the
inner classes can themselfves be \emph{overriden}, which is why they
are called virtual. There are many language mechanisms that provide
variants of virtual classes or similar mechanisms~\cite{McDirmid01Jiazzi,Aracic06CaesarJ,Smaragdakis98mixin,nystrom2006j}. The work by 
Nystrom on \emph{nested intersection}~\cite{nystrom2006j} uses a
form of intersection types to support the composition of
families of classes. Ostermann's \emph{delegation layers}~\cite{Ostermann02dynamically}
use delegation for doing dynamic composition in a system 
with virtual classes. This in contrast with most other approaches 
that use class-based composition, but closer to the dynamic
composition that we use in \name.

\begin{comment}
In contrast to type systems for virtual classes 
and similar mechanisms, the goal of our work is to study the type
systems and basic language mechanism to better support such design patterns. 


 some researchers have designed new type
system features such as virtual classes~\cite{ernst2006virtual}, polymorphic
variants~\cite{garrigue1998programming}, while others have shown employing
programming pattern such as object algebras~\cite{oliveira2012extensibility} by
using features within existing programming languages. Both of the two approaches
have drawbacks of some kind. The first approach often involves heavyweight
designs, while the second approach still sacrifices the readability for
extensibility.
\bruno{fill me in with more details and more references!}
\end{comment}

% Intersection types have been shown to be useful in designing languages that
% support modularity.~\cite{nystrom2006j}

\paragraph{Extensible records.}

%\george{Record field deletion is also possible.}

% http://elm-lang.org/learn/Records.elm

Encoding records using intersection types appear in
Reynolds~\cite{reynolds1997design} and Castagna et
al.~\cite{castagna1995calculus}. Although Dunfield also discusses this idea in
his paper \cite{dunfield2014elaborating}, he only provides an implementation but
not formalization. Very similar to our treatment of elaborating records is
Cardelli's work~\cite{cardelli1992extensible} on translating a calculus, named
$ F_{\subtype \rho}$, with extensible records to a simpler calculus that without
records primitives (in which case is $ F_{\subtype} $). But he does not consider
encoding multi-field records as intersections; hence his translation is more
heavyweight. Crary~\cite{crary1998simple} uses intersection types and
existential types to address the problem that arises when interpreting method
dispatch as self-application. But in his paper, intersection types are not used
to encode multi-field records.

Wand~\cite{wand1987complete} started the work on extensible records and proposes
row types~\cite{wand1989type} for records. Cardelli and
Mitchell~\cite{cardelli1990operations} defined three primitive operations on
records that are similar to ours: \emph{selection}, \emph{restriction}, and
\emph{extension}. The merge operator in \name plays the same role as extension.
Following Cardelli and Mitchell's approach,
Leijen~\cite{leijen2004first,leijen2005extensible} define record update in terms
of restriction and extension. Both Leijen's system and ours allows records that contain
duplicate labels. Arguably Leijen's system is stronger. For example, it supports
passing record labels as arguments to functions. He also shows encoding an
intersection types using first-class labels. \bruno{check carefully
  this text!}

\begin{comment}
Chlipala's
\texttt{Ur}~\cite{chlipala2010ur} explains record as type level
constructs.\bruno{What is the point of citing Chlipala's paper?}
\end{comment}

\begin{comment}
Our system can be adapted to simulate systems that support extensible
records but not intersection of ordinary types like \texttt{Int} and
\texttt{Float} by allowing only intersection of record types.

$ \turnsrec \tau $ states that $ \tau $ is a record type, or the intersection of
record types, and so forth.

\inferrule [RecBase] {} {\turnsrec \recty l \tau}

\inferrule [RecStep]
{\turnsrec \tau_1 \andalso \turnsrec \tau_2}
{\turnsrec \tau_1 \andop \tau_2}

\inferrule [Merge']
{\Gamma \turns e_1 : \tau_1 \yields {E_1} \andalso \turnsrec \tau_1 \\
 \Gamma \turns e_2 : \tau_2 \yields {E_2} \andalso \turnsrec \tau_2}
{\Gamma \turns e_1 \mergeop e_2 : \tau_1 \andop \tau_2 \yields {\pair {E_1} {E_2}}}

Of course our approach has its limitation as duplicated labels in a record are
allowed. This has been discussed in a larger issue by
Dunfield~\cite{dunfield2014elaborating}.

R{\'e}my~\cite{remy1989type}
\end{comment}
