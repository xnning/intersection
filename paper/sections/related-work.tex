\section{Related Work}

Strictly speaking, the system we have described is not intersection in the
set-theoretic sense, but rather ``implicit pairs'' whose introduction is
explicit by using the merge operator and elimination is implicit.

\subsection{Intersection Types with Parametric Polymorphism}

The closest work to ours is Pierce's work on a prototype compiler for a language
with both intersection types, union types, and parametric polymorphism. The
important difference with our system is that in his language there is no
explicit introduction construct like our merge operator. However, as shown in
Section 3, this feature is critical in supporting modularity and extensibility
because it allows dynamic composition of values.

\subsection{Other Type Systems with Intersection Types}

Pierce does not include any proof.

% Write a comprehensive F_{<:}
% The dot calculus
% Foundations of path-dependent types

Duunfield describes an elaboration type system that translates unrestricted
intersection and unions into $\lambda$-calculus terms. Although similar in this
spirit, our translation uses coercion explicitly in the case of term
applications. Besides, our translation does not have the undeterministic
$ \texttt{merge} $ rule. Instead, it is captured by the deteterministic
\rulename{TrMerge} rule.

Reynolds invented Forsythe in the 1980s. Our merge operator is analogous to
$ p_1, p_2 $. Castagna, and Dunfield describe elaborating multi-fields records
into merge of single-field records. As Dunfield has noted, in Forsythe merges
can be only used unambiguously. For instance, it is not allowed in Forsythe to
merge two functions. Reynolds and Castagna do not consider elaboration and
Dunfield do not formalize elaborating records.

Both Pierce and Dunfield's system include a subsumption rule, which states that
if an expression has been inferred of type $ t $, then it is also of any
supertype of $ t $. Our system does not have this rule.

\subsection{Type Systems for Modularity}

\subsection{Extensible Records and Row Types}

The idea of encoding records using intersection types is due to Castagna and
Reynolds (1996). Although Dunfield considers this idea, he does not proceed to
formalize elaborating records.

\begin{lstlisting}
\end{lstlisting}

\begin{lstlisting}
\end{lstlisting}

\subsection{Other Related Work}

\begin{itemize}

\item{\bf Elaborating simply-typed lambda calculus}

  Dunfield has introduced a type system with intersection polymorphism but no
  parametric polymorphism.

\end{itemize}

Nystrom et. al. OOPSLA 06

Applications:

- Object/Fold Algebras. How to support extensibility in an easier way.

See Datatypes a la Carte

- Mixins

- Lenses? Can intersection types help with lenses? Perhaps making the
types more natural and easy to understand/use?

- Embedded DSLs? Extensibility in DSLs? Composing multiple DSL interpretations?

http://www.cs.ox.ac.uk/jeremy.gibbons/publications/embedding.pdf
