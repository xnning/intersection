\section{Implementation}

We implemented all the functionalities of the \name as a contribution to the
open source community. Besides, we built a language on top of \name to
facilitate programming. We adopted a three-phase design to compile source
programs built into System $ F $ terms.

\begin{enumerate}
\item A \emph{type checking} phase that checks the usage of \name and other
  source features against an abstract syntax tree that follows strictly with the
  source syntax. We intentionally made type checking happen first, so as to make
  the error messages relevant for programmers, although such an approach has
  obviously complicated the implementation.

\item A \emph{desugaring} phase that translates well-typed source terms into
  \name terms. A number of source-level features such as multi-field records,
  recursive \texttt{let} definitions, type aliases are rewritten at this phase.
  The resulting program is just \name expressions with other minor features.

\item A \emph{compilation} phase that translates well-typed \name terms
  into System $ F $ ones.
\end{enumerate}

Phase 3 is what we have formalized in this paper.

\paragraph{Type aliases.}

We implement \emph{type alias} as in Scala and Haskell.

In fact, to make future adaptations easier, we implemented this feature on top
of the more general System $ F_{\omega} $.


\begin{lstlisting}
type T[A, B] = t; e
\end{lstlisting}

\subsection{Optimization.}

The reader might argue that translating merges as pairs can be inefficient in
practice. That is indeed the case. For example, the ternary merge
\begin{lstlisting}
1,,2,,3 
\end{lstlisting}
is elaborated into a nested tuple
\begin{lstlisting}
((1,2),3)
\end{lstlisting}
as the merge operator associates to the left.

Another issue is that every time a coercion is generated, an application will be
incurred at run-time.
