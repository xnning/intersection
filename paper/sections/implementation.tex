\section{Implementation}

We implemented the functionalities of the \name as a contribution to the open
source community. Besides, we built a source language on top of \name to
facilitate programming. We adopted a three-phase design to compile source
programs built into System $F$ terms.
\begin{enumerate}
\item A \emph{type checking} phase that checks the usage of \name features and
  other source language features against an abstract syntax tree that follows
  strictly with the source syntax.
  % We intentionally made type checking happen first, so as to make
  % the error messages relevant for programmers, although such an approach has
  % obviously complicated the implementation.

\item A \emph{desugaring} phase that translates well-typed source terms into
  \name terms. A number of source-level features including multi-field records,
  recursive \texttt{let} definitions, type aliases are rewritten at this phase.
  The resulting program is just \name expressions with other minor features.

\item A \emph{translation} phase that translates well-typed \name terms into
  System $F$ ones.
\end{enumerate}
Phase 3 is what we have formalized in this paper.

\paragraph{Other features.} We implemented \emph{type alias} as in Scala and
Haskell (called ``type synonyms''). In fact, to make future adaptations easier,
our source language is an extension of the more general System $F_{\omega}$.

\paragraph{Identity functions.} Our translation inserts identity functions
whenever subtyping or record operation occurs. But that is just for the sake of
the simplicity of formalization. In our current implementation, we introduced a
partial evaluator to eliminate the identity functions as another phase after the
translation phase. In another version of our implementation, partial evaluation
is weaved into the process of translation and unnecessary identity functions are
removed simultaneously during the translation.
