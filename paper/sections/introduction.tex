\section{Introduction}

\begin{minted}{Ruby}
  class Foo
    def init
      pi = Math::PI
      @var = "Pi is approx. #{pi}"
    end
  end
\end{minted}

-- Compare Scala:
-- merge[A,B] = new A with B

-- type IEval  = { eval :  Int }
-- type IPrint = { print : String }

-- F[\_]

We present a polymorphic calculus containing intersection types and records, and show
how this language can be used to solve various common tasks in functional
programming in a nicer way.

Intersection types provides a power mechanism for functional programming, in
particular for extensibility and allowing new forms of composition.

Prototype-based programming is one of the two major styles of object-oriented
programming, the other being class-based programming which is featured in
languages such as Java and C\#. It has gained increasing popularity recently
with the prominence of JavaScript in web applications. Prototype-based
programming supports highly dynamic behaviors at run time that are not possible
with traditional class-based programming. However, despite its flexibility,
prototype-based programming is often criticized over concerns of correctness and
safety. Furthermore, almost all prototype-based systems rely on the fact that
the language is dynamically typed and interpreted.

In summary, the contributions of this paper are:

\begin{itemize}

\item{elaboration typing rules which given a source expression with intersection
    types, typecheck and translate it into an ordinary F term. Prove a type
    preservation result: if a term $e$ has type $\tau$ in the source language,
    then the translated term $|e|$ is well-typed and has type $|\tau|$ in the
    target language.}

\item{present an algorithm for detecting incoherence which can be very important
    in practice.}

\item{explores the connection between intersection types and object algebra by
    showing various examples of encoding object algebra with intersection
    types.}

\end{itemize}
