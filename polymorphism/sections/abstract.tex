The combination of \emph{intersection types}, a \emph{merge operator} 
and \emph{parametric polymorphism} enables
important applications for programming. However such combination makes
it hard to achieve the desirable property of a \emph{coherent
semantics}: all valid reductions for the same expression should
have the same value. Recent work proposed \emph{disjoint
intersections types} as a means to ensure coherence in a simply typed
setting. However, the addition of parametric polymorphism was not
studied.

This paper presents \name: a calculus with 
\emph{disjoint intersection types}, a variant of \emph{parametric polymorphism} and a
\emph{merge operator}. \name is both type-safe and
coherent. The key difficulty in adding polymorphism is that, when a type variable occurs in an intersection
type, it is not statically known whether the instantiated type will be
disjoint to other components of the intersection. To address this
problem we propose \emph{disjoint polymorphism}: a constrained form of
parametric polymorphism, which allows programmers to specify
disjointness constraints for type variables. With disjoint
polymorphism the calculus remains very flexible in terms of programs
that can be written, while retaining coherence.

\begin{comment}
Coherence is achieved by ensuring that intersection types
are \emph{disjoint}. The approach works in the presence of parametric
polymorphism. However, parametric polymorphism makes the problem of coherence
significantly harder. When a type variable occurs in an intersection
type, it is not statically known whether the instantiated type will
be disjoint to other components of the intersection.
To address this problem we propose \emph{disjoint polymorphism}: a
constrained form of parametric polymorphism, that allows programmers
to specify disjointness constraints for type variables. With disjoint
polymorphism the calculus remains very flexible in terms of programs
that can be written with intersection types, while retaining coherence.
\end{comment}