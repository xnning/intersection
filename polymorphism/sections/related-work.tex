\section{Related Work}
\label{sec:related-work}
\paragraph{Intersection types, polymorphism and the merge operator.}
To our knowlegde no previous work has been done in a calculus which
includes parametric polymorphism, intersection types and a
\emph{merge} operator. However, various simply typed systems with 
intersections types and a merge operator have been studied in the past.
The merge operator was first introduced 
by Reynold's in the Forsythe~\cite{reynolds1997design}
language. The merge operator in Forsythe is coherent~\cite{reynolds1991coherence}, but it
has various restrictions to ensure coherence that are lifted in
\name. For example Forsythe merges cannot contain more than one function.
Castagna et al.~\cite{Castagna92calculus} studied a coherent calculus with a
special merge operator on that works on functions only. The goal was to model
\emph{function overloading}. Unlike Reynold's operator, multiple 
functions are allowed in merges, but non-functional types are
forbidden. Pierce's $F_\wedge$~\cite{pierce1991programming2} also 
studied a $glue$ operator that was esseintially a merge. However
Pierce's calculus is incoherent.
\begin{comment}
Concerning simply-typed systems, the oldest work dates back to the 1980s,
when Reynolds invented Forsythe~\cite{reynolds1997design}, with his
merge-like operator being noted as $p_1, p_2$.
Even though his calculus was proven to be coherent~\cite{reynolds1991coherence},
it uses a form of biased choice by having precedence in typing rules for intersections.
His system is also arguably more restrictive than ours since it forbids, for instance,
the merge of two functions.

Pierce~\cite{pierce1991programming2} made a comprehensive review
of coherence, but he was unable to prove it for his $F_\wedge$ calculus. 
He introduced a primitive $\code{glue}$ function as
a language extension which corresponds to our merge operator. 
However, users can ``glue'' two arbitrary values, which may lead to incoherence.

Castagna et al.~\cite{Castagna92calculus} proposed $\lamint$ to study
the overloading problem for functions. 
They have a special merge for functions and also a special function application
for these merges.
Their calculus is coherent and they also employ well-formedness conditions on their 
(functional) merges. 
These conditions are incompatible with systems like ours -- which allow arbitrary
intersections -- and thus are hard to compare in terms of expressiveness.
This means that $\lamint$ accepts different functional merges than \name and 
vice-versa.
\end{comment}
More recently, Dunfield~\cite{dunfield2014elaborating} formalised a system with
intersection types and a merge operator with a type-directed translation to the
simply-typed lambda calculus with pairs. Although Dunfield's calculus
is incoherent, it was the inspiration for the
\oldname~\cite{oliveira16disjoint}, which \name extends.

\oldname solves the coherence problem for a calculus similar to Dunfield's, by requiring that
intersection types can only be composed of \emph{disjoint} types.
For simplicity purposes, \oldname uses a specification for disjointness, which says that two types are 
disjoint if they do not share a common supertype.
\name does not use such specification as its adaptation to polymorphic types would require
using unification, making the specification arguably more complex than
the algorithmic rules (and defeating the purpose of having a
specification). \name's notion of disjointness is based on \oldname's more
general notion of disjointness concerning $\top$ types, called $\top$-disjointness, which 
states that two types $A$ and $B$ are disjoint if two conditions are satisfied:
%\begin{definition}[$\top$-disjointness]
%  Two types $A$ and $B$ are disjoint
%  (written $A \disjoint B$) if the following two conditions are satisfied:
\begin{enumerate}
  \item $(\text{not}~\toplike{A})~\text{and}~(\text{not}~\toplike{B}) $
  \item $\forall C.~\text{if}~ A \subtype C~\text{and}~B \subtype C~\text{then}~\toplike{C}$
\end{enumerate}
%\end{definition}
\noindent A major difference between the two calculus, is that $\top$-disjointness does not allow
$\top$ in intersections, while our calculus allows it.
In other words, condition (1) is not imposed by \name.
As a consequence, the set of well-formed \emph{top-like} types is a superset of \oldname's.
This is covered in greater detail in Section~\ref{sec:disjoint}.

\paragraph{Intersection types and polymorphism, without the merge operator.}
Recently, Castagna~\textit{et al.}~\cite{Castagna:2014} studied a very interesting
and coherent calculus that has polymorphism and set-theoretic type connectives (such as
intersections, unions, and negations). 
However, their calculus is based on a semantic subtyping relation due to their 
interpretation of intersection types.
The difference to our calculus, is that their intersections are used between function
types, allowing overloading (i.e. branching) of types.
For example, they can express a function whose domain is defined on any type, but executes different
code depending on that type:

\[
\lambda^{(\tyint \to \tybool) \wedge (\alpha \backslash \tyint \to \alpha \backslash \tyint)} 
x. (x \in \tyint) ? (x \text{ mod } 2) = 0 : x
\]

\noindent In our system we cannot express some of these intersections, namely the ones that
do not have disjoint co-domains.
However, \name accepts other kind of intersections which are not possible to express
in their calculus, as in merges of type $(\tyint \to \tybool) \inter (\tyint \to \tyint)$.
Just as~\cite{Castagna92calculus}, their work is focused on combining 
intersections with functions, whereas \name is concerned with merges between
arbitrary types.

\joao{elaborate more on this paragraph?}
Going in the direction of higher
kinds, Compagnoni and Pierce~\cite{compagnoni1996higher} added
intersection types to System $ F_{\omega} $ and used a new calculus,
$ F^{\omega}_{\wedge} $, to model multiple inheritance. 
In their system, types include the construct of intersection of types of the
same kind $ K $. 
Davies and Pfenning~\cite{davies2000intersection} studied the interactions between
intersection types and effects in call-by-value languages. 
They proposed a ``value restriction'' for intersection types, similar to
value restriction on parametric polymorphism.

Intersection types have also been adopted in object-oriented languages such as Scala,
Ceylon, Grace, Flow or TypeScript.
Generally speaking, the main difference between these languages and \name is the lack
of explicit instantiation through the merge operator.
This leads to several ad-hoc solutions at defining such operator in those languages,
as it has been discussed in Section~\ref{sec:intro} and Section~\ref{subsec:mixins}. 

\paragraph{Extensible records.}%
%http://elm-lang.org/learn/Records.elm
Encoding records using intersection types first appeared in 
Reynolds~\cite{reynolds1997design} and Castagna et al.~\cite{castagna1995calculus}. 
Although Dunfield also discussed this idea in his paper \cite{dunfield2014elaborating}, 
he only provided an implementation but not a formalization. 
A similar to our treatment of elaborating records is
Cardelli's work~\cite{cardelli1992extensible} on translating a calculus
with extensible records ($ F_{\subtype \rho}$) to a simpler calculus without
records primitives ($ F_{\subtype} $). 
However, he does not encode multi-field records as intersections; hence his translation is more
heavyweight. 
Crary~\cite{crary1998simple} used intersection types and
existential types to address the problem that arises when interpreting method
dispatch as self-application. 
But also in his paper, intersection types are not used to encode multi-field records.

Wand~\cite{wand1987complete} started the work on extensible records and proposed
row types~\cite{wand1989type} for records, together with a \emph{concatenation}
operator, which is standard in many calculi with extensible records
\cite{harper1991record,remy1992typing,wand1989type,sulzmann97designing,makholm05type,pottier2003constraint}.
Cardelli and Mitchell~\cite{cardelli1990operations} defined three primitive operations on
records that are standard in type-systems with record types: 
\emph{selection}, \emph{restriction}, and \emph{extension}. 
Most calculi are based on these three primitive operators (especially extension), such as
\cite{remy1993type,gaster1996polymorphic,jones99lightweight,leijen2004first,leijen2005extensible,blume2006extensible}.
%Following Cardelli and Mitchell's approach, which is based on extension and row types, 
%More recently, Leijen formalized two neat systems~\cite{leijen2004first,leijen2005extensible}.
%Just like our calculus, his systems allow records that contain duplicate labels, without the need to 
%override 
%However, Leijen's system is more sophisticated, since it relies on shadowing of labels. 
%Also, it supports passing record labels as arguments to functions. 
%He also showed an encoding of intersection types using first-class labels.
The merge operator in \name plays the same role as the concatenation operator, as its components 
may contain any (well-formed and disjoint) types.
One problem with concatenation is that typically systems are hard to use in practise, due to the
complicated set of constraints/filters which might arise from combining several, possibly 
polymorphic, records~\cite{leijen2005extensible}.
%All of the aforementioned record calculi are entirely concerned with the conceptual and implementation 
%issues of type-systems with records, using techniques as row types, predicates, flags, qualified types,
%cases, amongst others.
However, in \name, the implementation of records instead piggybacks on the disjointness
and intersection types features of the type-system.
The constraints on labels are encoded using disjoint quantification, and the coherent
type-system will statically ensure that records and their operations are safe.
This means that technical concerns w.r.t. records, i.e type inference, efficiency, etc.; are to be dealt with the 
compilation of disjoint intersection types as opposed to compiling extensible records directly.
%Thus, what is expressed in other systems by means of row types, predicates, flags, qualified types, etc; 
%in \name it is expressed by the use of disjoint intersection types.

% Chlipala's
% \texttt{Ur}~\cite{chlipala2010ur} explains record as type level
% constructs.\bruno{What is the point of citing Chlipala's paper?}

% Our system can be adapted to simulate systems that support extensible
% records but not intersection of ordinary types like \texttt{Int} and
% \texttt{Float} by allowing only intersection of record types.
%
% $ \turnsrec A $ states that $ A $ is a record type, or the intersection of
% record types, and so forth.
%
% \inferrule [RecBase] {} {\turnsrec \recordType l A}
%
% \inferrule [RecStep]
% {\turnsrec A_1 \andalso \turnsrec A_2}
% {\turnsrec A_1 \inter A_2}
%
% \inferrule [Merge']
% {\Gamma \turns e_1 : A_1 \yields {E_1} \andalso \turnsrec A_1 \\
%  \Gamma \turns e_2 : A_2 \yields {E_2} \andalso \turnsrec A_2}
% {\Gamma \turns e_1 \mergeop e_2 : A_1 \inter A_2 \yields {\pair {E_1} {E_2}}}
%
% Of course our approach has its limitation as duplicated labels in a record are
% allowed. This has been discussed in a larger issue by
% Dunfield~\cite{dunfield2014elaborating}.
%
% R{\'e}my~\cite{remy1989type}

\paragraph{Traits and mixins.}
Traits \cite{ducasse2006traits,fisher04atyped,odersky2005scalable} and mixins 
\cite{bracha1990mixin,ancona96algebraic,flatt1998classes,findler1998modular,ancona2003jam,bettini2004corecalculus} 
have become very popular in object-oriented languages. 
They allow a kind of object composition which is not allowed in
object-oriented type-systems with support for single inheritance.
One of the main differences between traits and mixins are the way in
which ambiguity of names is dealt.
The former usually rejects programs which compose classes with 
conflicting methods, whereas the latter assumes a resolution strategy,
usually language dependent. 
Our example demonstrated in Section~\ref{subsec:records} suggests that
\name can be used in place of traits, due to the semantics of the merge
operator.
However there is a subtle difference between traits and \name: equal 
names/labels are always rejected in the former, whereas in the latter only rejects
when their types are disjoint.
\joao{ref CZ? (similar spec to disjointess)}

\begin{comment}

\paragraph{Coherence}
Reynolds invented Forsythe~\cite{reynolds1997design} in the 1980s. Our
merge operator is analogous to his operator $p_1, p_2 $. Forsythe has
a coherent semantics. The result was proved formally by
Reynolds~\cite{reynolds1991coherence} in a lambda calculus with
intersection types and a merge operator. However there are two key
differences to our work. Firstly the way coherence is ensured is
rather ad-hoc. He has four different typing rules for the merge
operator, each accounting for various possibilities of what the types
of the first and second components are. In some cases the meaning of
the second component takes precedence (that is, is biased) over the
first component. The set of rules is restrictive and it forbids, for
instance, the merge of two functions (even when they a provably
disjoint). In contrast, disjointness in \name has a well-defined
specification and it is quite flexible. Secondly, Reynolds calculus
does not support universal quantification. It is unclear to us whether
his set of rules would still ensure disjointness in the presence of
universal quantification. Since some biased choice is allowed in
Reynold's calculus the issues illustrated in Section~\ref{subsec:polymorphism} could be a problem.

Pierce~\cite{pierce1991programming2} made a comprehensive review
of coherence, especially on Curien and Ghelli~\cite{curienl1990coherence} and
Reynolds' methods of proving coherence; but he was not able to prove coherence
for his $F_\wedge$ calculus. He introduced a primitive $\code{glue}$ function as
a language extension which corresponds to our merge operator. However, in his
system users can ``glue'' two arbitrary values, which can lead to incoherence.

Our work is largely inspired by
Dunfield~\cite{dunfield2014elaborating}. He described a similar
approach to ours: compiling a system with intersection types and a
merge operator into
ordinary $ \lambda $-calculus terms with pairs. One major difference
is that his system does not include parametric polymorphism, while
ours does not include unions. The calculus presented in
Section~\ref{sec:fi} can be seen as a relatively straightforward
extension of Dunfield's calculus with parametric
polymorphism. However, as acknowledged by Dunfield, his calculus lacks
of coherence. He discusses the issue of coherence throughout his
paper, mentioning biased choice as an option (albeit a rather
unsatisfying one). He also mentioned that the notion of disjoint
intersection could be a good way to address the problem, but he did not
pursue this option in his work. In contrast to his work, we developed a type
system with disjoint intersection types and proposed disjoint
quantification to guarantee coherence in our calculus.

% \url{http://homepages.inf.ed.ac.uk/gdp/publications/Sub_Par.pdf}

% \cite{plotkin1994subtyping}

% Also discussed intersection types!~\cite{malayeri2008integrating}.

% Pierce Ph.D thesis: F<: + /|
%        technical report: F + /|, closer to ours

% \cite{barbanera1995intersection}

\paragraph{Intersection types with polymorphism.}
Our type system combines intersection types and parametric polymorphism. Closest
to us is Pierce's work~\cite{pierce1991programming1} on a prototype
compiler for a language with both intersection types, union types, and
parametric polymorphism. Similarly to \name in his system universal
quantifiers do not support bounded quantification. However Pierce did not try to prove any
meta-theoretical results and his calculus does not have a merge
operator.  Pierce also studied a system where both intersection
types and bounded polymorphism are present in his Ph.D.
dissertation~\cite{pierce1991programming2} and a 1997
report~\cite{pierce1997intersection}.

Going in the direction of higher
kinds, Compagnoni and Pierce~\cite{compagnoni1996higher} added
intersection types to System $ F_{\omega} $ and used the new calculus,
$ F^{\omega}_{\wedge} $, to model multiple inheritance. In their
system, types include the construct of intersection of types of the
same kind $ K $. Davies and Pfenning
\cite{davies2000intersection} studied the interactions between
intersection types and effects in call-by-value languages. And they
proposed a ``value restriction'' for intersection types, similar to
value restriction on parametric polymorphism. Although they proposed a system with
parametric polymorphism, their subtyping rules are significantly different from ours,
since they consider parametric polymorphism
as the ``infinit analog'' of intersection polymorphism.

Recently,
Castagna et al.~\cite{Castagna:2014} studied an very expressive calculus that
has polymorphism and set-theoretic type connectives (such as intersections,
unions, and negations). As a result, in their calculus one is also able to
express a type variable that can be instantiated to any type other than
$\code{Int}$ as $\alpha \setminus \code{Int}$, which is syntactic sugar for
$\alpha \wedge \neg \code{Int}$. As a comparison, such a type will need a
disjoint quantifier, like $\fordis \alpha {\code{Int}} \alpha$, in our system.
Unfortunately their calculus does not include a merge operator like ours.

There have been attempts to provide a foundational calculus
for Scala that incorporates intersection
types~\cite{amin2014foundations,amin2012dependent}.
Although the minimal Scala-like calculus does not natively support
parametric polymorphism, it is possible to encode parametric
polymorphism with abstract type members. Thus it can be argued that
this calculus also supports intersection types and parametric
polymorphism. However, the type-soundness of a minimal Scala-like
calculus with intersection types and parametric polymorphism is not
yet proven. Recently, some form of intersection
types has been adopted in object-oriented languages such as Scala,
Ceylon, and Grace. Generally speaking,
the most significant difference to \name is that in all previous systems
there is no explicit introduction construct like our merge operator. As shown in
Section~\ref{subsec:OAs}, this feature is pivotal in supporting modularity
and extensibility because it allows dynamic composition of values.

%\begin{comment}
only allow intersections of concrete types (classes),
whereas our language allows intersections of type variables, such as
\texttt{A \& B}. Without that vehicle, we would not be able to define
the generic \texttt{merge} function (below) for all interpretations of
a given algebra, and would incur boilerplate code:

\begin{lstlisting}
let merge [A, B] (f: ExpAlg A) (g: ExpAlg B) = {
  lit (x : Int) = f.lit x ,, g.lit x,
  add (x : A & B) (y : A & B) =
    f.add x y ,, g.add x y
}
\end{lstlisting}
%\end{comment}


\paragraph{Other type systems with intersection types.}
% Although similar in spirit,
% our elaboration typing is simpler: we require subtyping in the case of
% applications, thus avoiding the subsumption rule. Besides, our treatment
% combines the merge rules ($ k $ ranges over $ \{1, 2\} $)
% \inferrule
% {\Gamma \turns e_k : A}
% {\Gamma \turns e_1 \mergeop e_2 : A}
% and the standard intersection-introduction rule
% \inferrule
% {\Gamma \turns e : A_1 \andalso \Gamma \turns e : A_2}
% {\Gamma \turns e : A_1 \inter A_2}
% into one rule:
% \inferrule [Merge]
% {\Gamma \turns e_1 : A_1 \andalso \Gamma \turns e_2 : A_2}
% {\Gamma \turns e_1 \mergeop e_2 : A_1 \inter A_2}
%Castagna, and Dunfield describe
%elaborating multi-fields records into merge of single-field records.
% Reynolds and Castagna do not consider elaboration and Dunfield do not
% formalize elaborating records.
% Both Pierce and Dunfield's system include a subsumption rule, which states that
% if an term has been inferred of type $ A $, then it is also of any
% supertype of $ A $. Our system does not have this rule.
Refinement
intersection~\cite{dunfield2007refined,davies2005practical,freeman1991refinement}
is the more conservative approach of adopting intersection types. It increases
only the expressiveness of types but not terms. But without a term-level
construct like ``merge'', it is not possible to encode various language
features. As an alternative to syntactic subtyping described in this paper,
Frisch et al.~\cite{frisch2008semantic} studied semantic subtyping. Semantic
subtyping seems to have important advantages over syntactic subtyping. One
worthy avenue for future work is to study languages with intersection types
and merge operator in a semantic subtyping setting.

%\begin{comment}
\paragraph{Extensibility.} One of our motivations to study systems
with intersections types is to better understand the
type system requirements needed to address extensibility problems.
A well-known problem in programming languages is the Expression
Problem~\cite{wadler1998expression}. In recent years there have been
various solutions to the Expression Problem in the literature. Mostly
the solutions are presented in a specific language, using the language
constructs of that language. For example, in Haskell, type classes~\cite{WadlerB89}
can be used to implement type-theoretic encodings of
datatypes~\cite{Hinze:2006}. It has been shown~\cite{finally-tagless}
that, when encodings of datatypes are modeled with type classes,
the subclassing mechanism of type classes can be used to achieve
extensibility and reuse of operations. Using such techniques provides
a solution to the Expression Problem. Similarly, in OO languages with
generics, it is possible to use generic interfaces and classes to
implement type-theoretic encodings of datatypes. Conventional
subtyping allows the interfaces and classes to be extended, which can
also be used to provide extensibility and reuse of operations. Using
such techniques, it is also possible to solve the Expression Problem
in OO languages~\cite{oliveira09modular,oliveira2012extensibility}.
It is even possible to solve the Expression Problem in theorem provers
like Coq, by exploiting Coq's type class mechanism~\cite{DelawareOS13}.
Nevertheless, although there is a clear connection between all those
techniques and type-theoretic encodings of datatypes, as far as we
know, no one has studied the expression problem from a more
type-theoretic point of view. As shown in Section~\ref{subsec:OAs}, a system
with intersection types, parametric polymorphism, the merge operator
and disjoint quantification can be used to explain type-theoretic
encodings with subtyping and extensibility.
%\end{comment}

%\begin{comment}
To improve support for extensibility various researchers have proposed
new OOP languages or programming mechanisms. It is interesting to
note that design patterns such as object algebras or modular visitors
provide a considerably different approach to extensibility when
compared to some previous proposals for language designs for
extensibility. Therefore the requirements in terms of type system
features are quite different.  One popular approach is \emph{family
  polymorphism}~\cite{Ernst01family}, which allows whole class hierarchies to be
captured as a family of classes. Such a family can be later reused to
create a derived family with potentially new class members, and
additional methods in the existing classes.  \emph{Virtual
  classes}~\cite{ernst2006virtual} are a concrete realization of this idea, where a
container class can hold nested inner \emph{virtual} classes (forming
the family of classes). In a subclass of the container class, the
inner classes can themselves be \emph{overriden}, which is why they
are called virtual. There are many language mechanisms that provide
variants of virtual classes or similar mechanisms~\cite{McDirmid01Jiazzi,Aracic06CaesarJ,Smaragdakis98mixin,nystrom2006j}. The work by
Nystrom on \emph{nested intersection}~\cite{nystrom2006j} uses a
form of intersection types to support the composition of
families of classes. Ostermann's \emph{delegation layers}~\cite{Ostermann02dynamically}
use delegation for doing dynamic composition in a system
with virtual classes. This in contrast with most other approaches
that use class-based composition, but closer to the dynamic
composition that we use in \namedis.
In contrast to type systems for virtual classes
and similar mechanisms, the goal of our work is to study the type
systems and basic language mechanism to better support such design patterns.
 some researchers have designed new type
system features such as virtual classes~\cite{ernst2006virtual}, polymorphic
variants~\cite{garrigue1998programming}, while others have shown employing
programming pattern such as object algebras~\cite{oliveira2012extensibility} by
using features within existing programming languages. Both of the two approaches
have drawbacks of some kind. The first approach often involves heavyweight
designs, while the second approach still sacrifices the readability for
extensibility.
\bruno{fill me in with more details and more references!}
%\end{comment}
% Intersection types have been shown to be useful in designing languages that
% support modularity.~\cite{nystrom2006j}

\end{comment}


