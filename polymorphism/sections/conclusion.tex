\section{Conclusion and Future Work}
\label{sec:conclusion}

This paper described \name: a System $F$-based language that combines
intersection types, parametric polymorphism and a merge operator.
The language is proved to be type-safe and coherent.
To ensure coherence the type system accepts only
disjoint intersections. To provide flexibility in the presence of parametric polymorphism,
universal quantification is extended with
disjointness constraints. We believe that disjoint intersection types
and disjoint quantification are intuitive, and at the same time
flexible enough to enable practical applications.

%We implemented the core functionalities of the \namedis as part of a JVM-based
%compiler. Based on the type system of \namedis, we have built an ML-like
%source language compiler that offers interoperability with Java (such as object
%creation and method calls). The source language is loosely based on the more
%general System $F_{\omega}$ and supports a
%number of other features, including records, mutually recursive
%\code{let} bindings, type aliases, algebraic data types, pattern matching, and
%first-class modules that are encoded using \code{letrec} and records.

For the future, we intend to create a prototype-based statically typed
source language based on \name.  We are also interested in extending
our work to systems with union types and a $\bot$ type. Union types
are also widely used in languages such as Ceylon or Flow, but
preserving coherence in the presence of union types is
challenging. The naive addition of $\bot$ seems to be problematic. 
The proofs for \name rely on the invariant that a type variable $\alpha$ can never be disjoint 
to another type that contains $\alpha$. The addition of $\bot$ breaks
this invariant, allowing us to derive, for example, $\jdis \Gamma
\alpha \alpha$.
Finally, we could study a similar system with implicit polymorphism.
Such system would require some changes in the subtyping and disjointness relations.
For instance, subtyping should allow  
${\for \alpha {\alpha \to \alpha}} \subtype \tyint \to \tyint$.
Consequently, the disjointness relation would have to be modified,
since valid statements in \name such as 
$\jdis \Gamma {\for \alpha {\alpha \to \alpha}} {\tyint \to \tyint}$ 
would no longer hold under the more powerful subtyping relation. 
