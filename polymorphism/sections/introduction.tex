\section{Introduction}
\label{sec:intro}

Intersection types~\cite{coppo1981functional,Pottinger80type} are a popular language feature for
modern languages, such as Microsoft's TypeScript~\cite{typescript}, 
Redhat's Ceylon~\cite{ceylon}, Facebook's Flow~\cite{flow} and
Scala~\cite{scala-overview}.
%\footnote{Limited forms of intersection types also exist in
%recent versions of Java~\cite{}.} 
In those languages a typical use of intersection
types, which has been known for a long time~\cite{comppier96}, 
is to model the subtyping aspects of OO-style multiple inheritance. 
For example, the following Scala declaration:

\begin{lstlisting}
class A extends B with C
\end{lstlisting}

\noindent says that the class \lstinline{A} implements \emph{both}
\lstinline{B} \emph{and} \lstinline{C}. The fact that \lstinline{A}
implements two interfaces/traits is captured by an intersection type
between \lstinline{B} and \lstinline{C} (denoted in Scala by
\lstinline{B with C}). Unlike a language like Java, where
\lstinline{implements} (which plays a similar role to
\lstinline{with}) would be a mere keyword, in Scala
intersection types are first class. For example, it is possible to define 
functions such as:

\begin{lstlisting}
def narrow(x : B with C) : B = x
\end{lstlisting}

\noindent taking an argument with an intersection
type \lstinline{B with C}. 

The existence of first-class intersections has lead to the
discovery of other interesting applications of intersection types. 
For example, TypeScript's documentation motivates intersection
types\footnote{\url{https://www.typescriptlang.org/docs/handbook/advanced-types.html}}
as follows:

\begin{quote}
\emph{You will mostly see intersection types used for mixins and other
concepts that don’t fit in the classic object-oriented mold. (There are a lot of these in JavaScript!)}
\end{quote}

\noindent Two points are worth emphasizing. Firstly, 
intersection types are being used to model concepts that are not
like the classical (class-based) object-oriented programming. Indeed, 
being a prototype-based language, JavaScript has a much more dynamic 
notion of object composition compared to class-based languages:
objects are composed at run-time, and their types are not necessarily
statically known. Secondly, the use of intersection types in
TypeScript is inspired by common programming patterns in the
(dynamically typed) JavaScript. This hints that intersection types are 
useful to capture certain programming patterns that are out-of-reach for
more conventional type systems without intersection types.

Central to TypeScript's use of intersection types for modelling such a
dynamic form of mixins is the function:

\begin{lstlisting}
function extend<T, U>(first: T, second: U) : T & U {...}
\end{lstlisting}

\noindent The name \emph{extend} is given as an analogy to the
\emph{extends} keyword commonly used in OO languages like Java.
The function takes two objects (\lstinline{first} and
\lstinline{second}) and produces an object with the intersection of
the types of the original objects. The implementation of
\emph{extend} relies on low-level (and type-unsafe) features of 
JavaScript. When a method is invoked on the new object resulting from 
the application of \lstinline{extend}, the new object tries to use the 
\lstinline{first} object to answer the method call and, if the method
invocation fails, it then uses the \lstinline{second} object to answer
the method call. 

The \emph{extend} function is essentially an encoding of the
\emph{merge operator}. The merge operator is used on some
calculi~\cite{reynolds1997design,reynolds1991coherence,castagna1995calculus,dunfield2014elaborating,oliveira16disjoint} as
an introduction form for intersection types. Similar encodings to
those in TypeScript have been proposed for Scala to enable
applications where the merge operator also plays a fundamental
role~\cite{oliveira2013feature,Rendel14OA}. 
Unfortunately, the merge operator is not directly
supported by TypeScript, Scala, Ceylon or Flow. There are two possible
reasons for such lack of support. One reason is simply that the merge operator is
not well-known: many calculi with intersection types in the literature
do not have explicit introduction forms for intersection types. The
other reason is that, while powerful, the merge operator is known
to introduce \emph{(in)coherence} problems~\cite{reynolds1991coherence,dunfield2014elaborating}.  
If care is not
taken, certain programs using the merge operator do not have a unique
semantics, which significantly complicates reasoning about programs.

Solutions to the problem of coherence in the presence 
of a merge operator exist for simply typed
calculi~\cite{reynolds1997design,reynolds1991coherence,castagna1995calculus,oliveira16disjoint}, 
but no prior work addresses polymorphism. Most recently, we
proposed using \emph{disjoint intersection types}~\cite{oliveira16disjoint} to 
guarantee coherence in \oldname: a simply typed calculus with intersection types and a merge 
operator. The key idea is to allow only disjoint types in
intersections. If two types are disjoint then there is no ambiguity in
selecting a value of the appropriate type from an intersection, 
guaranteeing coherence.

Combining parametric polymorphism with disjoint intersection
types, while retaining enough flexibility for practical applications,
is non-trivial. The key issue is that when a type variable occurs in an intersection
type it is not statically known whether the instantiated types will
be disjoint to other components of the intersection.
A naive way  to add polymorphism is to forbid 
type variables in intersections, since they may be instantiated with 
a type which is not disjoint to other types in an intersection.
Unfortunately this is too conservative and prevents many useful 
programs, including the \lstinline{extend} function, which uses an
intersection of two type variables \lstinline{T} and \lstinline{U}. 

%Clearly this seems to indicate that a more
%foundational approach is lacking. What is currently missing is a
%foundational core calculus that can capture the key ideas behind the
%various solutions.

This paper presents \name: a core calculus with 
\emph{disjoint intersection types}, a variant of \emph{parametric polymorphism} and a
\emph{merge operator}. The key innovation in the calculus is \emph{disjoint polymorphism}: a
constrained form of parametric polymorphism, which allows programmers
to specify disjointness constraints for type variables. With disjoint
polymorphism the calculus remains very flexible in terms of programs
that can be written with intersection types, while retaining
coherence. In \name the \lstinline{extend} function is implemented
as follows:

\begin{lstlisting}
let extend T (U * T) (first : T, second : U) : T & U = first ,, second 
\end{lstlisting}

\noindent From the typing point of view, the difference between
\lstinline{extend} in TypeScript and \name is that the type variable
\lstinline{U} now has a \emph{disjointness constraint}. The notation
\lstinline{U * T} means that the type variable U can be instantiated
to any types which are disjoint to the type \lstinline{T}. Unlike
TypeScript, the definition of \lstinline{extend} is trivial, type-safe
and guarantees coherence by using the built-in merge operator (\lstinline{,,}). 

The applicability of \name is illustrated with examples using \lstinline{extend} 
ported from TypeScript, and various operations on polymorphic
extensible records~\cite{leijen2005extensible,harper1991record,jones99lightweight}. 
The operations on polymorphic extensible records
show that \name can encode various operations of row
types~\cite{wand1987complete}. However, in contrast to various existing proposals for 
row types and extensible records, \name supports general intersections 
and not just record operations.

\name and the proofs of coherence and type-safety are formalized in
the Coq theorem prover~\cite{coq}. The proofs are complete except for a minor
(and trivially true) variable renaming lemma used to prove the
soundness between two subtyping relations used in the
formalization. The problem arizes from the combination of the locally
nameless representation of binding~\cite{aydemir-popl-08} and existential
quantification, which prevents a Coq proof for that lemma.

In summary, the contributions of this paper are:

\begin{itemize*}

\item {\bf Disjoint Polymorphism:} A novel form of universal
quantification where type variables can have disjointness
constraints. Disjoint polymorphism enables a flexible combination
of intersection types, the merge operator and parametric
polymorphism. 

\item {\bf Coq Formalization of \name and Proof of Coherence:} An
  elaboration semantics of System \name into System F is
  given. Type-soundness and coherence are proved in Coq.
  The proofs for these properties and all other lemmata found in this paper
  are available at: 

  {\small \url{https://github.com/jalpuim/disjoint-polymorphism}}
  %\footnote{{\bf Note to reviewers:} Due
  %  to the anonymous submission process, the Coq formalization is submitted as supplementary material.}.

\item {\bf Applications:} We show how \name provides basic support
 for dynamic mixins and various operations on polymorphic extensible records.

\end{itemize*}

\begin{comment}

There has been a remarkable number of works aimed at improving support
for extensibility in programming languages. The motivation behind this
line of work is simple, and it is captured quite elegantly by the
infamous \emph{Expression Problem~}\cite{wadler1998expression}: there
are \emph{two} common and desirable forms of extensibility, but most
mainstream languages can only support one form well. Unfortunately
the lack of support in the other form has significant
consequences in terms of code maintenance and software evolution.  As a
result researchers proposed various approaches to address the problem,
including: visions of new programming
models~\cite{Prehofer97,Tarr99ndegrees,Harrison93subject}; new
programming languages or language
extensions~\cite{McDirmid01Jiazzi,Aracic06CaesarJ,Smaragdakis98mixin,nystrom2006j},
and \emph{design patterns} that can be used with existing mainstream
languages~\cite{togersen:2004,Zenger-Odersky2005,oliveira09modular,oliveira2012extensibility}.

Some of the more recent work on extensibility is focused on design
patterns. Examples include \emph{Object
  Algebras}~\cite{oliveira2012extensibility}, \emph{Modular Visitors}~\cite{oliveira09modular,togersen:2004} or
Torgersen's~\cite{togersen:2004} four design patterns using generics. In those
approaches the idea is to use some advanced (but already available)
features, such as \emph{generics}~\cite{Bracha98making}, in combination with conventional
OOP features to model more extensible designs.  Those designs work in
modern OOP languages such as Java, C\#, or Scala.

Although such design patterns give practical benefits in terms of
extensibility, they also expose limitations in existing mainstream OOP
languages. In particular there are three pressing limitations:
1) lack of good mechanisms for
  \emph{object-level} composition; 2) \emph{conflation of
    (type) inheritance with subtyping}; 3) \emph{heavy reliance on generics}.

  The first limitation shows up, for example, in encodings of Feature-Oriented
  Programming~\cite{Prehofer97} or Attribute Grammars~\cite{Knuth1968} using Object
  Algebras~\cite{oliveira2013feature,rendel14attributes}. These programs are best
  expressed using a form of \emph{type-safe}, \emph{dynamic},
  \emph{delegation}-based composition. Although such form of
  composition can be encoded in languages like Scala, it requires the
  use of low-level reflection techniques, such as dynamic proxies,
  reflection or other forms of meta-programming. It is clear
  that better language support would be desirable.

  The second limitation shows up in designs for modelling
  modular or extensible visitors~\cite{togersen:2004,oliveira09modular}.  The vast majority of modern
  OOP languages combines type inheritance and subtyping.
  That is, a type extension induces a subtype. However
  as Cook et al.~\cite{cook1989inheritance} famously argued there are programs where
  ``\emph{subtyping is not inheritance}''. Interestingly
  not many programs have been previously reported in the literature
  where the distinction between subtyping and inheritance is
  relevant in practice. However, as shown in this paper, it turns out that this
  difference does show up in practice when designing modular
  (extensible) visitors.  We believe that modular visitors provide a
  compelling example where inheritance and subtyping should
  not be conflated!

  Finally, the third limitation is prevalent in many extensible
  designs~\cite{togersen:2004,Zenger-Odersky2005,oliveira09modular,oliveira2013feature,rendel14attributes}.
  Such designs rely on advanced features of generics,
  such as \emph{F-bounded polymorphism}~\cite{Canning89f-bounded}, \emph{variance
    annotations}~\cite{Igarashi06variant}, \emph{wildcards}~\cite{Torgersen04wildcards} and/or \emph{higher-kinded
    types}~\cite{Moors08generics} to achieve type-safety. Sadly, the amount of
  type-annotations, combined with the lack of understanding of these
  features, usually deters programmers from using such designs.

This paper presents System \namedis (pronounced \emph{f-and}): an extension of System F~\cite{Reynolds74f}
with intersection types and a merge operator~\cite{dunfield2014elaborating}.  The goal of
System \namedis is to study the \emph{minimal} foundational language
constructs that are needed to support various extensible designs,
while at the same time addressing the limitations of existing OOP
languages. To address the lack of good object-level composition
mechanisms, System \namedis uses the merge operator for dynamic
composition of values/objects. Moreover, in System \namedis (type-level)
extension is independent of subtyping, and it is possible for an
extension to be a supertype of a base object type. Furthermore,
intersection types and conventional subtyping can be used in many
cases instead of advanced features of generics. Indeed this paper
shows many previous designs in the literature can be encoded
without such advanced features of generics.


Technically speaking System \namedis is mainly inspired by the work of
Dunfield~\cite{dunfield2014elaborating}.  Dunfield showed how to model a simply typed
calculus with intersection types and a merge operator. The presence of
a merge operator adds significant expressiveness to the language,
allowing encodings for many other language constructs as syntactic
sugar. System \namedis differs from Dunfield's work in a few
ways. Firstly, it adds parametric polymorphism and formalizes an
extension for records to support a basic form of objects. Secondly,
the elaboration semantics into System F is done directly from the
source calculus with subtyping.
% In contrast, Dunfield has an additional step which eliminates subtyping.
Finally, a non-technical difference
is that System \namedis is aimed at studying issues of OOP languages and
extensibility, whereas Dunfield's work was aimed at Functional
Programming and he did not consider application to extensibility.
Like many other foundational formal models for OOP (for
example $F_{<:}$~\cite{CMMS}), System \namedis is purely functional and it uses
structural typing.

%%System \namedis is
%%formalized and implemented. Furthermore the paper illustrates how
%%various extensible designs can be encoded in System \namedis.


In summary, the contributions of this paper are:


\begin{itemize*}

\item {\bf A Minimal Core Language for Extensibility:} This paper
  identifies a minimal core language, System \namedis, capable of
  expressing various extensibility designs in the literature.
  System \namedis also addresses limitations of existing OOP
  languages that complicate extensible designs.

\item {\bf Formalization of System \namedis:} An elaboration semantics of
  System \namedis into System F is given, and type-soundness is proved.

\item {\bf Encodings of Extensible Designs:} Various encodings of
  extensible designs into System \namedis, including \emph{Object
    Algebras} and \emph{Modular Visitors}.

\item {\bf A Practical Example where ``Inheritance is not Subtyping''
    Matters:} This paper shows that modular/extensible visitors
  suffer from the ``inheritance is not subtyping problem''.
%% Moreover
%% with extensible visitors the extension should become a
%% \emph{supertype}, not a subtype. \bruno{extension with accept method}

\item {\bf Implementation:} An implementation of an
  extension of System \namedis, as well as the examples presented in the
  paper, are publicly available\footnote{{\bf Note to reviewers:} Due
    to the anonymous submission process, the code (and some machine
    checked proofs) is submitted as supplementary material.}.

\end{itemize*}

\end{comment}

\begin{comment}
\subsection{Other Notes}

finitary overloading: yes
but have other merits of intersection been explored?

-- Compare Scala:
-- merge[A,B] = new A with B

-- type IEval  = { eval :  Int }
-- type IPrint = { print : String }

-- F[\_]
\end{comment}

% \section{Introduction}

% Dunfield's work showed how many language features can be encoded in terms
% of intersection types with a merge operator. However two important
% questions were left open by Dunfield:

% \begin{enumerate}
% \item How to allow coherent programs only?

% \item If a restriction that allows coherent programs is in place, can
%   all coherent programs conform to the restriction?
% \end{enumerate}

% In other words question 1) asks whether we can find sufficient
% conditions to guarantee coherency; whereas question 2) asks
% whether those conditions are also necessary. In terms of technical
% lemmas that would correspond to:

% \begin{enumerate}

% \item Coherency theorem: $\Gamma \turns e : A \leadsto E_1 \wedge
%   \Gamma \turns e : A \leadsto E_2~\to~E_1 = E_2$.

% \item Completness of Coherency: ($\Gamma \turns_{old} e : A \leadsto E_1 \wedge
%   \Gamma \turns_{old} e : A \leadsto E_2~\to~E_1 = E_2) \to \Gamma
%   \turns e : A$.

% \end{enumerate}

% For these theorems we assume two type systems. On liberal type system
% that ensures type-safety, but not coherence ($\Gamma \turns_{old} e :
% A$); and another one that is both type-safe and coherent  ($\Gamma \turns e :
% A$). What needs to be shown for completness is that if a coherent
% program type-checks in the liberal type system, then it also
% type-checks in the restricted system.


% \special{papersize=8.5in,11in}
% \setlength{\pdfpageheight}{\paperheight}
% \setlength{\pdfpagewidth}{\paperwidth}

% \title{\namedis}

% \subsection{``Testsuite'' of examples}

% \begin{enumerate}

% \item $\lambda (x : Int * Int). (\lambda (z : Int) . z)~x$: This
%   example should not type-check because it leads to an ambigous choice
%   in the body of the lambda. In the current system the well-formedness
%   checks forbid such example.

% \item $\Lambda A.\Lambda B.\lambda (x:A).\lambda (y:B). (\lambda (z:A)
%   . z) (x,,y)$: This example should not type-check because it is not
%   guaranteed that the instantiation of A and B produces a well-formed
%   type. The TyMerge rule forbids it with the disjointness check.

% \item $\Lambda A.\Lambda B * A.\lambda (x:A).\lambda (y:B). (\lambda
%   (z:A) . z) (x,,y)$: This example should type-check because B is
%   guaranteed to be disjoint with A. Therefore instantiation should
%   produce a well-formed type.

% \item $(\lambda (z:Int) . z) ((1,,'c'),,(2,False))$: This example
%   should not type-check, since it leads to an ambigous lookup of
%   integers (can either be 1 or 2). The definition of disjointness is
%   crutial to prevent this example from type-checking. When
%   type-checking the large merge, the disjointness predicate will
%   detect that more than one integer exists in the merge.

% \item $(\lambda (f: Int \to Int \& Bool) . \lambda (g : Int \to Char \& Bool) . ((f,,g) : Int \to Bool)$:
%   This example
%   should not type-check, since it leads to an ambigous lookup of
%   functions. It shows that in order to check disjointness
%   of functions we must also check disjointness of the subcomponents.

% \item $(\lambda (f: Int \to Int) . \lambda (g : Bool \to Int) . ((f,,g) : Bool \& Int \to Int)$:
%   This example shows that whenever the return types overlap, so does the function type:
%   we can always find a common subtype for the argument types.
% \end{enumerate}

% \subsection{Achieving coherence}

% The crutial challenge lies in the generation of coercions, which can lead
% to different results due to multiple possible choices in the rules that
% can be used. In particular the rules subinter1 and subinter2 overlap and
% can result in coercions that are not equivalent. A simple example is:

% $(\lambda (x:Int) . x) (1,,2)$

% The result of this program can be either 1 or 2 depending on whether
% we chose subinter1 or subinter2.

% Therefore the challenge of coherence lies in ensuring that, for any given
% types A and B, the result of $A <: B$ always leads to the same (or semantically
% equivalent) coercions.

% It is clear that, in general, the following does not hold:

% $if~A <: B \leadsto C1~and~A <: B \leadsto C2~then~C1 = C2$

% We can see this with the example above. There are two possible coercions:\\

% \noindent $(Int\&Int) <: Int \leadsto \lambda (x,y). x$\\
% $(Int\&Int) <: Int \leadsto \lambda (x,y). y$\\

% However $\lambda (x,y). x$ and $\lambda (x,y). y$ are not semantically equivalent.

% One simple observation is that the use of the subtyping relation on the
% example uses an ill-formed type ($Int\&Int$). Since the type system can prevent
% such bad uses of ill-formed types, it could be that if we only allow well-formed
% types then the uses of the subtyping relation do produce equivalent coercions.
% Therefore the we postulate the following conjecture:

% $if~A <: B \leadsto C1~and~A <: B \leadsto C2~and~A, B~well~formed~then~C1 = C2$

% If the following conjecture does hold then it should be easy to prove that
% the translation is coherent.

% % \begin{mathpar}
% %   \inferrule
% %   {}
% %   {\jtype \epsilon {1 \mergeop 2} {\constraints {\tyint \disjoint \tyint} \tyint \inter \tyint}}
% % \end{mathpar}

% % \begin{definition}{(Disjointness)}
% % Two sets $S$ and $T$ are \emph{disjoint} if there does not exist an element $x$, such that $x \in S$ and $x \in T$.
% % \end{definition}

% % \begin{definition}{(Disjointness)}
% % Two types $A$ and $B$ are \emph{disjoint} if there does not exist an term $e$, which is not a merge, such that $\jtype \epsilon e A'$, $\jtype \epsilon e B'$, $A' \subtype A$, and $B' \subtype B$.
% % \end{definition}

% \section{Introduction}

% The benefit of a merge, compared to a pair, is that you don't need to explicitly extract an item out. For example, \lstinline@fst (1,'c')@

% \begin{definition}{Determinism}
% If $e : A_1 \hookrightarrow E_1$ and $e : A_2 \hookrightarrow E_2$,
% then $A_1 = A_2$ and $E_1 = E_2$.
% \end{definition}

% \emph{Coherence} is a property about the relation between syntax and semantics. We say a semantics is \emph{coherent} if the syntax of a term uniquely determines its semantics.

% \begin{definition}{Coherence}
% If $e_1 : A_1 \hookrightarrow E_1$ and $e_2 : A_2 \hookrightarrow E_2$,
% $E_1 \Downarrow v_1$ and $E_2 \Downarrow v_2$,
% then $v_1 = v_2$.
% \end{definition}

% \begin{definition}{Disjointness}
% Two types $A$ and $B$ are \emph{disjoint} (written as ``$\disjoint A B$'') if there does not exist a type $C$ such that $C \subtype A$ and $C \subtype B$ and $C \subtype A \inter B$.
% \end{definition}
