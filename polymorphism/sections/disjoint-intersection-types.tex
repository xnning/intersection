\section{Semantics, Coherence and Type-Safety}\label{sec:disjoint}
This section discusses the elaboration semantics of \name and proves
type-safety and coherence.
It will first show how the semantics is described by an elaboration to System $F$.
Then, it will discuss the necessary extensions to retain coherence, namely in the coercions of 
top-like types; coercive subtyping, and bidirectional type-system's
elaboration. 

\subsection{Target Language}
The dynamic semantics of the call-by-value \name is defined by means of
a type-directed translation to an extension of System $F$ with
pairs.
%The type-directed translation is also shown in
%Figure~\ref{}, where the resulting System F terms are highlighted in
%gray.
The syntax and typing of our target language is unsurprising: 
  \[
    \begin{array}{llrl}
      \text{Types}    & T & \Coloneqq & \alpha \mid \tyint \mid {T_1}
      \to {T_2} \mid \for \alpha T \mid \highlight {$()$} \mid
      \highlight {$\pair {T_1} {T_2}$} \\
      \text{Terms}    & E & \Coloneqq & x \mid i \mid \lamty x T E \mid \app {E_1} {E_2} \mid 
                                        \blam \alpha E \mid \tapp E T
                                        \mid \highlight {$()$} \mid \highlight {$ \pair {E_1} {E_2} $} 
                                        \mid \highlight {$ \proj 1 E
                                          $} \mid \highlight {$ \proj 2 E
                                          $} \\
      \text{Contexts} & G & \Coloneqq & \cdot \mid G, \alpha \mid G, x \oftype T \\
    \end{array}
  \]
The highlighted part shows its difference with the standard System $F$. 
The interested reader can find the formalization of the full target language
syntax and typing rules in our Coq development.

\paragraph{Type and context translation.}
Figure~\ref{fig:type-and-context-translation} defines the type translation
function $\im \cdot$ from \name types $A$ to target language types $T$. 
The notation $\im \cdot$ is also overloaded for context translation from \name
contexts $\Gamma$ to target language contexts $G$.

\begin{figure}[!t]
\begin{spacing}{0.5}
\begin{minipage}[t]{.5\textwidth}
  \framebox{$\im A = T$}

  \begin{align*}
    \im \alpha                &= \alpha \\
    \im \top                  &= () \\
    \im {A_1 \to A_2}         &= \im {A_1} \to \im {A_2} \\
    \im {\fordis \alpha A B}  &= \for \alpha \im B \\
    \im {A_1 \inter A_2}      &= \pair {\im {A_1}} {\im {A_2}} \\
  \end{align*}
\end{minipage}
\begin{minipage}[t]{.5\textwidth}
  \framebox{$\im \Gamma = G$}

  \begin{align*}
    \im \cdot                        &= \cdot \\
    \im {\Gamma, \alpha \disjoint A} &= \im \Gamma, \alpha \\
    \im {\Gamma, \alpha \oftype A}   &= \im \Gamma, \alpha \oftype \im A
  \end{align*}
\end{minipage}
\end{spacing}
  \caption{Type and context translation.}
  \label{fig:type-and-context-translation}
\end{figure}

% The rules given in this section are identical with those in
% Section~\ref{sec:fi}, except for the light blue part. The translation consists
% of four sets of rules, which are explained below:


\subsection{Coercive Subtyping and Coherence}

\paragraph{Coercive subtyping.}

The judgement
\[
A_1 \subtype A_2 \yields E
\]
extends the subtyping judgement in Figure~\ref{fig:fi-subtype} with a coercion
on the right hand side of $ \yields {} $. A coercion $ E $ is just a term
in the target language and is ensured to have type
$ \im {A_1} \to \im {A_2} $ (by Lemma~\ref{lemma:sub}). For example,
\[
\tyint \inter \alpha \subtype \alpha \yields {\lamty x {\im {\tyint \inter \tybool}} {\proj 2 x}}
\]
generates a target coercion function with type: $(\tyint,\alpha) \to \alpha$.

Rules \reflabel{\labelsubtop}, \reflabel{\labelsubvar}, \reflabel{\labelsubint},
\reflabel{\labelsubfun}, \reflabel{\labelsubinterl}, \reflabel{\labelsubinterr}, and
\reflabel{\labelsubinter} are taken directly from \oldname.
%In rule \reflabel{\labelsubtop} the coercion is the constant function of the unit term.
%In rules \reflabel{\labelsubvar}, \reflabel{\labelsubint}, coercions are just
%identity functions. 
%In \reflabel{\labelsubfun}, we elaborate the subtyping of
%parameter and return types by $\eta$-expanding $f$ to $\lamty x {\im {A_3}}
%{\app f x}$, applying $E_1$ to the argument and $E_2$ to the result. 
%Rules \reflabel{\labelsubinterl}, \reflabel{\labelsubinterr}, and
%\reflabel{\labelsubinter} elaborate intersection types.
%\reflabel{\labelsubinter} uses both coercions to form a pair. 
%Rules \reflabel{\labelsubinterl} and \reflabel{\labelsubinterr} reuse the coercion
%from the premises and create new ones that cater to the changes of the argument
%type in the conclusions. 
In rule \reflabel{\labelsubvar}, the coercion is simply the identity function.
Rule \reflabel{\labelsubforall} elaborates disjoint quantification, reusing only the coercion of 
subtyping between the bodies of both types. 
Rule \reflabel{\labelsubrec} elaborates records by simply reusing the coercion generated between
the inner types.
Finally, all rules produce type-correct coercions:

%But in the implementation one usually applies the rules sequentially with
%pattern matching, essentially defining a deterministic order of lookup.

% if we know $A_1$ is a subtype of $A_3$ and $C$ is a coercion from $A_1$
% to $A_3$, then we can conclude that $A_1 \inter A_2$ is also a subtype
% of $A_3$ and the new coercion is a function that takes a value $ x $ of type
% $A_1\inter A_2$, project $x$ on the first item, and apply $ C $ to it.

\defaultthmprf{
\begin{restatable}[Subtyping rules produce type-correct coercions]{lemma}{lemmasub}
  \label{lemma:sub}
  If $ A_1 \subtype A_2 \yields E $, then $ \jtype \cdot E {\im {A_1} \to \im {A_2}} $.
\end{restatable}}
{\noindent \emph{Proof.} By a straightforward induction on the derivation.}

% \george{Explain \reflabel{\labelsubforall} and distinction of kernel and all version.}

%\paragraph{Ordinary types}
%Atomic types are just those which are not intersection types, and are asserted by the judgement \[ \jatomic A \]

%In the left decomposition rules for intersections we introduce a requirement that
%$A_3$ is atomic. 
%The consequence of this requirement is that when $A_3$ is an intersection type, then
%the only rule that can be applied is \reflabel{\labelsubinter}.
%With the atomic constraint, one can guarantee that at any moment during the
%derivation of a subtyping relation, at most one rule can be used.

\paragraph{Unique coercions}

%The subtyping rules need some adjustment.
%Note that the $\bot$ type does not participate in subtyping since it holds no value.
%An important
%problem with the subtyping rules in Figure~\ref{fig:fi-type} is that the all three rules
%dealing with intersection types
%(\reflabel{\labelsubinterl} and \reflabel{\labelsubinterr} and \reflabel{\labelsubinter})
%overlap. Unfortunatelly,
%this means that different coercions may be given when checking the subtyping
%between two types, depending on which derivation is chosen. This is ultimately the reason
%for incoherence.
%There are two important types of overlap:
%
%\begin{enumerate}
%
%\item The left decomposition rules for intersections (\reflabel{\labelsubinterl} and \reflabel{\labelsubinterr}) overlap with each other.
%
%\item The left decomposition rules for intersections (\reflabel{\labelsubinterl} and \reflabel{\labelsubinterr})
%overlap with the right decomposition rules for intersections (\reflabel{\labelsubinter}).
%
%\end{enumerate}
%
%\noindent Fortunately, disjoint intersections (which are enforced by well-formedness)
%deal with problem 1): only one of the two left decomposition rules
%can be chosen for a disjoint intersection type. Since the two types in the intersection
%are disjoint it is impossible that both of the preconditions of the left decompositions are satisfied
%at the same time. 

In order to prove the type-system coherent, the subtyping relation also 
needs to be shown coherent. With disjoint polymorphism the following lemma holds:

\defaultthmprf{
\begin{lemma}[Unique subtype contributor]
  \label{lemma:unique-subtype-contributor}
  If $A_1 \inter A_2 \subtype B$, 
  where $A_1 \inter A_2$ and $B$ are well-formed types, and
  $B$ is not top-like,
  then it is not possible that the following holds at the same time:
  \begin{enumerate}
    \item $A_1 \subtype B$
    \item $A_2 \subtype B$
  \end{enumerate}
\end{lemma}}
{\noindent \emph{Proof.} By double induction: the first on the disjointness derivation (which follows from $A_1 \inter A_2$ being well-formed);
the second on type $B$.
The variable cases \reflabel{\labeldisvar} and \reflabel{\labeldissym} needed to show that, for any two well-formed 
and disjoint types $A$ and $B$, and $B$ is not toplike, then $A$ cannot be a subtype of $B$.}

Using this lemma, we can show that the coercion of a subtyping relation $A \subtype B$ 
is uniquely determined.
This fact is captured by the following lemma:

\defaultthmprf{
\begin{restatable}[Unique coercion]{lemma}{uniquecoercion}
  \label{lemma:unique-coercion}
  If $A \subtype B \yields {E_1}$ and $A \subtype B \yields {E_2}$, where $A$
  and $B$ are well-formed types, then $E_1 \equiv E_2$.
\end{restatable}}
{\noindent \emph{Proof.} By induction on the first derivation and case analysis on the second.} 

\begin{comment}
\paragraph{Expressiveness.}
Remarkably, our restrictions on subtyping do not sacrifice the expressiveness of
subtyping since we have the following two theorems:
\begin{theorem}
  If $A_1 \subtype A_3$, then $A_1 \inter A_2 \subtype A_3$.
\end{theorem}
\begin{theorem}
If $A_2 \subtype A_3$, then $A_1 \inter A_2 \subtype A_3$.
\end{theorem}

The interpretation of the theorem is that: even though the premise is made more
strict by the atomic condition, we can still derive the every subtyping relation
in the unrestricted system.
% \george{Explain why the proof shows this.}

Note that $A$ \emph{exclusive} or $B$ is true if and only if their truth value
differ. Next, we are going to investigate the minimal requirement (necessary and
sufficient conditions) such that the theorem holds.

If $A_1$ and $A_2$ in this setting are the same, for example,
$\tyint \inter \tyint \subtype \tyint$, obviously the theorem will
not hold since both the left $\tyint$ and the right $\tyint$ are a
subtype of $\tyint$.

We can try to rule out such possibilities by making the requirement of
well-formedness stronger. This suggests that the two types on the sides of
$\inter$ should not ``overlap''. In other words, they should be ``disjoint''. It
is easy to determine if two base types are disjoint. For example, $\tyint$
and $\tyint$ are not disjoint. Neither do $\tyint$ and $\code{Nat}$.
Also, types built with different constructors are disjoint. For example,
$\tyint$ and $\tyint \to \tyint$. For function types, disjointness
is harder to visualize. But bear in the mind that disjointness can defined by
the very requirement that the theorem holds.


This result is captured more formally by the following lemma:
\end{comment}
% \george{Note that $\bot$ does not participate in subtyping and why (because the
% empty set intersecting the empty set is still empty).}
% \george{What's the variance of the disjoint constraint? C.f. bounded
% polymorphism.}
% \george{Two points are being made here: 1) nondisjoint intersections, 2) atomic
% types. Show an offending example for each?}
%\subsection{Disjointness} Throughout the paper we already presented an intuitive
%definition for disjoiness. Here such definition is made a bit more precise, and
%well-suited to \namedis.
%
%\begin{definition}[Disjoint types]
%
%  Given a context $\Gamma$, two types $A$ and $B$ are said to be disjoint
%  (written $\jdis \Gamma A B$) if they do not share a common supertype. That is,
%  there does not exist a type $C$ such that $A \subtype C$ and that $B \subtype
%  C$. Note that we assume that all free type variables in $A$,
%    $B$ and $C$ are bound in $\Gamma$.
%
%  \[\jdis \Gamma A B \equiv \not \exists C.~A \subtype C \wedge B \subtype C\]
%
%\end{definition}
%
%To see this definition in action, $\tyint$ and $\tychar$ are disjoint,
%because there is
%no type that is a supertype of the both. On the other hand, $\tyint$ is not
%disjoint with itself, because $\tyint \subtype \tyint$. This implies that
%disjointness is not reflexive as subtyping is. Two types with different shapes
%are always disjoint, unless one of them is an intersection type. For example, a function
%type and a universally quantified type must be disjoint. But a function type and an intersection
%type may not be. Consider:
%\[ \tyint \to \tyint \quad \text{and} \quad (\tyint \to \tyint) \inter (\tystring \to \tystring) \]
%Those two types are not disjoint since $\tyint \to \tyint$ is their common supertype.

%\subsection{Syntax}
%
%\begin{figure}
%  \[
%    \begin{array}{l}
%      \begin{array}{llrll}
%        \text{Types}
%        & A, B & \Coloneqq & \alpha                  & \\
%        &      & \mid & \highlight {$\bot$}          & \\
%        &      & \mid & A \to B                      & \\
%        &      & \mid & \for {(\alpha \highlight {$\disjoint A$})} B  & \\
%        &      & \mid & A \inter B                   & \\
%
%        \\
%        \text{Terms}
%        & e & \Coloneqq & x                        & \\
%        &   & \mid & \lam {(x \oftype A)} e          & \\
%        &   & \mid & \app {e_1} {e_2}              & \\
%        &   & \mid & \blam {(\alpha \highlight {$\disjoint A$})} e  & \\
%        &   & \mid & \tapp e A                     & \\
%        &   & \mid & e_1 \mergeop e_2              & \\
%
%        \\
%        \text{Contexts}
%        & \Gamma & \Coloneqq & \cdot
%                   \mid \Gamma, \alpha \highlight {$\disjoint A$}
%                   \mid \Gamma, x \oftype A  & \\
%
%        \text{Syntactic sugar} & \blam \alpha e & \equiv & \blamdis \alpha \bot e & \\
%                               & \forall \alpha.~A & \equiv & \forall (\alpha * \bot)~.~A & \\
%      \end{array}
%    \end{array}
%  \]
%
%  \caption{Amendments of the rules.}
%  \label{fig:fi-syntax-dis}
%\end{figure}
%
%\begin{figure}
%  \framebox{$\im A = T$}
%
%  \begin{align*}
%    \im \bot                  &= () \\
%    \im {\fordis \alpha A B}  &= \for \alpha \im B \\
%  \end{align*}
%
%  \framebox{$\im \Gamma = G$}
%
%  \begin{align*}
%    \im {\Gamma, \alpha \disjoint A} &= \im \Gamma, \alpha \\
%  \end{align*}
%
%  \caption{Additional type and context translation.}
%  \label{fig:additional-type-and-context-translation}
%\end{figure}
%
%% \george{May also note on the scoping of type variables inside contexts.}
%
%Figure~\ref{fig:fi-syntax-dis} shows the updated syntax with the
%changes highlighted and Figure~\ref{fig:additional-type-and-context-translation}
%shows type and context translations for the new constructs.
%Note how similar the changes are to those needed
%to extend System $F$ with bounded quantification. First, type
%variables are now always associated with their disjointness
%constraints (like $\alpha \disjoint A$) in types, terms, and
%contexts. Second, the bottom type ($\bot$) is introduced so that
%universal quantification becomes a special case of disjoint
%quantification: $\blam \alpha e$ is really a syntactic sugar for
%$\blamdis \alpha \bot e$. The underlying idea is that any type is
%disjoint with the bottom type.  Note the analogy with bounded
%quantification, where the top type is the trivial upper bound in
%bounded quantification, while the bottom type is the trivial
%disjointness constraint in disjoint quantification.

%Indeed, \bruno{unfinidhed sentence}\george{Mabe show a diagram here to contrast
%with bounded polymorphism.}

\subsection{Top-like Types and their Coercions}
\begin{figure}[!t]
\begin{spacing}{1}
  \begin{mathpar}
    \formtoplike \\ %\framebox{$\jatomic A$} \\

    \ruletopltop \and \ruletoplinter \and \ruletoplfun \and \ruletoplforall
  \end{mathpar}
\begin{minipage}[t]{.5\textwidth}
  %\begin{center}
    \framebox{$\andcoerce{A}_{C} = T$}
  %\end{center}
  \begin{center} 
  \[
  \andcoerce{A}_{C} = 
  \begin{cases} 
        \toplike{A} & \andcoerce{A} \\ 
        %A = \top & () \\
        %A = A_1 \to A_2 \; \wedge \; \toplike{A_2} & \lam x \andcoerce{A_2}_{C} \\
        \text{otherwise} & {C} 
  \end{cases}
  \]
  \end{center}
\end{minipage}
\begin{minipage}[t]{.5\textwidth}
  %\begin{center}
    \framebox{$\andcoerce{A} = T$}
  %\end{center}

  \[
  \andcoerce{A} = 
  \begin{cases} 
        A = \top & () \\
        A = A_1 \to A_2 & \lam x \andcoerce{A_2} \\
        A = A_1 \inter A_2 & \pair {\andcoerce{A_1}} {\andcoerce{A_2}} \\
        A = \fordis \alpha B A & \blam \alpha {\andcoerce{A}}
  \end{cases}
  \]
\end{minipage}
\end{spacing}
  \caption{Top-like types and their coercions.}
  \label{fig:andcoercion}
\end{figure}

Lemma~\ref{lemma:unique-subtype-contributor}, which is fundamental in
the proof of coherence of subtyping, holds under the condition
that $B$ is not a \emph{top-like type}. Top-like types in \name
include $\top$ as well as other syntactically different types that
behave as $\top$ (such as $\top \inter \top$), or are applications
resulting in a top-like type. It is easy to see that the unique
subtyping contributor lemma is invalidated if no restriction on
top-like types exists. Since every type is a subtype of $\top$ (or
another top-like type) then, without the restriction, the lemma would
never hold.

\paragraph{Rules.}
\name's definition of top-like types extends that from \oldname.  The
rules that compose this unary relation, denoted as $\toplike{ . }$,
are presented at the top of Figure~\ref{fig:andcoercion}.  The only
new rule is \reflabel{\labeltoplforall}, which states that a
(disjoint) universal quantifier is top-like whenever its body is also
top-like.

\paragraph{Coercions for top-like types} Coercions for top-like types 
require special treatement for retaining coherence. Although Lemma~\ref{lemma:unique-subtype-contributor}
does not hold for top-like types, we can still ensure that that any 
coercions for top-like types are unique, \emph{even if multiple
  derivations exist}. The meta-function
$\andcoerce{A}$, shown at the bottom of Figure~\ref{fig:andcoercion},
 defines coercions for top-like types. With respect to \oldname the $\forall$ case is
now defined, and there is also a change in the $\inter$ case (covering types such as $\top
\inter \top$). With this definition, although two derivations
exist for type $Char\&Int <: \top$, they both generate the coercion 
$\lambda x . ()$.
 
\paragraph{Allowing overlapping top-like types in intersections} 
In \name $\top \& \top$ is a well-formed disjoint intersection type.
This may look odd at first, since all other types
cannot appear more than once in an intersection. Indeed in \oldname 
$\top \& \top$ is not well-formed. However, $\top$ types are special 
in that they have a \emph{unique} canonical top value, which is
translated to the unit value \lstinline{()} in the target language. In
other words a merge of two top types will always return the same value 
regardless of which component of the merge we chose. This is different 
from merges with other types of values, which do not have this
property.  
It turns out that allowing top-types in intersections is fundamental 
for the $\top$ type to behave correctly as the empty constraint, since
the empty constraint (i.e. $\top$) should be disjoint to every other
type. This explains the more liberal treatment of top types \name when
compared to \oldname.

\subsection{Elaboration of type-system and coherence} 
In order to prove the coherence result, we refer to the bidirectional type-system introduced in
Section~\ref{sec:fi}. 
The bidirectional type-system is elaborating, producing a term in the target language while
performing the typing derivation.

\paragraph{Key idea of the translation.}
This translation turns merges into usual pairs, similar to Dunfield's
elaboration approach~\cite{dunfield2014elaborating}.
It also translates the form of disjoint quantification and disjoint type application into regular (polymorphic) 
quantification and type application. 
For example, $\blamdis \alpha \tyint {\lam x {(x \mergeop 1)}}$ in \name will be translated into System $F$'s 
$\blam \alpha {\lam x {\pair x 1}}$.
\begin{comment}
This is achieved by means of the type-directed translation.
Note that, when performing type application, the type-system will ensure that the type being applied is compatible
with the disjoint constraint, otherwise the program is rejected.
For example, applying the term above to $\tychar$ would be accepted, and the generated term would translate this
into System $F$'s type application.
On the other hand, applying the term to $\tyint$ would be rejected, and hence the target term produced is meaningless.
\end{comment}

\begin{comment}
For example, \[ 1 \mergeop \code{"one"} \] becomes \pair 1
{\code{"one"}}. In usage, the pair will be coerced according to type
information. For example, consider the function application: \[ \app {(\lamty x
\tystring x)} {(1 \mergeop \code{"one"})} \] This expression will be translated to \[ \app
{(\lamty x \tystring x)} {(\app {(\lamty x {\pair \tyint \tystring} {\proj 2 x})}
{\pair 1 {\code{"one"}}})} \] The coercion in this case is $(\lamty x {\pair
\tyint \tystring} {\proj 2 x})$. It extracts the second item from the pair, since
the function expects a $\tystring$ but the translated argument is of type $\pair
\tyint \tystring$.
\end{comment}

\paragraph{The translation judgement.} 
The translation judgement $\jtype \Gamma e A \yields E$ extends the
typing judgement with an elaborated term on the right hand side of
$\yields {}$.  The translation ensures that $E$ has type $\im A$.  
The two rules for type abstraction (\reflabel{\labeltblam}) and type
application (\reflabel{\labelttapp}) generate the expected
corresponding coercions in System F.  
%In \name, subtyping means, for
%example, that one may pass more information to a function than what is
%required. This is not the case in System $F$.  The rule
%\reflabel{\labeltsub} is used to account for such differences: the
%coercion $E_{sub}$, derived from the subtyping relation, is applied to
%coerce the System F term into the right type.  
The coercions generated for \reflabel{\labeltrec} and \reflabel{\labeltprojr} 
simply erase the labels and translate the corresponding underlying term. 
All the remaining rules are ported from \oldname. 
%However, it may be noteworthy
%pointing out that, as usual, the rule \reflabel{\labeltmerge}
%translates merges into pairs.


%As an example, we show the coercion generated by the identity function:

%\[
%\cdot
%\inferrule* [right=\labeltmerge]
%  {...}
%  {\jinfer \Gamma {e_1 \mergeop e_2} {A \inter B} \yields {\pair {E_1} {E_2}}}
%}
%\]
%Note that we intentionally left out the annotation...

\paragraph{Type-safety}
The type-directed translation is type-safe. This property is captured
by the following two theorems.

\thmprf{0cm}{
\begin{restatable}[Type preservation]{theorem}{typepreservation}
  \label{theorem:type-preservation}
  We have that:
  \begin{itemize}
  \item If $ \jinfer \Gamma e A \yields E $, 
        then $ \jtype {\im \Gamma} E {\im A} \,$.
  \item If $ \jcheck \Gamma e A \yields E $, 
        then $ \jtype {\im \Gamma} E {\im A} $.
  \end{itemize}
\end{restatable}}{-0.1cm}
{\noindent \emph{Proof.} By structural induction on the term and the corresponding inference rule.}
{-0.1cm}
\defaultthmprf{
\begin{theorem}[Type safety]
  If $e$ is a well-typed \name term, then $e$ evaluates to some System $F$
  value $v$.
\end{theorem}}
{\noindent \emph{Proof.} Since we define the dynamic semantics of \name in terms of the composition of
  the type-directed translation and the dynamic semantics of System $F$, type safety follows immediately.}
%From the theoretical point-of-view, the end goal of this section is to show that the resulting system has
%a coherent (or unique) elaboration semantics:
%\begin{restatable}[Unique elaboration]{theorem}{uniqueelaboration}
%  \label{theorem:unique-elaboration}
%
%  If $\jtype \Gamma e {A_1} \yields {E_1}$ and $\jtype \Gamma e {A_2} \yields
%  {E_2}$, then $E_1 \equiv E_2$. (``$\equiv$'' means syntactical equality, up to
%  $\alpha$-equality.)
%
%\end{restatable}
%
%\noindent In other words, given a source term $e$, elaboration always produces
%the same target term $E$. The most important hurdle we need to overcome is that
%if $A \inter B \subtype C$, then either $A$ or $B$ contributes to that subtyping
%relation, resulting in two possible coercions.



%Figure~\ref{fig:fi-type-patch} shows modifications to Figure~\ref{fig:fi-type} in
%order to support disjoint intersection types and disjoint
%quantification. Only new rules or rules that different are shown.
%Importantly, the disjointness judgement appears in the well-formedness rule for intersection
%types and the typing rule for merges.

%\begin{figure}
%  \begin{mathpar}
%    \framebox{$\jatomic A$} \\
%
%    \inferrule*
%    {}
%    {\jatomic \bot}
%
%    \inferrule*
%    {}
%    {\jatomic {A \to B}}
%
%    \inferrule*
%    {}
%    {\jatomic {\fordis \alpha B A}}
%  \end{mathpar}
%
%  \begin{mathpar}
%    \formsub \\ \rulesubinterldis \and \rulesubinterrdis \and \rulesubforalldis
%  \end{mathpar}
%
%  \begin{mathpar}
%    \formwf \\ \rulewfforalldis \and \rulewfinterdis
%  \end{mathpar}
%
%  \begin{mathpar}
%    \formt \\ \ruletblamdis \and \rulettappdis \and \ruletmergedis
%  \end{mathpar}
%
%  \caption{Affected rules.}
%  \label{fig:fi-type-patch}
%\end{figure}

%\begin{figure}
%  \begin{mathpar}
%    \framebox{$\jatomic A$} \\
%
%    \inferrule*{}{\jatomic {A \to B}}
%
%    \inferrule*{}{\jatomic \alpha}
%
%    \inferrule*{}{\jatomic {\for \alpha A}}
%  \end{mathpar}
%
%  \begin{mathpar}
%    \formsub \\ \rulesubint \and \rulesubvar \and \rulesubfun \and \rulesubforall
%    \and \rulesubinter  \and \rulesubinterldis \and \rulesubinterrdis 
%  \end{mathpar}
%
%  \begin{mathpar}
%    \formwf \\ \rulewfint \and \rulewfvar \and \rulewffun \and \rulewfforall \and \rulewfinter
%  \end{mathpar}
%
%%  \begin{mathpar}
%%    \formt \\ \ruletvar \and \ruletlam \and \ruletapp \and
%%    \ruletblam \and \rulettapp \and \ruletmergedis 
%%  \end{mathpar}
%
%  \begin{mathpar}
%    \formbi \\ \bruletint \and \bruletvar \and \bruletapp \and
%    \brulettapp \and \bruletmergedis \and \bruletann 
%  \end{mathpar}
%
%  \begin{mathpar}
%    \formbc \\ \bruletlam \and  \bruletblam \and \bruletsub
%  \end{mathpar}
%
%  \caption{Rules for Naive system.}
%  \label{fig:fi-type-naive}
%\end{figure}

%\begin{figure}
%  \begin{mathpar}
%    \framebox{$\jatomic A$} \\
%
%    \inferrule*{}{\jatomic {\fordis \alpha B A}}
%  \end{mathpar}
%
%  \begin{mathpar}
%    \formsub \\ \rulesubtop \and \rulesubforalldis \and 
%    \rulesubinterlcoerce \and \rulesubinterrcoerce
%  \end{mathpar}
%
%  \begin{mathpar}
%    \formwf \\ \rulewftop \and \rulewfvardis \and \rulewfforalldis 
%  \end{mathpar}
%
%  \begin{mathpar}
%    \formbi \\ \brulettop \and \brulettappdis 
%  \end{mathpar}
%
%  \begin{mathpar}
%    \formbc \\ \bruletblamdis 
%  \end{mathpar}
%
%
%  \caption{Changes for Extended systems.}
%  \label{fig:fi-type-extended}
%\end{figure}

%\begin{figure}
%  \begin{mathpar}
%    \formsub \\ \rulesubforallext
%  \end{mathpar}
%
%  \caption{Rules for Extended system with improved ForAll.}
%  \label{fig:fi-type-extended_forall}
%\end{figure}

%\begin{figure}[t]
%  \begin{mathpar}
%    \formtoplike \\ %\framebox{$\jatomic A$} \\
%
%    \ruletopltop \and \ruletoplinter \and \ruletoplfun \and \ruletoplforall
%  \end{mathpar}
%
%
%  \begin{center}
%    \framebox{$\andcoerce{A}_{C} = T$}
%  \end{center}
%
%  \[
%  \andcoerce{A}_{C} = 
%  \begin{cases} 
%        \toplike{A} & \andcoerce{A} \\ 
%        %A = \top & () \\
%        %A = A_1 \to A_2 \; \wedge \; \toplike{A_2} & \lam x \andcoerce{A_2}_{C} \\
%        \text{otherwise} & {C} 
%  \end{cases}
%  \]
%
%  \begin{center}
%    \framebox{$\andcoerce{A} = T$}
%  \end{center}
%
%  \[
%  \andcoerce{A} = 
%  \begin{cases} 
%        A = \top & () \\
%        A = A_1 \to A_2 & \lam x \andcoerce{A_2} \\
%        A = A_1 \inter A_2 & \pair {\andcoerce{A_1}} {\andcoerce{A_2}} \\
%        A = \fordis \alpha B A & \blam \alpha {\andcoerce{A}}
%  \end{cases}
%  \]
%  \caption{Coercion generation considering Top-like types.}
%  \label{fig:andcoercion}
%\end{figure}



%\paragraph{Atomic Types.} The new system introduces atomic types. Essentially a type
%is atomic if it is any type, which is not an intersection type.
%The notion of atomic
%type will be helpful

\paragraph{Uniqueness of type-inference}
An important property of the bidirectional type-checking is that, given an expression $e$, if
it is possible to infer a type for it, then $e$ has a unique type.

\defaultthmprf{
\begin{restatable}[Uniqueness of type-inference]{theorem}{uniqueinference}
  \label{theorem:unique-inference}
  If $\jinfer \Gamma e {A_1} \yields {E_1}$ and $\jinfer \Gamma e {A_2} \yields {E_2}$, then ${A_1} = {A_2}$. 
\end{restatable}}
{\noindent \emph{Proof.} By structural induction on the term and the corresponding inference rule.}
\paragraph{Coherency of Elaboration}
Combining the previous results, we are finally able to show the central theorem:

\defaultthmprf{
\begin{restatable}[Unique elaboration]{theorem}{uniqueelaboration}
  \label{theorem:unique-elaboration}
  We have that:
  \begin{itemize*}
    \item If $\jinfer \Gamma e A \yields {E_1}$ and $\jinfer \Gamma e A \yields
          {E_2}$, then $E_1 \equiv E_2$. 
    \item If $\jcheck \Gamma e A \yields {E_1}$ and $\jcheck \Gamma e A \yields
          {E_2}$, then $E_1 \equiv E_2$.
  \end{itemize*}(``$\equiv$'' means syntactical equality, up to
  $\alpha$-equality.)
\end{restatable}}
{\noindent \emph{Proof.}
  By induction on the first derivation. The most important case is the
  subsumption rule:
%\begin{comment}
%  Note that three cases need special attention: \reflabel{\labelttapp}, \reflabel{\labeltsub} and \reflabel{\labeltapp}.
%  In the \reflabel{\labeltblam}:
%  \begin{mathpar}
%    \brulettappdis
%  \end{mathpar}
%\noindent we need to show that the constraint $B$ is unique, using Theorem~\ref{theorem:unique-inference}, 
%in order to apply the induction hypothesis.
%For \reflabel{\labeltapp}: 
%\begin{mathpar}
%\bruletapp
%\end{mathpar}
%\noindent we also need to show the uniqueness of $A_1$ using
%Theorem~\ref{theorem:unique-inference}, in order to apply the
%induction hypothesis.
%  Also, in the \reflabel{\labeltsub} rule:
%\end{comment}
\begin{mathpar}
\bruletsub
\end{mathpar}
\noindent We need to show that $E_{sub}$ is unique (by
  Lemma~\ref{lemma:unique-coercion}), and thus to show that $A$ is well-formed 
  (by Lemma~\ref{lemma:wellformed-typing}). Note that this is the
  place where stability of substitutions (used by
  Lemma~\ref{lemma:wellformed-typing}) plays a crutial role in
  guaranteeing coherence.
  We also need to show that $A$ is unique (by
Theorem~\ref{theorem:unique-inference}). 
Uniqueness of $A$ is needed to apply the induction hypothesis.}
