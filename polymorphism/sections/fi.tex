\section{The \name Calculus}\label{sec:fi}
This section presents the syntax, subtyping, and typing of \name: 
a calculus with intersection types, parametric polymorphism, records and a merge operator. 
This calculus is an extension of the \oldname calculus~\cite{oliveira16disjoint},
which is itself inspired by Dunfield's
calculus~\cite{dunfield2014elaborating}. \name extends \oldname with (disjoint) polymorphism.
%The novelty of \name is the addition of \emph{disjoint polymorphism}:
%a form of parametric polymorphism with disjointness contraints, which
%allows flexibility while at the same time retaining coherence. 
%As discussed in Section~\ref{overview} retaining
%coherence, while having an expressive form of polymorphism is non-trvial.
%Section~\ref{sec:disjoint} introduces \namedis, which shows the necessary changes
%for supporting disjoint intersection types and disjoint
%quantification and ensuring coherence.
Section~\ref{sec:alg-dis} introduces the necessary changes to the
definition of disjointness presented by Oliveira et al.~\cite{oliveira16disjoint} in
order to add disjoint polymorphism.

%\joao{we already say this in the introduction}
%All the meta-theory of \name has been mechanized in Coq, and is available in
%the supplementary materials submitted with the paper.

\subsection{Syntax}
The syntax of \name (with the differences to \oldname highlighted in gray) is: 
%are intersection types $A \inter B$ at the
%type-level and the ``merges'' $e_1 \mergeop e_2$ at the term level.

%TODO merge this figure with figure 5 (text too)
%\begin{figure}[!t]
\vspace{-15pt}
  \[
    \begin{array}{l}
      \begin{array}{llrll}
        \text{Types}
        & A, B & \!\!\Coloneqq & \!\top \mid \tyint \mid A \to B \mid A
                             \inter B \mid \highlight{$\alpha$} \mid \highlight{$\fordis \alpha A B$} \mid \highlight{$\recordType l A$} & \\ 

        \text{Terms}
        & e & \!\!\Coloneqq & \!\top \mid i \mid x \mid \lamty x A e \mid \app {e_1} {e_2} 
              \mid e_1 \mergeop e_2 \mid \!\highlight{$\blamdis \alpha A e$} \!\mid \!\highlight{$\tapp e A$} \!\mid 
              \!\highlight{$\recordCon l e$} \!\mid \!\highlight{$\recordProj e l$} & \\
        \text{Contexts}
        & \Gamma & \!\!\Coloneqq & \!\cdot
                   \mid \Gamma, \highlight{$\alpha \disjoint A$}
                   \mid \Gamma, x \oftype A  & \\
      \end{array}
    \end{array}
  \]

%  \caption{\name syntax.}
%  \label{fig:fi-syntax}
% \end{figure}

\paragraph{Types.} 
Metavariables $A$, $B$ range over types. 
Types include all constructs in \oldname: a top type $\top$; 
the type of integers $\tyint$;
function types $A \to B$; and intersection types $A \inter B$.
The main novelty are two standard constructs of System $F$ used to support
polymorphism: 
type variables $\alpha$ and disjoint (universal) quantification $\fordis \alpha A B$. 
Unlike traditional universal quantification, the disjoint
quantification includes a disjointness constraint associated to a type variable $\alpha$.
Finally, \name also includes singleton record types, which consist of a label $l$ and
an associated type $A$.
We will use $\subst {A} \alpha {B}$
to denote the capture-avoiding substitution of $A$ for $\alpha$ inside $B$ and
$\ftv \cdot$ for sets of free type variables. 

\paragraph{Terms.} 
Metavariables $e$ range over terms.  
Terms include all constructs in \oldname: a canonical top value $\top$; integer literals $i$;
variables $x$, lambda abstractions ($\lamty x A e$); applications 
($\app {e_1} {e_2}$); and the \emph{merge} of terms $e_1$ and $e_2 $ denoted as 
$e1 \mergeop e2$.
Terms are extended with two standard constructs in System $F$:
abstraction of type variables over terms $\blamdis \alpha A e$; and
application of terms to types $\tapp e A$. 
The former also includes an extra disjointness constraint tied to the type 
variable $\alpha$, due to disjoint quantification.
%If one regards $e_1$ and $e_2$ as objects, their merge will respond to
%every method that one or both of them have.
Singleton records consists of a label $l$ and an associated term $e$.
Finally, the accessor for a label $l$ in term $e$ is denoted as $\recordProj e l$.

\paragraph{Contexts.} Typing contexts $ \Gamma $ track bound type variables
$\alpha$ with disjointness constraints $A$; and variables $x$ with their type $A$. 
We will use $\subst {A} \alpha {\Gamma}$
to denote the capture-avoiding substitution of $A$ for $\alpha$ in the co-domain of
$\Gamma$ where the domain is a type variable (i.e all disjointness constraints).
Throughout this paper, we will assume that all contexts are
well-formed. Importantly, besides usual well-formedness conditions, in
well-formed contexts type variables must not appear free within its own disjointness constraint.
%All substitutions performed in environments must also lead to well-formed environments.
%In order to focus on the key features that make this language interesting, we do
%not include other forms such as type constants and fixpoints here. 
%However they can be included in the formalization in
%standard ways and we are using them in discussions and examples. %TODO are we?
\paragraph{Syntactic sugar}
In \name we may quantify a type variable and ommit its constraint. 
This means that its constraint is $\top$. 
For example, the function type $\forall \alpha. \alpha \to \alpha$ is syntactic sugar
for  $\fordis \alpha \top {\alpha \to \alpha}$.
This is discussed in more detail in Section~\ref{sec:disjoint}. 

% \paragraph{Discussion.} A natural question the reader might ask is that why we
% have excluded union types from the language. The answer is we found that
% intersection types alone are enough support extensible designs.

\subsection{Subtyping}
% In some calculi, the subtyping relation is external to the language: those
% calculi are indifferent to how the subtyping relation is defined. In \name, we
% take a syntactic approach, that is, subtyping is due to the syntax of types.
% However, this approach does not preclude integrating other forms of subtyping
% into our system. One is ``primitive'' subtyping relations such as natural
% numbers being a subtype of integers. The other is nominal subtyping relations
% that are explicitly declared by the programmer.


%\begin{figure}
%  \begin{mathpar}
%    \formsub \\
%    \rulesubvar \and \rulesubfun \and \rulesubforall \and \rulesubinter \and
%    \rulesubinterl \and \rulesubinterr
%  \end{mathpar}
%
%  \begin{mathpar}
%    \formwf \\
%    \rulewfvar \and \rulewffun \and \rulewfforall \and \rulewfinter
%  \end{mathpar}
%
%  \begin{mathpar}
%    \formt \\
%    \ruletvar \and \ruletlam \and \ruletapp \and \ruletblam \and \rulettapp \and
%    \ruletmerge
%  \end{mathpar}
%
%  \caption{The type system of \name.}
%  \label{fig:fi-type}
%\end{figure}

% Intersection types introduce natural subtyping relations among types. For
% example, $ \tyint \inter \tybool $ should be a subtype of $ \tyint $, since the former
% can be viewed as either $ \tyint $ or $ \tybool $. To summarize, the subtyping rules
% are standard except for three points listed below:
% \begin{enumerate}
% \item $ A_1 \inter A_2 $ is a subtype of $ A_3 $, if \emph{either} $ A_1 $ or
%   $ A_2 $ are subtypes of $ A_3 $,

% \item $ A_1 $ is a subtype of $ A_2 \inter A_3 $, if $ A_1 $ is a subtype of
%   both $ A_2 $ and $ A_3 $.

% \item $ \recordType {l_1} {A_1} $ is a subtype of $ \recordType {l_2} {A_2} $, if
%   $ l_1 $ and $ l_2 $ are identical and $ A_1 $ is a subtype of $ A_2 $.
% \end{enumerate}
% The first point is captured by two rules $ \reflabelsubinterl $ and
% $ \reflabelsubinterr $, whereas the second point by $ \reflabelsubinter $.
% Note that the last point means that record types are covariant in the type of
% the fields.

The subtyping rules of the form $A \subtype B$ are shown in 
Figure~\ref{fig:fi-subtype}. 
At the moment, the reader is advised to ignore the
gray-shaded parts, which will be explained later. 
Some rules are ported from \oldname: \reflabel{\labelsubtop}, 
\reflabel{\labelsubint},
\reflabel{\labelsubfun}, \reflabel{\labelsubinter}, \reflabel{\labelsubinterl} and
\reflabel{\labelsubinterr}.

\begin{figure}[t]
\begin{spacing}{0.5}
  \begin{mathpar}
    \framebox{$\jatomic A$} \\
    \inferrule*{}{\jatomic \tyint} \and 
    \inferrule*{}{\jatomic {A \to B}} \and
    \inferrule*{}{\jatomic \alpha} \and
    \inferrule*{}{\jatomic {\fordis \alpha B A}} \and
    \inferrule*{}{\jatomic {\recordType l A}}
  \end{mathpar}
  \begin{mathpar}
    \formsub \\ 
    \rulesubtop \and \rulesubinter \and 
    \rulesubint \and \rulesubinterlcoerce \and 
    \rulesubrec \and \rulesubinterrcoerce \and
    \rulesubvar  \and \rulesubfun \and 
    \rulesubforallext 
  \end{mathpar}
\end{spacing}
  \caption{Subtyping rules of \name.}
  \label{fig:fi-subtype}
\end{figure}



%There are three rules which rather straightforward: \reflabel{\labelsubtop}
%says that every type is a subtype of $\top$; \reflabel{\labelsubint} and 
%\reflabel{\labelsubvar} define subtyping as a reflexive relation on integers and
%type variables.
%The rule \reflabel{\labelsubfun} says that a function is contravariant in 
%its parameter type and covariant in its return type. 
%The three rules dealing with intersection types are just what one would expect 
%when interpreting types as sets. 
%Under this interpretation, for example, the rule \reflabel{\labelsubinter}
%says that if $A_1$ is both the subset of $A_2$ and the subset of $A_3$, then
%$A_1$ is also the subset of the intersection of $A_2$ and $A_3$.

\paragraph{Polymorphism and Records.}
The subtyping rules introduced by \name refer to polymorphic constructs and records. 
\reflabel{\labelsubvar} defines subtyping as a reflexive relation on type variables.
In \reflabel{\labelsubforall} a universal quantifier ($\forall$) 
is covariant in its body, and contravariant in its disjointness constraints.
The \reflabel{\labelsubrec} rule says that records are covariant
within their fields' types.
The subtyping relation uses an auxiliary unary $ordinary$ relation,
which identifies types that are not intersections. The $ordinary$ conditions on two of the intersection rules are necessary to 
produce unique coercions~\cite{oliveira16disjoint}. The $ordinary$
relation needed to be extended with respect to \oldname.
As shown at the top of Figure~\ref{fig:fi-subtype}, the new types it contains are 
type variables, universal quantifiers and record types.

\paragraph{Properties of Subtyping.} The subtyping relation is reflexive and transitive.
\thmprf{0cm}{
\begin{restatable}[Subtyping reflexivity]{lemma}{subrefl}
  \label{lemma:subrefl}
  For any type $A$, $A \subtype A$.
\end{restatable}}
{-0.1cm}
{\noindent \emph{Proof} By induction on $A$.}
{0cm}
%\noindent \emph{Proof.} By induction on $A$.
%\restatableproof{lemma}{Subtyping reflexivity}{subrefl}{lemma:subrefl}
%{For any type $A$, $A \subtype A$.}
%{By induction on $A$.}{-0.1cm}%
%\begin{prf}
%By induction on $A$.
%\end{prf}%
%\begin{restatable}[Subtyping transitivity]{lemma}{subtrans}
%  \label{lemma:subtrans}
%  If $A \subtype B$ and $B \subtype C$, then $A \subtype C$.
%\end{restatable}%
%\noindent \emph{Proof.} By double induction on both derivations.%
\restatableproof{lemma}{Subtyping transitivity}{subtrans}{lemma:subtrans}
{If $A \subtype B$ and $B \subtype C$, then $A \subtype C$.}
{By double induction on both derivations.}{-0.1cm}

%\begin{prf}
%By double induction on both derivations. 
%\end{prf}
%\bruno{Too much space waisted between Lemma and proof. reduce the
%  white space.}
%TODO example showing contravariance in disjointness constraints goes here or in the overview 
%section?
%\paragraph{Metatheory.} As other standard subtyping relations, we can show that
%subtyping defined by $\subtype$ is also reflexive and transitive.
%
%\begin{lemma}[Subtyping is reflexive] \label{lemma:sub-refl}
%  For all type $ A $, $ A \subtype A $.
%\end{lemma}
%
%\begin{lemma}[Subtyping is transitive] \label{lemma:sub-trans}
%  If $ A_1 \subtype A_2 $ and $ A_2 \subtype A_3 $,
%  then $ A_1 \subtype A_3 $.
%\end{lemma}
\subsection{Typing}

\begin{comment}
\begin{figure}[!t]
  \begin{mathpar}
    \formwf \\ \rulewfint \and \rulewfvardis \and \rulewffun \and \rulewfrec \and 
    \rulewftop \and \rulewfforalldis \and \rulewfinterdis 
  \end{mathpar}

  \caption{Well-formedness rules for types of \name.}
  \label{fig:wf}
\end{figure}
\end{comment}


%  \begin{mathpar}
%    \formt \\ \ruletvar \and \ruletlam \and \ruletapp \and
%    \ruletblam \and \rulettapp \and \ruletmergedis 
%  \end{mathpar}

\paragraph{Well-formedness.}
The well-formedness rules are shown in the top part of Figure~\ref{fig:fi-type}. 
The new rules over \oldname are \reflabel{\labelwfvar} and \reflabel{\labelwfforall}. 
Their definition is quite straightforward, but note that the constraint in the latter
must be well-formed.

\begin{figure}
  \begin{spacing}{1}
  \begin{mathpar}
    \formwf \\ \rulewfint \and \rulewfvardis \and \rulewfrec \and 
    \rulewffun \and \rulewftop \and \rulewfforalldis \and \rulewfinterdis 
  \end{mathpar}
  \begin{mathpar}
    \formbi \\ \brulettop \and \bruletint \and \bruletvar \and \bruletann \and 
    \bruletapp \and \brulettappdis \and \bruletmergedis \and \bruletrec \and 
    \bruletprojr \and \bruletblamdis 
  \end{mathpar}
  \begin{mathpar}
    \formbc \\ \bruletlam \and \bruletsub
  \end{mathpar}
  \end{spacing}
  \caption{Well-formedness and type system of \name.}
  \label{fig:fi-type}
\end{figure}

\paragraph{Typing rules.}
Our typing rules are formulated as a bi-directional type-system. 
Just as in \oldname, this ensures the type-system is not only syntax-directed, but
also that there is no type ambiguity: that is, inferred types are unique.
The typing rules are shown in the bottom part of Figure~\ref{fig:fi-type}. 
Again, the reader is advised to ignore the
gray-shaded parts, as these will be explained later. 
The typing judgements are of the form: $\jcheck \Gamma e A$ and  
$\jinfer \Gamma e A$.
They read: ``in the typing context $\Gamma$, the term $e$ can be
checked or inferred to
type $A$'', respectively. 
The rules ported from \oldname are the
check rules for $\top$ (\reflabel{\labelttop}), integers (\reflabel{\labeltint}), 
variables (\reflabel{\labeltvar}),  application (\reflabel{\labeltapp}), merge operator  
(\reflabel{\labeltmerge}), annotations (\reflabel{\labeltann}); and infer rules
for lambda abstractions (\reflabel{\labeltlam}), and the subsumption rule 
(\reflabel{\labeltsub}).

\paragraph{Disjoint quantification.}
The new rules, inspired by System $F$, are the infer rules for type
application \reflabel{\labelttapp}, and for type abstraction
\reflabel{\labeltblam}.  Type abstraction is introduced by the big
lambda $\blamdis \alpha A e$, eliminated by the usual type application
$\tapp e A$ (\reflabel{\labelttapp}).  The disjointness constraint is
added to the context in \reflabel{\labeltblam}. During a type application, the
type system makes sure that the type argument satisfies the
disjointness constraint.  Type application performs an extra check
ensuring that the type to be instantiated is compatible
(i.e. disjoint) with the constraint associated with the abstracted
variable.  This is important, as it will retain the desired coherence
of our type-system.  For ease of discussion, also in
\reflabel{\labeltblam}, we require the type variable introduced by the
quantifier to be fresh.  For programs with type variable shadowing,
this requirement can be met straighforwardly by variable renaming.

\paragraph{Records.}
Finally, $\reflabel{\labeltrec}$ and $\reflabel{\labeltprojr}$ deal with record types.
The former infers a type for a record with label $l$ if it can infer a type for the
inner expression; the latter says if one can infer a record type $\recordType l A$ 
from an expression $e$, then it is safe to access the field $l$, and infering type $A$.

