\section{Related Work}
\label{sec:related-work}

%*******************************************************************************
\paragraph{Coherence.}
%*******************************************************************************

Reynolds invented Forsythe~\cite{reynolds1997design} in the 1980s. Our merge
operator is analogous to his operator $p_1, p_2 $. Forsythe has a coherent
semantics. The result was proved formally by
Reynolds~\cite{reynolds1991coherence} in a lambda calculus with intersection
types and a merge operator. However  the way coherence is ensured is not general
enough. He has four different typing rules for the merge operator, each
accounting for various possibilities of what the types of the first and second
components are. In some cases the meaning of the second component takes
precedence (that is, is biased) over the first component. The set of rules is
restrictive and it forbids, for instance, the merge of two functions (even when
they a provably disjoint). In contrast, disjointness in \name has a well-defined
specification and it is quite flexible.

Pierce~\cite{pierce1991programming2} made a comprehensive review
of coherence, especially on Curien and Ghelli~\cite{curienl1990coherence} and
Reynolds' methods of proving coherence; but he was not able to prove coherence
for his $F_\wedge$ calculus. He introduced a primitive $\code{glue}$ function as
a language extension which corresponds to our merge operator. However, in his
system users can ``glue'' two arbitrary values, which can lead to incoherence.

Our work is largely inspired by Dunfield~\cite{dunfield2014elaborating}. He
described a similar approach to ours: compiling a system with intersection types
and a merge operator into ordinary $ \lambda $-calculus terms with pairs. One
major difference is that our system does not include unions. However, as
acknowledged by Dunfield, his calculus lacks of coherence. He discusses the
issue of coherence throughout his paper, mentioning biased choice as an option
(albeit a rather unsatisfying one). He also mentioned that the notion of
disjoint intersection could be a good way to address the problem, but he did not
pursue this option in his work.

% \url{http://homepages.inf.ed.ac.uk/gdp/publications/Sub_Par.pdf}

% \cite{plotkin1994subtyping}

% Also discussed intersection types!~\cite{malayeri2008integrating}.

% Pierce Ph.D thesis: F<: + /|
%        technical report: F + /|, closer to ours

% \cite{barbanera1995intersection}
%
% \paragraph{Intersection types with polymorphism.}
% Our type system combines intersection types and parametric polymorphism. Closest
% to us is Pierce's work~\cite{pierce1991programming1} on a prototype
% compiler for a language with both intersection types, union types, and
% parametric polymorphism. Similarly to \name in his system universal
% quantifiers do not support bounded quantification. However Pierce did not try to prove any
% meta-theoretical results and his calculus does not have a merge
% operator.  Pierce also studied a system where both intersection
% types and bounded polymorphism are present in his Ph.D.
% dissertation~\cite{pierce1991programming2} and a 1997
% report~\cite{pierce1997intersection}.

Going in the direction of higher
kinds, Compagnoni and Pierce~\cite{compagnoni1996higher} added
intersection types to System $ F_{\omega} $ and used the new calculus,
$ F^{\omega}_{\wedge} $, to model multiple inheritance. In their
system, types include the construct of intersection of types of the
same kind $ K $. Davies and Pfenning
\cite{davies2000intersection} studied the interactions between
intersection types and effects in call-by-value languages. And they
proposed a ``value restriction'' for intersection types, similar to
value restriction on parametric polymorphism.

Recently, Castagna et al.~\cite{Castagna:2014} studied an very expressive
calculus that has polymorphism and set-theoretic type connectives (such as
intersections, unions, and negations). As a result, in their calculus one is
also able to express a type variable that can be instantiated to any type other
than $\code{Int}$ as $\alpha \setminus \code{Int}$, which is syntactic sugar for
$\alpha \wedge \neg \code{Int}$. Unfortunately their calculus does not include a
merge operator like ours.

There have been attempts to provide a foundational calculus
for Scala that incorporates intersection
types~\cite{amin2014foundations,amin2012dependent}.
Although the minimal Scala-like calculus does not natively support
parametric polymorphism, it is possible to encode parametric
polymorphism with abstract type members. Thus it can be argued that
this calculus also supports intersection types and parametric
polymorphism. However, the type-soundness of a minimal Scala-like
calculus with intersection types and parametric polymorphism is not
yet proven. Recently, some form of intersection
types has been adopted in object-oriented languages such as Scala,
Ceylon, and Grace. Generally speaking,
the most significant difference to \name is that in all previous systems
there is no explicit introduction construct like our merge operator.

\begin{comment}
only allow intersections of concrete types (classes),
whereas our language allows intersections of type variables, such as
\texttt{A \& B}. Without that vehicle, we would not be able to define
the generic \texttt{merge} function (below) for all interpretations of
a given algebra, and would incur boilerplate code:

\begin{lstlisting}
let merge [A, B] (f: ExpAlg A) (g: ExpAlg B) = {
  lit (x : Int) = f.lit x ,, g.lit x,
  add (x : A & B) (y : A & B) =
    f.add x y ,, g.add x y
}
\end{lstlisting}
\end{comment}

%*******************************************************************************
\paragraph{Other Type Systems with Intersection Types.}
%*******************************************************************************

% Although similar in spirit,
% our elaboration typing is simpler: we require subtyping in the case of
% applications, thus avoiding the subsumption rule. Besides, our treatment
% combines the merge rules ($ k $ ranges over $ \{1, 2\} $)
% \inferrule
% {\Gamma \turns e_k : A}
% {\Gamma \turns e_1 \mergeop e_2 : A}
% and the standard intersection-introduction rule
% \inferrule
% {\Gamma \turns e : A_1 \andalso \Gamma \turns e : A_2}
% {\Gamma \turns e : A_1 \inter A_2}
% into one rule:
% \inferrule [Merge]
% {\Gamma \turns e_1 : A_1 \andalso \Gamma \turns e_2 : A_2}
% {\Gamma \turns e_1 \mergeop e_2 : A_1 \inter A_2}
%Castagna, and Dunfield describe
%elaborating multi-fields records into merge of single-field records.
% Reynolds and Castagna do not consider elaboration and Dunfield do not
% formalize elaborating records.
% Both Pierce and Dunfield's system include a subsumption rule, which states that
% if an term has been inferred of type $ A $, then it is also of any
% supertype of $ A $. Our system does not have this rule.
Refinement
intersection~\cite{dunfield2007refined,davies2005practical,freeman1991refinement}
is the more conservative approach of adopting intersection types. It increases
only the expressiveness of types but not terms. But without a term-level
construct like ``merge'', it is not possible to encode various language
features. As an alternative to syntactic subtyping described in this paper,
Frisch et al.~\cite{frisch2008semantic} studied semantic subtyping. Semantic
subtyping seems to have important advantages over syntactic subtyping. One
worthy avenue for future work is to study languages with intersection types
and merge operator in a semantic subtyping setting.

%*******************************************************************************
\paragraph{Traits and Trait Calculi.}
%*******************************************************************************

Since the seminal paper by Schärli~\emph{et al.} documented an implementation of
the trait mechanism in dynamically typed Smalltalk without providing a type
system, Fisher and Reppy~\cite{fisher2004typed} presented a statically typed
calculus that models traits. \name is not dedicated to traits; but rather, it
supports a source language that models traits. Compared to Fisher and Reppy's
calculus, \name is more lightweight. For example, self reference is not in the
language of \name. One reason for the difference is that Fisher and Reppy's
calculus supports \emph{classes} in addition to traits, and considers the
interaction between them, whereas our object oriented source language is
\emph{prototype}-based---the mechanism for code reuse is purely trait. Of
course, there have been many formalization of traits, such
as~\cite{scharli2003traitsformal}. But most of them are heavyweight and specific
to modeling traits, and therefore differ from our approach.

Bettini~\textit{et al.}'s prototype language,
SWRTJ~\cite{bettini2010prototypical} distinguishes, in their terminology,
``records'' and ``traits''---the former contain fields and the latter contain
methods. Since we try to model a pure object-oriented language, we have excluded
fields, which provide state reuse. In SWRTJ, traits themselves are not meant to
be the generator of instances. Instead, another construct, called ``classes''
are, and make use of traits.

\bruno{Our traits vs Scala traits?}

Scala also has the notion of ``traits''. However, unlike what its name suggests,
it is more similar to mixins~\cite{bracha1990mixin}, whereas our traits are
intended to model the original trait idea closely. Aside from that, Scala's
traits and our source language's traits have three major differences:

\begin{enumerate}

  \item Scala's trait cannot be instantiated but only mixed in to a class (which
  can be anonymous), whereas traits in our language can be instantiated
  directly.

  \item Scala's trait cannot take constructor parameters whereas ours can. As in
  the point example, our trait is itself a constructor and takes the x- and
  y-coördinates.

  \begin{lstlisting}
  trait Point(x: Int, y: Int) { self: Point (*$ \to $*)
    x() = x
    y() = y
  } in (* \ldots *)
  \end{lstlisting}

  \item Scala allows traits to contain methods whose body is not implemented.
  But our current language does not support this feature.

\end{enumerate}
