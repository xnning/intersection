\section{Disjointness and Coherence}
\label{sec:disjoint}
% \bruno{This section still needs to be cleaned up to remove stuff
%   related to polymorphism.}
Although the system shown in the previous section is type-safe, it is not
coherent, in the sense that one source program can be elaborated into two
different target programs. This section shows how to modify it so that it
guarantees coherence as well as type soundness. The result is a calculus named
\name. The key aspect is the notion of disjoint intersection.

%From the theoretical point-of-view, the end goal of this section is to show that the resulting system has
%a coherent (or unique) elaboration semantics:
%\begin{restatable}[Unique elaboration]{theorem}{uniqueelaboration}
%  \label{theorem:unique-elaboration}
%
%  If $\jtype \Gamma e {A_1} \yields {E_1}$ and $\jtype \Gamma e {A_2} \yields
%  {E_2}$, then $E_1 \equiv E_2$. (``$\equiv$'' means syntactical equality, up to
%  $\alpha$-equality.)
%
%\end{restatable}
%
%\noindent In other words, given a source term $e$, elaboration always produces
%the same target term $E$. The most important hurdle we need to overcome is that
%if $A \inter B \subtype C$, then either $A$ or $B$ contributes to that subtyping
%relation, resulting in two possible coercions.

\subsection{Disjointness} Throughout the paper we already presented an intuitive
definition for disjointness. Here such definition is made a bit more precise, and
well-suited to \name.

% \bruno{This definition is wrong and too complicated. Since there is no
% polymorphism, we do not need $\Gamma$.}
\begin{definition}[Disjoint types]

  Two types $A$ and $B$ are said to be disjoint
  (written $\jdis \Gamma A B$) if they do not share a common supertype. That is,
  there does not exist a type $C$ such that $A \subtype C$ and that $B \subtype
  C$.

  \[\jdis \Gamma A B \equiv \not \exists C.~A \subtype C \wedge B \subtype C\]

\end{definition}

For example, $\tyint$ and $\tychar$ are disjoint,
if there is
no type that is a supertype of the both (unless a top type is included). On the other hand, $\tyint$ is not
disjoint with itself, because $\tyint \subtype \tyint$. This implies that
disjointness is not reflexive as subtyping is. Two types with different ``shapes''
are always disjoint, unless one of them is an intersection type.
For example, a function type and an intersection
type may not be disjoint. Consider:
\[ \tyint \to \tyint \quad \text{and} \quad (\tyint \to \tyint) \inter (\tystring \to \tystring) \]
Those two types are not disjoint since $\tyint \to \tyint$ is their common supertype.

% \bruno{There are no changes in syntax, right? So the whole section
%   (and figure) can be dropped! Also, without polymorphism, we don't
%   need a bottom type.}
% \subsection{Syntax}
%
% \begin{figure}
%   \[
%     \begin{array}{l}
%       \begin{array}{llrll}
%         \text{Types}
%         & A, B & \Coloneqq & A \to B                 & \\
%         &      & \mid & A \inter B                   & \\
%
%         \\
%         \text{Terms}
%         & e & \Coloneqq & x                        & \\
%         &   & \mid & \lam {(x \oftype A)} e          & \\
%         &   & \mid & \app {e_1} {e_2}              & \\
%         &   & \mid & e_1 \mergeop e_2              & \\
%
%         \\
%         \text{Contexts}
%         & \Gamma & \Coloneqq & \cdot
%                    \mid \Gamma, x \oftype A  & \\
%       \end{array}
%     \end{array}
%   \]
%
%   \caption{Amendments of the rules.}
%   \label{fig:fi-syntax-dis}
% \end{figure}

% \george{May also note on the scoping of type variables inside contexts.}

% Figure~\ref{fig:fi-syntax-dis} shows the updated syntax with the
% changes highlighted.

% First, type
% variables are now always associated with their disjointness
% constraints (like $\alpha \disjoint A$) in types, terms, and
% contexts. Second, the bottom type ($\bot$) is introduced so that
% universal quantification becomes a special case of disjoint
% quantification: $\blam \alpha e$ is really a syntactic sugar for
% $\blamdis \alpha \bot e$. The underlying idea is that any type is
% disjoint with the bottom type.  Note the analogy with bounded
% quantification, where the top type is the trivial upper bound in
% bounded quantification, while the bottom type is the trivial
% disjointness constraint in disjoint quantification.

%Indeed, \bruno{unfinidhed sentence}\george{Mabe show a diagram here to contrast
%with bounded polymorphism.}

\subsection{Typing}

Figure~\ref{fig:fi-type-patch} shows modifications to Figure~\ref{fig:fi-type}
in order to support disjoint intersection types.
Only new rules or rules that different are shown. Importantly, the disjointness
judgment appears in the well-formedness rule for intersection types and the
typing rule for merges.

\begin{figure}
  \begin{mathpar}
    \framebox{$\jatomic A$} \\

    \inferrule*
    {}
    {\jatomic {A \to B}}
  \end{mathpar}

  \begin{mathpar}
    \formsub \\ \rulesubinterldis \and \rulesubinterrdis
  \end{mathpar}

  \begin{mathpar}
    \formwf \\ \rulewfinterdis
  \end{mathpar}

  \begin{mathpar}
    \formt \\ \ruletmergedis
  \end{mathpar}

  \caption{Affected rules.}
  \label{fig:fi-type-patch}
\end{figure}

%\paragraph{Atomic Types.} The new system introduces atomic types. Essentially a type
%is atomic if it is any type, which is not an intersection type.
%The notion of atomic
%type will be helpful

\paragraph{Well-formedness.}
We require that the two types of an intersection must be disjoint in their
context. Under the new rules, intersection types such as $\tyint \inter \tyint$
are no longer well-formed because the two types are not disjoint.

\paragraph{Metatheory.}
Since in this section we only restrict the type system
in the previous section, it is easy to see that type preservation and
type-safety still holds. Additionally, we can see that typing always produces a
well-formed type.

% \begin{restatable}[Instantiation]{lemma}{instantiation}
%   \label{lemma:instantiation}
%
%   If $\jwf {\Gamma, \alpha \disjoint B} C$, $\jwf \Gamma A$, $\jdis \Gamma A B$
%   then $\jwf \Gamma {\subst A \alpha C}$.
% \end{restatable}
%
% \begin{restatable}[Well-formed typing]{lemma}{wellformedtyping}
%   \label{lemma:wellformed-typing}
%
%   If $\jtype \Gamma e A$, then $\jwf \Gamma A$.
% \end{restatable}
%
% \begin{proof}
%   By induction on the derivation that leads to $\jtype \Gamma e A$ and applying
%   Lemma~\ref{lemma:instantiation} in the case of \reflabel{\labelttapp}.
% \end{proof}

% With our new definition of well-formed types, this result is nontrivial.
% In general, disjointness judgments are not invariant with respect to
% free-variable substitution. In other words, a careless substitution can violate
% the disjoint constraint in the context. For example, in the context $\alpha
% \disjoint \tyint$, $\alpha$ and $\tyint$ are disjoint:
% \[ \jdis {\alpha \disjoint \tyint} \alpha \tyint \]
% But after the substitution of $\tyint$ for $\alpha$ on the two types, the sentence
% \[ \jdis {\alpha \disjoint \tyint} \tyint \tyint \]
% is longer true since $\tyint$ is clearly not disjoint with itself.

\subsection{Subtyping}

% \george{Explain \reflabel{\labelsubforall} and distinction of kernel and all version.}

The subtyping rules need some adjustment. An important problem with the
subtyping rules in Figure~\ref{fig:fi-type} is that the all three rules dealing
with intersection types (\reflabel{\labelsubinterl} and
\reflabel{\labelsubinterr} and \reflabel{\labelsubinter}) overlap.
Unfortunately, this means that different coercions may be given when checking
the subtyping between two types, depending on which derivation is chosen. This
is the ultimately reason for incoherence. There are two important types of
overlap:

\begin{enumerate}

\item The left decomposition rules for intersections (\reflabel{\labelsubinterl}
and \reflabel{\labelsubinterr}) overlap with each other.

\item The left decomposition rules for intersections (\reflabel{\labelsubinterl}
and \reflabel{\labelsubinterr}) overlap with the right decomposition rules for
intersections \reflabel{\labelsubinter}.

\end{enumerate}

\noindent Fortunately, disjoint intersections (which are enforced by
well-formedness) deal with problem 1): only one of the two left decomposition
rules can be chosen for a disjoint intersection type. Since the two types in the
intersection are disjoint, it is impossible that both of the preconditions of
the left decompositions are satisfied at the same time. More formally, with
disjoint intersections, we have the following theorem:

\begin{lemma}[Unique subtype contributor]
  \label{lemma:unique-subtype-contributor}

  If $A_1 \inter A_2 \subtype B$, where $A_1 \inter A_2$ and $B$ are well-formed types,
  then it is not possible that the following holds at the same time:
  \begin{enumerate}
    \item $A_1 \subtype B$
    \item $A_2 \subtype B$
  \end{enumerate}
\end{lemma}

Unfortunately, disjoint intersections alone are insufficient to deal with
problem 2). In order to deal with problem 2), we introduce a distinction between
types, and atomic types.

\paragraph{Atomic Types.} Atomic types are just those which are not intersection
types, and are asserted by the judgment \[ \jatomic A \]

Since types in \name are simple, the only atomic type is the function type.
But in richer systems, it can also include, for example, the record type.
In the left decomposition rules for intersections we introduce a requirement
that $A_3$ is atomic. The consequence of this requirement is that when $A_3$ is
an intersection type, then the only rule that can be applied is
\reflabel{\labelsubinter}. With the atomic constraint, one can guarantee that at
any moment during the derivation of a subtyping relation, at most one rule can
be used. Consequently, the coercion of a subtyping relation $A \subtype B$ is
uniquely determined. This fact is captured by the following lemma:

\begin{restatable}[Unique coercion]{lemma}{uniquecoercion}
  \label{lemma:unique-coercion}

  If $A \subtype B \yields {E_1}$ and $A \subtype B \yields {E_2}$, where $A$
  and $B$ are well-formed types, then $E_1 \equiv E_2$.
\end{restatable}

\paragraph{No Loss of Expressiveness.}
Interestingly, our restrictions on subtyping do not sacrifice the expressiveness of
subtyping since we have the following two theorems:
\begin{theorem}
  If $A_1 \subtype A_3$, then $A_1 \inter A_2 \subtype A_3$.
\end{theorem}
\begin{theorem}
If $A_2 \subtype A_3$, then $A_1 \inter A_2 \subtype A_3$.
\end{theorem}

\noindent The interpretation of the two theorems is that: even though the
premise is made stricter by the atomic condition, we can still derive the every
subtyping relation which is valid in the unrestricted system.

% Note that $A$ \emph{exclusive} or $B$ is true if and only if their truth value
% differ. Next, we are going to investigate the minimal requirement (necessary and
% sufficient conditions) such that the theorem holds.
%
% If $A_1$ and $A_2$ in this setting are the same, for example,
% $\tyint \inter \tyint \subtype \tyint$, obviously the theorem will
% not hold since both the left $\tyint$ and the right $\tyint$ are a
% subtype of $\tyint$.
%
% We can try to rule out such possibilities by making the requirement of
% well-formedness stronger. This suggests that the two types on the sides of
% $\inter$ should not ``overlap''. In other words, they should be ``disjoint''. It
% is easy to determine if two base types are disjoint. For example, $\tyint$
% and $\tyint$ are not disjoint. Neither do $\tyint$ and $\code{Nat}$.
% Also, types built with different constructors are disjoint. For example,
% $\tyint$ and $\tyint \to \tyint$. For function types, disjointness
% is harder to visualize. But bear in the mind that disjointness can defined by
% the very requirement that the theorem holds.
%
%
% This result is captured more formally by the following lemma:

% \george{Note that $\bot$ does not participate in subtyping and why (because the
% empty set intersecting the empty set is still empty).}

% \george{What's the variance of the disjoint constraint? C.f. bounded
% polymorphism.}

% \george{Two points are being made here: 1) nondisjoint intersections, 2) atomic
% types. Show an offending example for each?}

\subsection{Coherence of the Elaboration}
Combining the previous results, we are able to show the central theorem:

\begin{restatable}[Unique elaboration]{theorem}{uniqueelaboration}
  \label{theorem:unique-elaboration}

  If $\jtype \Gamma e {A_1} \yields {E_1}$ and $\jtype \Gamma e {A_2} \yields
  {E_2}$, then $E_1 \equiv E_2$. (``$\equiv$'' means syntactical equality, up to
  $\alpha$-equality.)

\end{restatable}

\begin{proof}
  Note that the typing rules are already syntax-directed but the case of
  \reflabel{\labeltapp} (copied below) still needs special attention since we
  need to show that the generated coercion $E$ is unique.
  \begin{mathpar}
    \ruletapp
  \end{mathpar}
  Luckily we know that typing
  judgments give well-formed types, and thus $\jwf \Gamma {A_1}$ and $\jwf
  \Gamma {A_3}$. Therefore we are able to apply
  Lemma~\ref{lemma:unique-coercion} and conclude that $E$ is unique.

\end{proof}
