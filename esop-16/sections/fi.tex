\section{The \name Calculus}
\label{sec:fi}

This section presents the syntax, subtyping, typing, as well as the (incoherent)
semantics of \name: a calculus with intersection types and a merge operator.
This calculus is similar to Dunfield's calculus~\cite{dunfield2014elaborating}.
The differences are that, in Dunfield's language, the coercions are source
language terms and that function application allows the expression denoting the
function to have an intersection type. In contrast, in \name coercions are
target language terms and function applications do not allow the expression
denoting the applied function to have an intersection type. The later
difference will play an important role in achieving coherence.
Sections~\ref{sec:disjoint} and \ref{sec:alg-dis} will present the more fundamental contributions of this
paper by showing other necessary changes for supporting
disjoint intersection types and ensuring coherence.

\subsection{Syntax}

Figure~\ref{fig:fi-syntax} shows the syntax. The differences to the
$\lambda$-calculus, highlighted in gray, are intersection types $A \inter B$ at
the type-level and the ``merges'' $e_1 \mergeop e_2$ at the term level.

\begin{figure}[t]
  \[
    \begin{array}{l}
      \begin{array}{llrll}
        \text{Types}
        & A, B, C & \Coloneqq & \code{Int} & \\
        &         & \mid      & A \to B    & \\
        &         & \mid      & \highlight{$A \inter B$}  & \\

        \\
        \text{Terms}
        & e & \Coloneqq & i            & \\
        &   & \mid & x                 & \\
        &   & \mid & \lamty x A e      & \\
        &   & \mid & \app {e_1} {e_2}  & \\
        &   & \mid & \highlight{$e_1 \mergeop e_2$}  & \\

        \\
        \text{Contexts}
        & \Gamma & \Coloneqq & \cdot
                   \mid \Gamma, x \oftype A  & \\
      \end{array}
    \end{array}
  \]

  \caption{\name syntax.}
  \label{fig:fi-syntax}
\end{figure}

\paragraph{Types.} Metavariables $A$, $B$ range over types. Types include
function types $A \to B$. $A \inter B$ denotes the intersection of types $A$ and
$B$. We also include integer types $\code{Int}$.

\paragraph{Terms.} Metavariables $e$ range over terms.  Terms include standard
constructs: variables $x$; abstraction of terms over variables of a given type
$\lamty x A e$; and application of terms $e_1$ to terms $e_2$, written $\app
{e_1} {e_2}$. The expression $e_1 \mergeop e_2$ is the \emph{merge} of two terms
$e_1$ and $e_2$. Merges of terms correspond to intersections of types $A \inter
B$. In addition, we also include integer literals $i$.

\paragraph{Contexts.} Typing contexts $ \Gamma $ track bound variables $x$ with their type $A$.

% We use $\subst {A} \alpha {B}$
% to denote the capture-avoiding substitution of $A$ for $\alpha$ inside $B$ and
% $\ftv \cdot$ for sets of free type variables.

In order to focus on the key features that make this language interesting, we do
not include other forms such as type constants and fixpoints here. However they
can be included in the formalization in standard ways and we are using them in
discussions and examples. The formalization of nonessential features such as
records and record operations can be found in the full version of the paper.

% \paragraph{Discussion.} A natural question the reader might ask is that why we
% have excluded union types from the language. The answer is we found that
% intersection types alone are enough support extensible designs.

\subsection{Subtyping}

% In some calculi, the subtyping relation is external to the language: those
% calculi are indifferent to how the subtyping relation is defined. In \name, we
% take a syntactic approach, that is, subtyping is due to the syntax of types.
% However, this approach does not preclude integrating other forms of subtyping
% into our system. One is ``primitive'' subtyping relations such as natural
% numbers being a subtype of integers. The other is nominal subtyping relations
% that are explicitly declared by the programmer.

\begin{figure}
  \begin{mathpar}
    \formsub \\
    \rulesubvar \and \rulesubint \and \rulesubfun \and \rulesubinter \and
    \rulesubinterl \and \rulesubinterr
  \end{mathpar}

  \begin{mathpar}
    \formwf \\
    \rulewfint \and \rulewffun \and \rulewfinter
  \end{mathpar}

  \begin{mathpar}
    \formt \\
    \ruletvar \and \ruletint \and
    \ruletlam \and \ruletapp \and
    \ruletmerge
  \end{mathpar}

  \caption{The type system of \name.}
  \label{fig:fi-type}
\end{figure}\bruno{MAJOR PROBLEM: We are missing the well-formedness
  rules as well as their uses in Figure 2. These need to be added so
  that we can talk about coherence later!}

% Intersection types introduce natural subtyping relations among types. For
% example, $ \code{Int} \inter \code{Bool} $ should be a subtype of $ \code{Int} $, since the former
% can be viewed as either $ \code{Int} $ or $ \code{Bool} $. To summarize, the subtyping rules
% are standard except for three points listed below:
% \begin{enumerate}
% \item $ A_1 \inter A_2 $ is a subtype of $ A_3 $, if \emph{either} $ A_1 $ or
%   $ A_2 $ are subtypes of $ A_3 $,

% \item $ A_1 $ is a subtype of $ A_2 \inter A_3 $, if $ A_1 $ is a subtype of
%   both $ A_2 $ and $ A_3 $.

% \item $ \recordType {l_1} {A_1} $ is a subtype of $ \recordType {l_2} {A_2} $, if
%   $ l_1 $ and $ l_2 $ are identical and $ A_1 $ is a subtype of $ A_2 $.
% \end{enumerate}
% The first point is captured by two rules $ \reflabelsubinterl $ and
% $ \reflabelsubinterr $, whereas the second point by $ \reflabelsubinter $.
% Note that the last point means that record types are covariant in the type of
% the fields.

The subtyping rules of the form $A \subtype B$ are shown in the top part of
Figure~\ref{fig:fi-type}. At the moment, the reader is advised to ignore the
gray-shaded part in the rules, which will be explained later. The rule
\reflabel{\labelsubfun} says that a function is contravariant in its parameter
type and covariant in its return type. The three rules dealing with
intersection types are just what one would expect when interpreting types as
sets. Under this interpretation, for example, the rule \reflabel{\labelsubinter}
says that if $A_1$ is both the subset of $A_2$ and the subset of $A_3$, then
$A_1$ is also the subset of the intersection of $A_2$ and $A_3$.

\paragraph{Metatheory.} As other sane subtyping relations, we can show that
subtyping defined by $\subtype$ is also reflexive and transitive.

\begin{lemma}[Subtyping is reflexive] \label{lemma:sub-refl}
  For all type $ A $, $ A \subtype A $.
\end{lemma}

\begin{lemma}[Subtyping is transitive] \label{lemma:sub-trans}
  If $ A_1 \subtype A_2 $ and $ A_2 \subtype A_3 $,
  then $ A_1 \subtype A_3 $.
\end{lemma}

For the corresponding mechanized proofs in Coq, we refer the reader to the
repository associated with the paper\footnote{\url{https://github.com/zhiyuanshi/intersection/tree/master/esop-16}}.

\subsection{Typing}

The well-formedness rules are shown in the middle part of
Figure~\ref{fig:fi-type}. In addition to the standard rules,
\reflabel{\labelwfinter} is also not surprising. The typing rules are shown in
the bottom part of the figure. The typing judgment is of the form:
\[ \jtype \Gamma e A \]
It reads: ``in the typing context $\Gamma$, the term $e$ is of type
$A$''. The standard rules are those for variables
\reflabel{\labeltvar} and lambda abstractions
\reflabel{\labeltlam}. \reflabel{\labeltmerge} means that a merge
$e_1 \mergeop e_2$, is assigned an intersection type composed of the
resulting types of $e_1$ and $e_2$.

\paragraph{Typing Applications}
The rule that deserves more attention is the application rule
\reflabel{\labeltapp}.  This is where the most important difference between
\name and Dunfield calculus is. In Dunfield's calculus, if
\lstinline$f$ and \lstinline$g$ are both functions expecting an
integer, the following program typechecks:

\begin{lstlisting}
(f ,, g) 1
\end{lstlisting}

\noindent In our calculus, however, one needs manually extract the desired
function by passing \lstinline$f ,, g$ to a function. For example, if
\lstinline$f : Int$ $\to$  \lstinline$Bool$ and \lstinline$g : Int$ $\to$ \lstinline$Char$, we could select  \lstinline$f$
from the intersection as follows:

\begin{lstlisting}
((fun (h : Int (*$\to$*) Bool) (*$\to$*) h) (f ,, g)) 1
\end{lstlisting}

The reason for this design choice is that otherwise we could have
another source of semantic ambiguity. Consider the following program:

\begin{lstlisting}
(idInt ,, idChar) (1,,'c')
\end{lstlisting}

\noindent where \lstinline$idInt$ and \lstinline$idChar$ are,
respectively, identity functions on integers and on characters. Clearly both
merges in the program (\lstinline$(f ,, g)$ and \lstinline$(1,,'c')$)
are disjoint. Yet there are two possible outcomes for this program:
\lstinline$1$ or \lstinline$'c'$. Thus application requires a
different treatment, if we wish to have coherence.
Our design choice is simply to only employ subtyping in the argument
of the application, but not on the expression being applied.
It should be possible to design a ``smarter''
application rule, that still avoids the ambiguity problems that would
be present in Dunfield's calculus, while at the same time avoiding
explicitly extracting functions from intersections. However such rule
would necessarily be complicated, and rather ad-hoc.
So we opted for simplicity. In any case there is no expressiveness
loss (only loss of convinience), as it is always possibly to extract the function component from
a merge and then apply it.

\subsection{Semantics}

We define the dynamic semantics of the call-by-value \name by means of
a type-directed translation to an extension of $\lambda$-calculus with pairs.
%The type-directed translation is also shown in
%Figure~\ref{}, where the resulting System F terms are highlighted in
%gray.


\paragraph{Target Language.}
The syntax and typing of our target language is unsurprising. The syntax of the
target language is shown in Figure~\ref{fig:f-syntax}. The highlighted part
shows its difference with the $\lambda$-calculus. The typing rules can be found
in the extended version of the paper.

% \bruno{Does the target language does need unit? Any rules generating units? I
% don't think so, so it can be dropped.}

\paragraph{Key Idea of the Translation.}
This translation turns merges into usual pairs, similar to Dunfield's
elaboration approach~\cite{dunfield2014elaborating}.
For example, \[ 1 \mergeop \code{"one"} \] becomes \pair 1
{\code{"one"}}. In usage, the pair will be coerced according to type
information. For example, consider the function application: \[ \app {(\lamty x
{\code{String}} x)} {(1 \mergeop \code{"one"})} \] This expression will be translated to \[ \app
{(\lamty x {\code{String}} x)} {(\app {(\lamty x {\pair {\code{Int}} {\code{String}}} {\proj 2 x})}
{\pair 1 {\code{"one"}}})} \] The coercion in this case is $(\lamty x {\pair
{\code{Int}} {\code{String}}} {\proj 2 x})$. It extracts the second item from the pair, since
the function expects a $\code{String}$ but the translated argument is of type $\pair
{\code{Int}} {\code{String}}$.


\begin{figure}[t]
  \[
    \begin{array}{llrl}
      \text{Types}    & T & \Coloneqq & \code{Int} \\
                      &   & \mid      & {T_1} \to {T_2} \\
                      &   & \mid      & \highlight {$ \pair {T_1} {T_2} $} \\
      \text{Terms}    & E & \Coloneqq & x \\
                      &   & \mid      & i \\
                      &   & \mid      & \lamty x T E \\
                      &   & \mid      & \app {E_1} {E_2} \\
                      &   & \mid      & \highlight {$ \pair {E_1} {E_2} $} \\
                      &   & \mid      & \highlight {$ \proj k E $} \quad k \in \{ 1, 2 \} \\
      \text{Contexts} & G & \Coloneqq & \cdot \mid G, x \oftype T \\
    \end{array}
  \]
  \caption{Target language syntax.}
  \label{fig:f-syntax}
\end{figure}

\paragraph{Type and Context Translation.}

Figure~\ref{fig:type-and-context-translation} defines the type translation
function $\im \cdot$ from \name types $A$ to target language types $T$. The
notation $\im \cdot$ is also overloaded for context translation from \name
contexts $\Gamma$ to target language contexts $G$.

\begin{figure}[t]
  \framebox{$\im A = T$}

  \begin{align*}
    \im {\code{Int}}     &= \code{Int} \\
    \im {A_1 \to A_2}    &= \im {A_1} \to \im {A_2} \\
    \im {A_1 \inter A_2} &= \pair {\im {A_1}} {\im {A_2}} \\
  \end{align*}

  \framebox{$\im \Gamma = G$}

  \begin{align*}
    \im \cdot                      &= \cdot \\
    \im {\Gamma, \alpha \oftype A} &= \im \Gamma, \alpha \oftype \im A
  \end{align*}

  \caption{Type and context translation.}
  \label{fig:type-and-context-translation}
\end{figure}

% The rules given in this section are identical with those in
% Section~\ref{sec:fi}, except for the light blue part. The translation consists
% of four sets of rules, which are explained below:

\paragraph{Coercive subtyping.}

The judgment
\[
A_1 \subtype A_2 \yields E
\]
extends the subtyping judgment in Figure~\ref{fig:fi-type} with a coercion
on the right hand side of $ \yields {} $. A coercion $ E $ is just an term
in the target language and is ensured to have type
$ \im {A_1} \to \im {A_2} $ (by Lemma~\ref{lemma:sub}). For example,
\[
\code{Int} \inter \code{Bool} \subtype \code{Bool} \yields {\lamty x {\im {\code{Int} \inter \code{Bool}}} {\proj 2 x}}
\]
generates a coercion function with type: $\code{Int} \inter \code{Bool} \to \code{Bool}$.

In \reflabel{\labelsubfun}, we elaborate the subtyping of
parameter and return types by $\eta$-expanding $f$ to $\lamty x {\im {A_3}}
{\app f x}$, applying $E_1$ to the argument and $E_2$ to the result. Rules
\reflabel{\labelsubinterl}, \reflabel{\labelsubinterr}, and
\reflabel{\labelsubinter} elaborate intersection types.
\reflabel{\labelsubinter} uses both coercions to form a pair. Rules
\reflabel{\labelsubinterl} and \reflabel{\labelsubinterr} reuse the coercion
from the premises and create new ones that cater to the changes of the argument
type in the conclusions. Note that the two rules are overlapping and
hence a program can be elaborated differently, depending on which rule
is used. Finally, all rules produce type-correct coercions:

%But in the implementation one usually applies the rules sequentially with
%pattern matching, essentially defining a deterministic order of lookup.

% if we know $A_1$ is a subtype of $A_3$ and $C$ is a coercion from $A_1$
% to $A_3$, then we can conclude that $A_1 \inter A_2$ is also a subtype
% of $A_3$ and the new coercion is a function that takes a value $ x $ of type
% $A_1\inter A_2$, project $x$ on the first item, and apply $ C $ to it.

\begin{restatable}[Subtyping rules produce type-correct coercions]{lemma}{lemmasub}
  \label{lemma:sub}
  If $ A_1 \subtype A_2 \yields E $, then $ \jtype \cdot E {\im {A_1} \to \im {A_2}} $.
\end{restatable}

\begin{proof}
  By a straightforward induction on the derivation\footnote{The proofs of major lemmata and theorems can be found in the full version of the paper.\bruno{where??}}.
\end{proof}

\paragraph{The translation judgment.} The translation judgment $\jtype \Gamma e
A \yields E$ extends the typing judgment with an elaborated term on the right
hand side of $\yields {}$. The translation ensures that $E$ has type $\im A$. In
\name, one may pass more information to a function than what is required; but
not in ordinary $\lambda$-calculus. To account for this difference, in \reflabel{\labeltapp}, the
coercion $E$ from the subtyping relation is applied to the argument.
\reflabel{\labeltmerge} straightforwardly translates merges into pairs.

The type-directed translation is type-safe. This property is captured
by the following two theorems.

\begin{restatable}[Type preservation]{theorem}{typepreservation}
  \label{theorem:type-preservation}

  If $ \jtype \Gamma e A \yields E $,
  then $ \jtype {\im \Gamma} E {\im A} $.
\end{restatable}

\begin{proof}
  (Sketch) By structural induction on the term and the corresponding
  inference rule.
\end{proof}

\begin{theorem}[Type safety]
  If $e$ is a well-typed \name term, then $e$ evaluates to some $\lambda$-calculus
  value $v$.
\end{theorem}

\begin{proof}
  Since we define the dynamic semantics of \name in terms of the composition of
  the type-directed translation and the dynamic semantics of $\lambda$-calculus, type safety follows immediately.
\end{proof}
