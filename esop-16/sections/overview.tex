\section{Overview} \label{sec:overview}

This section introduces \name and its support for intersection types and the
merge operator. It then discusses the issue of coherence and shows how the
notion of disjoint intersection types achieve a coherent semantics.

Note that this section uses some syntactic sugar, as well as standard
programming language features, to illustrate the various concepts in
\name. Although the minimal core language that we formalize in
Section~\ref{sec:fi} does not present all such features, our implementation
supports them.

% It then shows that,
% with unrestricted intersection types, the system
% lacks \emph{coherence}. This motivates the introduction of
% disjoint intersection types and extending universal quantification to
% disjoint quantification, which is enough to ensure coherence.

\subsection{Intersection Types and the Merge Operator}
%%\subsection{Intersection Types, Merge and Polymorphism in \name}

Intersection types date back as early as Coppo et
al.'s work~\cite{coppo1981functional}. Since then various researchers have
studied intersection types, and some languages have adopted them in one
form or another.
%However, as we shall see in
%Section~\ref{subsec:incoherence}, it also introduces difficulties. In what follows
%intersection types and the merge operator are informally introduced.

\paragraph{Intersection types.}
The intersection of type $A$ and $B$ (denoted as \lstinline{A & B} in
\name) contains exactly those values
which can be used as either values of type $A$ or of type $B$. For instance,
consider the following program in \name:

\begin{lstlisting}
let x : Int & Char = (*$ \ldots $*) in -- definition omitted
let idInt (y : Int) : Int = y in
let idChar (y : Char) : Char = y in
(idInt x, idChar x)
\end{lstlisting}

\noindent If a value \lstinline{x} has type \lstinline{Int & Char} then
\lstinline{x} can be used as an integer or as a character. Therefore,
\lstinline{x} can be used as an argument to any function that takes
an integer as an argument, or any
function that take a character as an argument. In the program above
the functions \lstinline{idInt} and \lstinline{idChar} are the
identity functions on integers and characters, respectively.
Passing \lstinline{x} as an argument to either one (or both) of the
functions is valid.

\paragraph{Merge operator.}
In the previous program we deliberately did not show how to introduce values of
an intersection type. There are many variants of intersection types in the
literature. Our work follows a particular formulation, where intersection types
are introduced by a \emph{merge operator}. As
Dunfield~\cite{dunfield2014elaborating} has argued a merge operator adds
considerable expressiveness to a calculus. The merge operator allows two values
to be merged in a single intersection type. For example, an implementation of
\lstinline{x} is constructed in \name as follows:

\begin{lstlisting}
let x : Int & Char = 1,,'c' in (*$ \ldots $*)
\end{lstlisting}

\noindent In \name (following Dunfield's notation), the
merge of two values $v_1$ and $v_2$ is denoted as $v_1 ,, v_2$.

\paragraph{Merge operator and pairs.}
The merge operator is similar to the introduction construct on pairs.
An analogous implementation of \lstinline{x} with pairs would be:

\begin{lstlisting}
let xPair : (Int, Char) = (1, 'c') in (*$ \ldots $*)
\end{lstlisting}

\noindent The significant difference between intersection types with a
merge operator and pairs is in the elimination construct. With pairs
there are explicit eliminators (\lstinline{fst} and
\lstinline{snd}). These eliminators must be used to extract the
components of the right type. For example, in order to use
\lstinline{idInt} and \lstinline{idChar} with pairs, we would need to
write a program such as:

\begin{lstlisting}
(idInt (fst xPair), idChar (snd xPair))
\end{lstlisting}

\noindent In contrast the elimination of intersection types is done
implicitly, by following a type-directed process. For example,
when a value of type \lstinline{Int} is needed, but an intersection
of type \lstinline{Int & Char} is found, the compiler uses the
type system to extract the corresponding value.

\subsection{Incoherence}\label{subsec:incoherence}
Unfortunately the implicit nature of elimination for intersection
types built with a merge operator can lead to incoherence.
The merge operator combines two terms, of type $A$ and $B$
respectively, to form a term of type $A \inter B$. For example,
$1 \mergeop `c'$ is of type $\tyint \inter \tychar$. In this case, no
matter if $1 \mergeop `c'$ is used as $\tyint$ or $\tychar$, the result
of evaluation is always clear. However, with overlapping types, it is
not straightforward anymore to see the result. For example, what
should be the result of this program, which asks for an integer out of
a merge of two integers:
\begin{lstlisting}
(fun (x: Int) (*$ \to $*) x) (1,,2)
\end{lstlisting}
Should the result be \lstinline$1$ or \lstinline$2$?

If both results are accepted, we say that the semantics is \emph{incoherent}:
there are multiple possible meanings for the same valid program. Dunfield's
calculus~\cite{dunfield2014elaborating} is incoherent and accepts the program
above.

\paragraph{Getting around incoherence: biased choice.}
In a real implementation of Dunfield calculus a choice has to be made
on which value to compute. For example, one potential option is to
always take the left-most value matching the type in the
merge. Similarly, one could always take the right-most
value matching the type in the merge. Either way, the meaning
of a program will depend on a biased implementation choice,
which is clearly unsatisfying from the theoretical point of view
(although perhaps acceptable in practice).

\subsection{Restoring Coherence: Disjoint Intersection Types}\label{sec:restoring}
Coherence is a desirable property for a semantics. A semantics is said
to be coherent if any \emph{valid program} has exactly one
meaning~\cite{reynolds1991coherence} (that is, the semantics is not ambiguous).
One option to restore coherence is to reject programs which may have
multiple meanings.
%Of course, when rejecting programs it is important
%not to be too conservative, and reject too many programs which are
%actually coherent.
Analyzing the expression $1 \mergeop 2$, we can see that the reason
for incoherence is that there are multiple, overlapping, integers in the
merge. Generally speaking, if both terms can be assigned some type $C$,
both of them can be chosen as the meaning of the merge,
which leads to multiple meanings of a term.
Thus a natural option is to try to forbid such overlapping
values of the same type in a merge.

This is precisely the approach taken in \name. \name requires that the
two types of in intersection must be \emph{disjoint}.  However,
although disjointness seems a natural restriction to impose on
intersection types, it is not obvious to formalize it. Indeed Dunfield
has mentioned disjointness as an option to restore coherence, but he
left it for future work due to the non-triviality of the approach.

\paragraph{Searching for a definition of disjointness.}
The first step towards disjoint intersection types is to come up
with a definition of disjointness. A first attempt at such definition would
be to require that, given two types $A$ and $B$, both types are not
subtypes of each other. Thus, denoting disjointness as $A * B$, we would have:
\[A * B \equiv A \not<: B \wedge B \not<: A\]
At first sight this seems a reasonable definition and it does prevent
merges such as \lstinline{1,,2}. However some moments of thought are enough to realize that
such definition does not ensure disjointness. For example, consider
the following merge:

\begin{lstlisting}
((1,,'c') ,, (2,,True))
\end{lstlisting}

\noindent This merge has two components which are also intersection
types. The first component (\lstinline{(1,,'c')}) has type $\tyint \inter
\tychar$, whereas the second component (\lstinline{(2 ,, True)}) has type
$\tyint \inter \tybool$. Clearly,
\[ \tyint \inter \tychar \not \subtype \tyint \inter \tybool \wedge \tyint \inter \tybool \not \subtype \tyint \inter \tychar \]
Nevertheless the following program still leads to
incoherence:
\begin{lstlisting}
(fun (x: Int) (*$ \to $*) x) ((1,,'c'),,(2,,True))
\end{lstlisting}
as both \lstinline{1} or \lstinline{2} are possible outcomes
of the program. Although this attempt to define disjointness failed,
it did bring us some additional insight: although the types of the two
components of the merge are not subtypes of each other, they share
some types in common.

\paragraph{A proper definition of disjointness.} In order for two types
to be truly disjoint, they must not have any subcomponents sharing
the same type. In a system with intersection types this can be ensured
by requiring the two types do not share a common supertype. The
following definition captures this idea more formally.

\begin{definition}[Disjointness]
  Given two types $A$ and $B$, two types are disjoint
  (written $A \disjoint B$) if there is no type $C$ such that both $A$ and $B$ are
  subtypes of $C$:
  \[A \disjoint B \equiv \not\exists C.~A \subtype C \wedge B \subtype C\]
\end{definition}

\noindent This definition of disjointness prevents the problematic
merge. Since $Int$ is a common supertype of both $Int \& Char$ and
$Int \& Bool$, those two types are not disjoint.

\name's type system only accepts programs that use disjoint intersection
types. As shown in Section~\ref{sec:disjoint} disjoint intersection types will
play a crucial rule in guaranteeing that the semantics is coherent.

% \subsection{Parametric Polymorphism and Intersection Types}\label{subsec:polymorphism}
% Before we show how \name extends the idea of disjointness to parametric
% polymorphism, we discuss some non-trivial issues that arise from
% the interaction between parametric polymorphism and intersection types.
%The interaction between parametric polymorphism and
%intersection types when coherence is a goal is non-trivial.
%In particular biased choice .
%The key challenge is to have a type
%system that still ensures coherence, but at the same time is not too
%restrictive in the programs that can be accepted.
% Dunfield~\cite{} provides a
% good illustrative example of the issues that arise when combining
% disjoint intersection types and parametric polymorphism:
% \[\lambda x. {\bf let}~y = 0 \mergeop x~{\bf in}~x\]
% Consider the attempt to write
% the following polymorphic function in \name (we use
% uppercase Latin letters to denote type variables):
% \begin{lstlisting}
% let fst A B (x: A & B) = (fun (z:A) (*$ \to $*) z) x in (*$ \ldots $*)
% \end{lstlisting}
% The
% \code{fst} function is supposed to extract a value of type
% (\lstinline{A}) from the merge value $x$ (of type \lstinline{A&B}). However
% this function is problematic.  The reason is that when
% \lstinline{A} and \lstinline{B} are instantiated to non-disjoint
% types, then uses of \lstinline{fst} may lead to incoherence.
% For example, consider the following use of \lstinline{fst}:
% \begin{lstlisting}
% fst Int Int (1,,2)
% \end{lstlisting}
% \noindent This program is clearly incoherent as both
% $1$ and $2$ can be extracted from the merge and still match the type
% of the first argument of \lstinline{fst}.

% \paragraph{Biased choice breaks equational reasoning.} At first sight, one option
% to workaround the issue incoherence would be to bias the type-based merge lookup
% to the left or to the right (as discussed in
% Section~\ref{subsec:incoherence}). Unfortunately, biased choice is
% very problematic when parametric polymorphism is present in the language.
% To see the issue, suppose we chose to always pick the
% rightmost value in a merge when multiple values of same type exist.
% Intuitively, it would appear that the result of the use of
% \lstinline{fst} above is $2$. Indeed simple equational reasoning
% seems to validate such result:
% \begin{lstlisting}
%    fst Int Int (1,,2)
% (*$ \rightsquigarrow $*) (fun (z: Int) (*$ \to $*) z) (1,,2) -- (* \textnormal{By the definition of \code{fst}} *)
% (*$ \rightsquigarrow $*) (fun (z: Int) (*$ \to $*) z) 2      -- (* \textnormal{Right-biased coercion} *)
% (*$ \rightsquigarrow $*) 2                          -- (* \textnormal{By $\beta$-reduction} *)
% \end{lstlisting}
%
% However (assumming a straightforward implementation of right-biased
% choice) the result of the program would be 1! The reason for this has
% todo with \emph{when} the type-based lookup on the merge happens. In
% the case of \lstinline{fst}, lookup is triggered by a coercion
% function inserted in the definition of \lstinline{fst} at
% compile-time.
% In the definition of \lstinline$fst$ all it is known is that a
% value of type $A$ should be returned from a merge with an intersection
% type $A\&B$.  Clearly the only type-safe choice to coerce the value of
% type $A\&B$ into $A$ is to
% take the left component of the merge. This works perfectly for merges
% such as \lstinline$(1,,'c')$, where the types of the first and second components
% of the merge are disjoint. For the merge \lstinline$(1,,'c')$, if a integer lookup
% is needed, then \lstinline$1$ is the rightmost integer, which is consistent with the
% biased choice. Unfortunately, when given the merge \lstinline$(1,,2)$ the
% left-component (\lstinline$1$) is also picked up, even though in this case \lstinline$2$
% is the rightmost integer in the merge. Clearly this is inconsistent
% with the biased choice!
%
% Unfortunately this subtle interaction of polymorphism and type-based lookup
%  means that equational reasoning is broken!
% In the equational reasoning steps above, doing apparently correct
% substitutions lead us to a wrong result. This is a major problem for
% biased choice and a reason to dismiss it as a possible implementation
% choice for \name.

\paragraph{Conservatively rejecting intersections.}
To avoid incoherence, and the issues of biased choice, another option
is simply to reject programs where the
instantiations of type variables may lead to incoherent programs.
In this case the definition of \lstinline$fst$ would be rejected, since there
are indeed some cases that may lead to incoherent programs.
Unfortunately this is too restrictive and prevents many useful
programs.

We have built a source language that is desugared into \name. The most
central feature is the trait declaration. Trait can take several parameters,
which is then in scope in the body of the trait definition. In fact, trait
creation is dynamic, which means it can be contained inside a function.

%*******************************************************************************
\subsection{Traits}
%*******************************************************************************

Trait~\cite{scharli2003traits} is an alternative to inheritance as a mechanism
of code reuse in object-oriented programming. A trait is a collection of related
methods that characterizes only a specific aspect of the features of a class.
Therefore, programs using traits usually have a number of small traits instead
of fewer but larger classes in OO programs that use inheritance. Code reuse with
traits is easier with traits than with classes, since traits are usually shorter
and traits can be \emph{composed}. In fact, trait composition offers a variety
of possibilities: two traits can be ``added'' together (which is an symmetric
operation); methods can be removed from a trait; and traits provide conflict
detection, etc.

Although trait or similar feature is widely implemented in many programming
languages, there have not been a foundational calculus that models traits.
However, we are going to show that \name is able to model all the important
features of trait.

The simplest use case of traits is extracting a feature out from a model class.
The norm of a point is a generalized notion of distance of that point to the
origin. In the following example, we provide two definitions of norm via two
traits. Later we compose the definition of point and norm so that the norm have
different meanings. Composition of traits is anonymous in the sense that there
is no need to explicitly give a name to the composition.

\begin{lstlisting}
type Point = {x: () (*$ \to $*) Int, y: () (*$ \to $*) Int} in
trait Point(x: Int, y: Int) { self: Point (*$ \to $*)
  x() = x
  y() = y
} in
trait EuclideanNorm() { self: Point (*$ \to $*)
  norm() = Math.sqrt(self.x() * self.x() + self.y() * self.y())
} in
trait ManhattanNorm() { self: Point (*$ \to $*)
  norm() = Math.abs(self.x()) + Math.abs(self.y())
} in
console.log(new (Point(3,4) & EuclideanNorm()).norm()) -- This prints 5
console.log(new (Point(3,4) & ManhattanNorm()).norm()) -- This prints 7
\end{lstlisting}

The following example shows a counter object and how we could extend its
behavior so that it supports reset. First we define a \lstinline$Counter$ as a
type synonym for a record that contains a \lstinline$val$ method, which returns
the current counter value. Next we define a trait \lstinline$Counter$ that
contains two methods. The \lstinline$val$ method just returns the value that is
bound at the parameter of the trait, and \lstinline$incr$ returns a new counter.

\begin{lstlisting}
type Counter = { val: () (*$ \to $*) Int } in
trait Counter(val: Int) { self: Counter (*$ \to $*)
  val() = val
  incr() = new Counter(val + 1)
} in
trait Reset() { self: Counter (*$ \to $*)
  reset() = new Counter(0)
} in
let counter = new (Counter(0) & Reset())
in counter.incr()
\end{lstlisting}

In the above code, even though \lstinline$counter$ has a reset method, after we
call the \lstinline$incr$ method, the resulting object no longer has that.
Therefore, naturally we would like to override the \lstinline$incr$ method
inside \lstinline$Reset$.

\begin{lstlisting}
type Counter = { val: () (*$ \to $*) Int } in
trait Counter(val: Int) { self: Counter (*$ \to $*)
  val() = val
  incr() = new Counter(val + 1)
} in
trait Reset() { self: Counter (*$ \to $*)
  incr() = new (Counter(val + 1) & Reset())
  reset() = new Counter(0)
} in
let counter = new (Counter(0) & Reset())
in counter.incr()
\end{lstlisting}

However the modified code should not typecheck according to the specification of
traits, since both \lstinline$Counter$ and \lstinline$Reset$ contains a
conflicting \lstinline$incr$ method. The code also does not typecheck since it
violates the typing rule of \name. The programmer can resolve the conflict by
excluding the \lstinline$incr$ being overridden using the record exclusion
operator.

\begin{lstlisting}
(* \ldots *)
let counter = new (Counter(0) \ incr & Reset())
in counter.incr()
\end{lstlisting}

The next example, although a little bit contrived, illustrates that when two
traits are composed, any two methods in those two traits can refer to each
other via the self reference, just as if they were inside the same class.

\begin{lstlisting}
type EvenOdd = {
  even: Int (*$ \to $*) Bool,
  odd:  Int (*$ \to $*) Bool
} in
trait Even() { self: EvenOdd (*$ \to $*)
  even(n: Int) = if n == 0 then True else self.odd(n - 1)
} in
trait Odd() { self: EvenOdd (*$ \to $*)
  odd(n: Int) = if n == 0 then False else self.even(n - 1)
} in
new (Even() & Odd()).odd(42)
\end{lstlisting}

As a comparison, in Scala we would write

\begin{lstlisting}
trait Even { self: Even with Odd =>
  def even(n: Int): Boolean = if (n == 0) true else self.odd(n - 1)
}

trait Odd { self: Even with Odd =>
  def odd(n: Int): Boolean = if (n == 0) false else self.even(n - 1)
}
\end{lstlisting}

When the two traits are composed, conceptually it is as if that a new class were
being created on the fly by copying all the definitions inside those two traits.
If there is any unresolved conflict, the program will be rejected by the type
system. But one difference with the trait approach and the class approach is
that in our language we are able to compose traits \emph{dynamically} and then
instantiate them, which is impossible in traditional OO languages such as Java
since classes being instantiated must be known statically.

\begin{lstlisting}
class Point3D(x: Int, y: Int) { self: Point3D (*$ \to $*)
  x() = x
  y() = self.z()
  z() = self.x()
}
\end{lstlisting}

%*******************************************************************************
\subsection{Desugaring}
%*******************************************************************************

This subsection introduces how the source language is desugared into \name. A
more formal description can be found in the appendix. A trait in the source
language is translated into nothing but a normal term in \name. For example,

\begin{lstlisting}
trait Point(x: Int, y: Int) { self: Point (*$ \to $*)
  x() = x
  y() = self.z()
}
(* \ldots *)
\end{lstlisting}

becomes

\begin{lstlisting}
let Point (x: Int) (y: Int) (self: () (*$ \to $*) Point) = {
  x = (*$ \lambda $*)(_: ()) (*$ \to $*) x,
  y = (*$ \lambda $*)(_: ()) (*$ \to $*) (self ()).z()
} in
\end{lstlisting}

One difference is that the self reference becomes a thunk and all occurrences of
it have been replaced by \lstinline$self ()$ and the position of the self
reference in the parameter list is adjusted. In fact, \lstinline$self$ is not a
special keyword. It can be named as \lstinline$s$ but \lstinline$self$ is a
convention.

The body of the trait becomes a record whose labels are the method names.
\lstinline$Point$ has type:

\begin{lstlisting}
Int (*$ \to $*) Int (*$ \to $*) (() (*$ \to $*) Point) (*$ \to $*) Point
\end{lstlisting}

The syntax for construction such as \lstinline$Point(3,4)$ is just function
application in \name. And note that \lstinline$Point(3,4)$ is of type
\begin{lstlisting}
(() (*$ \to $*) Point) (*$ \to $*) Point
\end{lstlisting}

Therefore it is an open term. \lstinline$new$ instantiates a trait by taking the
fixpoint of its corresponding open term. In fact, \lstinline$new$ is just a function defined as

\begin{lstlisting}
let new[A] (f: (() (*$ \to $*) A) (*$ \to $*) A): A
  = let rec x : () (*$ \to $*) A = (*$ \lambda $*)(_: ()) (*$ \to $*) f x in x ()
in (* \ldots *)
\end{lstlisting}

Although \lstinline$new$ is defined as
a polymorphic function for convenience, we can always define specialized version
of it since \name does not support polymorphism.

The composition of traits in the source language is desugared using the merge
operator. The reason that traits built on \name have conflict detection for free
is that the merge operator is enforcing that the two terms being merged are
disjoint. For example,

\begin{lstlisting}
new[Point3D](Point(3,4) & Z(5))
\end{lstlisting}

is turned into

\begin{lstlisting}
new [Point3D] ((\self: Point3D) (*$ \to $*) ((Point 3 4 self) ,, (Z 5 self)))
\end{lstlisting}

% \subsection{Intersection Types in Existing Languages}
%
% What is an intersection type? The intersection of types $A$ and $B$
% contains exactly those values which can be used as either of type $A$
% or of type $B$.  Just as not all intersection of sets are nonempty,
% not all intersections of types are inhabited.  For example, the
% intersection of a base type $\tyint$ and a function type
% $\tyint \to \tyint$ is not inhabited.\bruno{put this text somewhere?}
%
% Since then various researchers have
% studied intersection types, and some languages have adopted in one
% form or another. However, while intersection types are already used
% in various languages, the lack of a merge operator removes
% considerable expressiveness.
%
%
% A number of OO languages, such as
% Java, C\#, Scala, and Ceylon\footnote{\url{http://ceylon-lang.org/}},
% already support intersection types to different degrees. Intersection
% types are particularly relevant for OOP as they can be used to model
% multiple interface inheritance. In Java, for example,
%
% \begin{lstlisting}
% interface AwithB extends A, B {}
% \end{lstlisting}
%
% \noindent introduces a new interface \lstinline{AwithB} that satisfies the interfaces of
% both \lstinline{A} and \lstinline{B}. Arguably such type can be considered as a nominal
% intersection type. Scala takes one step further by eliminating the
% need of a nominal type. For example, given two concrete traits, it is possible to
% use \emph{mixin composition} to create an object that implements both
% traits. Such an object has a (structural) intersection type:
%
% \begin{lstlisting}
% trait A
% trait B
%
% val newAB : A with B = new A with B
% \end{lstlisting}
%
% \noindent Scala also allows intersection of type parameters. For example:
% \begin{lstlisting}
% def merge[A,B] (x: A) (y: B) : A with B = ...
% \end{lstlisting}
% uses the anonymous intersection of two type parameters \lstinline{A} and
% \lstinline{B}. However, in Scala it is not possible to dynamically
% compose two objects. For example, the following code:
%
% \begin{lstlisting}
% // Invalid Scala code:
% def merge[A,B] (x: A) (y: B) : A with B = x with y
% \end{lstlisting}
%
% \noindent is rejected by the Scala compiler. The problem is that the
% \lstinline{with} construct for Scala expressions can only be used to
% mixin traits or classes, and not arbitrary objects. Note that in the
% definition \lstinline{newAB} both \lstinline{A} and \lstinline{B} are
% \emph{traits}, whereas in the definition of \lstinline{merge} the variables
% \lstinline{x} and \lstinline{y} denote \emph{objects}.
%
% This limitation essentially put intersection types in Scala in a second-class
% status. Although \lstinline{merge} returns an intersection type, it is
% hard to actually build values with such types. In essence an
% object-level introduction construct for intersection types is missing.
% As it turns out using low-level type-unsafe programming features such
% as dynamic proxies, reflection or other meta-programming techniques,
% it is possible to implement such an introduction
% construct in Scala~\cite{oliveira2013feature,rendel14attributes}. However, this
% is clearly a hack and it would be better to provide proper language
% support for such a feature.
%
% To address the limitations of intersection types in languages like
% Scala, \name allows intersecting any two terms at run time using a
% \emph{merge} operator (denoted by $ \mergeop $)~\cite{dunfield2014elaborating}.  With the merge
% operator it is trivial to implement the \lstinline{merge} function in \name:
%
% \begin{lstlisting}
% let merge[A, B * A] (x : A) (y : B) : A & B = x ,, y in (*$ \ldots $*)
% \end{lstlisting}
%
% \noindent In contrast to Scala's term-level \lstinline{with}
% construct, the operator \lstinline{,,} allows two arbitrary values \lstinline{x}
% and \lstinline{y} to be merged. The resulting type is a \emph{disjoint}
% intersection of the types of  \lstinline{x}
% and \lstinline{y} (\lstinline{A & B} in this case).
%
% A well-formed type is such that given any query type,
% it is always clear which subpart the query is referring to.
% In terms of rules, this notion of well-formedness is almost the same as the one in System $F$
% except for intersection types we require the two components to be disjoint.
%
