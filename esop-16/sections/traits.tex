%*******************************************************************************
\section{Dynamically Composable Traits} \label{sec:trait}
%*******************************************************************************

\bruno{This section suffers from having examples which are different
  to illustrate the various concepts. A better approach would be to
use related examples (say various things todo with points), to
illustrate the approach.}

\george{after the new keyword, types should be specified.}

As an application of disjoint intersection types, we show how to model a
simple, yet expressive form of dynamically composable
traits~\cite{scharli2003traits} in \name. Traits provide a
mechanism of code reuse in object-oriented programming, that
can be used as an alternative to multiple inheritance.
The interesting aspect about traits is the way conflicts that
typically arise in multiple-inheritance~\cite{} are dealt with.
Instead of trying to automatically resolve conflicts, traits
detect those conflicts and require programmers to explicitly resolve
them. This is where the relation to disjoint intersection types comes
in: the mechanism to detect incoherence of disjoint intersection types
provides us with the mechanism to detect conflicts in traits.

We demonstrate various trait
features in a simple OO language, and then show a straightforward
translation from that language to \name. \george{Need to note the difference
between class-based and prototype-based somewhere.}

\paragraph{Basic Traits.}

A trait is a collection of related methods that characterizes only a specific
perspective of the features of an object. Therefore, compared with
programs uses inheritance, programs using traits usually have a number of small
traits rather than fewer but larger classes. Code reuse with traits is easier
with traits than with classes, since traits are usually shorter and traits can
be \emph{composed}. In fact, trait composition offers a variety of
possibilities: two traits can be ``added'' together (which is an symmetric
operation); methods can be removed from a trait; and trait systems provide
conflict detection, etc.

The first example shows basic trait composition. Many social networking sites
allow users to ``upvote'' a comment and the number of upvotes that comment has
received is also displayed. We would like to separate the logic for upvotes from
comments so that it can be reused in other entities such as posts and sharings.
The code below defines a trait, \lstinline$Comment$, which
contains a single method \lstinline$content$.

\begin{lstlisting}
type Comment = { content: () (*$ \to $*) String } in
trait Comment(content: String) { self: Comment (*$ \to $*)
  content() = content
} in
\end{lstlisting}

\noindent Next, we create another trait, \lstinline$Up$, for tracking the number
of upvotes.

\begin{lstlisting}
type Up = { upvotes: () (*$ \to $*) Int } in
trait Up(upvotes: Int) { self: Up (*$ \to $*)
  upvotes() = upvotes
} in
\end{lstlisting}

\noindent Finally, we create a single object from those two traits and test its
functionalities.

\begin{lstlisting}
let comment = new[Comment&Up] (Comment("Have fun!"), Up(4))
in println(comment.content(), comment.upvotes())
-- Output: "Have fun!" 4
\end{lstlisting}

\bruno{Consider mixing explanation with the source code. Instead of
  presenting the full source and then explaining. Alternatively, make
this a figure.}

At this point the reader may wonder why there are duplicate declarations related
to \lstinline$Comment$ and \lstinline$Up$. In mainstream OO languages such as
Java, a class declaration such as \lstinline$class C { ... }$ does two things at
the same time:

\begin{itemize}
\item Declaring a \emph{template} for creating objects;
\item Declaring a new \emph{type}.
\end{itemize}

\noindent In contrast, trait declarations in this source language only does the
former. Back to our example, the purpose of declaring two types is just to use
them for type annotations of the self reference.  In traits literature, a trait
usually ``requires'' some methods and, based on that,  ``provides'' another set
of methods. In our examples, the type of \lstinline$self$ actually denotes what
methods are required.

A trait can expect parameters, which become in scope in the entire trait body.
For example, the \lstinline$Comment$ trait is parametrized by
\lstinline$content$, and the \lstinline$content$ method does nothing more than
returning the eponymous variable. In comparison, traits in Scala do not allow
taking parameters.

The origin of self references is always explicit. The \lstinline$Comment$ trait
requires that \lstinline$self$ be of type \lstinline$Comment$, which is defined
as a type synonym for a record in the first line. But the name ``self'' is
nothing special. In fact, \lstinline$self$ is just another parameter after the
preceding parameter list, and becomes in scope after the arrow. We could have
named it \lstinline$this$ or even \lstinline$s$, but that is generally
discouraged.

Creating an object is via the \lstinline$new$ keyword, similar to many OO
languages, except for one crucial novelty: we can create an object from multiple
traits. More precisely, the object is created from the \emph{composition} of
those traits. Therefore, we are able to call methods from different traits on a
single object.

\paragraph{Traits with Dependencies.} The following example shows that a trait
can depend on another trait. First we define the type of a point and a trait for
a standard point.

\begin{lstlisting}
type Point = {x: () (*$ \to $*) Int, y: () (*$ \to $*) Int} in
trait Point(x: Int, y: Int) { self: Point (*$ \to $*)
  x() = x
  y() = y
} in
\end{lstlisting}

The norm of a point can be defined as its distance to the origin. We provide two
definitions of norm via two traits.

\begin{lstlisting}
type Norm = { norm: () (*$ \to $*) Double } in
trait EuclideanNorm() { self: Point (*$ \to $*)
  norm() = Math.sqrt(self.x() * self.x() + self.y() * self.y())
} in

trait ManhattanNorm() { self: Point (*$ \to $*)
  norm() = Math.abs(self.x()) + Math.abs(self.y())
} in
\end{lstlisting}

Later we compose the definition of point and norm so that the norm have
different meanings. Composition of traits is anonymous in the sense that there
is no need to explicitly give a name to the composition.

\begin{lstlisting}
println(new[Point&Norm] (Point(3,4) & EuclideanNorm()).norm()) -- This prints 5
println(new[Point&Norm] (Point(3,4) & ManhattanNorm()).norm()) -- This prints 7
\end{lstlisting}

\paragraph{Detecting and Resolving Conflicts in Trait Composition.}
Traits usually supports explicit conflict detection and resolution.
In inheritance, one pattern is for the subclass to override methods defined in the parent.
The trait-based approach analog is excluding a method from a trait.
We show how the mechanism can be modeled in \name.
The following example shows a counter object and how we could extend its
behavior so that it supports reset. First we define a \lstinline$Counter$ as a
type synonym for a record that contains a \lstinline$val$ method, which returns
the current counter value. Next we define a trait \lstinline$Counter$ that
contains two methods. The \lstinline$val$ method just returns the value that is
bound at the parameter of the trait, and \lstinline$incr$ returns a new counter.

\begin{lstlisting}
type Counter = { val: () (*$ \to $*) Int } in
trait Counter(val: Int) { self: Counter (*$ \to $*)
  val() = val
  incr() = new[Counter] Counter(val + 1)
} in
type Reset = { reset: () (*$ \to $*) Counter } in
trait Reset() { self: Counter (*$ \to $*)
  reset() = new[Counter] Counter(0)
} in
let counter = new[Counter&Reset] (Counter(0) & Reset())
in counter.incr()
\end{lstlisting}

In the above code, even though \lstinline$counter$ has a reset method, after we
call the \lstinline$incr$ method, the resulting object no longer has that.
Therefore, naturally we would like to override the \lstinline$incr$ method
inside \lstinline$Reset$.

\begin{lstlisting}
type Counter = { val: () (*$ \to $*) Int } in
trait Counter(val: Int) { self: Counter (*$ \to $*)
  val() = val
  incr() = new Counter(val + 1)
} in
trait Reset() { self: Counter (*$ \to $*)
  incr() = new (Counter(val + 1) & Reset())
  reset() = new Counter(0)
} in
let counter = new (Counter(0) & Reset())
in counter.incr()
\end{lstlisting}

However the modified code should not typecheck according to the specification of
traits, since both \lstinline$Counter$ and \lstinline$Reset$ contains a
conflicting \lstinline$incr$ method. The code also does not typecheck since it
violates the typing rule of \name. The programmer can resolve the conflict by
excluding the \lstinline$incr$ being overridden using the record exclusion
operator.

\begin{lstlisting}
(* \ldots *)
let counter = new (Counter(0) \ incr & Reset())
in counter.incr()
\end{lstlisting}

\paragraph{Mutual Dependencies.}
The next example, although a little bit contrived, illustrates that when two
traits are composed, any two methods in those two traits can refer to each
other via the self reference, just as if they were inside the same class.

\begin{lstlisting}
type EvenOdd = {
  even: Int (*$ \to $*) Bool,
  odd:  Int (*$ \to $*) Bool
} in
trait Even() { self: EvenOdd (*$ \to $*)
  even(n: Int) = if n == 0 then True else self.odd(n - 1)
} in
trait Odd() { self: EvenOdd (*$ \to $*)
  odd(n: Int) = if n == 0 then False else self.even(n - 1)
} in
new (Even() & Odd()).odd(42)
\end{lstlisting}

When the two traits are composed, conceptually it is as if that a new class were
being created on the fly by copying all the definitions inside those two traits.
If there is any unresolved conflict, the program will be rejected by the type
system.

\paragraph{Dynamic Composition.}

One difference with the trait approach and the class approach is that in our
language we are able to compose traits \emph{dynamically} and then instantiate
them, which is impossible in traditional OO languages such as Java since classes
being instantiated must be known statically. Actually, since traits are just
terms, traits are first-class values and can be defined inside a function,
passed around or returned just as normal terms. The following function takes a
trait, \lstinline$log$, with unknown implementation and instantiate it.

\begin{lstlisting}
let f (log: trait[Log]) = new [Log] log
in (* \ldots *)
\end{lstlisting}

%*******************************************************************************
\subsection{Desugaring}
%*******************************************************************************

Of course, this whole section will lose its point if the source language cannot
be translated to \name and checked against the type system of \name. A more
formal description can be found in the appendix. The idea of trait translation
is inspired by the functional mixin semantics~\cite{cook1989denotational}, which was
proposed by Cook in an untyped setting. However, our translation is done context of a
statically-typed programming language, which is what provides the ability to statically
detect conflicts in traits.

\paragraph{Trait Declarations.} A trait in the source language is translated into
nothing but a normal term in \name. For example,

\begin{lstlisting}
trait Point(x: Int, y: Int) { self: Point (*$ \to $*)
  x() = x
  y() = self.z()
}
(* \ldots *)
\end{lstlisting}

becomes

\begin{lstlisting}
let Point (x: Int) (y: Int) (self: () (*$ \to $*) Point) = {
  x = (*$ \lambda $*)(_: ()) (*$ \to $*) x,
  y = (*$ \lambda $*)(_: ()) (*$ \to $*) (self ()).z()
} in
\end{lstlisting}

One difference is that the self reference becomes a thunk and all occurrences of
it have been replaced by \lstinline$self ()$ and the position of the self
reference in the parameter list is adjusted. In fact, \lstinline$self$ is not a
special keyword. It can have any name, but \lstinline$self$ is a
convention.

The body of the trait becomes a record whose labels are the method names.
\lstinline$Point$ has type:

\begin{lstlisting}
Int (*$ \to $*) Int (*$ \to $*) (() (*$ \to $*) Point) (*$ \to $*) Point
\end{lstlisting}

The syntax for construction such as \lstinline$Point(3,4)$ is just function
application in \name. And note that \lstinline$Point(3,4)$ is of type
\begin{lstlisting}
(() (*$ \to $*) Point) (*$ \to $*) Point
\end{lstlisting}

Therefore it is an open recursive term: the recursive call is passed as an argument.

\paragraph{The ``new'' Keyword.} \lstinline$new$ instantiates a trait by taking the
fixpoint of its corresponding open term. In fact, \lstinline$new$ is translated as
an inlined fixpoint. For example,

\begin{lstlisting}
new[Point] Point(3,4)
\end{lstlisting}

becomes

\begin{lstlisting}
let rec x : () (*$ \to $*) Point = (*$ \lambda $*)(_: ()) (*$ \to $*) Point 3 4 x in x ()
\end{lstlisting}

The composition of traits in the source language is desugared using the merge
operator. The reason that traits built on \name have conflict detection for free
is that the merge operator is enforcing that the two terms being merged are
disjoint. For example,

\begin{lstlisting}
new[Point3D] (Point(3,4) & Z(5))
\end{lstlisting}

is turned into

\george{Not correct}
\begin{lstlisting}
new ((\self: Point3D) (*$ \to $*) ((Point 3 4 self) ,, (Z 5 self)))
\end{lstlisting}

\paragraph{The ``trait'' Keyword.} The \lstinline$trait$ keyword expect a type
and is translated into an open type with that type. For example,

\begin{lstlisting}
trait[Point]
\end{lstlisting}

becomes

\begin{lstlisting}
(() (*$ \to $*) Point) (*$ \to $*) Point
\end{lstlisting}

% \subsection{Intersection Types in Existing Languages}
%
% What is an intersection type? The intersection of types $A$ and $B$
% contains exactly those values which can be used as either of type $A$
% or of type $B$.  Just as not all intersection of sets are nonempty,
% not all intersections of types are inhabited.  For example, the
% intersection of a base type $\code{Int}$ and a function type
% $\code{Int} \to \code{Int}$ is not inhabited.\bruno{put this text somewhere?}
%
% Since then various researchers have
% studied intersection types, and some languages have adopted in one
% form or another. However, while intersection types are already used
% in various languages, the lack of a merge operator removes
% considerable expressiveness.
%
%
% A number of OO languages, such as
% Java, C\#, Scala, and Ceylon\footnote{\url{http://ceylon-lang.org/}},
% already support intersection types to different degrees. Intersection
% types are particularly relevant for OOP as they can be used to model
% multiple interface inheritance. In Java, for example,
%
% \begin{lstlisting}
% interface AwithB extends A, B {}
% \end{lstlisting}
%
% \noindent introduces a new interface \lstinline{AwithB} that satisfies the interfaces of
% both \lstinline{A} and \lstinline{B}. Arguably such type can be considered as a nominal
% intersection type. Scala takes one step further by eliminating the
% need of a nominal type. For example, given two concrete traits, it is possible to
% use \emph{mixin composition} to create an object that implements both
% traits. Such an object has a (structural) intersection type:
%
% \begin{lstlisting}
% trait A
% trait B
%
% val newAB : A with B = new A with B
% \end{lstlisting}
%
% \noindent Scala also allows intersection of type parameters. For example:
% \begin{lstlisting}
% def merge[A,B] (x: A) (y: B) : A with B = ...
% \end{lstlisting}
% uses the anonymous intersection of two type parameters \lstinline{A} and
% \lstinline{B}. However, in Scala it is not possible to dynamically
% compose two objects. For example, the following code:
%
% \begin{lstlisting}
% // Invalid Scala code:
% def merge[A,B] (x: A) (y: B) : A with B = x with y
% \end{lstlisting}
%
% \noindent is rejected by the Scala compiler. The problem is that the
% \lstinline{with} construct for Scala expressions can only be used to
% mixin traits or classes, and not arbitrary objects. Note that in the
% definition \lstinline{newAB} both \lstinline{A} and \lstinline{B} are
% \emph{traits}, whereas in the definition of \lstinline{merge} the variables
% \lstinline{x} and \lstinline{y} denote \emph{objects}.
%
% This limitation essentially put intersection types in Scala in a second-class
% status. Although \lstinline{merge} returns an intersection type, it is
% hard to actually build values with such types. In essence an
% object-level introduction construct for intersection types is missing.
% As it turns out using low-level type-unsafe programming features such
% as dynamic proxies, reflection or other meta-programming techniques,
% it is possible to implement such an introduction
% construct in Scala~\cite{oliveira2013feature,rendel14attributes}. However, this
% is clearly a hack and it would be better to provide proper language
% support for such a feature.
%
% To address the limitations of intersection types in languages like
% Scala, \name allows intersecting any two terms at run time using a
% \emph{merge} operator (denoted by $ \mergeop $)~\cite{dunfield2014elaborating}.  With the merge
% operator it is trivial to implement the \lstinline{merge} function in \name:
%
% \begin{lstlisting}
% let merge[A, B * A] (x : A) (y : B) : A & B = x ,, y in (*$ \ldots $*)
% \end{lstlisting}
%
% \noindent In contrast to Scala's term-level \lstinline{with}
% construct, the operator \lstinline{,,} allows two arbitrary values \lstinline{x}
% and \lstinline{y} to be merged. The resulting type is a \emph{disjoint}
% intersection of the types of  \lstinline{x}
% and \lstinline{y} (\lstinline{A & B} in this case).
%
% A well-formed type is such that given any query type,
% it is always clear which subpart the query is referring to.
% In terms of rules, this notion of well-formedness is almost the same as the one in System $F$
% except for intersection types we require the two components to be disjoint.
%
