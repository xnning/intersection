\section{Conclusion and Future Work}
\label{sec:conclusion}

This paper described \name: a language that combines
intersection types and a merge operator.
The language is proved to be type-safe and coherent.
To ensure coherence the type system accepts only
disjoint intersections. We believe that disjoint intersection types are
intuitive, and at the same time expressive. We have shown the
applicability of disjoint intersection types to model a simple form of traits.

We implemented the core functionalities of the \name as part of a JVM-based
compiler. Based on the type system of \name, we have built an ML-like
source language compiler that offers interoperability with Java (such as object
creation and method calls). The source language is loosely based on the more
general System $F_{\omega}$ and supports a
number of other features, including records, polymorphism, mutually recursive
\code{let} bindings, type aliases, algebraic data types, pattern matching, and
first-class modules that are encoded using \code{letrec} and records.

For the future, we intend to improve our source language
and show the power of disjoint intersection types in large case
studies. One pressing challenge is to address the intersection between 
disjoint intersection types and polymorphism.
We are also interested in extending our work
to systems with a $\top$ type. This will also require an adjustment
to the notion of disjoint types. A suitable notion of
disjointness between two types $A$ and $B$ in the presence of $\top$
would be to require that the only common supertype of $A$ and $B$ is $\top$.
Finally we would like to study the
addition of union types. This will also require changes in our
notion of disjointness, since with union types there always exists
a type $A \union B$, which is the common supertype of two
types $A$ and $B$.

% Some immediate topics for
% further improvement of the results in this paper are discussed next.
%
% \paragraph{Union types.}
%
% If a type system ever contains union types (the counterpart of intersection
% types), with the following standard subtyping rules,
% \begin{mathpar}
%   \inferrule* [right=Sub\_Union\_1]
%     { }
%     {A \subtype A \union B}
%
%   \inferrule* [right=Sub\_Union\_2]
%     { }
%     {B \subtype A \union B}
% \end{mathpar}
% then no two types $A$ and $B$ can ever be disjoint, since there always exists
% the type $A \union B$, which is their common supertype. So it is reasonable to
% conjecture that such system cannot be coherent.
% \bruno{I wouldn't say this is a motivation: it sounds like we caould
%   not support union types, when I think this is not true. For example
% we could say something like: there does not exist an \emph{atomic} C ...}
%
%
% \paragraph{Implementation.}
%
% We implemented the core functionalities of the \name as part of a JVM-based
% compiler. Based on the type system of \name, we built an ML-like
% source language compiler that offers interoperability with Java (such as object
% creation and method calls). The source language is loosely based on the more
% general System $F_{\omega}$ and supports a
% number of other features, including records, mutually recursive
% \code{let} bindings, type aliases, algebraic data types, pattern matching, and
% first-class modules that are encoded using \code{letrec} and records.
%
% Relevant to this paper are the three phases in the compiler, which
% collectively turn source programs into System $F$:
%
% \begin{enumerate}
% \item A \emph{typechecking} phase that checks the usage of \name features and
%   other source language features against an abstract syntax tree that follows
%   the source syntax.
%
% \item A \emph{desugaring} phase that translates well-typed source terms into
%   \name terms. Source-level features such as multi-field records, type aliases
%   are removed at this phase. The resulting program is just an \name term
%   extended with some other constructs necessary for code generation.
%
% \item A \emph{translation} phase that turns well-typed \name terms into System
%   $F$ ones.
% \end{enumerate}
%
% Phase 3 is what we have formalized in this paper.
%
%
% \paragraph{Reduce the number of coercions.}
%
% Our translation inserts a coercion (many of them are identity functions)
% whenever subtyping occurs during a function application, which could mean
% notable run-time overhead. In the current implementation, we introduced a
% partial evaluator with three simple rewriting rules to eliminate the redundant
% identity functions as another compiler phase after the translation. In another
% version of our implementation, partial evaluation is weaved into the process of
% translation so that the unwanted identity functions are not introduced during
% the translation. Besides, since the order of the two types in a binary
% intersection does not matter, we may normalize them to avoid unnecessary
% coercions.
