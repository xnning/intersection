\section{Implementation}

We implemented the core functionalities of the \name as part of a JVM-based
compiler. The implementation supports record update instead of restriction as a
primitive; however the former is formalized with the same underlying idea of
elaborating records. Based on the type system of \name, we built an ML-like
source language compiler that offers interoperability with Java (such as object
creation and method calls). The source language is loosely based on the more
general System $F_{\omega}$ and supports a
number of other features, including multi-field records, mutually recursive
\code{let} bindings, type aliases, algebraic data types, pattern matching, and
first-class modules that are encoded with \code{letrec} and records.

Relevant to this paper are the three phases in the compiler, which 
collectively turn source programs into System $F$:

\begin{enumerate}
\item A \emph{typechecking} phase that checks the usage of \name features and
  other source language features against an abstract syntax tree that follows
  the source syntax.

\item A \emph{desugaring} phase that translates well-typed source terms into
  \name terms. Source-level features such as multi-field records, type aliases
  are removed at this phase. The resulting program is just an \name expression
  extended with some other constructs necessary for code generation.

\item A \emph{translation} phase that turns well-typed \name terms into System
  $F$ ones.
\end{enumerate}

Phase 3 is what we have formalized in this paper.

\paragraph{Removing identity functions.} Our translation inserts identity
functions whenever subtyping or record operation occurs, which could mean
notable run-time overhead. But in practice this is not an issue. In the current
implementation, we introduced a partial evaluator with three simple rewriting
rules to eliminate the redundant identity functions as another compiler phase
after the translation. In another version of our implementation, partial
evaluation is weaved into the process of translation so that the unwanted
identity functions are not introduced during the translation.
