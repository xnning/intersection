\section{Proofs}

\paragraph{Notation.} We sketch most of our proofs in two-column style: on the
left are the intermediate results and on the right are the justification (for
the previous intermediate result to reach the corresponding left-hand side).

% \subsection{Subtyping}

% \begin{theorem}[Subtyping is reflexive]
%   If $\tau$ is a well-formed type, then $\tau \subtype \tau$.
% \end{theorem}

% \begin{proof}
%   \george{TODO}
% \end{proof}

% \begin{theorem}[Subtyping is transitive]
%   If $\tau_1$, $\tau_2$, $\tau_3$ are well-formed types, and
%   $\tau_1 \subtype \tau_2$ and $\tau_2 \subtype \tau_3$, then
%   $\tau_1 \subtype \tau_3$.
% \end{theorem}

% \begin{proof}
%   \george{TODO}
% \end{proof}

% \bruno{fix numbering of lemmas}
% \bruno{reflexitivity and transitivity missing. You can do a proof
%   sketch instead of a full proof. Just say in 1 or 2 sentences what is
% the main idea. You can mention that we have a full proof in Coq.}
% \bruno{target type system is missing 3 cases: Tunit; Tproj1; TProj2}

% \subsection{Elaboration}

\lemmasub*
\begin{proof}
  By structural induction of the derivation.

  \begin{itemize}

  \item \textbf{Case}
    \begin{flalign*}
      & \ruleSubVar &
    \end{flalign*}

    \begin{tabular}{rll}
      & $ \judgeTarget {\epsilon} {\lam x {\im \alpha} x} {\alpha \to \alpha} $ & By $ \ruleLabelTargetvar $ and $ \ruleLabelTargetlam $
    \end{tabular} \\

  \item \textbf{Case}
    \begin{flalign*}
      & \ruleSubTop &
    \end{flalign*}

    \begin{tabular}{rll}
      & $ \judgeTarget {\epsilon} {\lam x {\im \tau} ()} {\im \tau \to ()} $ & By $ \ruleLabelTargetvar $ and $ \ruleLabelTargetlam $ \\
      & $ \judgeTarget {\epsilon} {\lam x {\im \tau} ()} {\im \tau \to \im \top} $ & By the definition of $ \im \cdot $
    \end{tabular} \\

  \item \textbf{Case}
    \begin{flalign*}
      & \ruleSubFun &
    \end{flalign*}

    \begin{tabular}{rll}
      & $ \judgeTarget \epsilon {C_1} {\im {\tau_3} \to \im {\tau_1}} $ & By i.h. \\
      & $ \judgeTarget \epsilon {C_2} {\im {\tau_2} \to \im {\tau_4}} $ & By i.h. \\
      & $ \judgeTarget {\epsilon, x \hast {\im {\tau_3}}} x {\im {\tau_3}} $ & By $\ruleLabelTargetvar$ \\
      & $ \tau_3 \subtype \tau_1 \yields {C_1} $ & Premise \\
      & $ \judgeTarget {\epsilon, x \hast {\im {\tau_3}}} {\app {C_1} x} {\im {\tau_1}} $ & By $\ruleLabelTargetApp$ \\
      & $ \judgeTarget {\epsilon, f \hast {\im {\tau_1 \to \tau_2}}} f {\im {\tau_1 \to \tau_2}} $ & By $\ruleLabelTargetvar$ \\
      & $ \judgeTarget {\epsilon, f \hast {\im {\tau_1 \to \tau_2}}} f {\im {\tau_1} \to \im {\tau_2}} $ & By the definition of $ \im \cdot $ \\
      & $ \judgeTarget {\epsilon, f \hast {\im {\tau_1 \to \tau_2}}, x \hast {\im {\tau_3}}} {\app f {(\app {C_1} x})} {\im {\tau_2}} $ & By $\ruleLabelTargetApp$ \\
      & $ \judgeTarget {\epsilon, f \hast {\im {\tau_1 \to \tau_2}}, x \hast {\im {\tau_3}}} {\app {C_2} {(\app f {(\app {C_1} x})})} {\im {\tau_4}} $ & By $\ruleLabelTargetApp$ \\
      & $ \judgeTarget \epsilon {\lam f {\im {\tau_1 \to \tau_2}} {\lam x {\im {\tau_3}} {\app {C_2} {(\app f {(\app {C_1} x)})}}}} {\im {\tau_1 \to \tau_2} \to \im {\tau_3} \to \im {\tau_4}} $ & By applying $\ruleLabelTargetlam$ twice \\
      & $ \judgeTarget \epsilon {\lam f {\im {\tau_1 \to \tau_2}} {\lam x {\im {\tau_3}} {\app {C_2} {(\app f {(\app {C_1} x)})}}}} {\im {\tau_1 \to \tau_2} \to \im {\tau_3 \to \tau_4}} $ & By the definition of $\im \cdot$
    \end{tabular} \\

  \item \textbf{Case}
    \begin{flalign*}
      & \ruleSubForall &
    \end{flalign*}

    \begin{tabular}{rll}
      & $ \judgeTarget \epsilon C {\im {\tau_1} \to {\im {\subst {\alpha_1} {\alpha_2} \tau_2}}} $ & By i.h. \\
      & $ \judgeTarget {\epsilon, f \hast {\im {\for {\alpha_1} {\tau_1}}}} f {\im {\for {\alpha_1} {\tau_1}}} $ & By $\ruleLabelTargetvar$ \\
      & $ \judgeTarget {\epsilon, f \hast {\im {\for {\alpha_1} {\tau_1}}}} f {\for {\alpha_1} {\im {\tau_1}}} $ & By the definition of $ \im \cdot $ \\
      & $ \judgeTarget {\epsilon, f \hast {\im {\for {\alpha_1} {\tau_1}}}, \alpha} {\tapp f \alpha} {\subst \alpha {\alpha_1} {\im {\tau_1}}} $ & By $\ruleLabelTargetvar$ and $\ruleLabelTargetTApp$ \\
      & $ \judgeTarget {\epsilon, f \hast {\im {\for {\alpha_1} {\tau_1}}}, \alpha} {\app C {(\tapp f \alpha)}} {\subst \alpha {\alpha_1} {\im {\subst {\alpha_1} {\alpha_2} {\tau_2}}}} $ & By $\ruleLabelTargetApp$ \\
      & $ \judgeTarget {\epsilon, f \hast {\im {\for {\alpha_1} {\tau_1}}}} {\blam \alpha {\app C {(\tapp f \alpha)}}} {\for {\alpha_2} {\im {\tau_2}}} $ & By $\ruleLabelTargetBLam$ and substitution \george{Substitution is problematic} \\
      & $ \judgeTarget \epsilon {\lam f {\im {\for {\alpha_1} {\tau_1}}} {\blam \alpha {\app C {(\tapp f \alpha)}}}} {\im {\for {\alpha_1} {\tau_1}} \to \for {\alpha_2} {\im {\tau_2}}} $ & By $\ruleLabelTargetlam$ \\
      & $ \judgeTarget \epsilon {\lam f {\im {\for {\alpha_1} {\tau_1}}} {\blam \alpha {\app C {(\tapp f \alpha)}}}} {\im {\for {\alpha_1} {\tau_1}} \to \im {\for {\alpha_2} {\tau_2}}} $ & By the definition of $\im \cdot$
    \end{tabular} \\

  \item \textbf{Case}
    \begin{flalign*}
      & \ruleSubAnd &
    \end{flalign*}

    \begin{tabular}{rll}
      & $\judgeTarget {\epsilon, x \hast \im {\tau_1}} x {\im {\tau_1}}$ & By $\ruleLabelTargetvar$ \\
      & $\judgeTarget \epsilon {C_1} {\im {\tau_1} \to \im {\tau_2}}$ & By i.h. \\
      & $\judgeTarget {\epsilon, x \hast \im {\tau_1}} {\app {C_1} x} {\im {\tau_2}}$ & By $\ruleLabelTargetApp$ and weakening \\
      & $\judgeTarget {\epsilon, x \hast \im {\tau_1}} {\app {C_2} x} {\im {\tau_3}}$ & Similar \\
      & $\judgeTarget {\epsilon, x \hast \im {\tau_1}} {\pair {\app {C_1} x} {\app {C_2} x}} {\pair {\im {\tau_2}} {\im {\tau_3}}}$ & By $\ruleLabelTargetPair$ \\
      & $\judgeTarget {\epsilon, x \hast \im {\tau_1}} {\pair {\app {C_1} x} {\app {C_2} x}} {\im {\tau_2 \andOp \tau_3}}$ & By the definition of $\im \cdot$ \\
      & $\judgeTarget \epsilon {\lam x {\im {\tau_1}} {\pair {\app {C_1} x} {\app {C_2} x}}} {\im {\tau_1} \to {\im {\tau_2 \andOp \tau_3}}}$ & By $\ruleLabelTargetlam$
    \end{tabular} \\

  \item \textbf{Case}
    \begin{flalign*}
      & \ruleSubAndleft &
    \end{flalign*}

    \begin{tabular}{rll}
      & $ \judgeTarget {\epsilon, x \hast \im {\tau_1 \andOp \tau_2}} x {\im {\tau_1 \andOp \tau_2}} $ & By $\ruleLabelTargetvar$ \\
      & $ \judgeTarget {\epsilon, x \hast \im {\tau_1 \andOp \tau_2}} x {\pair {\im {\tau_1}} {\im {\tau_2}}} $ & By the definition of $\im \cdot$ \\
      & $ \judgeTarget {\epsilon, x \hast \im {\tau_1 \andOp \tau_2}} {\proj 1 x} {\im {\tau_1}} $ & By $\ruleLabelTargetProjLeft$ \\
      & $ \judgeTarget \epsilon C {\im {\tau_1} \to \im {\tau_3}} $ & By i.h. \\
      & $ \judgeTarget {\epsilon, x \hast \im {\tau_1 \andOp \tau_2}} C {\im {\tau_1} \to \im {\tau_3}} $ & By weakening \\
      & $ \judgeTarget {\epsilon, x \hast \im {\tau_1 \andOp \tau_2}} {\app C {(\proj 1 x)}} {\im {\tau_3}} $ & By $\ruleLabelTargetApp$ \\
      & $ \judgeTarget \epsilon {\lam x {\im {\tau_1 \andOp \tau_2}} {\app C {(\proj 1 x)}}} {\im {\tau_1 \andOp \tau_2} \to \im {\tau_3}} $ & By $\ruleLabelTargetlam$
    \end{tabular} \\

  \item \textbf{Case}
    \begin{flalign*}
      & \ruleSubAndright &
    \end{flalign*}

    By symmetry with the above case. \\

  \item \textbf{Case}
    \begin{flalign*}
      & \ruleSubRec &
    \end{flalign*}

    \begin{tabular}{rll}
      & $ \tau_1 \subtype \tau_2 \yields C $ & Premise \\
      & $ \judgeTarget \epsilon C {\im {\tau_1} \to \im {\tau_2}} $ & By i.h. \\
      & $ \judgeTarget {\epsilon, x \hast \im {\recordType l {\tau_1}}} x {\im {\recordType l {\tau_1}}} $ & By $ \ruleLabelTargetvar $ \\
      & $ \judgeTarget {\epsilon, x \hast \im {\recordType l {\tau_1}}} x {\im {\tau_1}} $ & By the definition of $ \im \cdot $ \\
      & $ \judgeTarget {\epsilon, x \hast \im {\recordType l {\tau_1}}} {\app C x} {\im {\tau_2}} $ & By $ \ruleLabelTargetApp $ \\
      & $ \judgeTarget {\epsilon, x \hast \im {\recordType l {\tau_1}}} {\app C x} {\im {\recordType l {\tau_2}}} $ & By the definition of $ \im \cdot $ \\
      & $ \judgeTarget \epsilon {\lam x {\im {\recordType l {\tau_1}}} {\app C x}} {\im {\recordType l {\tau_1}} \to \im {\recordType l {\tau_2}}} $ & By $ \ruleLabelTargetlam $
    \end{tabular} \\

  \end{itemize}

\end{proof}

\lemmaselect*
\begin{proof}
  By structural induction of the derivation.

  \begin{itemize}

  \item \textbf{Case}
    \begin{flalign*}
      & \ruleGetElab &
    \end{flalign*}

    \begin{tabular}{rll}
      & $ \judgeTarget \epsilon {\lam x {\im {\recordType l \tau}} x} {\im {\recordType l \tau} \to \im {\recordType l \tau}} $ & By $ \ruleLabelTargetlam $ and $\ruleLabelTargetvar$ \\
      & $ \judgeTarget \epsilon {\lam x {\im {\recordType l \tau}} x} {\im {\recordType l \tau} \to \im \tau} $ & By the definition of $ \im \cdot $
    \end{tabular} \\

  \item \textbf{Case}
    \begin{flalign*}
      & \ruleGetLeftElab &
    \end{flalign*}

    \begin{tabular}{rll}
      & $ \judgeTarget {\epsilon, x \hast \im {\tau_1 \andOp \tau_2}} x {\im {\tau_1 \andOp \tau_2}} $ & By $ \ruleLabelTargetvar $ \\
      & $ \judgeTarget {\epsilon, x \hast \im {\tau_1 \andOp \tau_2}} x {\pair {\im {\tau_1}} {\im {\tau_2}}} $ & By the definition of $\im \cdot$ \\
      & $ \judgeTarget {\epsilon, x \hast \im {\tau_1 \andOp \tau_2}} {\proj 1 x} {\im {\tau_1}} $ & By $\ruleLabelTargetProjLeft$ \\
      & $ \judgeTarget \epsilon C {\im {\tau_1} \to \im \tau} $ & By i.h. \\
      & $ \judgeTarget {\epsilon, x \hast \im {\tau_1 \andOp \tau_2}} C {\im {\tau_1} \to \im \tau} $ & By weakening \\
      & $ \judgeTarget {\epsilon, x \hast \im {\tau_1 \andOp \tau_2}} {\app C {(\proj 1 x)}} {\im \tau} $ & By $\ruleLabelTargetApp$ \\
      & $ \judgeTarget \epsilon {\lam x {\im {\tau_1 \andOp \tau_2}} {\app C {(\proj 1 x)}}} {\im {\tau_1 \andOp \tau_2} \to \im \tau} $ & By $ \ruleLabelTargetlam $
    \end{tabular} \\

  \item \textbf{Case}
    \begin{flalign*}
      & \ruleGetRightElab &
    \end{flalign*}

    By symmetry with the above case. \\

\end{itemize}
\end{proof}


\lemmarestrict*
\begin{proof}
  By structural induction of the derivation.

  \begin{itemize}

  \item \textbf{Case}
    \begin{flalign*}
      & \ruleRestrictElab &
    \end{flalign*}

    \begin{tabular}{rll}
      & $ \judgeTarget \epsilon {\lam x {\im {\recordType l \tau}} {()}} {\im {\recordType l \tau} \to ()} $ & By $\ruleLabelTargetUnit$ and $\ruleLabelTargetlam$ \\
      & $ \judgeTarget \epsilon {\lam x {\im {\recordType l \tau}} {()}} {\im {\recordType l \tau} \to \im \top} $ & By the definition of $\im \cdot$
    \end{tabular} \\

  \item \textbf{Case}
    \begin{flalign*}
      & \ruleRestrictLeftElab &
    \end{flalign*}

    \begin{tabular}{rll}
      & $\judgeRestrict {\tau_1} l \tau \yields C$ & Premise \\
      & $\judgeTarget \epsilon C {\im {\tau_1} \to \im \tau} $ & By i.h. \\
      & $\judgeTarget {\epsilon, x \hast \im {\tau_1 \andOp \tau_2}} x {\im {\tau_1 \andOp \tau_2}} $ & By $\ruleLabelTargetvar$ \\
      & $\judgeTarget {\epsilon, x \hast \im {\tau_1 \andOp \tau_2}} x {\pair {\im {\tau_1}} {\im {\tau_2}}} $ & By the definition of $\im \cdot$ \\
      & $\judgeTarget {\epsilon, x \hast \im {\tau_1 \andOp \tau_2}} {\proj 1 x} {\im {\tau_1}} $ & By $\ruleLabelTargetProjLeft$ \\
      & $\judgeTarget {\epsilon, x \hast \im {\tau_1 \andOp \tau_2}} {\proj 2 x} {\im {\tau_2}} $ & By $\ruleLabelTargetProjRight$ \\
      & $\judgeTarget {\epsilon, x \hast \im {\tau_1 \andOp \tau_2}} {\app C {(\proj 1 x)}} {\im {\tau}} $ & By $\ruleLabelTargetApp$ \\
      & $\judgeTarget {\epsilon, x \hast \im {\tau_1 \andOp \tau_2}} {\pair {\app C {(\proj 1 x)}} {\proj 2 x}} {\pair {\im {\tau}} {\im {\tau_2}}} $ & By $\ruleLabelTargetPair$ \\
      & $\judgeTarget {\epsilon, x \hast \im {\tau_1 \andOp \tau_2}} {\pair {\app C {(\proj 1 x)}} {\proj 2 x}} {\im {\tau \andOp \tau_2}} $ & By the definition of $\im \cdot$ \\
      & $\judgeTarget \epsilon {\lam x {\im {\tau_1
          \andOp \tau_2}} {\pair {\app C {(\proj 1 x)}} {\proj 2 x}}} {\im {\tau_1 \andOp \tau_2} \to \im {\tau \andOp \tau_2}} $ & By $\ruleLabelTargetlam$
    \end{tabular} \\

  \item \textbf{Case}
    \begin{flalign*}
      & \ruleRestrictRightElab &
    \end{flalign*}

    By symmetry with the above case. \\

\end{itemize}
\end{proof}

\begin{comment}
\begin{lemma}[\ruleLabelUpdate~rules produce type-correct coercion] \label{lemma:update-correct}
  If $ \judgeUpdate \tau l {\tau_1 \yields E} {\tau_2} {\tau_3} \yields C $ and $
  \judgeTarget \Gamma E {\im {\tau_1}} $ for some $ \Gamma $, then
  $ \judgeTarget \Gamma C {\im \tau \to \im {\tau_2}} $.
\end{lemma}

\begin{proof}
  By structural induction of the derivation.

  \begin{itemize}

  \item \textbf{Case}
    \begin{flalign*}
      & \ruleUpdateElab &
    \end{flalign*}

    \begin{tabular}{rll}
      & $ \judgeTarget \Gamma {\lam \_ {\im {\recordType l \tau}} E} {\im {\recordType l \tau}} \to \im {\tau_1} $ & By $ \ruleLabelTargetlam $, $ \ruleLabelTargetvar $, and the hypothesis
    \end{tabular} \\

  \item \textbf{Case}
    \begin{flalign*}
      & \ruleUpdateLeftElab &
    \end{flalign*}

    \begin{tabular}{rll}
      & $ \judgeTarget {\Gamma, x \hast \im {\tau_1 \andOp \tau_2}} x {\im {\tau_1 \andOp \tau_2}} $ & By $\ruleLabelTargetvar$ \\
      & $ \judgeTarget {\Gamma, x \hast \im {\tau_1 \andOp \tau_2}} x {\pair {\im {\tau_1}} {\im {\tau_2}}} $ & By the definition of $\im \cdot$ \\
      & $ \judgeTarget {\Gamma, x \hast \im {\tau_1 \andOp \tau_2}} {\proj 1 x} {\im {\tau_1}} $ & By $\ruleLabelTargetProjLeft$ \\
      & $ \judgeTarget \Gamma C {\im {\tau_1} \to \im {\tau_3}} $ & By i.h. \\
      & $ \judgeTarget {\Gamma, x \hast \im {\tau_1 \andOp \tau_2}} C {{\im {\tau_1}} \to \im {\tau_3}} $ & By weakening \\
      & $ \judgeTarget {\Gamma, x \hast \im {\tau_1 \andOp \tau_2}} {\app C {(\proj 1 x)}} {\im {\tau_3}} $ & By $\ruleLabelTargetApp$ \\
      & $ \judgeTarget \Gamma {\lam x {\im {\tau_1 \andOp \tau_2}} {\app C {(\proj 1 x)}}} {\im {\tau_1 \andOp \tau_2} \to \im {\tau_3}} $ & By $\ruleLabelTargetlam$
    \end{tabular} \\

  \item \textbf{Case}
    \begin{flalign*}
      & \ruleUpdateRightElab &
    \end{flalign*}

    By symmetry with the above case. \\

  \end{itemize}
\end{proof}
\end{comment}

\begin{lemma}[Preservation of well-formedness under type translation] \label{lemma:preserve-wf}
  If $ \judgeSourceWF \gamma \tau $, then $ \judgeTargetWF {\im \gamma} {\im \tau} $.
\end{lemma}

\begin{proof}
  By structural induction of the derivation. The only case to consider is $ \ruleLabelSourceWF $.

  \begin{itemize}

  \item \textbf{Case}

    \begin{flalign*}
      & \ruleSourceWF &
    \end{flalign*}

    \begin{tabular}{rll}
      & $ \ftv \tau \in \gamma $ & Premise \\
      & $ \ftv {\im \tau} \in \im \gamma $ & By the definition of $ \im \cdot $ \\
      & $ \judgeTargetWF {\im \gamma} {\im \tau} $ & By $ \ruleLabelTargetWF $
    \end{tabular} \\

  \end{itemize}
\end{proof}

\theorempreservation*
\begin{proof}
  By structural induction of the derivation.

  \begin{itemize}

  \item \textbf{Case}
    \begin{flalign*}
      & \ruleSourceVarElab &
    \end{flalign*}

    \begin{tabular}{rll}
     & $ (x,\tau) \in \gamma $ & Premise \\
     & $ (x,\im \tau) \in \im \gamma $ & By the definition of $ \im \cdot $ \\
     & $ \judgeTarget {\im \gamma} x {\im \tau} $ & By $ \ruleLabelTargetvar $
    \end{tabular} \\

  \item \textbf{Case}
    \begin{flalign*}
      & \ruleSourceTopElab &
    \end{flalign*}

    \begin{tabular}{rll}
      & $\judgeTarget {\im \gamma} {()} {()} $ & By $\ruleLabelTargetUnit$ \\
      & $\judgeTarget {\im \gamma} {()} {\im \top}$ & By the definition of $ \im \cdot$
    \end{tabular} \\

  \item \textbf{Case}
    \begin{flalign*}
      & \ruleSourceLamElab &
    \end{flalign*}

    \begin{tabular}{rll}
      & $ \judgeSource {\gamma, x \hast \tau} e {\tau_1} \yields E $ & Premise \\
      & $ \judgeTarget {\im {\gamma, x \hast \tau}} E {\im {\tau_1}} $ & By i.h. \\
      & $ \judgeTarget {\im \gamma, x \hast \im \tau} E {\im {\tau_1}} $ & By the definition of $ \im \cdot $ \\
      & $ \judgeTarget {\im \gamma} {\lam x {\im \tau} E} {\im \tau \to \im {\tau_1}} $ & By $ \ruleLabelTargetlam $ \\
      & $ \judgeTarget {\im \gamma} {\lam x {\im \tau} E} {\im {\tau \to \tau_1}} $ & By the definition of $ \im \cdot $
    \end{tabular} \\

  \item \textbf{Case}
    \begin{flalign*}
      & \ruleSourceAppElab &
    \end{flalign*}

    \begin{tabular}{rll}
     & $ \judgeSource \gamma {e_1} {\tau_1 \to \tau_2} \yields {E_1} $  & Premise \\
     & $ \judgeTarget {\im \gamma} {E_1} {\im {\tau_1 \to \tau_2}} $ & By i.h. \\
     & $ \judgeSource \gamma {e_2} {\tau_3} \yields {E_2} $ & Premise \\
     & $ \judgeTarget {\im \gamma} {E_2} {\im {\tau_3}} $ & By i.h. \\
     & $ \tau_3 \subtype \tau_1 \yields C $ & Premise \\
     & $ \judgeTarget \epsilon C {\im {\tau_3} \to \im {\tau_1}} $ & By Lemma~\ref{lemma:sub} \\
     & $ \judgeTarget {\im \gamma} {\app {E_1} {(\app C E_2)}} {\im {\tau_2}} $ & By $ \ruleLabelTargetApp $ and the definition of $ \im \cdot $
    \end{tabular} \\

  \item \textbf{Case}
    \begin{flalign*}
      & \ruleSourceBLamElab &
    \end{flalign*}

    \begin{tabular}{rll}
      & $ \judgeSource {\gamma, \alpha} e \tau \yields E $ & Premise \\
      & $ \judgeTarget {\im {\gamma, \alpha}} E {\im \tau} $ & By i.h. \\
      & $ \judgeTarget {\im \gamma, \alpha} E {\im \tau} $ & By the definition of $ \im \cdot $ \\
      & $ \judgeTarget {\im \gamma} {\blam \alpha E} {\for \alpha {\im \tau}} $ & By $ \ruleLabelTargetBLam $ \\
      & $ \judgeTarget {\im \gamma} {\blam \alpha E} {\im {\for \alpha \tau}} $ & By the definition of $ \im \cdot $
    \end{tabular} \\

  \item \textbf{Case}
    \begin{flalign*}
      & \ruleSourceTAppElab &
    \end{flalign*}

    \begin{tabular}{rll}
     & $ \judgeSource \gamma e {\for \alpha \tau_1} \yields E $ & Premise \\
     & $ \judgeTarget {\im \gamma} E {\im {\for \alpha \tau_1}} $ & By i.h. \\
     & $ \judgeTarget {\im \gamma} E {\for \alpha \im {\tau_1}} $ & By the definition of $ \im \cdot $ \\
     & $ \judgeSourceWF \gamma \tau $ & Premise \\
     & $ \judgeTargetWF {\im \gamma} {\im \tau} $ & By Lemma~\ref{lemma:preserve-wf} \\
     & $ \judgeTarget \gamma {\tapp E {\im \tau}} {\subst {\im \tau} \alpha {\im {\tau_1}}} $ & By $ \ruleLabelTargetTApp $ \\
     & $ \judgeTarget \gamma {\tapp E {\im \tau}} {\im {\subst \tau \alpha {\tau_1}}} $ & By substitution lemma
    \end{tabular} \\

  \item \textbf{Case}
    \begin{flalign*}
      & \ruleSourceMergeElab &
    \end{flalign*}

    \begin{tabular}{rll}
      & $ \judgeSource \gamma {e_1} {\tau_1} \yields {E_1} $ & Premise \\
      & $ \judgeTarget {\im \gamma} {E_1} {\im {\tau_1}} $ & By i.h. \\
      & $ \judgeTarget {\im \gamma} {E_2} {\im {\tau_2}} $ & Similar to the above \\
      & $ \judgeTarget {\im \gamma} {\pair {E_1} {E_2}} {\pair {\im {\tau_1}} {\im {\tau_2}}} $ & By $ \ruleLabelTargetPair $ \\
      & $ \judgeTarget {\im \gamma} {\pair {E_1} {E_2}} {\im {\tau_1 \andOp \tau_2}} $ & By the definition of $ \im \cdot $
    \end{tabular} \\

  \item \textbf{Case}
    \begin{flalign*}
      & \ruleSourcerecordConstructElab &
    \end{flalign*}

    \begin{tabular}{rll}
      & $ \judgeSource \gamma e \tau \yields E $ & Premise \\
      & $ \judgeTarget {\im \gamma} E {\im \tau} $ & By i.h. \\
      & $ \judgeTarget {\im \gamma} E {\im {\recordType l \tau}} $ & By the definition of $ \im \cdot $
    \end{tabular} \\

  \item \textbf{Case}
    \begin{flalign*}
      & \ruleSourceRecSelectElab &
    \end{flalign*}

    \begin{tabular}{rll}
     & $ \judgeSelect \tau l {\tau_1} \yields C $ & Premise \\
     & $ \judgeTarget \epsilon C {\im \tau \to \im {\tau_1}} $ & By Lemma~\ref{lemma:select} \\
     & $ \judgeTarget {\im \gamma} C {\im \tau \to \im {\tau_1}} $ & By weakening \\
     & $ \judgeSource \gamma e \tau \yields E $ & Premise \\
     & $ \judgeTarget {\im \gamma} E {\im \tau} $ & By i.h. \\
     & $ \judgeTarget {\im \gamma} {\app C E} {\im {\tau_1}} $ & By $ \ruleLabelTargetApp $
    \end{tabular} \\

  \item \textbf{Case}
    \begin{flalign*}
      & \ruleSourceRecRestrictElab &
    \end{flalign*}

    \begin{tabular}{rll}
     & $ \judgeRestrict \tau l {\tau_1} \yields C $ & Premise \\
     & $ \judgeTarget \epsilon C {\im \tau \to \im {\tau_1}} $ & By Lemma~\ref{lemma:restrict} \\
     & $ \judgeTarget {\im \gamma} C {\im \tau \to \im {\tau_1}} $ & By weakening \\
     & $ \judgeSource \gamma e \tau \yields E $ & Premise \\
     & $ \judgeTarget {\im \gamma} E {\im \tau} $ & By i.h. \\
     & $ \judgeTarget {\im \gamma} {\app C E} {\im {\tau_1}} $ & By $ \ruleLabelTargetApp $
    \end{tabular} \\

  % \item \textbf{Case}
  %   \begin{flalign*}
  %     & \ruleSourcerecordUpdateelab &
  %   \end{flalign*}

  %   \begin{tabular}{rll}
  %     & $ \judgeSource \gamma {e_1} {\tau_1} \yields {E_1} $ & Premise \\
  %     & $ \judgeTarget {\im \gamma} {E_1} {\im {\tau_1}} $ & By i.h. \\
  %     & $ \judgeUpdate \tau l {\tau_1 \yields {E_1}} {\tau_2} {\tau_3} \yields C $ & Premise \\
  %     & $ \judgeTarget {\im \gamma} C {\im \tau \to \im {\tau_2}} $ & By Lemma~\ref{lemma:update-correct} \\
  %     & $ \judgeSource \gamma e \tau \yields E $ & Premise \\
  %     & $ \judgeTarget {\im \gamma} E {\im \tau} $ & By i.h. \\
  %     & $ \judgeTarget {\im \gamma} {\app C E} {\im {\tau_2}} $ & By $ \ruleLabelTargetApp $
  %   \end{tabular} \\

  \end{itemize}
\end{proof}