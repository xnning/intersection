\documentclass{article}

% Remote packages

% For pdflatex, replaced by fontspec:
\usepackage{tgpagella}
\usepackage[T1]{fontenc}
\usepackage[utf8]{inputenc}

% For xelatex or lualatex
% \usepackage{fontspec}
% \setmainfont{Times New Roman}

\usepackage{amsmath}
\usepackage{amsthm}
\usepackage{amssymb}
\usepackage{mathtools} % For \Coloneqq
\usepackage{bm}        % Bold symbols in maths mode
\usepackage{fixltx2e}
\usepackage{stmaryrd}
\usepackage[dvipsnames]{xcolor}
\usepackage{listings} % For code listings
% \usepackage{minted}
% \usemintedstyle{murphy}
\usepackage{fancyvrb}
\usepackage{url}
\usepackage{xspace}
\usepackage{comment}

% Typography
\usepackage[euler-digits,euler-hat-accent]{eulervm}

% Copied from the FCore paper:
\usepackage[colorlinks=true,allcolors=black,breaklinks,draft=false]{hyperref}   % hyperlinks, including DOIs and URLs in bibliography
% known bug: http://tex.stackexchange.com/questions/1522/pdfendlink-ended-up-in-different-nesting-level-than-pdfstartlink

% Figures with borders
% http://en.wikibooks.org/wiki/LaTeX/Floats,_Figures_and_Captions
% \usepackage{float}
% \floatstyle{boxed}
% \restylefloat{figure}

% Local packages

\usepackage{styles/bcprules}    % by Benjamin C. Pierce
\usepackage{styles/cmll}
\usepackage{styles/mathpartir}  % by Didier Rémy


% ! Always load mathastext last
% http://mirrors.ibiblio.org/CTAN/macros/latex/contrib/mathastext/mathastext.pdf
% \renewcommand\familydefault\ttdefault
% \usepackage{mathastext}
% \renewcommand\familydefault\rmdefault


\newtheorem{theorem}{Theorem}
\newtheorem{lemma}{Lemma}

% Define macros immediately before the \begin{document} command
% math-mode versions of \rlap, etc
% from Alexander Perlis, "A complement to \smash, \llap, and lap"
%   http://math.arizona.edu/~aprl/publications/mathclap/
\def\clap#1{\hbox to 0pt{\hss#1\hss}}
\def\mathllap{\mathpalette\mathllapinternal}
\def\mathrlap{\mathpalette\mathrlapinternal}
\def\mathclap{\mathpalette\mathclapinternal}
\def\mathllapinternal#1#2{\llap{$\mathsurround=0pt#1{#2}$}}
\def\mathrlapinternal#1#2{\rlap{$\mathsurround=0pt#1{#2}$}}
\def\mathclapinternal#1#2{\clap{$\mathsurround=0pt#1{#2}$}}

% math-mode versions of \rlap, etc
% from Alexander Perlis, "A complement to \smash, \llap, and lap"
%   http://math.arizona.edu/~aprl/publications/mathclap/
\def\clap#1{\hbox to 0pt{\hss#1\hss}}
\def\mathllap{\mathpalette\mathllapinternal}
\def\mathrlap{\mathpalette\mathrlapinternal}
\def\mathclap{\mathpalette\mathclapinternal}
\def\mathllapinternal#1#2{\llap{$\mathsurround=0pt#1{#2}$}}
\def\mathrlapinternal#1#2{\rlap{$\mathsurround=0pt#1{#2}$}}
\def\mathclapinternal#1#2{\clap{$\mathsurround=0pt#1{#2}$}}

\newcommand{\horizontalrule}{
  \begin{center}
    \line(1,0){250}
  \end{center}
}

\newcommand{\code}[1]{\texttt{#1}}
\definecolor{light-gray}{gray}{0.9}
\newcommand{\highlight}[1]{\colorbox{light-gray}{#1}}

\newcommand{\im}[1]{\lvert #1 \rvert}
\newcommand{\powerset}[1]{\mathcal{P}(#1)}
\newcommand{\universe}{\mathbb{U}}

\newcommand{\turns}{\vdash}
\newcommand{\oftype}{\!:\!}
\newcommand{\subtype}{<:}
\newcommand{\commonsuper}{\Uparrow}
\newcommand{\commonsub}{\Downarrow}
\newcommand{\disjoint}{*}
\newcommand{\disjointimpl}{*_\textnormal{i}}
\newcommand{\disjointax}{*_\textnormal{ax}}
\newcommand{\subst}[2]{\lbrack #2 := #1 \rbrack~}

\newcommand{\yields}[1]{\highlight{$\; \hookrightarrow #1$}}
% \newcommand{\yields}[1]{}

\newcommand{\ftv}[1]{\textsf{ftv}(#1)}

\newcommand{\binderspace}{\,}
\newcommand{\appspace}{\;}

\newcommand{\inter}{\&}
\newcommand{\union}{|}
\newcommand{\for}[2]{\forall #1.\binderspace #2}
\newcommand{\fordis}[3]{\for {(#1 \disjoint #2)} {#3}}
\newcommand{\lam}[2]{\lambda #1.\binderspace #2}
\newcommand{\lamty}[3]{\lam {(#1 \oftype #2)} #3}
\newcommand{\blam}[2]{\Lambda #1.\binderspace #2}
\newcommand{\blamdis}[3]{\blam {(#1 \disjoint #2)} #3}
\newcommand{\mergeop}{,,}
\newcommand{\app}[2]{#1 \; #2}
\newcommand{\tapp}[2]{#1 \appspace #2}
\newcommand{\pair}[2]{(#1, #2)}
\newcommand{\proj}[2]{{\code{proj}}_{#1} #2}
\newcommand{\fst}[1]{\app {\code{fst}} {#1}}
\newcommand{\snd}[1]{\app {\code{snd}} {#1}}
\newcommand{\recordType}[2]{\{ #1 : #2 \}}
\newcommand{\recordCon}[2]{\{ #1 = #2 \}}

\newcommand{\true}{\code{True}}
\newcommand{\tyint}{\code{Int}}
\newcommand{\tybool}{\code{Bool}}
\newcommand{\tychar}{\code{Char}}
\newcommand{\tystring}{\code{String}}

% Judgements
\newcommand{\jwf}[2]{#1 \turns #2}
\newcommand{\jatomic}[1]{#1 \ \textnormal{atomic}}
\newcommand{\jtype}[3]{\turns #2 \ \oftype \ #3}
\newcommand{\jdis}[3]{\turns #2 \disjoint #3}
\newcommand{\jdisimpl}[3]{\turns #2 \disjointimpl #3}

\newcommand{\reflabel}[1]{(\textsc{#1})}

% \newcommand{\name}{$ \lambda_{\&} $\xspace}
\newcommand{\name}{\xspace[name]\xspace}

\newcommand{\authornote}[3]{\textcolor{#2}{\textsc{#1}: #3}}
\newcommand\bruno[1]{\authornote{bruno}{Red}{#1}}
\newcommand\george[1]{\authornote{george}{Blue}{#1}}

\newcommand{\formsub}{\framebox{$ A \subtype B \yields E $}}

\newcommand{\makelabelsub}[1]{Sub\_#1}

\newcommand{\labelsubint}{\makelabelsub Int}
\newcommand{\rulesubint}{
  \inferrule* [right=\labelsubint]
    { }
    {\tyint \subtype \tyint \yields {\lamty x {\im \alpha} x}}
}

\newcommand{\labelsubtop}{\makelabelsub Top}
\newcommand{\rulesubtop}{
  \inferrule* [right=\labelsubtop]
    { }
    {A \subtype \top \yields {\lamty x {\im A} \unit}}
}

\newcommand{\labelsubvar}{\makelabelsub Var}
\newcommand{\rulesubvar}{
  \inferrule* [right=\labelsubvar]
    { }
    {\alpha \subtype \alpha \yields {\lamty x {\im \alpha} x}}
}

\newcommand{\labelsubfun}{\makelabelsub Fun}
\newcommand{\rulesubfun}{
  \inferrule* [right=\labelsubfun]
    {{B_1} \subtype {A_1} \yields {E_1} \\
     {A_2} \subtype {B_2} \yields {E_2}}
    {{A_1 \to A_2} \subtype {B_1 \to B_2}
    \yields
        {\lamty f {\im {A_1 \to A_2}}
        {\lamty x {\im {B_1}}
            {\app {E_2} {(\app f {(\app {E_1} x)})}}}}}
}

\newcommand{\labelsubforall}{\makelabelsub Forall}
\newcommand{\rulesubforall}{
  \inferrule* [right=\labelsubforall]
    {{A_1} \subtype {A_2} \yields E}
    {\for {\alpha} {A_1} \subtype \for {\alpha} {A_2}
      \yields
        {\lamty f {\im {\for {\alpha} {A_1}}}
          {\blam \alpha {\app E {(\app f \alpha)}}}}}
}

\newcommand{\rulesubforalldis}{
  \inferrule* [right=\labelsubforall]
    {{B_1} \subtype {B_2} \yields E}
    {\fordis {\alpha} A {B_1} \subtype \fordis {\alpha} A {B_2}
      \yields
        {\lamty f {\im {\for {\alpha} {B_1}}}
          {\blam \alpha {\app E {(\app f \alpha)}}}}}
}

\newcommand{\rulesubforallext}{
  \inferrule* [right=\labelsubforall]
    {{B_1} \subtype {B_2} \yields E \\
     {A_2} \subtype {A_1} \yields {E_d}}
    {\fordis {\alpha} {A_1} {B_1} \subtype \fordis {\alpha} {A_2} {B_2}
      \yields
        {\lamty f {\im {\for {\alpha} {B_1}}}
          {\blam \alpha {\app E {(\app f \alpha)}}}}}
}

\newcommand{\labelsubinter}{\makelabelsub Inter}
\newcommand{\rulesubinter}{
  \inferrule* [right=\labelsubinter]
    {{A_1} \subtype {A_2} \yields {E_1} \\
     {A_1} \subtype {A_3} \yields {E_2}}
    {{A_1} \subtype {A_2 \inter A_3}
      \yields
        {\lamty x {\im {A_1}}
          {\pair {\app {E_1} x} {\app {E_2} x}}}}
}

\newcommand{\labelsubinterl}{\makelabelsub {Inter\_1}}
\newcommand{\rulesubinterl}{
  \inferrule* [right=\labelsubinterl]
    {{A_1} \subtype {A_3} \yields E}
    {{A_1 \inter A_2} \subtype {A_3}
      \yields
        {\lamty x {\im {A_1 \inter A_2}}
          {\app E {(\proj 1 x)}}}}
}

\newcommand{\rulesubinterldis}{
\inferrule* [right=\labelsubinterl]
    {{A_1} \subtype {A_3} \yields E \\ \jatomic {A_3}}
    {{A_1 \inter A_2} \subtype {A_3}
      \yields
        {\lamty x {\im {A_1 \inter A_2}}
          {\app E {(\proj 1 x)}}}}
}

\newcommand{\labelsubinterr}{\makelabelsub {Inter\_2}}
\newcommand{\rulesubinterr}{
  \inferrule* [right=\labelsubinterr]
    {{A_2} \subtype {A_3} \yields E}
    {{A_1 \inter A_2} \subtype {A_3}
      \yields
        {\lamty x {\im {A_1 \inter A_2}}
          {\app E {(\proj 2 x)}}}}
}

\newcommand{\rulesubinterrdis}{
  \inferrule* [right=\labelsubinterr]
    {{A_2} \subtype {A_3} \yields E \\ \jatomic {A_3}}
    {{A_1 \inter A_2} \subtype {A_3}
      \yields
        {\lamty x {\im {A_1 \inter A_2}}
          {\app E {(\proj 2 x)}}}}
}

\newcommand{\rulesubinterlcoerce}{
  \inferrule* [right=\labelsubinterl]
    {{A_1} \subtype {A_3} \yields E \\ \jatomic {A_3} }
    {{A_1 \inter A_2} \subtype {A_3}
      \yields
        { \lamty x {\im {A_1 \inter A_2}} {\andcoerce{A_3}_{{(
          {\app E {(\proj 1 x)}})}}}}}
}

\newcommand{\rulesubinterrcoerce}{
  \inferrule* [right=\labelsubinterr]
    {{A_2} \subtype {A_3} \yields E \\ \jatomic {A_3} }
    {{A_1 \inter A_2} \subtype {A_3}
      \yields 
        { \lamty x {\im {A_1 \inter A_2}} {\andcoerce{A_3}_{{(
          {\app E {(\proj 2 x)}})}}}}}
}

\newcommand{\ruleevarelab} {
\inferrule* [right=$\rulelabelevar$]
  {(x,\tau) \in \gamma}
  {\judgee \gamma x \tau \yields x}
}

\newcommand{\ruleelamelab} {
\inferrule* [right=$\rulelabelelam$]
  {\judgee {\gamma, x \hast \tau} e {\tau_1} \yields E \andalso \judgeewf \gamma \tau}
  {\judgee \gamma {\lam x \tau e} {\tau \to \tau_1} \yields {\lam x {\im \tau} E}}
}

\newcommand{\ruleeappelab}{
\inferrule* [right=$\rulelabeleapp$]
  {\judgee \gamma {e_1} {\tau_1 \to \tau_2} \yields {E_1} \\
   \judgee \gamma {e_2} {\tau_3} \yields {E_2} \andalso
   \tau_3 \subtype \tau_1 \yields C}
  {\judgee \gamma {\app {e_1} {e_2}} {\tau_2} \yields {\app {E_1} {(\app C E_2)}}}
}

\newcommand{\ruleeblamelab}{
\inferrule* [right=$\rulelabeleblam$]
  {\judgee {\gamma, \alpha} e \tau \yields E}
  {\judgee \gamma {\blam \alpha e} {\for \alpha \tau} \yields {\blam \alpha E}}
}

\newcommand{\ruleetappelab}{
\inferrule* [right=$\rulelabeletapp$]
  {\judgee \gamma e {\for \alpha {\tau_1}} \yields E \andalso \judgeewf \gamma \tau}
  {\judgee \gamma {\tapp e \tau} {\subst \tau \alpha \tau_1} \yields {\tapp E {\im \tau}}}
}

\newcommand{\ruleemergeelab}{
\inferrule* [right=$\rulelabelemerge$]
  {\judgee \gamma {e_1} {\tau_1} \yields {E_1} \andalso
   \judgee \gamma {e_2} {\tau_2} \yields {E_2}}
  {\judgee \gamma {e_1 \mergeop e_2} {\tau_1 \andop \tau_2} \yields {\pair {E_1} {E_2}}}
}

\newcommand{\ruleerecconelab}{
\inferrule* [right=$\rulelabelereccon$]
  {\judgee \gamma e \tau \yields E}
  {\judgee \gamma {\reccon l e} {\recty l \tau} \yields E}
}

\newcommand{\ruleerecprojelab}{
\inferrule* [right=$\rulelabelerecproj$]
  {\judgee \gamma e \tau \yields E \andalso
   \judgeget \tau l {\tau_1} \yields C}
  {\judgee \gamma {e.l} {\tau_1} \yields {\app C E}}
}

\newcommand{\ruleerecupdelab}{
\inferrule* [right=$\rulelabelerecupd$]
  {\judgee \gamma e \tau \yields E \andalso
   \judgee \gamma {e_1} {\tau_1} \yields {E_1} \\
   \judgeput \tau l {\tau_1 \yields {E_1}} {\tau_2} {\tau_3} \yields C \andalso
   \tau_1 \subtype \tau_3}
  {\judgee \gamma {\recupd e l {e_1}} {\tau_2} \yields {\app C E}}
}
\newcommand{\ruleget}{
  \inferrule* [right=$\rulelabelget$]
  { }
  {\judgeget {\recty l \tau} l \tau}
}

\newcommand{\rulegetelab}{
  \inferrule* [right=$\rulelabelget$]
  { }
  {\judgeget {\recty l \tau} l \tau \yields {\lam x {\im {\recty l \tau}} x}}
}

\newcommand{\rulegetleft}{
  \inferrule* [right=$\rulelabelgetleft$]
  {\judgeget {\tau_1} l \tau}
  {\judgeget {\tau_1 \andop \tau_2} l \tau}
}

\newcommand{\rulegetleftelab}{
  \inferrule* [right=$\rulelabelgetleft$]
  {\judgeget {\tau_1} l \tau \yields C}
  {\judgeget {\tau_1 \andop \tau_2} l \tau \yields {\lam x {\im {\tau_1
          \andop \tau_2}} {\app C {(\proj 1 x)}}}}
}

\newcommand{\rulegetright}{
  \inferrule* [right=$\rulelabelgetright$]
  {\judgeget {\tau_2} l \tau}
  {\judgeget {\tau_1 \andop \tau_2} l \tau}
}

\newcommand{\rulegetrightelab}{
  \inferrule* [right=$\rulelabelgetright$]
  {\judgeget {\tau_2} l \tau \yields C}
  {\judgeget {\tau_1 \andop \tau_2} l \tau \yields {\lam x {\im {\tau_1
          \andop \tau_2}} {\app C {(\proj 2 x)}}}}
}

\newcommand{\ruleput}{
\inferrule* [right=$\rulelabelput$]
  { }
  {\judgeput {\recty l \tau} l {\tau_1} {\recty l {\tau_1}} \tau}
}

\newcommand{\ruleputelab}{
\inferrule* [right=$\rulelabelput$]
  { }
  {\judgeput {\recty l \tau} l {\tau_1 \yields E} {\recty l {\tau_1}} \tau
  \yields {\lam \_ {\im {\recty l \tau}} E}}
}

\newcommand{\ruleputleft}{
\inferrule* [right=$\rulelabelputleft$]
  {\judgeput {\tau_1} l \tau {\tau_3} {\tau_4}}
  {\judgeput {\tau_1 \andop \tau_2} l \tau {\tau_3 \andop \tau_2} {\tau_4}}
}

\newcommand{\ruleputleftelab}{
\inferrule* [right=$\rulelabelputleft$]
  {\judgeput {\tau_1} l {\tau \yields E} {\tau_3} {\tau_4} \yields C}
  {\judgeput {\tau_1 \andop \tau_2} l {\tau \yields E} {\tau_3 \andop \tau_2} {\tau_4}
  \yields {\lam x {\im {\tau_1 \andop \tau_2}} {\app C {(\proj 1 x)}}}}
}

\newcommand{\ruleputright}{
\inferrule* [right=$\rulelabelputright$]
  {\judgeput {\tau_2} l \tau {\tau_3} {\tau_4}}
  {\judgeput {\tau_1 \andop \tau_2} l \tau {\tau_1 \andop \tau_3} {\tau_4}}
}

\newcommand{\ruleputrightelab}{
\inferrule* [right=$\rulelabelputright$]
  {\judgeput {\tau_2} l {\tau \yields E} {\tau_3} {\tau_4} \yields C}
  {\judgeput {\tau_1 \andop \tau_2} l {\tau \yields E} {\tau_1 \andop \tau_3} {\tau_4}
  \yields {\lam x {\im {\tau_1 \andop \tau_2}} {\app C {(\proj 2 x)}}}}
}
\newcommand{\makelabeltgt}[1]{Tgt\_#1}

% Target WF
\newcommand{\formtgtwf}{\framebox{$ \jwf G T $}}

\newcommand{\makelabeltgtwf}[1]{\makelabeltgt {WF\_#1}}

\newcommand{\labeltgtwffv}{\makelabeltgtwf FV}
\newcommand{\ruletgtwffv} {
\inferrule* [right=\labeltgtwffv]
  {\ftv T \in G}
  {\jwf G T}
}

% Target typing
\newcommand{\formtgt}{\framebox{$ \jtype G E T $}}

\newcommand{\makelabeltgtty}[1]{\makelabeltgt {Ty\_#1}}

\newcommand{\labeltgtvar}{\makelabeltgtty Var}
\newcommand{\ruletgtvar} {
\inferrule* [right=\labeltgtvar]
  {(x,T) \in \Gamma}
  {\jtype \Gamma x T}
}

\newcommand{\labeltgtlam}{\makelabeltgtty Lam}
\newcommand{\ruletgtlam} {
\inferrule* [right=\labeltgtlam]
  {\jtype {\Gamma, x \oftype T} E {T_1} \andalso \jwf \Gamma T}
  {\jtype \Gamma {\lamty x T E} {T \to T_1}}
}

\newcommand{\labeltgtapp}{\makelabeltgtty App}
\newcommand{\ruletgtapp}{
\inferrule* [right=\labeltgtapp]
  {\jtype \Gamma {E_1} {T_1 \to T_2} \andalso \jtype \Gamma {E_2} {T_1}}
  {\jtype \Gamma {\app {E_1} {E_2}} {T_2}}
}

\newcommand{\labeltgtblam}{\makelabeltgtty BLam}
\newcommand{\ruletgtblam}{
\inferrule* [right=\labeltgtblam]
  {\jtype {\Gamma, \alpha} E T}
  {\jtype \Gamma {\blam \alpha E} {\for \alpha T}}
}

\newcommand{\labeltgttapp}{\makelabeltgtty TApp}
\newcommand{\ruletgttapp}{
\inferrule* [right=\labeltgttapp]
  {\jtype \Gamma E {\for \alpha {T_1}} \andalso \jwf \Gamma T}
  {\jtype \Gamma {\tapp E T} {\subst T \alpha T_1}}
}

\newcommand{\labeltgtpair}{\makelabeltgtty Pair}
\newcommand{\ruletgtpair}{
\inferrule* [right=\labeltgtpair]
  {\jtype \Gamma {E_1} {T_1} \andalso \jtype \Gamma {E_2} {T_2}}
  {\jtype \Gamma {\pair {E_1} {E_2}} {\pair {T_1} {T_2}}}
}

\newcommand{\labeltgtprojl}{\makelabeltgtty {Proj\_1}}
\newcommand{\ruletgtprojl}{
\inferrule* [right=\labeltgtprojl]
  {\jtype \Gamma E {\pair {T_1} {T_2}}}
  {\jtype \Gamma {\proj 1 E} {T_1}}
}

\newcommand{\labeltgtprojr}{\makelabeltgtty {Proj\_2}}
\newcommand{\ruletgtprojr}{
\inferrule* [right=\labeltgtprojr]
  {\jtype \Gamma E {\pair {T_1} {T_2}}}
  {\jtype \Gamma {\proj 2 E} {T_2}}
}

\begin{mathpar}
  \framebox{$ \judgeSourceWF \gamma \tau $}

  \ruleSourceWFVar

  \ruleSourceWFTop

  \ruleSourceWFFun

  \ruleSourceWFForall

  \ruleSourceWFAnd

  \ruleSourceWFRec
\end{mathpar}

\newcommand{\name}{{\bf $F_{\&}$}\xspace}
%%\newcommand{\Name}{{\bf fi}}

\newcommand{\target}{{\bf f}\xspace}
\newcommand{\Target}{{\bf f}\xspace}

\newcommand{\authornote}[3]{{\color{#2} {\sc #1}: #3}}
\newcommand\bruno[1]{\authornote{bruno}{red}{#1}}
\newcommand\george[1]{\authornote{george}{blue}{#1}}

\lstdefinelanguage{scala}{
  morekeywords={abstract,case,catch,class,def,%
    do,else,extends,false,final,finally,%
    for,if,implicit,import,match,mixin,%
    new,null,object,override,package,%
    private,protected,requires,return,sealed,%
    super,this,throw,trait,true,try,%
    type,val,var,while,with,yield},
  otherkeywords={=>,<-,<\%,<:,>:,\#,@},
  sensitive=true,
  morecomment=[l]{//},
  morecomment=[n]{/*}{*/},
  morestring=[b]",
  morestring=[b]',
  morestring=[b]"""
}

\lstdefinelanguage{F2J}{
  morekeywords={let,rec,type,in},
  otherkeywords={->},
  sensitive=true,
  morecomment=[l]{--},
  morestring=[b]", % 'b' means inside a string delimiters are escaped by a backslash.
  morestring=[b]'
}

\lstset{ %
  % language=F2J,                % choose the language of the code
  columns=flexible,
  lineskip=-1pt,
  basicstyle=\ttfamily\small,       % the size of the fonts that are used for the code
  numbers=none,                   % where to put the line-numbers
  stepnumber=1,                   % the step between two line-numbers. If it's 1 each line will be numbered
  numbersep=5pt,                  % how far the line-numbers are from the code
  backgroundcolor=\color{white},  % choose the background color. You must add \usepackage{color}
  showspaces=false,               % show spaces adding particular underscores
  showstringspaces=false,         % underline spaces within strings
  showtabs=false,                 % show tabs within strings adding particular underscores
  tabsize=2,                  % sets default tabsize to 2 spaces
  captionpos=none,                   % sets the caption-position to bottom
  breaklines=true,                % sets automatic line breaking
  breakatwhitespace=false,        % sets if automatic breaks should only happen at whitespace
  title=\lstname,                 % show the filename of files included with \lstinputlisting; also try caption instead of title
  escapeinside={(*}{*)},          % if you want to add a comment within your code
  keywordstyle=\ttfamily\bfseries,
% commentstyle=\color{Gray}
% stringstyle=\color{Green}
}


\usepackage{fixltx2e}

% Typography
\usepackage{fontspec}
\setmainfont{Times New Roman}
\usepackage[euler-digits,euler-hat-accent]{eulervm}

\usepackage{listings}

\usepackage{url}

\begin{document}
\title{Supplementary Materials}
\maketitle

This document describes how to find and use the following artifacts associated
with the paper ``System $F_\&$: A Simple Core Language for Extensibility'':

\begin{itemize}
  \item Implementation of the compiler
  \item Runnable code examples written in our source language
  \item Mechanized proofs in Coq
\end{itemize}

\section*{Implementation}

The implementation of the compiler (in Haskell) is publicly available at:
\url{https://github.com/hkuplg/fcore.git}. To build and install the compiler,
simply follow the instructions at \texttt{README.md} at the project root. Besides,
what may be of special interest to the reader of this paper is the module
\texttt{Simplify}, which translates $ F_\&$ to a variant of System $F$. It is
located (relative to the project root) at \texttt{lib/simplify}.
Also, the definition of the abstract syntax tree of System $F_\&$
is at \texttt{lib/SystemFI.hs}.

\section*{Code examples}

Two code examples that are used in the paper (Section 3) can be found at:
\url{https://github.com/zhiyuanshi/intersection/tree/master/src}. One is named
\texttt{ObjectAlgebra.sf}, the other \texttt{Visitor.sf}. To run the examples,
you need a working installation of the compiler described in the previous
section. Here is how you would try out the examples at command line
(\texttt{f2j} is the name of the compiler; passing the \texttt{-r} flag
additionally runs the program):
\begin{lstlisting}
$ f2j ObjectAlgebra.sf -r
ObjectAlgebra using [Naive]
Compiling to Java source code ( ./ObjectAlgebra$.java )
7 + 2 = 9
\end{lstlisting}

\section*{Coq proofs}

Coq proofs can be found at:
\url{https://github.com/zhiyuanshi/intersection/blob/master/coq/Inter.v}

\end{document}
