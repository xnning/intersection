\section{An Overview of System \name}

% \bruno{mention (and cite) that there are a number of OO languages supporting
% intersection types }
%\george{Change the examples later to something very simple.}
%\bruno{Mention that we use some syntactic sugar, which is part
%of our implementation.}

% \bruno{Syntax in the examples is not being used consistently!
% That is why it is better to import code from a file. You have to
% carefully go over every example and see whether the examples
% are actually valid syntax in our language. I fixed some of these,
% but there may be some more.}

This section provides the reader with the necessary intuition for
\name by informally introducing the main features of the language.
The features of \name are also contrasted with features of mainstream
languages such as Java or Scala. Note that this section uses some
common syntactic sugar in programming languages, which is
part of our implementation.

%will do so by introducing a simple source language, which will be used in
%Section~\ref{sec:application}. \george{From the practice of similar papers, I
% think a figure listing the source syntax is necessary.}

\subsection{Intersection Types in Existing Languages}\label{subsec:interScala}

A number of OO languages, such as Java, C\#, Scala, and
Ceylon\footnote{\url{http://ceylon-lang.org/}}, already support intersection
types to different degrees. In Java, for example,

\begin{lstlisting}
interface AwithB extends A, B {}
\end{lstlisting}

\noindent introduces a new interface \lstinline{AwithB} that satisfies the interfaces of
both \lstinline{A} and \lstinline{B}. Arguably such type can be considered as a nominal
intersection type. Scala takes one step further by eliminating the
need of a nominal type. For example, given two concrete traits, it is possible to
use \emph{mixin composition} to create an object that implements both
traits. Such an object has a (structural) intersection type:

\begin{lstlisting}
trait A
trait B

val newAB : A with B = new A with B
\end{lstlisting}

\noindent Scala also allows intersection of type parameters. For example:
\begin{lstlisting}
def merge[A,B] (x: A) (y: B) : A with B = ...
\end{lstlisting}
uses the annonymous intersection of two type parameters \lstinline{A} and
\lstinline{B}. However, in Scala it is not possible to dynamically
compose two objects. For example, the following code:

\begin{lstlisting}
// Invalid Scala code:
def merge[A,B] (x: A) (y: B) : A with B = x with y
\end{lstlisting}

\noindent is rejected by the Scala compiler. The problem is that the
\lstinline{with} construct for Scala expressions can only be used to
mixin traits or classes, and not arbitrary objects. Note that in the
definition \lstinline{newAB} both \lstinline{A} and \lstinline{B} are
\emph{traits}, whereas in the definition of \lstinline{merge} the variables
\lstinline{x} and \lstinline{y} denote \emph{objects}.

\begin{comment}
A common limitation in those languages, though, is that there is no introduction
construct at the term level for intersection types. In Java and Scala, we cannot
create an instance of class type \lstinline{A & B} with
\begin{lstlisting}
  new A() & B()
\end{lstlisting}
\end{comment}

This limitation essentially put intersection types in Scala in a second-class
status. Although \lstinline{merge} returns an intersection type, it is
hard to actually build values with such types. In essense an
object-level introduction contruct for intersection types is missing.
As it turns out using low-level type-unsafe programming features such
as dynamic proxies, reflection or other meta-programming techniques,
it is possible to implement such an introduction
construct in Scala~\cite{oliveira2013feature,rendel14attributes}. However, this
is clearly a hack and it would be better to provide proper language
support for such a feature.

\begin{comment}
This is in
contrast, there are term-level introduction construct for function types (with
lambdas) and universal quantification (with big lambdas) in most core
calculi.
\end{comment}

% The key constructs are the ``merge'' operator, denoted by $ \mergeOp $
% at the term level and the corresponding type intersection operator, denoted by
% $ \andOp $ at the type level.
% \bruno{We take a particular approach to intersection types, namely that of
% Dundfield. Some other approaches are different. The \emph{key feature} in
% Dunfield's approach is the $\mergeOp$ operator, which allows for run-time
% value/object composition.}

% The addition of intersection types to System $ F $ has a number of consequences,
% which we will explore one by one in the following subsections.
% \bruno{Don't mention prototype-based inheritance in the summary, unless you
% are going to mention it later in the section. \emph{More generally don't
% mention things that you don't talk about later!}}.

\subsection{Intersection Types in \name}

% What is an intersection type? One classic view is from set-theoretic
% interpretation of types: $ A \andOp B $ stands for the intersection of
% the set of values of $ A $ and $ B $. A more practical view, adopted in this
% paper, regards types as a kind of interface: a value of type
% $ A \andOp B $ satisfies both of the interfaces of $ A $ and $ B $. For
% example, \lstinline{eval : Int} is the interface that supports evaluation to
% integers, while \lstinline{eval : Int & print : String } supports both
% evaluation and pretty printing. Those interfaces are akin to interfaces in
% Java or traits in Scala. But one key difference is that they are unnamed in
% \name.

To address the limitations of intersection types in languages like
Scala, \name allows intersecting any two terms at run time using a
\emph{merge} operator (denoted by $ \mergeOp $)~\cite{dunfield2014elaborating}.  With the merge
operator it is trivial to implement the \lstinline{merge} function in \name:

\begin{lstlisting}
let merge[A,B] (x : A) (y : B) : A & B = x ,, y;
\end{lstlisting}

\noindent In contrast to Scala's expression-level \lstinline{with}
construct, the operator \lstinline{,,} allows two arbitrary values \lstinline{x}
and \lstinline{y} to be merged. The resulting type is an
intersection of the types of  \lstinline{x}
and \lstinline{y} (\lstinline{A & B} in this case).

\begin{comment}
The following table
summarizes the extent of support for intersection types in Java,
Scala, and \name.

\hspace{-13pt}\begin{tabular}{ l | c | c | c  }
                                   & Java       & Scala      & \name      \\ \hline
  Basic intersection types         & \checkmark & \checkmark & \checkmark \\ \hline
  Anonymous intersection types     &            & \checkmark & \checkmark \\ \hline
  Intersection of type parameters  &            & \checkmark & \checkmark \\ \hline
  Term-level intersection          &            &            & \checkmark
\end{tabular} \\
\end{comment}

%In \name the central addition to the type system of System $F$ are
%intersection types.

\paragraph{Intersection types for overloading.}
A typical use-case for intersection types is to do
\emph{overloading}. The benefit is that programmers can use the same
operation on different types and delegate the task of choosing a
concrete implementation to the type system. For example, we can define
a \lstinline{show} function that takes either an integer or a boolean
and returns its string representation. In other words, \lstinline{show} is also
\emph{both} a function from integers to strings as well as a function
from boolean to strings.  Therefore, in \name \lstinline{show} should be of
following type:
\begin{lstlisting}
show : (Int -> String) & (Bool -> String)
\end{lstlisting}
Assuming that the following two functions are available:
\begin{lstlisting}
showInt : Int  -> String
showBool : Bool -> String
\end{lstlisting}

\noindent The overloaded \lstinline{show} function is defined by
merging the \lstinline{showInt} and \lstinline{showBool} using the
merge operator:

\begin{lstlisting}
let show = showInt ,, showBool;
\end{lstlisting}
% \bruno{This is not a proper sentence. Polish text.}

To illustrate the usage, consider the function application \lstinline{show 100}.
The type system will pick the first component of \lstinline{show}, namely
\lstinline{showInt}, as the implementation being applied to \lstinline{100}
because the type of \lstinline{showInt} is compatible with \lstinline{100}, but
\lstinline{showBool} is not. This example shows that one may regard
intersections in our system as ``implicit pairs'' whose introduction is explicit
by the merge operator and elimination is implicit (with no source-level
construct for elimination).

% The merge construct in the original function is elaborated into a pair in the
% target language:

% \begin{verbatim}
% show = (showInt, showBool)
% \end{verbatim}

% In the target language where there is no intersection types, the application
% of the integer \texttt{1} to this function does not typecheck. However, we may
% rescue this situtation by inserting a coercion that extracts the first item
% out of this pair.

% Thus \texttt{show 1} in FI corresponds to \texttt{(fst show) 1} in F.

% While elaborating intersection types, this paper is the first that presents a
% type system that incorporates both parametric polymorphism and intersection
% polymorphism.

% Describe intersection types, encoding records with Intersecion types

% \lstinputlisting[linerange=-]{} % APPLY:linerange=MIXIN_LIB

\paragraph{Subtyping.} The previous example exploits a natural
subtyping relation on intersection types. That is the type

\begin{lstlisting}
(Int -> String) & (Bool -> String)
\end{lstlisting}

\noindent is a \emph{subtype} of both \lstinline{Int -> String} and
\lstinline{Bool -> String}. This is why \lstinline{show} can take
\lstinline{100} as its argument. Generally speaking an intersection type
\lstinline{A & B} is a subtype of both \lstinline{A} and \lstinline{B}.
Moreover, subtyping of intersection types in \name is purely structural
and it enjoys of properties such as \emph{idempotence},
\emph{commutativity} and \emph{associativity}
(equality is defined as bidirectional subtyping relation): \\

% \noindent {\bf commutativity: } $A \andOp B = B \andOp A$  \\
% {\bf associativity:~~~~~~}$A \andOp (B \andOp C) = (A \andOp B) \andOp C$

\begin{tabular}{llrcl}
  \textbf{Idempotent}  & $A \andOp A$            & $=$ & $A$ \\
  \textbf{Commutative} & $A \andOp B$            & $=$ & $B \andOp A$ \\
  \textbf{Associative} & $(A \andOp B) \andOp C$ & $=$ & $A \andOp (B \andOp C)$
\end{tabular}

% \bruno{I think this will be a good place to talk about the following material:}
% \bruno{Talk about Inheritance is not Subtyping.
% Describe type inheritance and subtyping, show that they don't necessarelly
% go along together in our language. You may need to write some Java code, to
% illustrate differences. \emph{We support contravariant argument types!}
% }
% \bruno{Related to the previous point, don't forget to mention that
% there are nominal languages, that also separate inheritance from subtyping!
% See Klaus Ostermann's paper \& ``Inheritance is not Subtyping''.
% }

\paragraph{Type-inheritance} In \name type-inheritance is decoupled
from subtyping. Type-inheritance, as formally defined by Cook et
al.~\cite{cook1989inheritance}, is:
\[
G = F + \{ l_1:\tau_1, \ldots, l_n:\tau_n \}
\]
Here $G$ is a new type defined by adding fields to an existing type $F$.
The argument that ``Inheritance is not subtyping'' means that structually, $G$ can
be: 1) a subtype of $F$, 2) a supertype of $F$, or 3) $G$ and $F$ can have no
relation at all! If none of $l_1, \ldots, l_n$ is present in $F$, $G$ can be
safely regarded as a subtype of $F$, since $G$ always contains the fields
required by the signature $F$; if all of $l_1, \ldots, l_n$ are present in $F$
and each of their types is a supertype of the corresponding type, $G$ becomes
instead a supertype of $F$; Conceivably, the third possibility will make $F$ and
$G$ have no relation. However, most OO languages conflate the
subtyping and type-inheritance, allowing only 1). This leads
to cases where the type system prevents expressing semantically
correct programs. For example, consider a program that consists of the two
types:
\begin{lstlisting}
type A = {x1: Int};
type F = {f: A -> Int};
\end{lstlisting}
Later one wishes to add a field to form a new type \lstinline$B$ and refine the
parameter type of \lstinline$f$ by using \lstinline$B$ instead:
\begin{lstlisting}
type B = {x1: Int, x2: Int}; -- B <: A
type G = {f: B -> Int};      -- G :> F
\end{lstlisting}
Clearly, \lstinline$B$ should be a subtype of \lstinline$A$. But it is also
desirable to have the property that \lstinline$G$ is a \emph{supertype} of
\lstinline$F$, since the refinement happens at a parameter position. In OO
languages such as Java and Scala, the idiom of making \lstinline$G$ type-inherit
$F$ is through \lstinline$G extends F$, which only works when $G$ is a
subtype of $F$.
%Unfortunately, $G$ has become a subtype
%of $F$.

Although in nominal systems there is the well-known tension between inheritance
and subtyping, it is worth noting that extending this benefit to nominal
subtyping is possible. Ostermann~\cite{ostermann2008nominal} proposed a flexible
nominal system that untangles inheritance and subtyping. For example, users may
declare supertype of an existing class type while at the same time inheriting
the implementation, and vice versa.

% \george{I think points regarding the designs of type system should prob. be addressed
%   somewhere else. The purpose of this section is just introducing the source
%   language and preparing the reader for the next section.}

% Subtyping in \name is syntactical and structural.

\subsection{Generalized Records with Intersection Types}

Following Reynolds~\cite{reynolds1997design} and Castagna et
al.~\cite{castagna1995calculus}, \name leverages intersection types to type
extensible records. The idea is that a multi-field record can be encoded as
merges of single-field records, and multi-field record types as intersections.
Therefore in \name, there are only single-field record constructs.
%The result
%are orthogonal addition to the existing type system and a more general notion of
%records.

Conventionally, record operations work only on record types. But in \name
such operations can occur on \emph{any} type. The
reason is that multi-field records in \name do not have a proper
record types. Instead, their types are intersections.

\subsubsection{Record Operations}

To illustrate the various operations on records, we consider a record
with three fields and with the following type:

\begin{lstlisting}
{open : Int, high : Int, low : Int}
\end{lstlisting}

\noindent Note that this type is just syntactic sugar for:

\begin{lstlisting}
{open : Int} & {high : Int} & {low : Int}
\end{lstlisting}

\noindent That is a multi-field record type is desugared as
intersections of single-field record types.

\name has three primitive operations related to records: \emph{construction},
\emph{selection}, and \emph{restriction}. \emph{Extension}, described in many
other record systems, is delegated to the merge operator. Working with records
is type-safe: the type system prevents accessing or removing a field that does
not exist.

\paragraph{Construction.} The usual notation for constructing records
\begin{lstlisting}
{open=192, high=195, low=189}
\end{lstlisting}
is a shorthand for merges of single-field records
\begin{lstlisting}
{open=192} ,, {high=195} ,, {low=189}
\end{lstlisting}

\paragraph{Selection.}
Fields are extracted using the dot notation. For example,
\begin{lstlisting}
{open=192, high=195, low=189}.open
\end{lstlisting}
selects the value of the field labelled \lstinline{open} from the record.

\paragraph{Restriction.}
Restriction $e \restrictOp l$ removes a field $l$ from an expression $e$. If $e$
contains multiple fields labelled $l$, only the last field will be removed.
Restriction was chosen to be a primitive because other common
record operations can be defined in terms of restriction. For example, we can encode record
\emph{update} as a restriction followed by a merge. The following example
shows updating the \lstinline{high} field of a quote:
\begin{lstlisting}
{open=192, high=195, low=189} \ high ,, {high=196}
\end{lstlisting}
Similarly, \emph{renaming} of a field can be simulated by:
\begin{lstlisting}
{open=192, high=195, low=189} \ high ,, {dayHigh=196}
\end{lstlisting}

\begin{comment}
Refinement of fields is also possible, in the sense
that the type of a new value can be a subtype of that of the old
one.~\footnote{The subtyping restriction is not required for the \name to be
  coherent and it leaves option open for language designers.}
\bruno{Are we going to show an example of this?}
\end{comment}

\paragraph{Extension.}
Extension, just as construction, is performed with the merge operator
($ \mergeOp $). The following, for example, adds a \lstinline{close}
field to the record:
\begin{lstlisting}
{open=192, high=195, low=189} ,, {close=195}
\end{lstlisting}

\paragraph{Generalized records.}
In addition, a record is just a normal term and can be merged with any other
term, for example, the following program is valid:
\begin{lstlisting}
let mixed: Int & {x: Int} = 1 ,, {x=2};
\end{lstlisting}
and it is still possible to use record operations on
\lstinline{mixed}. That is because a record type of the form $ \recordType
l \tau $ can be thought as a normal type $ \tau $ tagged by the label
$ l $. For instance, \lstinline$mixed.x$ extracts the value of the field
\lstinline{x} from \lstinline{mixed}.

\subsubsection{Record Subtyping}

Subtyping of record types is both in width and in depth, as one might
expect: \lstinline${x: Int, y: Int}$ is a subtype of
\lstinline${x: Int}$; and if \lstinline$T1$ is a subtype of
\lstinline$T2$, then \lstinline${l: T1}$ is also a subtype of
\lstinline${l: T2}$.

% \bruno{I would call this a labelled type. Record types can be viewed as
% intersections of labelled types.}
% \bruno{Not a proper sentence! }

% \lstinline{e1} and \lstinline{e2} are two expressions that support both
% evaluation and pretty printing and each has type \lstinline{eval : Int, print :
%   String}. \lstinline{add} takes two expressions and computes their
% sum\bruno{Don't inline examples in the text. Give proper separate code
%   examples.}. Note that in order to compute a sum, \lstinline{add} only requires
% that the two expressions support evaluation and hence the type of the parameter
% \lstinline{eval : Int}. As a result, the type of \lstinline{e1} and
% \lstinline{e2} are not exactly the same with that of the parameters of
% \lstinline{add}. However, under a structural type system, this program should
% typecheck anyway because the arguments being passed has more information than
% required. In other words, \lstinline{eval : Int, print : String} is a subtype of
% \lstinline{eval : Int}.

% How is this subtyping relation derived? In \name, multi-field record types are
% excluded from the type system because \lstinline{eval : Int, print : String} can
% be encoded as \lstinline{eval : Int & print : String}. And by one of subtyping
% rules derives that \lstinline{eval : Int & print : String} is a subtype of
% \lstinline{eval : Int}.

% \bruno{Examples illustrating update and elimination of intersections
%   are missing!}

\subsection{\hspace{-5pt}Intersection~Types~and~Parametric~Polymorphism}

The combination of intersection types and parametric polymorphism
makes System \name quite expressive. In particular this combination
enables an encoding of a simple form of \emph{bounded universal
  quantification}~\cite{cardelli1985understanding}.

%In the same way that System \name extends System $ F $ with
%intersection types, System $ F_{\subtype} $ extends System $ F $ with bounded
%polymorphism. $ F_{\subtype} $~\cite{pierce2002types} allows giving an
%upper bound to the type variable in type abstractions.

\paragraph{Bounded quantification and loss of information.}
The idea of bounded universal quantification was discussed in the
seminal paper by Cardelli and Wegner
~\cite{cardelli1985understanding}. They show that bounded quantifiers
are useful because they are able to solve the ``loss of information''
problem. The extension of System $ F $ with intersection types is able
to address the same problem effectively. Suppose we have the following
definitions:
\begin{lstlisting}{language=scala}
let user = {name = "George", admin = true};
let id(user: {name: String}) = user;
\end{lstlisting}
Under a structural type system, passing \lstinline{user}
to \lstinline{id} is allowed.
In other words, the type of \lstinline{user}
is of a subtype of the expected parameter type of \lstinline{id}.
However there is a problem: what if programmers
want to access the \texttt{admin} field later. For example:
\begin{lstlisting}{language=scala}
(id user).admin
\end{lstlisting}

They cannot do so as the above will not typecheck. After going through the
function, the resulting value has the type:

\begin{lstlisting}{language=scala}
{name: String}
\end{lstlisting}
This is rather undesired because the value does have an \texttt{admin} field!

Bounded polymorphism enables the \lstinline{id} function to return the exact type of the
argument so that there is no problem in accessing the \texttt{admin} field later.
Consider the example below:
\begin{lstlisting}{language=scala}
let id[A <: {name: String}] (user: A) = user;
(id [{name: String, admin: Bool}] user).admin
\end{lstlisting}

\noindent This piece of pseudo-code, which is not valid in \name due
to the use of bounded polymorphism, illustrates the idea. Instead of
giving the \lstinline{user} argument a concrete type, the
\lstinline{id} function specifies that such an argument is a subtype
of \{\lstinline{name: String}\}. With such a type the function
\lstinline{id} avoids losing information about the argument type.


\paragraph{Encoding bounded polymorphism in \name.}
\name does not have bounded polymorphism. However the same effect can
be achieved with a combination of intersection types and parametric polymorphism:
\begin{lstlisting}{language=F2J}
let id[A] (user: A & {name: String}) = user;
(id [{admin: Bool}] user).admin
\end{lstlisting}
By requiring the type of the argument to be an intersection type of a type
parameter and the upper bound and passing the type information, we make sure
that we can still access the \lstinline{admin} field later.

Therefore, this technique allows \name to encode a simple form
of bounded quantification. This is good because it means that \name
can express many common idioms that require bounded quantification
without complicating the core calculus with native support for bounded quantifiers.

% \cite{pierce1997intersection}

% \bruno{I don't think this has been shown. What we can say is: we can encode a
% form of bounded polymorphism with intersection types.}




