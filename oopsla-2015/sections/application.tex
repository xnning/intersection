\section{Applications to Extensibility} \label{sec:application}

% Structural subtyping facilitates reuse~\cite{malayeri2008integrating}.
% \bruno{orphan sentence!}

% \bruno{Make sure that the important code in the paper is reused from a script
%   and not inlined directly in the text.}

This section shows that, although \name is a minimal language, its
features are enough for encoding extensible designs that been
presented in mainstream languages. Moreover \name addresses
limitations of those languages, making those designs significantly
simpler. There are two main advantages of \name over existing
languages:

\begin{enumerate}
\item \name supports dynamic composition of intersecting values.
\item \name does not couple type inheritance and subtyping. Moreover
  \name supports contravariant parameter types in the subtyping relation.
\end{enumerate}

These two features avoid the use of low-level programming techniques,
and make the designs less reliant on advanced features of generics. 

\begin{comment}
Various solutions have been proposed to deal with the extensibility problems and
many rely on heavyweight language features such as abstract methods and classes
in Java. These two features can be used to improve existing designs of modular
programs.

\bruno{I would like to see a story about Church Encodings in
  \name. Can you look at Pierce's papers and try to write something
  along those lines? That will be a good intro for object algebras and
visitors!}


\url{http://repository.cmu.edu/cgi/viewcontent.cgi?article=3018&context=compsci}
Church encoding allows modelling algebraic data types. 
\end{comment}

% Introduce the expression problem



% There has been recently a lightweight solution to the expression problem that
% takes advantage of covariant return types in Java. We show that \name is able
% to solve the expression problem in the same spirit.

% - Object/Fold Algebras. How to support extensibility in an easier way.

% See Datatypes a la Carte

% - Mixins

% - Lenses? Can intersection types help with lenses? Perhaps making the
% types more natural and easy to understand/use?

% - Embedded DSLs? Extensibility in DSLs? Composing multiple DSL interpretations?

% http://www.cs.ox.ac.uk/jeremy.gibbons/publications/embedding.pdf

% \bruno{You already talk about overloading in the previous section. Need to
% decide where to put the text!}

% Dunfield~\cite{dunfield2014elaborating} notes that using merges as a mechanism
% of overloading is not as powerful as type classes.

% Multiple inheritance?
% Algebra -> P1,2
% Visitor -> P2

% Yanlin
% Mixin

% \begin{lstlisting}
% let merge A B (f : ExpAlg A) (g : ExpAlg B) = {
%   lit = \(x : Int). f.lit x ,, g.lit x,
%   add = \(x : A & B). \(y : A & B). f.add x y ,, g.add x y
% };
% \end{lstlisting}

\subsection{Object Algebras}\label{subsec:OAs}

% Object algebras provide an alternative to \emph{algebraic data types}
% (ADT).\bruno{We are targeting an OO crowd. Mentioning algebraic
%   datatypes is not going to be very useful there.}

%  For example, the
% following Haskell definition of the type of simple expressions
% \begin{lstlisting}{language=haskell}
% data Exp where
%   Lit :: Int -> Exp
%   Add :: Exp -> Exp -> Exp
% \end{lstlisting}
% can be expressed by the \emph{interface} of an object algebra of
% simple expressions:
% \begin{lstlisting}{language=scala}
% trait ExpAlg[E] {
%   def lit(x: Int): E
%   def add(e1: E, e2: E): E
% }
% \end{lstlisting}
% Similar to ADT, data constructors in object algebras are represented by functions such as
% \lstinline{lit} and \lstinline{add} inside an interface \lstinline{ExpAlg}.
% Different with ADT, the type of the expression itself is abtracted by a type
% parameter \lstinline{E}.

% which can be expressed similarly in \name as:
% \begin{lstlisting}{language=F2J}
% type ExpAlg E = {
%   lit : Int -> E,
%   add : E -> E -> E
% }
% \end{lstlisting}

% Introduce Scala's intersection types

% Scala supports intersection types via the \lstinline{with} keyword. The type
% \lstinline{A with B} expresses the combined interface of \lstinline{A} and
% \lstinline{B}. The idea is similar to
% \begin{lstlisting}{language=java}
% interface AwithB extends A, B {}
% \end{lstlisting}
% in Java.
% \footnote{However, Java would require the \lstinline{A} and \lstinline{B} to be
%   concrete types, whereas in Scala, there is no such restriction.}

% The value level counterpart are functions of the type \lstinline
% {A => B => A with B}. \footnote{FIXME}

Oliveira and Cook~\cite{oliveira2012extensibility} proposed a design pattern that can solve the
Expression Problem in languages like Java. An advantage of the pattern
over previous solutions is that it is relatively lightweight in terms
of type system features. In a latter paper, Oliveira et al.~\cite{oliveira2013feature}
noted some limitations of the original design pattern and proposed 
some new techniques that generalized the original pattern, allowing it 
to express programs in a Feature-Oriented Programming~\cite{Prehofer97} style.
Key to these techniques was the ability to dynamically compose object
algebras.

Unfortunatelly, dynamic composition of object algebras is
non-trivial. At the type-level it is possible to express the resulting
type of the composition using intersection types. Thus, it is still
possible to solve that part problem nicely in a language like Scala (which
has basic support for intersection types). However, the dynamic
composition itself cannot be easily encoded in Scala. The fundamental 
issue is that Scala lacks a \lstinline{merge} operator (see the
discussion in Section~\ref{subsec:interScala}). Although both Oliveira et al.~\cite{oliveira2013feature} and
Rendell et al.~\cite{rendel14attributes} have shown that such a \lstinline{merge} operator can
be encoded in Scala, the encoding fundamentally relies in low-level
programming techniques such as dynamic proxies, reflection or
meta-programming. 

Because \name supports a \lstinline{merge} operator natively, dynamic
object algebra composition becomes easy to encode. The remainder of
this section shows how object algebras and object algebra composition
can be encoded in \name. We will illustrate this point 
step-by-step by solving the Expression Problem. 
%%Prior knowledge of object algebras is not assumed.
 
% can be cumbersome and
% language support for intersection types would solve that problem. 
% Our type system is just a simple extension of System $ F $; yet surprisingly, it
% is able to solve the limitations of using object algebras in languages such as
% Java and Scala.

\paragraph{A simple system of arithmetic expressions.} 
In the Expression Problem, the idea is to start with a very simple
system modeling arithmetic expressions and evaluation.
The initial system considers expressions with two variants (literals and
addition) and one operation (evaluation). Here is an interface that supports
evaluation:
\begin{comment}
  \begin{lstlisting}{language=F2J}
    type IEval = {eval: Int};
  \end{lstlisting}
\end{comment}
\lstinputlisting[linerange=4-4]{../src/ObjectAlgebra.sf} % APPLY:linerange=OA_IEVAL

\noindent In \name the interfaces of objects (or object types) are expressed as
a record type. A \lstinline{type} declaration allow us to create a
simple alias for a type.  In this case \lstinline{IEval} is an alias
for \{\lstinline{eval: Int}\}.

With object algebras, the idea is to create an object algebra
interface, \lstinline$ExpAlg$, for expression types with the two
variants. This interface has a fixed number of variants, but abstracts over the
type of the interpretation \lstinline$E$.

\begin{comment}
  \begin{lstlisting}{language=F2J}
    type ExpAlg[E] = {
      lit: Int -> E, 
      add: E -> E -> E
    };
  \end{lstlisting}
\end{comment}

\lstinputlisting[linerange=8-11]{../src/ObjectAlgebra.sf} % APPLY:linerange=OA_EXPALG

% whereas
%\lstinline$ExpAlg[IEval & IPrint]$ will be the type of object algebras that
%support both evaluation and pretty printing.
% In \name, record types are structural and hence any value that satisfies this
% interface is of type \lstinline$IEval$ or of a subtype of \lstinline$IEval$.
% \footnote{Should be mentioned in S2.}
Having defined the interfaces, we can implement that object algebra interface
with \lstinline$evalAlg$, which is an object algebra for evaluation. 
\begin{comment}
  \begin{lstlisting}{language=F2J}
    let evalAlg: ExpAlg[IEval] = {
      lit = \(x: Int) -> {eval = x},
      add = \(x: IEval) (y: IEval) -> {
        eval = x.eval + y.eval
      }
    };
  \end{lstlisting}
\end{comment}

\lstinputlisting[linerange=15-20]{../src/ObjectAlgebra.sf} % APPLY:linerange=OA_EVALALG
In this example we implement a record, where the two operations 
\lstinline{lit} and \lstinline{add} return a record with type \lstinline{IEval}.
The type \lstinline$ExpAlg[IEval]$ is the type of object algebras
supporting evaluation. However, the one interesting point
of object algebras is that other operations can be supported as
well. 

\paragraph{Add a subtraction variant.} The point of the Expression
Problem is to support the addition of new features to the existing
program, without modifying existing code. 
The first feature is adding a new variant, such as subtraction. We can do so by
simply intersecting the original types and merging with the original values:

\begin{comment}
  \begin{lstlisting}{language=F2J}
    type SubExpAlg[E] = 
    ExpAlg[E] & {sub: E -> E -> E};
    let subEvalAlg = evalAlg ,, {
      sub = \(x: IEval) (y: IEval) -> { 
        eval = x.eval - y.eval 
      }
    };
  \end{lstlisting}
\end{comment}

\lstinputlisting[linerange=24-31]{../src/ObjectAlgebra.sf} % APPLY:linerange=OA_SUBEXPALG

\noindent Note that here intersection types are used to model \emph{type
  inheritance} and the merge operator models a basic form of
\emph{dynamic implementation inheritance}. 

\paragraph{Add a pretty printing operation.}
A second extension is adding a new operation, such as pretty printing. 
Similar to evaluation, the interface of the pretty printing feature
is modeled as:
\begin{comment}
  \begin{lstlisting}{language=F2J}
    type IPrint = {print : String};
  \end{lstlisting}
\end{comment}
\lstinputlisting[linerange=35-35]{../src/ObjectAlgebra.sf} % APPLY:linerange=OA_IPRINT
The implementation of pretty printing for expressions that support literals,
addition, and subtraction is:
\begin{comment}
  \begin{lstlisting}{language=F2J}
    let printAlg : SubExpAlg[IPrint] = {
      lit = \(x: Int) -> {print = x.toString()},
      add = \(x: IPrint) (y: IPrint) -> {
        print = x.print ++ " + " ++ y.print
      },
      sub = \(x: IPrint) (y: IPrint) -> {
        print = x.print ++ " - " ++ y.print
      }
    };
  \end{lstlisting}
\end{comment}
\lstinputlisting[linerange=39-47]{../src/ObjectAlgebra.sf} % APPLY:linerange=OA_PRINTALG

\paragraph{Usage.}
With the definitions above, values are created using the
appropriate algebras. For example, the expression \lstinline{7 + 2} 
is encoded as follows:
\begin{comment}
  \begin{lstlisting}{language=F2J}
    let e1[E] (f: SubExpAlg[E]) = 
    f.add (f.lit 7) (f.lit 2);
  \end{lstlisting}
\end{comment}
\lstinputlisting[linerange=51-52]{../src/ObjectAlgebra.sf} % APPLY:linerange=OA_E1

\noindent The expressions are unusual in the sense that they are
functions that take an extra argument \lstinline$f$. The extra
argument is an object algebra that uses the functions in the record
(\lstinline$lit$, \lstinline$add$ and \lstinline$sub$) as factory
methods for creating values. Moreover, the algebras themselves are
abstracted over the allowed operations such as evaluation and pretty
printing by requiring the expression functions to take an extra
argument \lstinline$E$.

\paragraph{Dynamic object algebra composition.}
To obtain an expression that supports both evaluation and pretty
printing, a mechanism to combine the evaluation and printing
algebras is needed. \name allows such composition: the \lstinline$combine$
function, which takes two object algebras to create a combined algebra. It
does so by constructing a new object algebra where each field is a
function that delegates the input to the two algebra parameters.
\begin{comment}
  \begin{lstlisting}{language=F2J}
    let combine[A,B](f: ExpAlg[A])(g: ExpAlg[B]) : 
    ExpAlg[A&B] = {
      lit = \(x: Int) -> f.lit x ,, g.lit x,
      add = \(x: A & B) (y: A & B) ->
      f.add x y ,, g.add x y
    }
  \end{lstlisting}
\end{comment}
\lstinputlisting[linerange=58-63]{../src/ObjectAlgebra.sf} % APPLY:linerange=OA_COMBINE

\begin{comment}
  \begin{lstlisting}{language=F2J}
    let newAlg = 
    combine[IEval,IPrint] subEvalAlg printAlg;
    let o = e1[IEval&IPrint] newAlg;
    o.print ++ " = " ++ o.eval.toString()
  \end{lstlisting}
\end{comment}

\lstinputlisting[linerange=67-69]{../src/ObjectAlgebra.sf} % APPLY:linerange=OA_USAGE

Note that \lstinline$o$ is a single object that supports both
evaluation and printing, as the output of the program is
\begin{lstlisting}
> 7 + 2 = 9
\end{lstlisting}

In contrast to the Scala solutions available in the
literature, \name is able to express object algebra
composition very directly by using the merge operator. 

\subsection{Back to Visitors}

Object Algebras are closely related to the visitor
pattern~\cite{gamma1994design}.  Indeed, object algebra interfaces are just
\emph{internal
  visitors}~\cite{oliveira09modular,oliveira2012extensibility}. 
What distinguishes object algebras
from the traditional visitor pattern is the lack of a composite
interface with an \lstinline{accept} method, which 
is both a blessing and a curse.  On
the one hand the trouble with composite interfaces with an
\lstinline{accept} method is that they make adding new variants to the
visitor pattern very hard. Although extensible versions of the visitor
pattern are possible, they usually require complex types using
advanced features of generics~\cite{togersen:2004,oliveira2012extensibility}. On the other hand, the lack of
such composite interfaces makes object algebras harder to use than
visitors. As illustrated in Section~\ref{subsec:OAs}, constructing expressions
with object algebras can only be done using a function parametrized by
an object algebra.

The remainder of the section shows that in \name there is no need to
have a dillema between extensibility using simple types and
usability: \emph{in \name it is possible to have extensible visitors with
simple types}! The key to achieve this is to have type inheritance
decoupled from subtyping, and allowing contravariant parameter type 
refinement. 

\subsubsection{The Problem with Extensible Visitors}
\begin{comment}
is another design pattern that
facilitates extensibility. The gist of this pattern is to decouple an object
hierarchy from the behaviors of each object. In other words, objects no longer
contain operations but instead ``accept'' visitors to perform operations on
them. The visitor pattern tackles at the very problem that makes programming in
traditional OO styles hard to add a new operation, since operations are defined
inside object classes that represent data structures. As a result, using the
visitor pattern allows adding new operations to existing structures without
modifying code of structures, a style enjoyed by functional
programming. 
\end{comment}
We illustrate the problem with extensible visitors using Scala.
The type for expressions is defined in Scala as:
\begin{lstlisting}{language=scala}
trait Exp {
  def accept[E](v: ExpAlg[E]): E
}
\end{lstlisting}
The trait \lstinline{Exp} has only one method 
\lstinline$accept$, which takes an internal visitor (or object
algebra) as an argument. 
Here the type \lstinline{ExpAlg[E]} is the Scala analogous of the
corresponding type defined in Section~\ref{subsec:OAs} in \name. In terms 
of the visitor pattern, \lstinline{ExpAlg} defines the visit methods 
for all variants. 

\begin{comment}
The actual shape of the expressions
(i.e., variants) is determined by the type of the visitor, which we define in
another trait:
\begin{lstlisting}{language=scala}
trait ExpVisitor[E] {
  def lit(x: Int): E
  def add(e1: E, e2: E): E
}
\end{lstlisting}
\end{comment}

\paragraph{Adding a new variant.}
The difficulties arise when a new variant, such as subtraction is
added. To do so an extended visitor interface analogous to
\lstinline$SubExpAlg$ is needed. Moreover a corresponding composite 
interface \lstinline$SubExp$ is needed as well:
%For example, \lstinline$SubExp$ and \lstinline$SubExpAlg$
%together represent the type of expressions supporting subtraction, in addition
%to literal and addition.
\begin{lstlisting}{language=scala}
trait SubExp extends Exp {
  def accept[E](v: SubExpAlg[E]): E
}
\end{lstlisting}
%We extend the original visitor and add the new case.
The body of \lstinline{Exp} and \lstinline{SubExp} are almost the same: they
both contain an \lstinline{accept} method that takes an object algebra 
and returns a value of the type \lstinline{E}. The only difference in
\lstinline{SubExp} the object algebra
\lstinline{v} is of type \lstinline{SubExpAlg[E]}, which is a subtype of
\lstinline{ExpAlg[E]}. 

\paragraph{Inheritance is not subtyping.}
Since \lstinline{v} appears in parameter position of
\lstinline{accept} and function parameters are naturally contravariant,
\lstinline{SubExp[E]} should be a \emph{supertype} (and not a subtype)
of \lstinline{Exp[E]}.  However, in Scala every extension induces a
subtype. In other words type inheritance and subtyping always go along
together.  To ensure type-soundness Scala (and other common OO
languages) forbids any kind of type-refinement on method parameter
types.  In other words method parameter types are invariant.
The consequence of this is that Scala is not capable of expressing
that \lstinline{SubExp[E]} is an extension and a supertype of
\lstinline{Exp}. Such kind of extension is an example where
``\emph{inheritance is not subtyping}``~\cite{cook1989inheritance}.

\begin{comment}
However,
such subtyping relation does not fit well in Scala because inheritance implies
subtyping in such languages \footnote{It is still possible to encode
  contravariant parameter types in Scala but doing so would require some
  technique.\bruno{what technique?}}. As \lstinline{SubExp[E]} extends
\lstinline{Exp[E]}, the former becomes a subtype of the latter. This suffers
from the ``Inheritance is Not Subtyping'' problem.
\end{comment}

\subsection{Extensible Visitors in \name}
Such limitation does not exist in \name. For example, we can define the similar
interfaces for \lstinline{Exp} and \lstinline{SubExp}:
\begin{comment}
  \begin{lstlisting}{language=F2J}
    type Exp    = {
      accept: forall A. ExpAlg[A] -> A
    };
    type SubExp = {
      accept: forall A. SubExpAlg[A] -> A
    };
  \end{lstlisting}
\end{comment}
\lstinputlisting[linerange=7-12]{../src/Visitor.sf} % APPLY:linerange=V_EXP_SUBEXP
\name support contravariant parameter type refinement, which means that
\lstinline{SubExp} is a supertype of \lstinline{Exp}. Using these
types we first define two data constructors for simple expressions:
\begin{comment}
  \begin{lstlisting}{language=F2J}
    let lit (n: Int): Exp = {
      accept = /\E -> \(f: ExpAlg[E]) -> f.lit n
    };
    let add (e1: Exp) (e2: Exp): Exp = {
      accept = /\E -> \(f: ExpAlg[E]) -> 
      f.add (e1.accept[E] f) (e2.accept[E] f)
    };
  \end{lstlisting}
\end{comment}
\lstinputlisting[linerange=23-29]{../src/Visitor.sf} % APPLY:linerange=V_LITADD

\noindent Both \lstinline{lit} and \lstinline{add} build values of type
\lstinline{Exp} and use object algebras of type \lstinline{ExpAlg[E]}. 
However, subtraction requires a value of type \lstinline{SubExp} to be created:
\begin{comment}
  \begin{lstlisting}{language=F2J}
    let sub (e1: SubExp) (e2: SubExp): SubExp ={ 
      accept = /\E -> \(f : SubExpAlg[E]) ->
      f.sub (e1.accept[E] f) (e2.accept[E] f) 
    };
  \end{lstlisting}
\end{comment}
\lstinputlisting[linerange=36-39]{../src/Visitor.sf} % APPLY:linerange=V_SUB

% One big benefit of using the visitor pattern is that programmers are able to
% define structures in a more natural way. 

\paragraph{Usage.} With visitors constructing expressions is quite simple:

\begin{comment}
  \begin{lstlisting}{language=F2J}
    e2 = sub (lit 7) (lit 2)
  \end{lstlisting}
\end{comment}
\lstinputlisting[linerange=49-50]{../src/Visitor.sf} % APPLY:linerange=V_USAGE
The result is \lstinline$"7 - 2"$. Note that the programmer is able to pass \lstinline{lit 2}, which is of type \lstinline{Exp},
to \lstinline{sub}, which expects a \lstinline{SubExp}. The types are compatible
because because \lstinline$Exp$ is a \emph{subtype} of \lstinline$SubExp$. Code
reuse is achieved since we can use the constructors from \lstinline$Exp$ as the
constructor for \lstinline$SubExp$. In Scala, we would have to define two
literal constructors, one for \lstinline$Exp$ and another for
\lstinline$SubExp$. 

Compared to object algebras, the addition of the composite structure
allows values to be created much more intuitively, without any
drawback! All the code developed with object algebras works right
away with visitors. 

Finally note that in terms of typing, this solution does not require
any advanced use of generics. This is in sharp constrast with previous 
proposals for extensible visitors in the literature.

% \subsection{Yanlin stuff}
% \bruno{This can be dropped.}

% This subsection presents yet another lightweight solution to the Expression
% Problem, inspired by the recent work by Wang. It has been shown that
% contravariant return types allows refinement of the types of extended
% expressions.

% First, we define the type of expressions that support evaluation and implement
% two constructors:
% \begin{lstlisting}
% type Exp = { eval: Int }
% let lit (n: Int) = { eval = n }
% let add (e1: Exp) (e2: Exp)
%   = { eval = e1.eval + e2.eval }
% \end{lstlisting}

% If we would like to add a new operation, say pretty printing, it is nothing more
% than refining the original \lstinline{Exp} interface by \emph{intersecting} the
% original type with the new \lstinline{print} interface using the \lstinline{&}
% primitive and \emph{merging} the original data constructors using the \lstinline{,,}
% primitive.
% \begin{lstlisting}
% type ExpExt = Exp & { print: String }
% let litExt (n: Int) = lit n ,, { print = n.toString() }
% let addExt (e1: ExpExt) (e2: ExpExt)
%   = add e1 e2 ,,
%     { print = e1.print.concat(" + ").concat(e2.print) }
% \end{lstlisting}

% Now we can construct expressions using the constructors defined above:
% \begin{lstlisting}
% let e1: ExpExt = addExt (litExt 2) (litExt 3)
% let e2: Exp = add (lit 2) (lit 4)
% \end{lstlisting}
% \lstinline{e1} is an expression capable of both evaluation and printing, while
% \lstinline{e2} supports evaluation only.

% We can also add a new variant to our expression:
% \begin{lstlisting}
% let sub (e1: Exp) (e2: Exp) = { eval = e1.eval - e2.eval }
% let subExt (e1: ExpExt) (e2: ExpExt)
%   = sub e1 e2 ,, { print = e1.print.concat(" - ").concat(e2.print) }
% \end{lstlisting}

% Finally we are able to manipulate our expressions with the power of both
% subtraction and pretty printing.
% \begin{lstlisting}
% (subExt e1 e1).print
% \end{lstlisting}

% \subsection{Mixins}

% \bruno{Still not convinced by this section. Change to the record-based example.}
% Mixins are useful programming technique wildly adopted in dynamic programming
% languages such as JavaScript and Ruby. But obviously it is the programmers'
% responsbility to make sure that the mixin does not try to access methods or
% fields that are not present in the base class.

% In Haskell, one is also able to write programs in mixin style using records.
% However, this approach has a serious drawback: since there is no subtyping in
% Haskell, it is not possible to refine the mixin by adding more fields to the
% records. This means that the type of the family of the mixins has to be
% determined upfront, which undermines extensibility.

% \name is able to overcome both of the problems: it allows composing mixins
% that (1) extends the base behavior, (2) while ensuring type safety.

% The figure defines a mini mixin library. The apostrophe in front of types
% denotes call-by-name arguments similar to the \lstinline{=>} notation in the
% Scala language.

% \begin{lstlisting}{language=F2J}
% type Mixin S = 'S -> 'S -> S;
% let zero S (super : 'S) (this : 'S) : S = super;
% let rec mixin S (f : Mixin S) : S
%   = let m = mixin S in f (\ (_ : Unit). m f) (\ (_ : Unit). m f);
% let extends S (f : Mixin S) (g : Mixin S) : Mixin S
%   = \ (super : 'S). \ (this  : 'S). f (\ (d : Unit). g super this) this;
% \end{lstlisting}

% We define a factorial function in mixin style and make a \lstinline{noisy} mixin
% that prints ``Hello'' and delegates to its superclass. Then the two functions
% are composed using the \lstinline{mixin} and \lstinline{extends} combinators.
% The result is the \lstinline{noisyFact} function that prints ``Hello'' every
% time it is called and computes factorial.
% \begin{lstlisting}{language=F2J}
% let fact (super : 'Int -> Int) (this : 'Int -> Int) : Int -> Int
%   = \ (n : Int). if n == 0 then 1 else n * this (n - 1)
% let noisy (super : 'Int -> Int) (this : 'Int -> Int) : Int -> Int
%   = \ (n : Int). { println("Hello"); super n }
% let noisyFact = mixin (Int -> Int) (extends (Int -> Int) foolish fact)
% noisy 5
% \end{lstlisting}

% \subsection{Composing Mixins and Object Algebras}
