\documentclass[prodmode,acmtoplas]{acmsmall}

\usepackage{color}
\usepackage{graphicx}
\usepackage{latexsym}
\usepackage{stmaryrd}
\usepackage{amsmath}
\usepackage{amssymb}
% \usepackage{amsthm}
\usepackage{stmaryrd}
\usepackage{mathpartir}
\usepackage{wasysym}
\usepackage{url}
%%\usepackage{algorithm, algpseudocode}
\usepackage{fancyvrb}
%\usepackage[[style=1]{mdframed}
\usepackage{lipsum}
\usepackage{comment}

\definecolor{light}{gray}{.75}

% \mdfdefinestyle{MyFrame}{%
%     linecolor=white,
% %%    outerlinewidth=2pt,
% %%    roundcorner=20pt,
% %%    innertopmargin=\baselineskip,
% %%    innerbottommargin=\baselineskip,
% %%    innerrightmargin=20pt,
% %%    innerleftmargin=20pt,
%     backgroundcolor=light}

\newcommand{\gbox}[1]{\colorbox{light}{$\!\!#1\!\!$}}

%%\bibliographystyle{plain}

%% for comments
\newcommand{\yank}[1]{\textbf{yank:}}
\newcommand{\todoc}[2]{{\textcolor{#1} {\textbf{[[#2]]}}}}
\newcommand{\todo}[1]{{\todoc{red}{\textbf{[[#1]]}}}}

\newcommand{\todored}[1]{\todoc{red}  {\textbf{[[#1]]}}}
\newcommand{\todoblue}[1]{\todoc{blue}{\textbf{[[#1]]}}}
\newcommand{\todogreen}[1]{\todoc{green}{\textbf{[[#1]]}}}

\newcommand{\TODO}[1]{\todored{#1}}

\newcommand{\wc}[1]{\todogreen{WC: #1}}
\newcommand{\wt}[1]{\todogreen{WT: #1}}
\newcommand{\tom}[1]{\marginpar{\textcolor{red}{TOM}}\textcolor{red}{#1}}
\newcommand{\bruno}[1]{\todogreen{BRUNO: #1}}

% \newcommand{\wc}[1]{}
% \newcommand{\wt}[1]{}
% \newcommand{\tom}[1]{}
% \newcommand{\bruno}[1]{}

%%% Local Variables: 
%%% mode: latex
%%% TeX-master: "Main"
%%% End: 

% math-mode versions of \rlap, etc
% from Alexander Perlis, "A complement to \smash, \llap, and lap"
%   http://math.arizona.edu/~aprl/publications/mathclap/
\def\clap#1{\hbox to 0pt{\hss#1\hss}}
\def\mathllap{\mathpalette\mathllapinternal}
\def\mathrlap{\mathpalette\mathrlapinternal}
\def\mathclap{\mathpalette\mathclapinternal}
\def\mathllapinternal#1#2{\llap{$\mathsurround=0pt#1{#2}$}}
\def\mathrlapinternal#1#2{\rlap{$\mathsurround=0pt#1{#2}$}}
\def\mathclapinternal#1#2{\clap{$\mathsurround=0pt#1{#2}$}}

% math-mode versions of \rlap, etc
% from Alexander Perlis, "A complement to \smash, \llap, and lap"
%   http://math.arizona.edu/~aprl/publications/mathclap/
\def\clap#1{\hbox to 0pt{\hss#1\hss}}
\def\mathllap{\mathpalette\mathllapinternal}
\def\mathrlap{\mathpalette\mathrlapinternal}
\def\mathclap{\mathpalette\mathclapinternal}
\def\mathllapinternal#1#2{\llap{$\mathsurround=0pt#1{#2}$}}
\def\mathrlapinternal#1#2{\rlap{$\mathsurround=0pt#1{#2}$}}
\def\mathclapinternal#1#2{\clap{$\mathsurround=0pt#1{#2}$}}

\newcommand{\horizontalrule}{
  \begin{center}
    \line(1,0){250}
  \end{center}
}

\newcommand{\code}[1]{\texttt{#1}}
\definecolor{light-gray}{gray}{0.9}
\newcommand{\highlight}[1]{\colorbox{light-gray}{#1}}

\newcommand{\im}[1]{\lvert #1 \rvert}
\newcommand{\powerset}[1]{\mathcal{P}(#1)}
\newcommand{\universe}{\mathbb{U}}

\newcommand{\turns}{\vdash}
\newcommand{\oftype}{\!:\!}
\newcommand{\subtype}{<:}
\newcommand{\commonsuper}{\Uparrow}
\newcommand{\commonsub}{\Downarrow}
\newcommand{\disjoint}{*}
\newcommand{\disjointimpl}{*_\textnormal{i}}
\newcommand{\disjointax}{*_\textnormal{ax}}
\newcommand{\subst}[2]{\lbrack #2 := #1 \rbrack~}

\newcommand{\yields}[1]{\highlight{$\; \hookrightarrow #1$}}
% \newcommand{\yields}[1]{}

\newcommand{\ftv}[1]{\textsf{ftv}(#1)}

\newcommand{\binderspace}{\,}
\newcommand{\appspace}{\;}

\newcommand{\inter}{\&}
\newcommand{\union}{|}
\newcommand{\for}[2]{\forall #1.\binderspace #2}
\newcommand{\fordis}[3]{\for {(#1 \disjoint #2)} {#3}}
\newcommand{\lam}[2]{\lambda #1.\binderspace #2}
\newcommand{\lamty}[3]{\lam {(#1 \oftype #2)} #3}
\newcommand{\blam}[2]{\Lambda #1.\binderspace #2}
\newcommand{\blamdis}[3]{\blam {(#1 \disjoint #2)} #3}
\newcommand{\mergeop}{,,}
\newcommand{\app}[2]{#1 \; #2}
\newcommand{\tapp}[2]{#1 \appspace #2}
\newcommand{\pair}[2]{(#1, #2)}
\newcommand{\proj}[2]{{\code{proj}}_{#1} #2}
\newcommand{\fst}[1]{\app {\code{fst}} {#1}}
\newcommand{\snd}[1]{\app {\code{snd}} {#1}}
\newcommand{\recordType}[2]{\{ #1 : #2 \}}
\newcommand{\recordCon}[2]{\{ #1 = #2 \}}

\newcommand{\true}{\code{True}}
\newcommand{\tyint}{\code{Int}}
\newcommand{\tybool}{\code{Bool}}
\newcommand{\tychar}{\code{Char}}
\newcommand{\tystring}{\code{String}}

% Judgements
\newcommand{\jwf}[2]{#1 \turns #2}
\newcommand{\jatomic}[1]{#1 \ \textnormal{atomic}}
\newcommand{\jtype}[3]{\turns #2 \ \oftype \ #3}
\newcommand{\jdis}[3]{\turns #2 \disjoint #3}
\newcommand{\jdisimpl}[3]{\turns #2 \disjointimpl #3}

\newcommand{\reflabel}[1]{(\textsc{#1})}

% \newcommand{\name}{$ \lambda_{\&} $\xspace}
\newcommand{\name}{\xspace[name]\xspace}

\newcommand{\authornote}[3]{\textcolor{#2}{\textsc{#1}: #3}}
\newcommand\bruno[1]{\authornote{bruno}{Red}{#1}}
\newcommand\george[1]{\authornote{george}{Blue}{#1}}

%%include lhs2TeX.fmt
%% ODER: format ==         = "\mathrel{==}"
%% ODER: format /=         = "\neq "
%
%
\makeatletter
\@ifundefined{lhs2tex.lhs2tex.sty.read}%
  {\@namedef{lhs2tex.lhs2tex.sty.read}{}%
   \newcommand\SkipToFmtEnd{}%
   \newcommand\EndFmtInput{}%
   \long\def\SkipToFmtEnd#1\EndFmtInput{}%
  }\SkipToFmtEnd

\newcommand\ReadOnlyOnce[1]{\@ifundefined{#1}{\@namedef{#1}{}}\SkipToFmtEnd}
\usepackage{amstext}
\usepackage{amssymb}
\usepackage{stmaryrd}
\DeclareFontFamily{OT1}{cmtex}{}
\DeclareFontShape{OT1}{cmtex}{m}{n}
  {<5><6><7><8>cmtex8
   <9>cmtex9
   <10><10.95><12><14.4><17.28><20.74><24.88>cmtex10}{}
\DeclareFontShape{OT1}{cmtex}{m}{it}
  {<-> ssub * cmtt/m/it}{}
\newcommand{\texfamily}{\fontfamily{cmtex}\selectfont}
\DeclareFontShape{OT1}{cmtt}{bx}{n}
  {<5><6><7><8>cmtt8
   <9>cmbtt9
   <10><10.95><12><14.4><17.28><20.74><24.88>cmbtt10}{}
\DeclareFontShape{OT1}{cmtex}{bx}{n}
  {<-> ssub * cmtt/bx/n}{}
\newcommand{\tex}[1]{\text{\texfamily#1}}	% NEU

\newcommand{\Sp}{\hskip.33334em\relax}


\newcommand{\Conid}[1]{\mathit{#1}}
\newcommand{\Varid}[1]{\mathit{#1}}
\newcommand{\anonymous}{\kern0.06em \vbox{\hrule\@width.5em}}
\newcommand{\plus}{\mathbin{+\!\!\!+}}
\newcommand{\bind}{\mathbin{>\!\!\!>\mkern-6.7mu=}}
\newcommand{\rbind}{\mathbin{=\mkern-6.7mu<\!\!\!<}}% suggested by Neil Mitchell
\newcommand{\sequ}{\mathbin{>\!\!\!>}}
\renewcommand{\leq}{\leqslant}
\renewcommand{\geq}{\geqslant}
\usepackage{polytable}

%mathindent has to be defined
\@ifundefined{mathindent}%
  {\newdimen\mathindent\mathindent\leftmargini}%
  {}%

\def\resethooks{%
  \global\let\SaveRestoreHook\empty
  \global\let\ColumnHook\empty}
\newcommand*{\savecolumns}[1][default]%
  {\g@addto@macro\SaveRestoreHook{\savecolumns[#1]}}
\newcommand*{\restorecolumns}[1][default]%
  {\g@addto@macro\SaveRestoreHook{\restorecolumns[#1]}}
\newcommand*{\aligncolumn}[2]%
  {\g@addto@macro\ColumnHook{\column{#1}{#2}}}

\resethooks

\newcommand{\onelinecommentchars}{\quad-{}- }
\newcommand{\commentbeginchars}{\enskip\{-}
\newcommand{\commentendchars}{-\}\enskip}

\newcommand{\visiblecomments}{%
  \let\onelinecomment=\onelinecommentchars
  \let\commentbegin=\commentbeginchars
  \let\commentend=\commentendchars}

\newcommand{\invisiblecomments}{%
  \let\onelinecomment=\empty
  \let\commentbegin=\empty
  \let\commentend=\empty}

\visiblecomments

\newlength{\blanklineskip}
\setlength{\blanklineskip}{0.66084ex}

\newcommand{\hsindent}[1]{\quad}% default is fixed indentation
\let\hspre\empty
\let\hspost\empty
\newcommand{\NB}{\textbf{NB}}
\newcommand{\Todo}[1]{$\langle$\textbf{To do:}~#1$\rangle$}

\EndFmtInput
\makeatother
%
%
%
%
%
%
% This package provides two environments suitable to take the place
% of hscode, called "plainhscode" and "arrayhscode". 
%
% The plain environment surrounds each code block by vertical space,
% and it uses \abovedisplayskip and \belowdisplayskip to get spacing
% similar to formulas. Note that if these dimensions are changed,
% the spacing around displayed math formulas changes as well.
% All code is indented using \leftskip.
%
% Changed 19.08.2004 to reflect changes in colorcode. Should work with
% CodeGroup.sty.
%
\ReadOnlyOnce{polycode.fmt}%
\makeatletter

\newcommand{\hsnewpar}[1]%
  {{\parskip=0pt\parindent=0pt\par\vskip #1\noindent}}

% can be used, for instance, to redefine the code size, by setting the
% command to \small or something alike
\newcommand{\hscodestyle}{}

% The command \sethscode can be used to switch the code formatting
% behaviour by mapping the hscode environment in the subst directive
% to a new LaTeX environment.

\newcommand{\sethscode}[1]%
  {\expandafter\let\expandafter\hscode\csname #1\endcsname
   \expandafter\let\expandafter\endhscode\csname end#1\endcsname}

% "compatibility" mode restores the non-polycode.fmt layout.

\newenvironment{compathscode}%
  {\par\noindent
   \advance\leftskip\mathindent
   \hscodestyle
   \let\\=\@normalcr
   \(\pboxed}%
  {\endpboxed\)%
   \par\noindent
   \ignorespacesafterend}

\newcommand{\compaths}{\sethscode{compathscode}}

% "plain" mode is the proposed default.

\newenvironment{plainhscode}%
  {\hsnewpar\abovedisplayskip
   \advance\leftskip\mathindent
   \hscodestyle
   \let\\=\@normalcr
   \(\pboxed}%
  {\endpboxed\)%
   \hsnewpar\belowdisplayskip
   \ignorespacesafterend}

% Here, we make plainhscode the default environment.

\newcommand{\plainhs}{\sethscode{plainhscode}}
\plainhs

% The arrayhscode is like plain, but makes use of polytable's
% parray environment which disallows page breaks in code blocks.

\newenvironment{arrayhscode}%
  {\hsnewpar\abovedisplayskip
   \advance\leftskip\mathindent
   \hscodestyle
   \let\\=\@normalcr
   \(\parray}%
  {\endparray\)%
   \hsnewpar\belowdisplayskip
   \ignorespacesafterend}

\newcommand{\arrayhs}{\sethscode{arrayhscode}}

% The mathhscode environment also makes use of polytable's parray 
% environment. It is supposed to be used only inside math mode 
% (I used it to typeset the type rules in my thesis).

\newenvironment{mathhscode}%
  {\parray}{\endparray}

\newcommand{\mathhs}{\sethscode{mathhscode}}

% texths is similar to mathhs, but works in text mode.

\newenvironment{texthscode}%
  {\(\parray}{\endparray\)}

\newcommand{\texths}{\sethscode{texthscode}}

% The framed environment places code in a framed box.

\def\codeframewidth{\arrayrulewidth}
\RequirePackage{calc}

\newenvironment{framedhscode}%
  {\parskip=\abovedisplayskip\par\noindent
   \hscodestyle
   \arrayrulewidth=\codeframewidth
   \tabular{@{}|p{\linewidth-2\arraycolsep-2\arrayrulewidth-2pt}|@{}}%
   \hline\framedhslinecorrect\\{-1.5ex}%
   \let\endoflinesave=\\
   \let\\=\@normalcr
   \(\pboxed}%
  {\endpboxed\)%
   \framedhslinecorrect\endoflinesave{.5ex}\hline
   \endtabular
   \parskip=\belowdisplayskip\par\noindent
   \ignorespacesafterend}

\newcommand{\framedhslinecorrect}[2]%
  {#1[#2]}

\newcommand{\framedhs}{\sethscode{framedhscode}}

% The inlinehscode environment is an experimental environment
% that can be used to typeset displayed code inline.

\newenvironment{inlinehscode}%
  {\(\def\column##1##2{}%
   \let\>\undefined\let\<\undefined\let\\\undefined
   \newcommand\>[1][]{}\newcommand\<[1][]{}\newcommand\\[1][]{}%
   \def\fromto##1##2##3{##3}%
   \def\nextline{}}{\) }%

\newcommand{\inlinehs}{\sethscode{inlinehscode}}

% The joincode environment is a separate environment that
% can be used to surround and thereby connect multiple code
% blocks.

\newenvironment{joincode}%
  {\let\orighscode=\hscode
   \let\origendhscode=\endhscode
   \def\endhscode{\def\hscode{\endgroup\def\@currenvir{hscode}\\}\begingroup}
   %\let\SaveRestoreHook=\empty
   %\let\ColumnHook=\empty
   %\let\resethooks=\empty
   \orighscode\def\hscode{\endgroup\def\@currenvir{hscode}}}%
  {\origendhscode
   \global\let\hscode=\orighscode
   \global\let\endhscode=\origendhscode}%

\makeatother
\EndFmtInput
%
%%\input{lemmas}

% \newenvironment{example}{\par\noindent\textit{Example}\quad}{}

% Metadata Information
\acmVolume{1}
\acmNumber{1}
\acmArticle{1}
\acmYear{2012}
\acmMonth{1}

% Document starts
\begin{document}

% Page heads
%\markboth{B. Oliveira et al.}{The Implicit Calculus: A New Foundation for Generic Programming}

% Title portion
\title{Formalization of System F to Java Translation}
\author{DRAFT}

\begin{abstract}
TODO
\end{abstract}

\category{D.3.2}{Programming Languages}   
                {Language Classifications}
                [Functional Languages, Object-Oriented Languages]
\category{F.3.3}{Logics and Meanings of Programs}   
                {Studies of Program Constructs}
                []

\terms{Languages}

\keywords{
Implicit parameters, type classes, C++ concepts, generic programming,
Haskell, Scala}

\acmformat{Oliveira, B. C. d. S., Schrijvers, T., Choi, W., Lee, W., Yi, K., Wadler, P.
201?. The implicit calculus: a new foundation for generic programming.}

\begin{bottomstuff}
This work is supported by \ldots.

\end{bottomstuff}

\maketitle

%include lhs2TeX.fmt
%include polycode.fmt
%include forall.fmt
%include Rule.fmt

\section{Type-Inference for the Simply Typed Lambda Calculus}

The syntax of System F is as follows: 
{\small
  \[ \begin{array}{llrl}
    \text{Types} & T & ::= & \alpha \mid T \arrow T 
    \mid \forall \alpha. T \\ 
    \text{Expressions} & E & ::=  & x \mid y \mid \lambda (x:T) . E \mid E\;E
    \mid \Lambda \alpha . E \mid E\;T 
  \end{array} \]}

\newcommand{\typeSrc}{T}
\newcommand{\envSrc}{G}
\figtwocol{f:syntax}{System F}{
\small
\bda{l}

\ba{llrl}
    \textbf{Types} & \type & ::= & \alpha \mid \type \arrow \type 
    \mid \forall \alpha. \type \\ 
    \textbf{Expressions} & e & ::=  & x \mid \lambda (x:\type) . e \mid e\;e \mid {\bf fix}~y . \lambda (x : \type_1) : \type_2 . e 
    \mid \Lambda \alpha . e \mid e\;\type 
\ea
\\ \\

\ba{llrl} 
\textbf{Type Environments} & \Gamma & ::= & \epsilon \mid \Gamma, \relation{x}{\type} \\
%%\textbf{Fixpoint Environments} & \Theta & ::= & \epsilon \mid \relation{x}{\type} 
\ea 
\\ \\

\textbf{Type System}
\\ \\

\ba{lc}
\multicolumn{2}{l}{\myruleform{\Gamma \turns  \relation{e}{\type}}} \\ \\

  (\texttt{F-Var}) & 
\myirule{
           (x : \type) \in \Gamma
 }{
            \Gamma \turns x : \type
} \\ \\

  (\texttt{F-Abs}) & 
\myirule{
 %%         \Gamma_1 \turns \Gamma_2 \\
          \Gamma_2, x : \type_1 \turns e : \type_2
 }{
          \Gamma_1 \turns \lambda (x:\type_1).e : \type_1 \rightarrow \type_2
} \\ \\

  (\texttt{F-App}) & 
\myirule{
  \Gamma \turns e_1 : \type_2 \rightarrow \type_1 \\
           \Gamma \turns e_2 : \type_2
          }{
           \Gamma \turns e_1 \, e_2 : \type_1
} \\ \\

%%  (\texttt{F-Fix2}) & 
%%\myirule{
%%           (x:\type); \Gamma \turns e [y \mapsto x]: \type
%% }{
%%           (y:\type); \Gamma \turns {\bf fix} (x:\type).e : \type
%%} \\ \\

  (\texttt{F-TApp}) & 
\myirule{
  \Gamma \turns e : \forall \alpha. \type_2
           }{
            \Gamma \turns e \, \type_1 : \type_2[\type_1/\alpha]
} \\ \\

  (\texttt{F-TAbs}) & 
\myirule{
   \Gamma, \alpha \turns e : \type
            }{
             \Gamma \turns \Lambda \alpha.e : \forall \alpha. \type 
} \\ \\

  (\texttt{F-Fix}) & 
\myirule{
           \Gamma, (y : \type_1 \rightarrow \type_2) \turns \lambda (x : \type_1) . e : \type_1 \rightarrow \type_2
 }{
           \Gamma \turns {\bf fix}~y. \lambda (x : \type_1) : \type_2 . e : \type_1 \rightarrow \type_2
} \\ \\

%%\multicolumn{2}{l}{\myruleform{\Theta; \Gamma_1 \turns \Gamma_2}} \\ \\
%%
%%(\texttt{F-Nothing}) & 
%%\myirule{
%% }{
%%            \epsilon; \Gamma \turns \Gamma 
%%} \\ \\

%%(\texttt{F-Just}) & 
%%\myirule{
%% }{
%%            (x : \type); \Gamma \turns \Gamma, (x :\type) 
%%} \\ \\


\ea

\eda
}

\figtwocol{f:syntax}{Closure F}{
\small
\bda{l}

\ba{llrl}
    \textbf{Types} & T & ::= & \alpha \mid \forall \Delta. T \\ 
    \textbf{Expressions} & E & ::=  & x \mid \lambda \Delta . E \mid E\;E \mid E\;T \mid {\bf fix} (x : \forall \Delta . T) . E
\ea
\\ \\

\ba{llrl} 
\textbf{Type Environments} & \Delta & ::= & \epsilon \mid \Delta
(\relation{x}{T}) \mid \Delta [ x : T ] \mid \Delta \alpha \\
%%\textbf{Fixpoint Environments} & \Theta & ::= & \epsilon \mid \relation{x}{\type} 
\ea 
\\ \\

\textbf{Type System}
\\ \\

\ba{lc}
\multicolumn{2}{l}{\myruleform{\Theta; \Delta \turns  \relation{E}{T}}} \\ \\

  (\texttt{C-Var}) & 
\myirule{
           (x : T) \in \Delta
 }{
            \Delta \turns x : T
} \\ \\

  (\texttt{C-RVar}) & 
\myirule{
           [x : T] \in \Delta
 }{
            \Delta \turns x : T
} \\ \\

  (\texttt{C-Abs}) & 
\myirule{
           \Delta_1 \uplus \Delta_2 \turns E : T
 }{
           \Delta_1 \turns \lambda \Delta_2.E : \forall \Delta_2. T
} \\ \\

  (\texttt{C-App}) & 
\myirule{
  \Delta_1 \turns E_1 : \forall (\relation{x}{T_2}) \Delta_2. T_1 ~~~~~~~
           \Delta_1 \turns E_2 : T_2 ~~~~~~~
           \Delta_2; T_1 \Downarrow T_3
          }{
           \Delta_1 \turns E_1 \, E_2 : T_3
} \\ \\

  (\texttt{C-TApp}) & 
\myirule{
  \Delta_1 \turns E : \forall \alpha \Delta_2. T_2  ~~~~~~~
           \Delta_2; T_2 \Downarrow T_3
           }{
            \Delta_1 \turns E \, T_1 : T_2[T_1/\alpha]
} \\ \\

  (\texttt{C-Fix}) & 
\myirule{
  \Delta_1 [x : \forall \Delta_2 . T] \turns \lambda \Delta_2 . E : \forall \Delta_2 . T  ~~~~~~~
           }{
            \Delta_1 \turns {\bf fix} (x : \forall \Delta_2 . T) . E : \forall \Delta_2 . T
} \\ \\

\multicolumn{2}{l}{\myruleform{\Delta; T_1 \Downarrow  T_2}} \\ \\

  (\texttt{D-Empty}) & 
\myirule{}{
           \epsilon; T \Downarrow T
} \\ \\

  (\texttt{D-NonEmpty}) & 
\myirule{}{
           \Delta; T \Downarrow \forall \Delta. T
} \\ \\

\ea

\eda
}


\figtwocol{f:syntax}{Syntax-Directed Translation from System F to
  Closure F}{
\small
\bda{l}
\ba{lc}
\multicolumn{2}{l}{\myruleform{e \leadsto  E}} \\ \\

  (\texttt{FC-Var}) & 
\myirule{}{
            x \leadsto x
} \\ \\

  (\texttt{FC-App}) & 
\myirule{
  e_1 \leadsto  E_1 ~~~~~~~~~~
           e_2 \leadsto E_2
          }{
           e_1 \, e_2 \leadsto E_1 \, E_2
} \\ \\

  (\texttt{FC-TApp}) & 
\myirule{
  e \leadsto E 
           }{
            e \, \type \leadsto E \, |T|
} \\ \\

  (\texttt{FC-Fix}) & 
\myirule{
  \epsilon \turns \lambda (x : \type_1) . e \leadsto \lambda \Delta . E \\
  T = adjust(\Delta,\type_2) 
           }{
            {\bf fix}~y . \lambda (x : \type_1) : \type_2 . e  \leadsto {\bf fix} (y : \forall \Delta . T) . E
} \\ \\

  (\texttt{FC-Rest}) & 
\myirule{
  \epsilon \turns e \leadsto E 
           }{
            e \leadsto E 
} \\ \\

\multicolumn{2}{l}{\myruleform{\Delta \turns e \leadsto  E}} \\ \\

  (\texttt{FC-Abs}) & 
\myirule{
           \Delta (x : |\type_1|) \turns e \leadsto E
 }{
           \Delta \turns \lambda x:\type_1.e \leadsto E
} \\ \\

  (\texttt{FC-TAbs}) & 
\myirule{
   \Delta \alpha \turns e \leadsto E
            }{
             \Delta \turns \Lambda \alpha.e \leadsto E 
} \\ \\

  (\texttt{FC-Rest2}) & 
\myirule{
  e \leadsto E 
           }{
            \Delta \turns e \leadsto \lambda \Delta. E 
} \\ \\

%%\multicolumn{2}{l}{\myruleform{\Delta \turns \type \leadsto  T}} \\ \\



\ea \\ \\

%%\ba{lcl}
%%|\alpha| & = & \alpha \\
%%|\tyint| & = & \tyint \\
%%|\type_1 \arrow \type_2| & = & |\rulet_1| \arrow |\rulet_2| \\
%%|\forall \alpha. \rulet| & = & \forall \alpha. |\rulet| \\
%%|\rulet_1 \iarrow \rulet_2| & = & |\rulet_1| \arrow |\rulet_2| %\\
%%\ea


\eda
}


\figtwocol{f:syntax}{Type-Directed Translation from Closure F to Java}{
\small
\bda{l}

\ba{lc}
\multicolumn{2}{l}{\myruleform{\Delta \turns  E : T \leadsto J \textbf{~in~} S}} \\ \\

  (\texttt{CJ-Var}) & 
\myirule{
           (x : T) \in \Delta
 }{
            \Delta \turns x : T \leadsto x.a \textbf{~in~} \{\}
} \\ \\

  (\texttt{CJ-RVar}) & 
\myirule{
           [x : T] \in \Delta
 }{
            \Delta \turns x : T \leadsto x \textbf{~in~} \{\}
} \\ \\

  (\texttt{CJ-Abs}) & 
\myirule{
           \epsilon ;\Delta_1 \uplus \Delta_2; \Delta_2 \turns E : T \leadsto J \textbf{~in~} S
 }{
           \Delta_1 \turns \lambda \Delta_2.E : \forall \Delta_2. T \leadsto J \textbf{~in~} S
} \\ \\


  (\texttt{CJ-App}) & 
\myirule{
  \Delta_1 \turns E_1 : \forall (\relation{x}{T_2}) \Delta_2. T_1
  \leadsto J_1 \textbf{~in~} S_1 ~~~~~~~
           \Delta_1 \turns E_2 : T_2 \leadsto J_2 \textbf{~in~} S_2 ~~~~~~~
           \Delta_2; T_1 \Downarrow T_3
          }{
           \Delta_1 \turns E_1 \, E_2 : T_3 \leadsto e_l\textbf{~in~}
           S_1 \uplus S_2 \uplus \{Closure~f = ((Closure)~J_1).clone(); f.a =
           J_2; f.apply(); T_3~e_l = f.out;\}
} \\ \\

  (\texttt{CJ-TApp}) & 
\myirule{
  \Delta_1 \turns E : \forall \alpha \Delta_2. T_2  \leadsto J
  \textbf{~in~} S ~~~~~~~
           \Delta_2; T_2 \Downarrow T_3 
           }{
            \Delta_1 \turns E \, T_1 : T_3[T_1/\alpha] \leadsto J
            \textbf{~in~} S
} \\ \\

  (\texttt{CJ-Fix}) & 
\myirule{
  [x : \forall \Delta_2 . T]; \Delta_1 [x : \forall \Delta_2 . T] \uplus \Delta_2; \Delta_2 \turns E : \forall \Delta_2 . T \leadsto J \textbf{~in~} S  ~~~~~~~
           }{
            \Delta_1 \turns {\bf fix} (x : \forall \Delta_2 . T) . E : \forall \Delta_2 . T \leadsto J \textbf{~in~} S
} \\ \\

\multicolumn{2}{l}{\myruleform{\Theta; \Delta_1; \Delta_2 \turns  E : T \leadsto J \textbf{~in~} S}} \\ \\

  (\texttt{CJD-Empty}) & 
\myirule{
           \Delta \turns \relation{E}{T} \leadsto  J \textbf{~in~} S
}{
           \Theta; \Delta; \epsilon \turns \relation{E}{T} \leadsto  J \textbf{~in~} S
} \\ \\

  (\texttt{CJD-Bind1}) & 
\myirule{
           %nolambda(\Delta_2) \\
           (x,E_2) = update(\Theta,y,E_1) \\
           \epsilon; \Delta_1; \Delta_2 \turns \relation{E_2}{T} \leadsto  J \textbf{~in~} S
}{
           \Theta_1; \Delta_1; (\relation{y}{T_1})\; \Delta_2 \turns
           \relation{E}{T} \leadsto  f \textbf{~in~} \{\\
              \textbf{class}~FUN~\textbf{extends}~Closure \{ \\
                   Closure~x = \textbf{this}; \\
                   \textbf{void}~apply() \{S; out = J;\} \\
                   \textbf{Closure}~clone() \{Closure~c = new~FUN(); c.x = \textbf{this}.x; c.apply(); \textbf{return}~c;\}                     
              \};\\
              Closure~f = new~FUN(); \}
} \\ \\

  (\texttt{CJD-Bind2}) & 
\myirule{
           \Theta; \Delta_1; \Delta_2 \turns \relation{E}{T} \leadsto  J \textbf{~in~} S
}{
           \Theta; \Delta_1; \alpha\; \Delta_2 \turns \relation{E}{T} \leadsto  J \textbf{~in~} S
} \\ \\

&
update(\epsilon,y,E) = (y,E) \\ &
update([x : \forall \Delta_2 . T],y,E) = (x, E[y \mapsto x])\\ \\

\ea


\eda


}


\figtwocol{f:syntax}{Optimized Type-Directed Translation from Closure
  F to Java (Different rules and additional rules only)}{
\small
\bda{l}

\ba{lc}

  (\texttt{CJ-App}) & 
\myirule{
  \Delta_1 \turns E_1 : \forall (\relation{x}{T_2}) \Delta_2. T_1
  \leadsto J_1 \textbf{~in~} S_1 ~~~~~~~
           \Delta_1 \turns E_2 : T_2 \leadsto J_2 \textbf{~in~} S_2~~~~~~~
           \Delta_2; f; T_1 \Downarrow T_3 \leadsto S_3
          }{
           \Delta_1 \turns E_1 \, E_2 : T_3 \leadsto f.out \textbf{~in~}
           S_1 \uplus S_2 \uplus \{Closure~f = ((Closure)~J_1).clone(); f.a =
           J_2;\} \uplus S_3
} \\ \\

  (\texttt{CJ-TApp}) & 
\myirule{
  \Delta_1 \turns E : \forall \alpha \Delta_2. T_2  \leadsto J
  \textbf{~in~} S_1 ~~~~~~~
           \Delta_2; J; T_2 \Downarrow T_3 \leadsto S_2 
           }{
            \Delta_1 \turns E \, T_1 : T_2[T_1/\alpha] \leadsto J
            \textbf{~in~} S_1 \uplus S_2
} \\ \\


  (\texttt{CJD-Bind1Closure}) & 
\myirule{
           \Delta_1; \Delta_2 \turns \relation{E}{T} \leadsto f_1 \textbf{~in~} S
}{
           \Delta_1; (\relation{x}{T_1})\; \Delta_2 \turns
           \relation{E}{T} \leadsto  f_2 \textbf{~in~} \{\\
              \textbf{class}~FUN~\textbf{extends}~Closure \{ \\
                   Closure~x = \textbf{this}; \\
                   \textbf{void}~apply() \{S; out = J;\} \\
                   \textbf{Closure}~clone() \{Closure~c = new~FUN(); c.x = \textbf{this}.x; c.apply(); \textbf{return}~c;\}                     
              \};\\
              Closure~f_2 = new~FUN(); \}
} \\ \\

  (\texttt{CJD-Bind1Other}) & 
\myirule{
           \Delta_1; \Delta_2 \turns \relation{E}{T} \leadsto  J \textbf{~in~} S
}{
           \Delta_1; (\relation{x}{T_1})\; \Delta_2 \turns
           \relation{E}{T} \leadsto  f \textbf{~in~} \{\\
              \textbf{class}~FUN~\textbf{extends}~Closure \{ \\
                   Closure~x = \textbf{this}; \\
                   \textbf{void}~apply() \{S; out = J;\} \\
                   \textbf{Closure}~clone() \{Closure~c = new~FUN(); c.x = \textbf{this}.x; c.apply(); \textbf{return}~c;\}                     
              \};\\
              Closure~f = new~FUN(); \}
} \\ \\


\multicolumn{2}{l}{\myruleform{\Delta; J; T_1 \Downarrow  T_2 \leadsto
  S}} \\ \\

  (\texttt{D-Empty}) & 
\myirule{}{
           \epsilon; J; T \Downarrow T \leadsto \{J.apply(); e_l = J.out;\} 
} \\ \\

  (\texttt{D-NonEmpty}) & 
\myirule{}{
           \Delta; J; T \Downarrow \forall \Delta. T \leadsto \{\}
} \\ \\

\ea

\eda
}

\figtwocol{f:syntax}{Tail Call Optimization when Translating from Closure
  F to Java (Different rules and additional rules only)}{
\small
\bda{l}

\ba{lc}

  (\texttt{CJ-App}) & 
\myirule{
  \Delta \turns E_1\; E_2 : T
  \leadsto J \textbf{~in~} S_1; \Sigma ~~~~~~~ \Sigma \downarrow S_2
          }{
           \Delta \turns E_1\; E_2 : T \leadsto J \textbf{~in~}
           S_1 \uplus S_2 
} \\ \\

\multicolumn{2}{l}{\myruleform{\Delta \turns  E : T \leadsto J
    \textbf{~in~} S; \Sigma}} \\ \\

  (\texttt{CJ-App-$\Sigma$}) & 
\myirule{
  \Delta_1 \turns E_1 : \forall (\relation{x}{T_2}) \Delta_2. T_1
  \leadsto J_1 \textbf{~in~} S_1; \Sigma_1 \\
           \Delta_1 \turns E_2 : T_2 \leadsto J_2 \textbf{~in~} S_2;
           \Sigma_2 \\
           \Delta_2; f; T_1 \Downarrow T_3 \leadsto S_3 
          }{
           \Delta_1 \turns E_1 \, E_2 : T_3 \leadsto f.out \textbf{~in~}
           S_1 \uplus S_2; \Sigma_1 \uplus \Sigma_2; \{Closure~f = (Closure)~J_1; f.a =
           J_2;\}; S_3
} \\ \\

 (\texttt{CJ-Rest-$\Sigma$}) & 
\myirule{\Delta \turns E : T \leadsto J \textbf{~in~}
           S 
          }{
           \Delta \turns E : T \leadsto J \textbf{~in~}
           S; \epsilon
} \\ \\

\multicolumn{2}{l}{\myruleform{\Sigma \downarrow  S}} \\ \\

  (\texttt{$\Sigma$-Empty}) & 
\myirule{}{
           \epsilon \downarrow \{\}
} \\ \\

  (\texttt{$\Sigma$-One}) & 
\myirule{}{
           \epsilon; S_1; S_2 \downarrow S_1 \uplus S_2
} \\ \\

  (\texttt{$\Sigma$-Many1}) & 
\myirule{
          \Sigma \downarrow S_3
}{
           \Sigma; S_1; \{\} \downarrow S_1 \uplus S_3
} \\ \\

  (\texttt{$\Sigma$-Many2}) & 
\myirule{
          \Sigma \downarrow S_3
}{
           \Sigma; S_1; S_2 \downarrow S_1 \uplus \{Stack.push({\bf new}~
           Closure()\{{\bf void}~apply() \{S_3\}\});\} \uplus S_2
} \\ \\

\multicolumn{2}{l}{\myruleform{\Delta; J; T_1 \Downarrow  T_2 \leadsto
  S}} \\ \\

  (\texttt{D-Empty}) & 
\myirule{}{
           \epsilon; J; T \Downarrow T \leadsto \{Stack.push(J);\}
} \\ \\

  (\texttt{D-NonEmpty}) & 
\myirule{}{
           \Delta; J; T \Downarrow \forall \Delta. T \leadsto \{\}
} \\ \\

\ea

\eda
}

\figtwocol{f:syntax}{Unboxing Translation Rules from Closure F to Java}{
\small
\bda{l}

\ba{lc}

  (\texttt{CJ-UnboxAux1}) & 
\myirule{
           T_1 \Downarrow Int ~~~~~~~ T_2 \Downarrow Int
 }{
            T_1;T_2 \Downarrow ClosureIntInt
} \\ \\

  (\texttt{CJ-UnboxAux2}) & 
\myirule{
           T_1 \Downarrow Int
 }{
            T_1;T_2 \Downarrow ClosureIntA
} \\ \\

  (\texttt{CJ-UnboxAux3}) & 
\myirule{
           T_2 \Downarrow Int
 }{
            T_1;T_2 \Downarrow ClosureAInt
} \\ \\

  (\texttt{CJ-UnboxAux0}) & 
\myirule{}{
            T_1;T_2 \Downarrow Closure
} \\ \\


  (\texttt{CJ-TAppEnv}) & 
\myirule{
  \Delta_1, \Omega_1 \turns E : \forall \alpha \Delta_2. T_2  \leadsto J
  \textbf{~in~} S_1, \Omega_1' ~~~~~~~
           \Delta_2; J; T_2 \Downarrow T_3 \leadsto S_2 
           }{
            \Delta_1, \Omega_1 \turns E \, T_1 : T_2\leadsto J
            \textbf{~in~} S_1 \uplus S_2, \Omega_1' \cup \{ T_1 \mapsto \alpha \}
} \\ \\

  (\texttt{CJ-WrapFun}) & 
\myirule{
           \Delta_1, \Omega, T_2 \turns T_2 \Omega \Downarrow \forall (\relation{x}{T_{A}}) \Delta_2. T_{B}   \\
           T_{A};T_{b} \Downarrow CT \in \{ClosureIntInt, ClosureIntA, ClosureAInt\} \\
           \Delta_1, \Omega \turns E_2 : (T_2 \Omega)~~~~~~~\\
           \Delta_1, \Omega_1 \turns E_1 : \forall (\relation{x}{T_{2P}}) \Delta_2. T_1
 ~~~~~~~\\
           T_2 \Omega \neq T_{2P} ~~~~~~~
          }{
           \Delta_1, \Omega \turns (E_2 : T_{2P})\Omega  \leadsto new~Closure \{\\
              \textbf{void}~apply() \{\\
                exp2.a = (T_{A})~wrappedf.a;\\
                exp2.apply();\\
                wrappedf.out = exp2.out;
              \}
           \};
} \\ \\

  (\texttt{CJ-WrapId}) & 
\myirule{
           \Delta_1, \Omega \turns E_1 : \forall (\relation{x}{T_{2P}}) \Delta_2. T_1
   }{
            \Delta_1, \Omega \turns (E_2 : T_{2P})\Omega  \leadsto exp2
} \\ \\

  (\texttt{CJ-Unbox}) & 
\myirule{
       \Delta_1, \Omega_1 \turns E_1 : \forall (\relation{x}{T_2}) \Delta_2. T_1 \leadsto J_1 \textbf{~in~} S_1, \Omega_1' ~~~~~~~ T_1;T_2 \Downarrow CT1\\
       \Delta_1, \Omega_2 \turns E_2 : T_{2P} \leadsto J_2 \textbf{~in~} S_2, \Omega_2' ~~~~~~~\Omega = \Omega_1' \cup \Omega_2'\\
       T_{2P} \Downarrow T_{2P}' \turns TX = T_{2P}' \Omega~~~~~~~\\\Delta_1, \Omega \turns (E_2 : T_{2}) \Omega \leadsto WRAP~~~~~~~\\
       \Delta_2; f; T_1 \Downarrow T_3 \leadsto S_3
          }{
           \Delta_1, \Omega_1 \turns E_1 \, E_2 : T_{3} \Omega \leadsto f.out \textbf{~in~}
           S_1 \uplus S_2 \uplus \{CT1~f = (CT1)~J_1; \\
           TX exp2 = (TX) J_2; \\
           T_2 exp2wrap = (T_2) WRAP; \\
           f.a = exp2wrap; f.apply();\} \uplus S_3, \Omega
} \\ \\

  (\texttt{*}) & 

            {All~other~rules~preserve~\Omega}
 \\ \\

\ea

\eda
}






% Bibliography
\bibliographystyle{acmsmall}
\bibliography{papers}


% History dates
\received{February 2007}{March 2009}{June 2009}

\end{document}
